\documentclass[10pt,letterpaper]{article} 
\usepackage{tikz}
\usepackage{tools}
\usepackage{enumitem,caption}
\usepackage{listings}
\lstset{language=Python}
%\lstset{frame=lines}
%\lstset{caption={Insert code directly in your document}}
\lstset{label={lst:code_direct}}
\lstset{basicstyle=\footnotesize}

%\usepackage{graphicx}‎‎
%\usefonttheme{serif}‎
%\usepackage{ptext}‎
%\usepackage{xepersian}
%\settextfont{B Nazanin}
\usepackage{lipsum}
\setlength{\parindent}{0pt}
\newcommand{\pf}{$\blacksquare$}

\newcommand{\Span}{\text{Span}}
\newcommand{\NF}{\text{NF}}
\newcommand{\EDFA}{\text{EDFA}}
\newcommand{\ASE}{\text{ASE}}

\newcommand{\bns}{\textit{broadcast-and-select}  architecture}
\newcommand{\Bns}{\textit{Broadcast-and-select} architecture}

\newcommand{\rns}{\textit{route-and-select} architecture}
\newcommand{\Rns}{\textit{Route-and-select} architecture}

\newcounter{QuestionNumber}
\setcounter{QuestionNumber}{1}

\newcommand{\temp}{{\color{red}{temp}}}

\newcommand{\Q}{
\textbf{Question \theQuestionNumber)}
\stepcounter{QuestionNumber}
}
\newcommand{\EX}{\Bbb E}
\newcommand{\nl}{\newline\newline}
\begin{document}
\large
\begin{center}
In the name of beauty

The 7th problem set of Optical Networks course
\hl
\end{center}

\Q

\begin{enumerate}[label=\alph*-]
\item
Fiber birefringence is the phenomenon of periodic alteration of signal polarization (e.g. between linear and circular polarizations). The minimum length of fiber at which a full birefringence takes place and signal initial polarization is recovered is referred to as beat length.
\item
The RMSA (or more notably, the RMLSA) is the problem of joint allocation of path (as a sequence of nodes and links), spectrum (or wavelength) and modulation level to demands in network. This process can be performed phyiscal-layer-impairment-aware (PLI-aware), where the impairments of network components (such as optical fiber or switches) are considered as parameters of design.
\item
Dispersion is known to decrease the power peaks of signal thereby spreading signal in time. From a non-linear point of view, this behavior leads to non-linear mitigation to some extent, which is desirable. The advent of signal processing has made it much easier to perform dispersion compensation on signal.
\end{enumerate}

\Q

\begin{enumerate}[label=\alph*-]
\item
The netgains of links and spans are set to 1 through a full compensation of fiber loss with amplifiers. This compensation yields two results. First, the signal power remains unchanged throughout the transmission. Second, noises of active elements and of non-linear source accumulate (i.e. sum up) linearly at the end of transmission. So, if we denote the signal power by $P$ and noise variance at the end of the $i$th link and all the links by $\sigma^2_i$ and $\sigma^2_\text{total}$ respectively, we end up with
$$
\sigma^2=\sum_{i=1}^{N_l}\sigma^2_i,
$$
where dividing both sides on $P$, yields the expected result.
\item
The SNR of a signal with power $P$ can be expressed as
$$
\text{SNR}=\frac{P}{\sigma^2_\text{ASE}+\eta_\text{NLI}P^3},
$$
where the optimum SNR takes place at power $P^*$ at which
$$
\frac{d\text{SNR}}{dP}\Big|_{P=P^*}=0.
$$
A bit of calculus yields
$$
P^*=\sqrt[3]{\frac{\sigma^2_\text{ASE}}{2\eta_\text{NLI}}}
$$
and
$$
\text{SNR}_\text{opt}=\frac{2}{3}\frac{1}{\sqrt[3]{2\eta_\text{NLI}\sigma^4_\text{ASE}}}.
$$
\end{enumerate}

\Q (Linear solution of Manakov equation)

\begin{enumerate}[label=\alph*-]
\item
The Fourier transform of Manakov equation gives
$$
\frac{\partial A(z,\omega)}{\partial z}
=
\Gamma(\omega) A(z,\omega)
-
j\frac{8}{9}\gamma
(A_x(z,\omega)*A_x^*(z,-\omega)+A_y(z,\omega)*A_y^*(z,-\omega))*A(z,t)
,
$$
where
$$
\Gamma(\omega)=-\alpha-j\frac{\omega^2}{2}\beta_2
.
$$
\item
The Manakov equation after setting $\gamma=0$, gives a first-order linear differential equation as
$$
\frac{\partial A(z,\omega)}{\partial z}
=
\Gamma(\omega) A(z,\omega)
$$
with the following answer
$$
A(z,\omega)=C(\omega)e^{z\Gamma(\omega)},
$$
where setting $z=0$ we obtain $C(\omega)=A(0,\omega)$ and finally
$$
A(z,\omega)
=
A(0,\omega)
e^{z\Gamma(\omega)}
.
$$
By replacing $A(0,\omega)=(A_x(0,\omega),A_y(0,\omega))$, we obtain
$$
A(z,\omega)
=
e^{z\Gamma(\omega)}
\sum_i c_i\delta(f-f_i)
=
\sum_i c_ie^{z\Gamma(f_i)}\delta(f-f_i)
=
\sum_i d_i(z)\delta(f-f_i)
,
$$
where $d_i(z)=(d_{x,i}(z),d_{y,i}(z))$ and
$$
d_{x,i}(z)=c_{x,i}e^{z\Gamma(f_i)}
\quad,
\quad
d_{y,i}(z)=c_{y,i}e^{z\Gamma(f_i)}.
$$
\end{enumerate}

\Q (Non-linear solution of Manakov equation; perturbation approach)

\[\begin{split}
\text{Fourier transform of }
j\frac{8}{9}\gamma |A_\text{LIN}(z,t)|^2A_\text{LIN}(z,t)
\end{split}\]


\begin{enumerate}[label=\alph*-]
\item
\[\begin{split}
\text{Fourier transform of }
&
j\frac{8}{9}\gamma |A_\text{LIN}(z,t)|^2A_\text{LIN}(z,t)
\\&=
j\frac{8}{9}\gamma
\left[A_x(z,\omega)*A_x^*(z,-\omega)+A_y(z,\omega)*A_y^*(z,-\omega)\right]A(z,t)
\\&=
j\frac{8}{9}\gamma
\big[\sum_i d_{x,i}(z)\delta(f-f_i)*\sum_i d_{x,i}(z)^*\delta(f+f_i)
\\&+\sum_i d_{y,i}(z)\delta(f-f_i)*\sum_i d_{y,i}^*(z)\delta(f+f_i)\big]*\sum_i d_i(z)\delta(f-f_i)
\\&=
j\frac{8}{9}\gamma
\big[
\sum_{i,j} d_{x,i}(z)d^*_{x,j}(z)\delta(f-f_i+f_j)
\\&+
\sum_{i,j} d_{y,i}(z)d^*_{y,j}(z)\delta(f-f_i+f_j)
\big]*\sum_i d_i(z)\delta(f-f_i)
\\&=
j\frac{8}{9}\gamma
\big[
\sum_{i,j,k} d_{x,i}(z)d^*_{x,j}(z)d_k(z)\delta(f-f_i+f_j-f_k)
\\&+
\sum_{i,j,k} d_{y,i}(z)d^*_{y,j}(z)d_k(z)\delta(f-f_i+f_j-f_k)
\big]
\\&=
j\frac{8}{9}\gamma
\sum_{i,j,k} \big[d_{x,i}(z)d^*_{x,j}(z)d_k(z)
\\&+
d_{y,i}(z)d^*_{y,j}(z)d_k(z)\big]\delta(f-f_i+f_j-f_k).
\end{split}\]
By defining 
$
g_{i,j,k}(z)\triangleq
-j\frac{8}{9}\gamma
d_{x,i}(z)d^*_{x,j}(z)d_k(z)
+
d_{y,i}(z)d^*_{y,j}(z)d_k(z)
$
we can write
\[\begin{split}
\text{Fourier transform of }
&
j\frac{8}{9}\gamma |A_\text{LIN}(z,t)|^2A_\text{LIN}(z,t)
\\&=
-
\sum_{i,j,k} g_{i,j,k}(z)\delta(f-f_i+f_j-f_k).
\end{split}\]
\item
After evaluating the non-linear part of Manakov equation, we can write it in the following form:
$$
\frac{\partial A(z,\omega)}{\partial z}
=
\Gamma(\omega) A(z,\omega)
+Q_\text{NLI}(z,\omega),
$$
where
$$
Q_\text{NLI}(z,\omega)=
\sum_{i,j,k} g_{i,j,k}(z)\delta(f-f_i+f_j-f_k)
,
$$
therefore
\[\begin{split}
&
\frac{\partial A_\text{NLI}(z,\omega)}{\partial z}
=
\Gamma(\omega) A_\text{NLI}(z,\omega)
+Q_\text{NLI}(z,\omega)
\implies\\&
e^{-z\Gamma(\omega)}\frac{\partial A_\text{NLI}(z,\omega)}{\partial z}
-
\Gamma(\omega)e^{-z\Gamma(\omega)} A_\text{NLI}(z,\omega)
=e^{-z\Gamma(\omega)}Q_\text{NLI}(z,\omega)
\implies\\&
\frac{\partial}{\partial z}\left[e^{-z\Gamma(\omega)}A_\text{NLI}(z,\omega)\right]
=e^{-z\Gamma(\omega)}Q_\text{NLI}(z,\omega)
\implies
\\&
e^{-z\Gamma(\omega)}A_\text{NLI}(z,\omega)=
C(f)+
\int_0^z e^{-u\Gamma(\omega)}Q_\text{NLI}(u,\omega)du
\implies
\\&
e^{-z\Gamma(\omega)}A_\text{NLI}(z,\omega)=
A_\text{NLI}(0,\omega)+
\int_0^z e^{-u\Gamma(\omega)}Q_\text{NLI}(u,\omega)du
\\&
A_\text{NLI}(z,\omega)=
e^{z\Gamma(\omega)}\int_0^z e^{-u\Gamma(\omega)}Q_\text{NLI}(u,\omega)du
.
\end{split}\]
The last implication comes from the fact that we have no non-linear electrical field at the beginning of an optical fiber and hence $A_\text{NLI}(0,\omega)=0$.

By substituting the result of part (a) in the above equation we finally obtain
\[\begin{split}
A_\text{NLI}(z,\omega)&=
e^{z\Gamma(\omega)}\int_0^z e^{-u\Gamma(\omega)}Q_\text{NLI}(u,\omega)du
\\&=
e^{z\Gamma(\omega)}\int_0^z e^{-u\Gamma(\omega)}\sum_{i,j,k} g_{i,j,k}(u)\delta(f-f_i+f_j-f_k)du
\\&=
e^{z\Gamma(\omega)}
\sum_{i,j,k}
\int_0^z e^{-u\Gamma(\omega)}g_{i,j,k}(u)\delta(f-f_i+f_j-f_k)du
\\&=
e^{z\Gamma(\omega)}
\sum_{i,j,k}
\int_0^z e^{-u\Gamma(f_i-f_j+f_k)}g_{i,j,k}(u)\delta(f-f_i+f_j-f_k)du
\\&=
\sum_{i,j,k}
\left[\int_0^z e^{-u\Gamma(f_i-f_j+f_k)}g_{i,j,k}(u)du\right]
e^{z\Gamma(\omega)}\delta(f-f_i+f_j-f_k)
\\&=
\sum_{i,j,k}
\left[\int_0^z e^{-u\Gamma(f_i-f_j+f_k)}g_{i,j,k}(u)du\right]
e^{z\Gamma(f_i-f_j+f_k)}\delta(f-f_i+f_j-f_k),
\end{split}\]
from which by defining $h_{i,j,k}(z)\triangleq \left[\int_0^z e^{-u\Gamma(f_i-f_j+f_k)}g_{i,j,k}(u)du\right]
e^{z\Gamma(f_i-f_j+f_k)}$ we obtain
$$
A_\text{NLI}(z,f)=\sum_{i,j,k} h_{i,j,k}(z)\delta(f-f_i+f_j-f_k)
.
$$
This above equation introduces the concept of FWM, where a set of frequency components of signal lead to emerging residual frequencies of from $f_i-f_j+f_k$.
\end{enumerate}

\end{document}