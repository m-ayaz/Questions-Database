\documentclass[10pt,letterpaper]{article} 
\usepackage{tikz}
\usepackage{tools}
\usepackage{enumitem}
%\usepackage{graphicx}‎‎
%\usefonttheme{serif}‎
%\usepackage{ptext}‎
%\usepackage{xepersian}
%\settextfont{B Nazanin}
\usepackage{lipsum}
\setlength{\parindent}{0pt}
\setlength{\parskip}{3mm}
\newcommand{\pf}{$\blacksquare$}

\newcommand{\bns}{\textit{broadcast-and-select}  architecture}
\newcommand{\Bns}{\textit{Broadcast-and-select} architecture}

\newcommand{\rns}{\textit{route-and-select} architecture}
\newcommand{\Rns}{\textit{Route-and-select} architecture}

\newcounter{QuestionNumber}
\setcounter{QuestionNumber}{1}

\newcommand{\Q}{
\textbf{Question \theQuestionNumber)}
\stepcounter{QuestionNumber}
}
\newcommand{\EX}{\Bbb E}
\newcommand{\nl}{\newline\newline}
\begin{document}
\Large
\begin{center}
In the name of beauty

The 3rd problem set of Optical Networks course
\hl
\end{center}
%\theQuestionNumber
%\stepcounter{QuestionNumber}
%\theQuestionNumber
%\Q
%
%A fiber optical communication system employs a 1.3$\mu$m 1$n$GaAsP semiconductor as the transmitter. To minimize dispersion, a single-mode fiber is used.
%\begin{enumerate}[label=\alph*.]
%\item
%Determine the largest possible core diameter for the fiber with a core refractive index of $1.500$ and a cladding refractive index of $1.495$.
%\item
%How many modes can propagate in the fiber if the transmitter is replaced with an AlGaAs semiconductor laser with an emission wavelength of 0.85 $\mu $m or a HeNe-laser with an emission wavelength of 0.63 $\mu$m?
%\end{enumerate}
%For waveguide dispersion, make the (realistic) assumption that the material dispersion can be neglected ($d n\over d\omega$=0).
%\nl
%\Q
%
%A $1$ km long polarization-maintaining single-mode fiber exhibits a birefringence of $\Delta n =6\times 10^{-4}$. Calculate the differential group delay (DGD) for this fiber at $1.55\mu m$ assuming that the average mode index $\bar n = 1.45$ and ${d\bar n\over d\lambda}=-0.01 \mu m^{-1}$ at this wavelength.
%\nl
\Q

\begin{enumerate}[label=\alph*.]
\item
A single mode fiber is measured to have 
$
\lambda^2 {d^2 n\over d\lambda^2}=0.02
$
 at $0.8 \mu m$. Calculate the dispersion $\beta_2$ and $D$. Express your final results for $\beta_2$ and $D$ in units of $\text{ps}^2/\text{km}$ and ps/nm/km, respectively.
\item
Calculate the power launched into a 40-km-long single-mode fiber for which
the SPM-induced nonlinear phase shift becomes 180°. Assume $\lambda=1.55\mu m$,
$A_\text{eff} = 40 \mu m^2$, $\alpha = 0.2$ dB/km, and $\bar n_2 = 2.6 \times 10^{-20} m^2/W$.
%\item
%A 1.55-$\mu m$ continuous-wave signal with 6-dBm power is launched into a fiber with 50-$\mu m^2$ effective mode area ($A_\text{eff}$). After what fiber length would the nonlinear phase shift induced by SPM become $2\pi$? Assume $\bar n_2= 2.6 \times 10^{-20} m^2/W$ and neglect fiber losses.
\end{enumerate}

(The nonlinear phase shift induced by SPM ($\phi_\text{NL}$) can be obtained from
$
\phi_\text{NL}=\gamma P_\text{in}L_\text{eff}
$
where
$
\gamma={2\pi \bar n_2\over A_\text{eff}\cdot \lambda}
$.)

\Q

Assume that we use a fiber with $D=25 ps/(nm\cdot km)$, length $L$ and  attenuation $\alpha_\text{DCF}=0.21$ dB/km for transmission. To compensate dispersion at receiver, we make use of a DCF
\footnote{
Dispersion-Compensating Fiber
}
 with dispersion parameter $D_\text{DCF}=16 ps/(nm\cdot km)$, attenuation $\alpha_\text{DCF}=0.21$ dB/km and length $L_\text{DCF}$ which is employed after fiber 1. The DCF compensates the dispersion completely, hence 
$$
D\cdot L+D_\text{DCF}\cdot L_\text{DCF}=0
$$
The signal is allowed to drop $20\text{dB}$ before it is amplified.
Calculate the attenuation-limited transmission length, without amplifiers (Note that the total length over which the signal is attenuated is $L_\text{tot}=L_\text{DCF}+L_\text{Fiber}$). Calculate $L_\text{Fiber}$ and $L_\text{DCF}$ at this maximum transmission length.

\Q

A $0.88\mu m$ communications system transmits data over a $10km$ single-mode fiber by using $10ns$ (FWHM) Gaussian pulses. Determine the maximum bit rate if the light-emitting diode (LED) has a spectral FWHM of $30nm$ with Gaussian pulse shape. Use $D=-80 ps/(km\cdot nm)$.

(The maximum bit rate can be obtained from:
$$
BL|D|\sigma_\lambda\le{1\over 4}
$$
where $L$ is the fiber length, $\sigma_\lambda$ is the LED spectral FWHM and $D$ is the dispersion parameter.)

\Q

Consider the following form of NLSE
\footnote{
Non-Linear Shr\o dinger equation
}
 with $\gamma=0$:
$$
{\partial A\over \partial z}+{\alpha\over 2}A-j{\beta_2\over 2}\cdot{\partial^2 A\over \partial t^2}=n(t)
$$
where $n(t)$ denotes the noise process at the input of the optical fiber.
\begin{enumerate}[label=\alph*.]
\item
Show that the general solution in frequency domain to this differential equation, in case of $n(t)=0$ and at the end of a fiber of length $L$ is
$$
A(z,\omega)=A(0,\omega)\exp\left(-\left[{\alpha\over 2}+j{\beta_2\over 2}\omega^2\right]z\right)
$$
(Hint: consider the frequency domain of the NLSE)
\item
Assuming a non-zero noise process $n(t)$ with the Fourier transform of $N(f)$ is given at the input of the fiber. Show that the above PDE has the following solution for a fiber of length $L$
$$
A(z,\omega)=A(0,\omega)\exp\left(-\left[{\alpha\over 2}+j{\beta_2\over 2}\omega^2\right]z\right)+F(z,\omega)
$$
by finding $F(z,\omega)$ in terms of $\alpha$, $\beta_2$ and $N(f)$.
%Find $1\over e$-intensity spectral half width $\Delta \omega_0$ ($\Delta \omega_0$ should be measured where the intensity of the frequency domain $A(0,\omega)$ reaches $1\over e$ of its maximum, hence first try to find the Fourier transform of $A(0,t)$).
\end{enumerate}

\Q

We have two transmitters and two types of fiber:

\begin{itemize}
\item
Transmitter 1 : $\lambda=1.55\mu m$, Gaussian spectral shape, spectral width (RMS) $0.67 nm$, no chirp.

\item
Transmitter 2 : $\lambda=1.55\mu m$, negligible spectral width, no chirp.

\item
Fiber 1: $\alpha=0.21 \text{dB}/\text{km}$, $D=25 ps/(nm\cdot km)$

\item
Fiber 2: $\alpha=0.23 \text{dB}/\text{km}$, $D=0 ps/(nm\cdot km)$, $S=0.06 ps/(nm^2\cdot km)$
\end{itemize}
We are transmitting Gaussian pulses, $T_0=80 ps$, at a bit rate of $750$ Mbit/s.
\begin{enumerate}[label=\alph*.]
\item
Calculate the dispersion-limited transmission length for all combination of fibers and transmitters.
\item
If the signal is allowed to drop $20\text{dB}$ before it is amplified, calculate the attenuation-limited transmission length.
\item
Which combination (of the 4 possible combinations) gives the maximum transmission length, without amplifiers (considering both attenuation and dispersion)?
\end{enumerate}
(Hint:
 The dispersion parameter $D$ and dispersion slope $S$ are related to $\beta_2$ and $\beta_3$ as
$$
D=-\left({2\pi c\over \lambda^2}\right)\beta_2
$$
$$
S=\left({2\pi c\over \lambda^2}\right)^2\beta_3+\left({4\pi c\over \lambda^3}\right)\beta_2
$$
and the bit rate-transmission length product limited inequalities for calculating dispersion-limited transmission length are
$$
BL|D|\sigma_\lambda\le{1\over 4}\quad,\quad D\ne 0
$$
$$
BL|S|\sigma_\lambda^2\le{1\over \sqrt8}\quad,\quad D= 0
$$
for large spectral widths (of order $\sim nm$) and 
$$
B\sqrt{|\beta_2| L}\le{1\over 4}
$$
$$
B\cdot(|\beta_3| L)^{1\over 3}\le{0.324}
$$
for negligible spectral widths. Clearly, if a system bears both attenuation and dispersion, the limited transmission length is the minimum of dispersion-limited transmission length and attenuation-limited transmission length.)

\Q

For practical reasons such as finite memory considerations, continuous-domain implementation of a signal is impossible. To this reason, an analog signal (with continuity in both amplitude and time) is first converted to an equivalent digital signal (where discretization is performed again in both amplitude and time) and then, DSP simulates all the desired performance on a computer. More details are mentioned in your second simulation assignment on SSFM, but the key points are worth noting.

In a p2p transmission, multiple users, $M$ of them, wish to exploit a shared media for communication, each of which, transmitting a sequence of modulation symbols on a channel of index $m$. The tx. output of each user would be
$$
A_m(t)=\sum_{l=-\infty}^\infty b_{m,l}s(t-\frac{l}{R})e^{\imath 2\pi mR t},
$$
yielding the following total transmitter output,
$$
A(t)=\sum_{m=1}^MA_m(t)=\sum_{m=1}^M\sum_{l=-\infty}^\infty b_{m,l}s(t-\frac{l}{R})e^{\imath 2\pi mR t}.
$$
in which, $b_{m,k}$ is the sequence of symbols sent by the $m$-th user, $s(t)$ is the sinc shaping pulse with symbol rate of $R$ which is equal to the channel bandwidth. It is implicitly assmed that the $m$-th user is assigned the $m$-th channel of bandwith $R$ and center frequency $mR$.
\begin{enumerate}[label=\alph*.]
\item
For simplicity, assume $M=2$ and that each user sends only a symbol, giving the following tx. output,
$$
A(t)=\sum_{m=1}^2 b_{m,0}s(t)e^{\imath 2\pi mR t}.
$$
Derive and sketch the Fourier transform of $A(t)$ for $b_{1,0}=b_{2,0}=1$.
\item
For simulation purposes, the signal $A(t)$ is sampled at a rate of $F_s$ samples per second. How large should $F_s$ at least be to maintain the signal information without loss due to frequency aliasing? (Hint: Use Shannon-Nyquist criterion for this part.)
\item
Assume once again
$$
A(t)=\sum_{m=1}^2 b_{m,0}s(t)e^{\imath 2\pi mR t}.
$$
Two sampled versions of $A(t)$ are given with sampling frequencies $R$ and $2R$ as
$$
\hat A_1[k]=A(\frac{k}{R})\quad,\quad \hat A_2[k]=A(\frac{k}{2R}),
$$
where $k$ in the discrete time index. Plot the Fourier transforms of $\hat A_1[k]$ and $\hat A_2[k]$ and show by investigating the equation of $\hat A_1[k]$ that the modulation symbols of the two independent users are not uniquely recoverable due to aliasing.
%
%\item
%How much is the broadening ratio? (Divide the $1\over e$-intensity spectral half width of output to that to input). Find the optimum $T_{0,\text{opt}}$ for minimum broadening ratio.
\end{enumerate}
%\Q
%
%Show that a chirped Gaussian pulse is initially compressed inside a single-mode fiber when $\beta_2 C<0$. Derive expressions for the minimum width and the fiber length at which the minimum occurs.
%\nl
%\Q
%
%A $1.55\mu m$ unchirped Gaussian pulse of $50ps$ width (FWHM) is launched into a single-mode fiber. Calculate the FWHM after $50km$ if the fiber has a dispersion of $D=16 ps/(km\cdot nm)$.
%\nl
%\Q
%
%A $0.88\mu m$ communications system transmits data over a $10km$ single-mode fiber by using $10ns$ (FWHM) Gaussian pulses. Determine the maximum bit rate if the light-emitting diode (LED) has a spectral FWHM of $30nm$ with Gaussian pulse shape. Use $D=-80 ps/(km\cdot nm)$.
%
%(The maximum bit rate can be obtained from:
%$$
%BL|D|\sigma_\lambda\le{1\over 4}
%$$
%where $L$ is the fiber length, $\sigma_\lambda$ is the LED spectral FWHM and $D$ is the dispersion parameter.)
%\nl
%\Q
%
%We have two transmitters and two types of fiber:
%
%\begin{itemize}
%\item
%Transmitter 1 : $\lambda=1.55\mu m$, Gaussian spectral shape, spectral width (RMS) $0.67 nm$, no chirp.
%
%\item
%Transmitter 2 : $\lambda=1.55\mu m$, negligible spectral width, no chirp.
%
%\item
%Fiber 1: $\alpha=0.21 \text{dB}/\text{km}$, $D=25 ps/(nm\cdot km)$
%
%\item
%Fiber 2: $\alpha=0.23 \text{dB}/\text{km}$, $D=0 ps/(nm\cdot km)$, $S=0.06 ps/(nm^2\cdot km)$
%\end{itemize}
%We are transmitting Gaussian pulses, $T_0=80 ps$, at a bit rate of $750$ Mbit/s.
%\begin{enumerate}[label=\alph*.]
%\item
%Calculate the dispersion-limited transmission length for all combination of fibers and transmitters.
%\item
%If the signal is allowed to drop $20\text{dB}$ before it is amplified, calculate the attenuation-limited transmission length.
%\item
%Which combination (of the 4 possible combinations) gives the maximum transmission length, without amplifiers (considering both attenuation and dispersion)?
%\end{enumerate}
%(Hint:
% The dispersion parameter $D$ and dispersion slope $S$ are related to $\beta_2$ and $\beta_3$ as
%$$
%D=-\left({2\pi c\over \lambda^2}\right)\beta_2
%$$
%$$
%S=\left({2\pi c\over \lambda^2}\right)^2\beta_3+\left({4\pi c\over \lambda^3}\right)\beta_2
%$$
%and the bit rate-transmission length product limited inequalities for calculating dispersion-limited transmission length are
%$$
%BL|D|\sigma_\lambda\le{1\over 4}\quad,\quad D\ne 0
%$$
%$$
%BL|S|\sigma_\lambda^2\le{1\over \sqrt8}\quad,\quad D= 0
%$$
%for large spectral widths (of order $\sim nm$) and 
%$$
%B\sqrt{|\beta_2| L}\le{1\over 4}
%$$
%$$
%B\cdot(|\beta_3| L)^{1\over 3}\le{0.324}
%$$
%for negligible spectral widths. Clearly, if a system bears both attenuation and dispersion, the limited transmission length is the minimum of dispersion-limited transmission length and attenuation-limited transmission length.)
%\nl
%\Q
%
%Assume that we use a fiber with $D=25 ps/(nm\cdot km)$, length $L$ and  attenuation $\alpha_\text{DCF}=0.21$ dB/km for transmission. To compensate dispersion at receiver, we make use of a DCF
%\footnote{
%Dispersion-Compensating Fiber
%}
% with dispersion parameter $D_\text{DCF}=16 ps/(nm\cdot km)$, attenuation $\alpha_\text{DCF}=0.21$ dB/km and length $L_\text{DCF}$ which is employed after fiber 1. The DCF compensates the dispersion completely, hence 
%$$
%D\cdot L+D_\text{DCF}\cdot L_\text{DCF}=0
%$$
%Once again, the signal is allowed to drop $20\text{dB}$ before it is amplified.
%\begin{enumerate}[label=\alph*.]
%\item
%Calculate the attenuation-limited transmission length, without amplifiers (this question can be solved in a similar manner to part c of the previous question. Note that the total length over which the signal is attenuated is $L_\text{tot}=L_\text{DCF}+L_\text{Fiber}$).
%\item
%Given the fiber lengths in part a, assume that we also use an EDFA
%\footnote{
%Erbium-Doped Fiber Amplifier
%}
%with noise factor 5.5 dB to fully compensate the total fiber attenuation. How much is the ASE
%\footnote{
%Amplified Spontaneous Emission
%}
% noise power added to the signal by the EDFA?
%\end{enumerate}
%
%%\nl
%\Q
%
%A $1.3 \mu m$ lightwave system uses a 50-km fiber link and requires at least $0.3 \mu W$
%at the receiver. The fiber loss is 0.5 dB/km. Fiber is spliced every 5 km and has
%two connectors of 1-dB loss at both ends. Splice loss is only 0.2 dB. Determine
%the minimum power that must be launched into the fiber.
%
%(Hint: splices and connectors only impose attenuation to the signal power and have no other impairments, so they can be treated in a similar manner to amplifiers without noise.)
%%%\nl
%%\Q
%%
%%\begin{enumerate}[label=\alph*.]
%%\item
%%Show that when $N$ pairwise distinct signal channels propagate through an optical fiber, the number of generated FWM frequencies follows the relation $$M={N^2(N-1)\over 2}$$
%%(Hint: Consider a triplet of frequencies $f_i,f_j,f_k$. Enumerate the FWM frequencies once for $i\ne k$, $j\ne k$ and $i\ne j$ and once again for $i=j\ne k$)
%%\item
%%Provide all the possible FWM frequencies given 
%%$\lambda_1=1551.72 nm$, $\lambda_2=1552.52 nm$ and $\lambda_3=1553.32 nm$ launched into an optical fiber.
%%\end{enumerate}
%%\Q
%%
%%Consider the following form of Manakov's equation with $\gamma=0$:
%%$$
%%{\partial A\over \partial z}+{\alpha\over 2}A-j\gamma P\cdot A=0
%%$$
%%where $A=\begin{bmatrix}A_x\\A_y\end{bmatrix}$ and $P=|A_x|^2+|A_y|^2$. 
%%
%%This equation can be splitted to two scalar equations:
%%$$
%%{\partial A_x\over \partial z}+{\alpha\over 2}A_x-j\gamma P\cdot A_x=0
%%$$
%%$$
%%{\partial A_y\over \partial z}+{\alpha\over 2}A_y-j\gamma P\cdot A_y=0
%%$$
%%\begin{enumerate}[label=\alph*.]
%%\item
%%Show that the total power $P$, follows from a 1st-order differential equation as
%%$$
%%{\partial P\over \partial z}=-\alpha P
%%$$
%%which has the following general solution:
%%$$
%%P=P(z,t)=P(0,t)e^{-\alpha z}
%%$$
%%(Hint: for this part, proceed  by multiplying the two scalar differential equations in $E_x$ and $E_y$, respectively. Consider the two resulted differential equations with their conjugates)
%%\item
%%By substituting the general solution of part a. in the Manakov's equation, the differential equation can be re-written as
%%$$
%%{\partial A\over \partial z}+{\alpha\over 2}A-j\gamma P(0,t)e^{-\alpha z} A=0
%%$$
%%which is a 1st-order differential equation. Show that the general solution to this equation at the end of a fiber of length $L$ (thereby substituting $z=L$ in the general solution) is
%%$$
%%A(z,t)=A(0,t)\exp\left[-{\alpha\over 2}L+j\gamma P(0,t)L_{\text{eff}}\right]
%%$$
%%where $L_{\text{eff}}={1-e^{-\alpha L}\over \alpha}$. The quantity $\gamma P(0,t)L_{\text{eff}}$ stands for the non-linear phase shift.
%%\end{enumerate}
%%\Q
%%Calculate the non-linear phase shift expression derived in part b of the preceding problem for an input continuous signal $A(0,t)=\sqrt{P_0}\ \text{sech}\left({t\over T_0}\right)$ after it has propagated through a $25$ km-long fiber. Assume $\alpha=0.2$ dB/km, $\gamma=2 \  W^{-1}/km)$ and 5 ps pulses (FWHM) with a $20$ mW peak power.
%%\nl
%%\Q
%%
%%\begin{enumerate}[label=\alph*.]
%%\item
%%Calculate the power launched into a 40-km-long single-mode fiber for which
%%the SPM-induced nonlinear phase shift becomes 180°. Assume $\lambda=1.55\mu m$,
%%$A_\text{eff} = 40 \mu m^2$, $\alpha = 0.2$ dB/km, and $\bar n_2 = 2.6 \times 10^{-20} m^2/W$.
%%\item
%%A 1.55-$\mu m$ continuous-wave signal with 6-dBm power is launched into a fiber with 50-$\mu m^2$ effective mode area ($A_\text{eff}$). After what fiber length would the nonlinear phase shift induced by SPM become $2\pi$? Assume $\bar n_2= 2.6 \times 10^{-20} m^2/W$ and neglect fiber losses.
%%\end{enumerate}
%%
%%(The nonlinear phase shift induced by SPM ($\phi_\text{NL}$) can be obtained from
%%$$
%%\phi_\text{NL}=\gamma P_\text{in}L_\text{eff}
%%$$
%%where
%%$$
%%\gamma={2\pi \bar n_2\over A_\text{eff}\cdot \lambda}
%%$$
%%)
\end{document}