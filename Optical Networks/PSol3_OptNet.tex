\documentclass[10pt,letterpaper]{article} 
\usepackage{tikz}
\usepackage{tools}
\usepackage{enumitem}
%\usepackage{graphicx}‎‎
%\usefonttheme{serif}‎
%\usepackage{ptext}‎
%\usepackage{xepersian}
%\settextfont{B Nazanin}
\usepackage{lipsum}
\setlength{\parindent}{0pt}
\newcommand{\pf}{$\blacksquare$}

\newcommand{\bns}{\textit{broadcast-and-select}  architecture}
\newcommand{\Bns}{\textit{Broadcast-and-select} architecture}

\newcommand{\rns}{\textit{route-and-select} architecture}
\newcommand{\Rns}{\textit{Route-and-select} architecture}

\newcounter{QuestionNumber}
\setcounter{QuestionNumber}{1}

\newcommand{\Q}{
\textbf{Question \theQuestionNumber)}
\stepcounter{QuestionNumber}
}

\newcommand{\EX}{\Bbb E}
\newcommand{\nl}{\newline\newline}
\begin{document}
\Large
\begin{center}
In the name of beauty

The 3rd problem set solution of Optical Networks course
\hl
\end{center}
\Q

a. The numerial aperture is given by
$$
\text{NA}=\sqrt{n_1^2-n_2^2}
$$
with $n_1=1.5$ and $n_2=1.4950$, we obtain NA=0.12 .

b.
$$
BL< {n_2\over n_1^2}{c\over \Delta}
$$
where 
$$
\Delta={\text{NA}^2\over 2n_1^2}
$$
Substituting yields
$$
L<6.23 \text{km}
$$
\nl
\Q

Note that
$$
D={\lambda\over c}{d^2n\over d\lambda^2}=-{2\pi c \over\lambda^2}\beta_2
$$
hence
$$
D=83 \text{ ps}/(\text{nm}\cdot\text{km})
$$
and 
$$
\beta_2=-28.29 \text{ ps}^2/\text{km}
$$
\Q

a. The FT of the partial differential equation is 
\eqn{
{\partial A(z,\omega)\over \partial z}&=-{\alpha\over 2}A(z,\omega)+j{\beta_2\over 2}\omega^2A(z,\omega)
\\&=[-{\alpha\over 2}+j{\beta_2\over 2}\omega^2]A(z,\omega)
}{}
which has the following solution
\eqn{
A(z,\omega)=C_1e^{-{\alpha\over 2}z+j{\beta_2\over 2}\omega^2z}
=A(0,\omega)e^{-{\alpha\over 2}z+j{\beta_2\over 2}\omega^2z}
}{}
The chirped part of the exponent is not considered since it does not influence the amplitude, therefore we must have
$$
-{1\over 2}({t\over T_0})^2=-{1\over 2}\implies t=\pm T_0
$$

b. The FT of $A(0,t)$ can be written as:
$$
A(0,\omega)=\sqrt{2\pi T_0^2\over 1+iC}\exp\left(-{\omega^2T_0^2\over 2(1+iC)}\right)
$$
The $1\over e$-intensity condition yields:
$$
\left|\exp\left(-{\omega^2T_0^2\over 2(1+iC)}\right)\right|={1\over \sqrt e}
$$
since when the spectral intensity increases by a factor of $\alpha$, the amplitude of the original pulse increases by a factor of $\sqrt\alpha$. The latter equality gives us
$$
\Delta\omega_0={\sqrt{1+C^2}\over T_0}
$$
\Q

At the end of the fiber we obtain:
\eqn{
A(z,\omega)&=A(0,\omega)e^{-{\alpha\over 2}z+j{\beta_2\over 2}\omega^2z}
\\&=A_0e^{-{\omega^2T_0^2\over 2+2jC}}e^{-{\alpha\over 2}L+j{\beta_2\over 2}\omega^2L}
\\&=
A_0e^{-{\alpha\over 2}L}e^{-{\omega^2T_0^2\over 2+2jC}}e^{j{\beta_2\over 2}\omega^2L}
\\&=
A_0e^{-{\alpha\over 2}L}e^{-{\omega^2\over 2}({T_0^2\over 1+jC}-j\beta_2L)}
\\&=
A_0e^{-{\alpha\over 2}L}e^{-{\omega^2\over 2}
{T_0^2-j\beta_2L+\beta_2CL\over 1+jC}
}
\\&=
A_0e^{-{\alpha\over 2}L}e^{-{\omega^2\over 2}
{(T_0^2+\beta_2CL)^2+(\beta_2L)^2\over (1+jC)(T_0^2+j\beta_2L+\beta_2CL)}
}
\\&=
A_0e^{-{\alpha\over 2}L}e^{-{\omega^2\over 2}
{(T_0^2+\beta_2CL)^2+(\beta_2L)^2\over T_0^2(1+jC_1)}
}
}{}
hence 
$$
T_1=T_0\sqrt{\left(1+{\beta_2CL\over T_0^2}\right)^2+\left({\beta_2L\over T_0^2}\right)^2}
$$
and 
$$
C_1=C+{\beta_2 L\over T_0^2}(1+C^2)
$$
\Q

The compression factor is
$$
\sqrt{\left(1+{\beta_2CL\over T_0^2}\right)^2+\left({\beta_2L\over T_0^2}\right)^2}
=
\sqrt{1+{2\beta_2CL\over T_0^2}+{(\beta_2C)^2L^2+\beta_2^2L^2\over T_0^4}}
$$
Since $|\beta_2C|<<1$ and $\left|{\beta_2C\over T_0^2}\right|<<1$, the term ${2\beta_2CL\over T_0^2}$ becomes dominant. Hence
$$
{T_1\over T_0}\approx \sqrt{1+{2\beta_2CL\over T_0^2}}
$$
which is greater than $1$ for $\beta_2 C>0$ and less than 1 for $\beta_2C<0$.

For finding the optimum length at which the width is minimized, we must differentiate the compressing factor w.r.t. $L$ as the fiber length. The zero-derivation equation is:
$$
C\left(1+{\beta_2CL\over T_0^2}\right)+\left({\beta_2L\over T_0^2}\right)=0
$$
which yields
$$
L_\text{opt}=-{T_0^2C\over \beta_2 (1+C^2)}
$$
which is valid for $\beta_2 C<0$. The minimum width is then derived as
$$
T_{1,\min}=T_0{1\over \sqrt {1+C^2}}
$$
\nl
\Q

Based on the equation 3.2.8 of the Agrawal's textbook, we have 
$$
T_\text{FWHM}=2\sqrt{\ln 2}T_0
$$
hence
\qn{
&T_0=30\text{ps}
\\& L=50 \text{km}
\\&\beta_2=-20.4 \text{ps}^2/\text{km}
\\&C=0
}
and we obtain
$$
T_{1,\text{FWHM}}=73.03 \text{ps}
$$
%$$D={-2\pi c\over\lambda^2}\beta_2$$
%$$
%\beta_2=-2.04\times 10^{-26}
%$$

\Q

\qn{
& L=10\text{ km}
\\&\sigma_\lambda=30\text{ nm}
\\& D=-80 \text{ ps}/(\text{nm}\cdot\text{km})
}
therefore
$$
B_{\max}=10.42MHz
$$

\Q

a.
\qn{
&\text{Fiber 1 : }\beta_2=-31.87 \text{ps}^2/\text{km}
\\&\text{Fiber 2 : }\beta_3=0.0992 \text{ps}^3/\text{km}
}

\qn{
&\text{Transmitter 1 , Fiber 1 : }L_\text{Dispersion}=19.9 \text{ km}
\\&\text{Transmitter 1 , Fiber 2 : }L_\text{Dispersion}=17502 \text{ km}
\\&\text{Transmitter 2 , Fiber 1 : }L_\text{Dispersion}=3486 \text{ km}
\\&\text{Transmitter 2 , Fiber 2 : }L_\text{Dispersion}=812717419\text{ km}
}

b.

\qn{
&\text{Transmitter 1 , Fiber 1 : }L_\text{Attenuation}=95.24 \text{ km}
\\&\text{Transmitter 1 , Fiber 2 : }L_\text{Attenuation}=86.96 \text{ km}
\\&\text{Transmitter 2 , Fiber 1 : }L_\text{Attenuation}=95.24 \text{ km}
\\&\text{Transmitter 2 , Fiber 2 : }L_\text{Attenuation}=86.96\text{ km}
}

c.

The combinations
\qn{
&\text{Transmitter 1 , Fiber 2}
\\&\text{Transmitter 2 , Fiber 2}
}
lead to maximum optical reach due to both attenuation and dispersion, i.e, when fiber 2 is used.
\nl
\Q

a. The transmission length should be $95.24$ km, just like part c- of the previous question. Since we have
$$
25L=16L_{\text{DCF}}
$$
we obtain
\qn{
& L=37.17\text{ km}
\\& L_\text{DCF}=58.07\text{ km}
}
b.

If signal is dropped 20 dB due to total attenuation, we need an amplifier of gain 20 dB to fully compensate for the fiber loss. Hence:
\qn{
&G_\text{dB}=20\implies G=100
\\&\text{NF}=5.5\implies F=3.55
}
By using the following relation for ASE noise power spectral density:
$$
\sigma^2_{\text{ASE,PSD}}={1\over 2}h\nu_\text{opt} GF
$$
where $\nu_\text{opt}={c\over \lambda}$, we finally calculate
 $$
\sigma^2_{\text{ASE,PSD}}=22.69 {\mu\text{W}\over\text{THz}}
$$
\Q

The total splice loss is $1+1+0.2=2.2$ dB. Also, the total fiber loss can be given by $50 \text{km}\times 0.5{\text{dB}\over \text{km}}=25\text{dB}$. Since there are 9 intermediate splices, the total loss imposed on signal is
$$
\text{Loss}=9\times 2.2+25=44.8\text{dB}
$$
The minimum sensitivity power is $0.3\mu\text{W}\equiv -5.23\text{dBu}$, giving the least launch power as 
$$
P=-5.23 \text{dBu}+44.8\text{dB}=39.57\text{dBu}=9.57\text{dBm}\equiv 9.06 \text{mW}
$$
\end{document}