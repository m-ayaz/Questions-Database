\documentclass[10pt,letterpaper]{article} 
%\usepackage{tikz}
%\usepackage{tools}
\usepackage{amsmath,amssymb,geometry,graphicx,enumitem}
%\usefonttheme{serif}‎
%\usepackage{ptext}‎
\usepackage{xepersian}
%\settextfont{B Nazanin}
%\usepackage{lipsum}
\setlength{\parindent}{0mm}
\setlength{\parskip}{3mm}
%\newcommand{\pf}{$\blacksquare$}
\newcounter{questionnumber}
\setcounter{questionnumber}{1}
\newcommand{\Q}{
\textbf{سوال \thequestionnumber)}
\stepcounter{questionnumber}
}
\newcommand{\eqn}[1]{
\[
\begin{split}
#1
\end{split}
\]
}
%\newcommand{\EX}{\Bbb E}
%\newcommand{\nl}{\newline\newline}


\begin{document}
\large
\begin{center}
به نام زیبایی

پاسخ تمرینات سری دوم درس شبکه های مخابرات نوری

\hrulefill
\end{center}

\Q

برای فیبری به طول 50 کیلومتر و تضعیف 
$
0.5\text{dB/km}
$،
مقدار تلفات کل برابر 
$
25\text{dB}
$
خواهد بود. از طرفی، در هر 5 کیلومتر طول فیبر، یک مفصل قرار دارد که دارای تلفات
$
0.2\text{dB}
$
است. چون تعداد این مفصل ها برابر 9 است، در نتیجه، تلفات کلی که به ترتیب از فیبر، مفصل های وسط فیبر و کانکتورهای دو سر فیبر بر سیگنال تحمیل می‌شود برابر است با
$
25\text{dB}
$،
$
1.8\text{dB}
$
و
$
2\text{dB}
$.
بنابراین، کل تلفات تحمیلی بر سیگنال، برابر است با
$
28.8\text{dB}
$.
از آنجا که توان ورودی گیرنده باید از آستانه حساسیت 
$
0.3\mu\text{W}\equiv -35.23\text{dBm}
$
بیشتر باشد، در نتیجه، حداقل توان مجاز جهت ارسال سیگنال برابر
$
-35.23\text{dBm}+28.8\text{dB}=-6.43\text{dBm}
$
خواهد بود.

\Q

الف) داریم:
\eqn{
\lambda&=1.55\times 10^{-6}
\\
A_{eff}&=40\times 10^{-12}
\\
\bar n_2&=2.6\times 10^{-20}
\\
L_{eff}&=\frac{1-e^{-\alpha L}}{\alpha}=18273.
\\
\phi_{NL}&=\pi
}
بنابراین، طبق رابطه‌ی 
$
\gamma=\frac{2\pi \bar n_2}{A_{eff}\cdot\lambda}
$
داریم
$
\gamma=2.63\times 10^{-3}
$.
پس، از رابطه‌ی 
$
\phi_{NL}=\gamma P_{in}L_{eff}
$،
مقدار
$
P_{in}
$
به صورت زیر به دست می آید:
$$
P_{in}=\frac{\phi_{NL}}{\gamma L_{eff}}=0.065\text{W}.
$$

ب) مشابه قسمت قبل، برای 
$
\gamma
$
به مقدار
$
2.11\times 10^{-3}
$
می رسیم. در حالتی که فیبر بدون تلفات باشد، 
$
L_{eff}=L
$.
بنابراین، با جایگذاری در رابطه‌ی 
$
\phi_{NL}=\gamma P_{in}L_{eff}
$
خواهیم داشت:
$$
L=L_{eff}=\frac{\phi_{NL}}{P_{in}\gamma}=748.74\text{km}.
$$

\Q

مجموع تلفات فیبر های 
SMF
و
NZ-DSF
عبارتست از
$
\alpha_{SMF}L_{SMF}+\alpha_{NZ-DSF}L_{NZ-DSF}
$.
این مقدار باید حداکثر برابر
20dB
باشد. بنابراین
$$
0.21[L_{SMF}(km)+L_{NZ-DSF}(km)]=20.
$$
در نتیجه،
$$
L_{SMF}+L_{NZ-DSF}=95.24km.
$$
از طرفی، پاشندگی باید به طور کامل جبران شود و از شرط جبرانسازی پاشندگی خواهیم داشت:
$$
25L_{SMF}=16L_{NZ-DSF}
.
$$
با حل این دو معادله، طول فیبرها به صورت زیر است:
\eqn{
&L_{SMF}=37.17km
\\&L_{NZ-DSF}=58.07km
.
}

ب) نویز EDFA از رابطه‌ی 
$
\sigma^2=h\nu GF
$
به دست می‌آید که
$
h
$
ثابت پلانک ($6.626\times 10^{-34}$)، 
$
\nu
$
فرکانس حامل‌های نوری،
$
F
$
عدد نویز تقویت کننده و 
$
G
$
بهره‌ی تقویت کننده است. فرکانس حامل‌های نوری در طول موج 1550 نانومتر عبارت است از
$$
\nu=\frac{c}{\lambda}=\frac{3\times10^8}{1550\times 10^{-9}}=193.55THz
.
$$
همچنین، بهره‌ی تقویت کننده، با تلفات فیبرها برابر است. بنابراین
$
G=20dB\equiv 100
$.
مقدار عدد نویز تقویت کننده عبارت است از
$
F=5.5dB\equiv 3.55
$
.
بنابراین، اندازه‌ی طیف توان نویز تقویت کننده برابر است با
$$
45.53\frac{\mu W}{THz}.
$$

\Q

فرض کنیم بهره‌ی تقویت کننده‌های Raman و EDFA، به ترتیب برابر
$
G_{Raman}
$
و
$
G_{EDFA}
$
باشد. به دلیل آن که این دو تقویت کننده باید تلفات فیبر را به طور کامل جبران کنند، در نتیجه 
$$
G_{Raman}(dB)+G_{EDFA}(dB)=20.
$$
از طرفی، عدد نویز کلی ناشی از دو تقویت کننده از رابطه‌ی زیر به دست می‌آید:
\eqn{
NF=NF_{Raman}+\frac{NF_{EDFA}-1}{G_{Raman}}.
}
مقدار 
$
NF_{EDFA}
$
به طور ثابت برابر
$
6dB\equiv 4
$
و رابطه‌ی
$
NF_{Raman}
$
با بهره‌ی تقویت کننده به صورت 
$
10^{-0.025G_{Raman}(dB)+2.425}
$
است. در نتیجه
\eqn{
NF=
10^{-0.025G_{Raman}(dB)+2.425}
+3\cdot 10^{-0.1G_{Raman}(dB)}.
}
واضح است که این تابع بر حسب 
$
G_{Raman}
$
نزولی اکید است. در نتیجه کمترین مقدار آن به ازای بیشترین مقدار 
$
G_{Raman}
$
که 
18dB
است، اتفاق می افتد. بنابراین، مقدار بهره‌ی 
EDFA
برابر
2dB
و مقدار عدد نویز کلی برابر 
19.75dB
خواهد بود.

\Q

الف) از آنجا که بهره‌ی کلی تمام لینک ها برابر 1 است، توان سیگنال در تمام لینک ها با هم برابر است. همچنین، واریانس نویز ناشی از تمام لینک ها، در انتهای آخرین لینک با یکدیگر جمع می شوند. بنابراین، واریانس نویز کلی‌ای که به سیگنال تحمیل می شود برابر است با
$$
\sigma^2=\sum_{i=1}^N\sigma^2_i
$$
که
$
\sigma^2_i
$،
توان نویز تحمیلی به سیگنال توسط لینک 
$
i
$
ام است. از طرفی، به دلیل اینکه توان سیگنال در خروجی تمام لینک ها با هم برابر است، مقدار SNR ارسال در انتهای لینک 
$
i
$
ام و همچنین، در انتهای تمام لینک ها به ترتیب برابر 
$
\frac{P}{\sigma^2_i}
$
و
$
\frac{P}{\sigma^2}
$
خواهد بود. بنابراین
\eqn{
\frac{1}{SNR}&=\frac{\sigma^2}{P}=\frac{1}{P}\sum_{i=1}^N\sigma^2_i=\sum_{i=1}^N\frac{\sigma^2_i}{P}=\sum_{i=1}^N\frac{1}{SNR_i}.
}

ب) رابطه‌ی 
SNR
به صورت زیر است:
\eqn{
SNR=\frac{P}{\sigma^2_{ASE}+\eta_{NLI}P^3}
.
}
برای به دست آوردن بیشترین SNR، باید مشتق SNR نسبت به $P$ را برابر صفر قرار دهیم. در نتیجه،
\eqn{
&\frac{d SNR}{dP}=0\implies
\sigma^2_{ASE}+\eta_{NLI}P^3-3\eta_{NLI}P^3=0
\\&\implies
\sigma^2_{ASE}-2\eta_{NLI}P^3=0
\implies
P_{opt}=\sqrt[3]{\frac{\sigma^2_{ASE}}{2\eta_{NLI}}}.
}


\end{document}