\documentclass[10pt,letterpaper]{article} 
\usepackage{tikz}
\usepackage{tools}
\usepackage{enumitem}
%\usepackage{graphicx}‎‎
%\usefonttheme{serif}‎
%\usepackage{ptext}‎
%\usepackage{xepersian}
%\settextfont{B Nazanin}
\usepackage{lipsum}
\setlength{\parindent}{0pt}
\newcommand{\pf}{$\blacksquare$}

\newcommand{\bns}{\textit{broadcast-and-select}  architecture}
\newcommand{\Bns}{\textit{Broadcast-and-select} architecture}

\newcommand{\rns}{\textit{route-and-select} architecture}
\newcommand{\Rns}{\textit{Route-and-select} architecture}

\newcounter{QuestionNumber}
\setcounter{QuestionNumber}{1}

\newcommand{\Q}{
\textbf{Question \theQuestionNumber)}
\stepcounter{QuestionNumber}
}

\newcommand{\EX}{\Bbb E}
\newcommand{\nl}{\newline\newline}
\begin{document}
\Large
\begin{center}
In the name of beauty

The 4th problem set solution of Optical Networks course
\hl
\end{center}
\Q

a.
 Given $n$ distinct frequencies $f_1,f_2,\cdots ,f_n$, the FWM frequencies may be produced through the relation $f_1+f_2-f_3$ when all $f_1$, $f_2$ and $f_3$ are not simultaneously equal. Picking up two distinct frequencies $f_1$ and $f_2$ from the set in $\binom{n}{2}$ ways, the possible FWM cases are
\qn{
&2f_1-f_2
\\&2f_2-f_1
}
which yields $2\binom{n}{2}$ different cases. When all the three frequencies are different in $\binom{n}{3}$, we obtain the following cases for FWM:
\qn{
&f_1+f_2-f_3
\\&f_1+f_3-f_2
\\&f_3+f_2-f_1
}
with a total of $3\binom{n}{3}$ different cases. Summing up, leaves us with $2\binom{n}{2}+3\binom{n}{3}={n^2(n-1)\over 2}$ total possible FWM frequency components.

b. 
\qn{
&f_1=193.2THz\quad f_2=193.2THz\quad f_3=193.1THz\quad:\quad f_{\text{FWM}}=193.3THz
\\&f_1=193.2THz\quad f_2=193.2THz\quad f_3=193.0THz\quad:\quad f_{\text{FWM}}=193.4THz
\\&f_1=193.1THz\quad f_2=193.2THz\quad f_3=193.0THz\quad:\quad f_{\text{FWM}}=193.3THz
\\&f_1=193.1THz\quad f_2=193.1THz\quad f_3=193.2THz\quad:\quad f_{\text{FWM}}=193.0THz
\\&f_1=193.1THz\quad f_2=193.1THz\quad f_3=193.0THz\quad:\quad f_{\text{FWM}}=193.2THz
\\&f_1=193.0THz\quad f_2=193.2THz\quad f_3=193.1THz\quad:\quad f_{\text{FWM}}=193.1THz
\\&f_1=193.0THz\quad f_2=193.1THz\quad f_3=193.2THz\quad:\quad f_{\text{FWM}}=192.9THz
\\&f_1=193.0THz\quad f_2=193.0THz\quad f_3=193.2THz\quad:\quad f_{\text{FWM}}=192.8THz
\\&f_1=193.0THz\quad f_2=193.0THz\quad f_3=193.1THz\quad:\quad f_{\text{FWM}}=192.9THz
}
\Q

a. By multiplying the first equation in $A_x^*$, the second one in $A_y^*$ and considering their complex conjugates, we obtain four equations:

\qn{
&A_x^H{\partial A_x\over \partial z}+{\alpha\over 2}A_x^HA_x-j\gamma PA_x^HA_x=0
\\
&A_x^T{\partial A_x^*\over \partial z}+{\alpha\over 2}A_x^TA_x^*+j\gamma PA_x^TA_x^*=0
\\
&A_y^H{\partial A_y\over \partial z}+{\alpha\over 2}A_y^HA_y-j\gamma PA_y^HA_y=0
\\
&A_y^T{\partial A_y^*\over \partial z}+{\alpha\over 2}A_y^TA_y^*+j\gamma PA_y^TA_y^*=0
}
where $(\cdot)^T$ and $(\cdot)^H$ denote the transpose and Hermitian (transpose+complex conjugate) operators.

By summing up all the equations and substituting $P=|A_x|^2+|A_y|^2$, the imaginary parts of the PDEs vanish and we finally obtain what we want:
$$
{\partial P\over \partial z}=-\alpha P
$$
with the following solution:
$$
P=P(z,t)=P(0,t)e^{-\alpha z}
$$

b. The PDE can be re-written as
\qn{
&{\partial A_x\over A_x\cdot \partial z}+{\alpha\over 2}-j\gamma P(0,t)e^{-\alpha z} =0
\\&{\partial A_y\over A_y\cdot \partial z}+{\alpha\over 2}-j\gamma P(0,t)e^{-\alpha z} =0
}
which by integration w.r.t. $x$ and $y$ respectively yields
\qn{
&
\ln A_x+{\alpha\over 2}z+j{\gamma\over \alpha} P(0,t)e^{-\alpha z}+C_1=0
\\&
\ln A_y+{\alpha\over 2}z+j{\gamma\over \alpha} P(0,t)e^{-\alpha z}+C_2=0
}
or equivalently
\qn{
&
A_x=e^{-{\alpha\over 2}z-j{\gamma\over \alpha} P(0,t)e^{-\alpha z}+C_1}
\\&
A_y=e^{-{\alpha\over 2}z-j{\gamma\over \alpha} P(0,t)e^{-\alpha z}+C_2}
}
Substituting $z=0$ leads to 
\qn{
&
A_x(0,t)=e^{-j{\gamma\over \alpha} P(0,t)+C_1}
\\&
A_y(0,t)=e^{-j{\gamma\over \alpha} P(0,t)+C_2}
}
By finding and replacing the constants $C_1$ and $C_2$, the result is immediately concluded $\blacksquare$.
\nl
\Q

The linear value of $\alpha$ is given by
$$
\alpha_\text{Linear}={\alpha_{dB}\over 4.343}=4.61\times 10^{-5} {1\over \text{m}}
$$
hence
\qn{
L_{\text{eff}}={1-\exp(-\alpha L)\over \alpha}=14.85 \text{km}
}
and by substituting, we obtain
\qn{
\phi_{\text{NL}}=\gamma L_{\text{eff}}P=0.59\text{rad}=34.03^\circ
}
\nl
\Q

a.
\qn{
&\gamma=2.63 {\text{W}^{-1}\over \text{km}}
\\& L_{\text{eff}}=18.27 \text{km}
\\& \phi_{\text{NL}}=\pi
}
therefore
$$
P=65.38 \text{mW}\equiv 18.15 \text{dBm}
$$

b.
\qn{
&\gamma=2.11 {\text{W}^{-1}\over \text{km}}
\\& L_{\text{eff}}=L
\\& \phi_{\text{NL}}=2\pi
\\&P=6\text{dBm}\equiv 3.98\text{mW}
}
which yield
$$
L=748\text{km}
$$
\end{document}