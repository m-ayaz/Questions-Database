\documentclass[10pt,letterpaper]{article} 
\usepackage{tikz}
\usepackage{tools}
\usepackage{enumitem}
%\usepackage{graphicx}‎‎
%\usefonttheme{serif}‎
%\usepackage{ptext}‎
%\usepackage{xepersian}
%\settextfont{B Nazanin}
\usepackage{lipsum}
\setlength{\parindent}{0pt}
\setlength{\parskip}{3mm}
\newcommand{\pf}{$\blacksquare$}

\newcommand{\bns}{\textit{broadcast-and-select}  architecture}
\newcommand{\Bns}{\textit{Broadcast-and-select} architecture}

\newcommand{\rns}{\textit{route-and-select} architecture}
\newcommand{\Rns}{\textit{Route-and-select} architecture}

\newcounter{QuestionNumber}
\setcounter{QuestionNumber}{1}

\newcommand{\Q}{
\textbf{Question \theQuestionNumber)}
\stepcounter{QuestionNumber}
}
\newcommand{\EX}{\Bbb E}
\newcommand{\nl}{\newline\newline}
\begin{document}
\Large
\begin{center}
In the name of beauty

The 3rd problem set solution of Optical Networks course
\hl
\end{center}
%\theQuestionNumber
%\stepcounter{QuestionNumber}
%\theQuestionNumber
%\Q
%
%A fiber optical communication system employs a 1.3$\mu$m 1$n$GaAsP semiconductor as the transmitter. To minimize dispersion, a single-mode fiber is used.
%\begin{enumerate}[label=\alph*.]
%\item
%Determine the largest possible core diameter for the fiber with a core refractive index of $1.500$ and a cladding refractive index of $1.495$.
%\item
%How many modes can propagate in the fiber if the transmitter is replaced with an AlGaAs semiconductor laser with an emission wavelength of 0.85 $\mu $m or a HeNe-laser with an emission wavelength of 0.63 $\mu$m?
%\end{enumerate}
%For waveguide dispersion, make the (realistic) assumption that the material dispersion can be neglected ($d n\over d\omega$=0).
%\nl
%\Q
%
%A $1$ km long polarization-maintaining single-mode fiber exhibits a birefringence of $\Delta n =6\times 10^{-4}$. Calculate the differential group delay (DGD) for this fiber at $1.55\mu m$ assuming that the average mode index $\bar n = 1.45$ and ${d\bar n\over d\lambda}=-0.01 \mu m^{-1}$ at this wavelength.
%\nl
\Q

\begin{enumerate}[label=\alph*.]
\item
Note that
$$
D={\lambda\over c}{d^2n\over d\lambda^2}=-{2\pi c \over\lambda^2}\beta_2
$$
hence
$$
D=83 \text{ ps}/(\text{nm}\cdot\text{km})
$$
and 
$$
\beta_2=-28.29 \text{ ps}^2/\text{km}
$$
\item
\qn{
&\gamma=2.63 {\text{W}^{-1}\over \text{km}}
\\& L_{\text{eff}}=18.27 \text{km}
\\& \phi_{\text{NL}}=\pi
}
therefore
$$
P=65.38 \text{mW}\equiv 18.15 \text{dBm}
$$
%\item
%A 1.55-$\mu m$ continuous-wave signal with 6-dBm power is launched into a fiber with 50-$\mu m^2$ effective mode area ($A_\text{eff}$). After what fiber length would the nonlinear phase shift induced by SPM become $2\pi$? Assume $\bar n_2= 2.6 \times 10^{-20} m^2/W$ and neglect fiber losses.
\end{enumerate}

\Q

For full dispersion compensation we must have
$$
25L_\text{Fiber}=16L_\text{DCF}
$$
and due to attenuation we must have
$$
L_\text{Fiber}+L_\text{DCF}=20\text{dB}/0.21\text{dB/km}=95.24\text{km},
$$
hence
$
L_\text{Fiber}=37.17\text{km}\quad,\quad
L_\text{DCF}=58.07\text{km}
.
$

\Q

\qn{
& L=10\text{ km}
\\&\sigma_\lambda=30\text{ nm}
\\& D=-80 \text{ ps}/(\text{nm}\cdot\text{km})
}
therefore
$$
B_{\max}=10.42MHz
$$

\Q

%Consider the following form of NLSE
%\footnote{
%Non-Linear Shr\o dinger equation
%}
% with $\gamma=0$:
%$$
%{\partial A\over \partial z}+{\alpha\over 2}A-j{\beta_2\over 2}\cdot{\partial^2 A\over \partial t^2}=n(t)
%$$
%where $n(t)$ denotes the noise process at the input of the optical fiber.
\begin{enumerate}[label=\alph*.]
\item
\eqn{
{\partial A(z,\omega)\over \partial z}&=-{\alpha\over 2}A(z,\omega)-j{\beta_2\over 2}\omega^2A(z,\omega)
\\&=[-{\alpha\over 2}-j{\beta_2\over 2}\omega^2]A(z,\omega)
}{}
which has the following solution
\eqn{
A(z,\omega)=C_1e^{-{\alpha\over 2}z-j{\beta_2\over 2}\omega^2z}
=A(0,\omega)e^{-{\alpha\over 2}z-j{\beta_2\over 2}\omega^2z}
}{}
\item
By defining $\Gamma(\omega)={\alpha\over 2}z+j{\beta_2\over 2}\omega^2$ and $B(z,\omega)=e^{-z\Gamma(\omega)}A(z,\omega)$ we obtain
$$
A_z(z,\omega)=\Gamma(\omega)A(z,\omega)+N(\omega)
$$
and consequently
$$
B_z(z,\omega)=e^{-z\Gamma(\omega)}N(\omega),
$$
hence
$$
B(z,\omega)=C_1(\omega)-e^{-z\Gamma(\omega)}\frac{N(\omega)}{\Gamma(\omega)}.
$$
Since $B(0,\omega)=A(0,\omega)$, we have
$$
B(z,\omega)=A(0,\omega)+(1-e^{-z\Gamma(\omega)})\frac{N(\omega)}{\Gamma(\omega)},
$$
which yields
$$
A(z,\omega)=A(0,\omega)e^{z\Gamma(\omega)}+(e^{z\Gamma(\omega)}-1)\frac{N(\omega)}{\Gamma(\omega)}.
$$
It is finally concluded that
$$
F(z,\omega)=
(e^{z\Gamma(\omega)}-1)\frac{N(\omega)}{\Gamma(\omega)}
.
$$
%Find $1\over e$-intensity spectral half width $\Delta \omega_0$ ($\Delta \omega_0$ should be measured where the intensity of the frequency domain $A(0,\omega)$ reaches $1\over e$ of its maximum, hence first try to find the Fourier transform of $A(0,t)$).
\end{enumerate}

\Q

a.
\qn{
&\text{Fiber 1 : }\beta_2=-31.87 \text{ps}^2/\text{km}
\\&\text{Fiber 2 : }\beta_3=0.0992 \text{ps}^3/\text{km}
}

\qn{
&\text{Transmitter 1 , Fiber 1 : }L_\text{Dispersion}=19.9 \text{ km}
\\&\text{Transmitter 1 , Fiber 2 : }L_\text{Dispersion}=17502 \text{ km}
\\&\text{Transmitter 2 , Fiber 1 : }L_\text{Dispersion}=3486 \text{ km}
\\&\text{Transmitter 2 , Fiber 2 : }L_\text{Dispersion}=812717419\text{ km}
}

b.

\qn{
&\text{Transmitter 1 , Fiber 1 : }L_\text{Attenuation}=95.24 \text{ km}
\\&\text{Transmitter 1 , Fiber 2 : }L_\text{Attenuation}=86.96 \text{ km}
\\&\text{Transmitter 2 , Fiber 1 : }L_\text{Attenuation}=95.24 \text{ km}
\\&\text{Transmitter 2 , Fiber 2 : }L_\text{Attenuation}=86.96\text{ km}
}

c.

The combinations
\qn{
&\text{Transmitter 1 , Fiber 2}
\\&\text{Transmitter 2 , Fiber 2}
}
lead to maximum optical reach due to both attenuation and dispersion, i.e, when fiber 2 is used.

\Q


\begin{enumerate}[label=\alph*.]
\item
$$
A(f)=\sum_{m=1}^2 s(f-mR).
$$
\item
Since the total bandwidth over which the signal spectrum takes on non-zero values is $2R$, the sampling frequency must be at least as much as this amount to avoid aliasing.
\item
$$
\hat A_1[k]=
\sum_{m=1}^2 b_{m,0}s(k/R)e^{\imath 2\pi mR k/R}
=
\sum_{m=1}^2 b_{m,0}s(k/R)
=
[b_{1,0}+b_{2,0}]s(k/R)
$$
$$
\hat A_2[k]=
\sum_{m=1}^2 b_{m,0}s(k/2R)e^{\imath 2\pi mR k/(2R)}
=
\sum_{m=1}^2 b_{m,0}s(k/2R)(-1)^{mk},
$$
hence
$$
\hat A_1(f)=2[b_{1,0}+b_{2,0}]
$$
and
$$
\hat A_2(f)=2Rb_{1,0}\sum_k S(2Rf-(2k+1)R)
+
2Rb_{2,0}\sum_k S(2Rf-2kR)
$$
\end{enumerate}

\end{document}