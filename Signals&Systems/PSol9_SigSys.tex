\documentclass[10pt,letterpaper]{article} 
\usepackage{tikz}
\usepackage{toolsper}
%\usepackage{graphicx}‎‎
%\usefonttheme{serif}‎
%\usepackage{ptext}‎
%\usepackage{xepersian}
%\settextfont{B Nazanin}
\usepackage{lipsum}
\setlength{\parindent}{0pt}
\newcommand{\pf}{$\blacksquare$}
\newcommand{\EX}{\Bbb E}
\newcommand{\nl}{\newline\newline}
\newcounter{QuestionNumber}
\setcounter{QuestionNumber}{1}

\newcommand{\wid}{40mm}
%\newcommand

\newcommand{\Q}{
\textbf{
سوال \theQuestionNumber)
}
\stepcounter{QuestionNumber}
}
\begin{document}
\Large
\begin{center}
به نام زیبایی

پاسخ تمرینات سری نهم سیگنال ها و سیستم ها
\hl
\end{center}
\Q

\textbf{راه اول:}
کانولوشن دایروی عبارتست از
$$
y[n]=\sum_{k=<N>}x[k]h[n-k]
$$
از آنجا که در کانولوشن دایره ای باید دوره‌ی تناوب هر دو سیگنال برابر باشد، تناوب هر دو سیگنال را برابر 12 می گیریم؛ در این صورت:
$$
y[n]=\sum_{k=<12>}x[k]h[n-k]
$$
اکنون می خواهیم ارتباط کانولوشن دایروی را با خطی پیدا کنیم. به طور مثال برای محاسبه‌ی 
$
y[0]
$
 می‌توان نوشت:
$$
y[0]=\sum_{k=<12>}x[k]h[-k]=\sum_{k=<12>}x[k]h[12-k]
$$
این مانند این است که کانولوشن خطی یک دوره‌ی تناوب از $x[n]$ را با یک دوره‌ی تناوب از $h[n]$ در نقطه‌ی 0 (یا 12 به دلیل متناوب بودن هر دو سیگنال با دوره‌ی 12 و طبق تساوی بالا) محاسبه کنیم؛ به عبارت دیگر، دو سیگنال شکل زیر را در هم ضرب کرده و سپس جمع نمونه های حاصل را محاسبه کنیم:
\begin{figure}[h!]
\begin{subfigure}{0.49\textwidth}
\includegraphics[width=80mm]{PSol9_Q1_2.eps}
\end{subfigure}
\begin{subfigure}{0.49\textwidth}
\includegraphics[width=80mm]{PSol9_Q1_1.eps}
\end{subfigure}
\end{figure}
در اینصورت 
$
y[0]=4
$
.

اگر بخواهیم $y[1]$ را محاسبه کنیم، باید تساوی زیر را در نظر بگیریم:
$$
y[0]=\sum_{k=<12>}x[k]h[1-k]=\sum_{k=<12>}x[k]h[13-k]
$$
به عبارت دیگر، سیگنالهای زیر را در هم ضرب کرده ایم:
\newpage
\begin{figure}[h!]
\begin{subfigure}{0.49\textwidth}
\includegraphics[width=80mm]{PSol9_Q1_3.eps}
\end{subfigure}
\begin{subfigure}{0.49\textwidth}
\includegraphics[width=80mm]{PSol9_Q1_1.eps}
\end{subfigure}
\end{figure}
دیده می شود که سیگنال $h[1-n]$ از روی $h[-n]$ به این ترتیب ساخته شده است که 1 واحد به راست شیفت دایره ای خورده است؛ یعنی نمونه‌ی 11 ام سیگنال $h[-n]$ به جایگاه 0 منتقل شده و سایر قسمت های سیگنال $h[-n]$ نیز 1 واحد به راست رفته اند. با محاسبه‌ی مجموع فوق دوباره داریم
$$
y[1]=4
$$
به همین ترتیب سایر نمونه های $y[n]$ نیز برابر 4 محاسبه می شوند و می توان نوشت:
$$
y[n]=4
$$
\textbf{راه دوم:}
با توجه به رابطه‌ی 
$$
y[n]=\sum_{k=<N>}x[k]h[n-k]
$$
اگر $h[n]$ با دوره‌ی $N_1$ متناوب باشد که 
$
N_1|12
$
 ، آنگاه
$$
y[n+N_1]=\sum_{k=<N>}x[k]h[n+N_1-k]=\sum_{k=<N>}x[k]h[n-k]=y[n]
$$
یعنی $y[n]$ نیز با دوره‌ی $N_1$ متناوب است. از طرف با تعویض جای $h$ و $x$، با فرض آنکه $x[n]$ با دوره‌ی $N_2$ متناوب باشد، $y[n]$ نیز با دوره‌ی $N_2$ متناوب است. پس $y[n]$ باید با دوره‌ی 
$
\gcd(N_1,N_2)
$
 متناوب باشد که در اینجا چون 
$
\gcd(N_1,N_2)=1
$
 در نتیجه سیگنال $y[n]$ ثابت است. مقدار ثابت را به راحتی می توان به دست آورد.

\textbf{راه سوم:}
با محاسبه‌ی سری فوریه‌ی دو سیگنال، به غیر از مضارب 12 در مکانهایی که 
$
a_k
$
 غیر صفر است، 
$
b_k
$
 صفر است و بالعکس. پس حاصل ضرب 
$
a_kb_k
$
 فقط در مضارب 12 غیر صفر می شود که معادل با سیگنال ثابت است.
\nl
\Q

$h[n]$
 شامل سه ضربه در زمان است؛ بنابراین
$$
y[n]=x[n+1]-x[n]+x[n-1]
$$
\begin{figure}[h!]
\centering
\includegraphics[width=80mm]{PSol9_Q2.eps}
\end{figure}
\Q

الف)

از شرط اول طبق خواص سری فوریه خواهیم داشت:
$$
x[n]=-x[n](-1)^n
$$
که نشان می دهد نمونه های زوج $x[n]$ برابر صفرند؛ پس سیگنال $x[n]$ در یک دوره تناوب مانند زیر است:
\begin{figure}[h!]
\centering
\includegraphics[width=80mm]{PSol9_Q3_1.eps}
\end{figure}
\nl
ب) ضرایب سری فوریه‌ی 
$
x[n-1]
$
 برابر است با:
$$
a_ke^{-jk{\pi\over 4}}
$$
بنابراین اگر ضرایب فوریه‌ی $y[n]$ را $b_k$ بنامیم، آنگاه:
\qn{
b_k&={1\over 2}[a_ke^{-jk{\pi\over 4}}+a_{k-4}e^{-j(k-4){\pi\over 4}}]
\\&={1\over 2}[a_ke^{-jk{\pi\over 4}}-a_{k}(-1)^ke^{-jk{\pi\over 4}}]
}{}
بنابراین
$$
f[k]=
{1\over 2}e^{-jk{\pi\over 4}}[1-(-1)^k]
$$
\Q

الف) طبق خواص می توان نوشت:
$$
b_k=a_k[1-(-1)^k]
$$
ب) اگر ضرایب سری فوریه‌ی این سیگنال را $b_k$ بنامیم، در این صورت $b_k$ با دوره‌ی $N/2$ متناوب است. در اینصورت
\qn{
b_k={2\over N}\sum_{n=0}^{{N\over 2}-1}\{x[n]+x[n+N/2]\}e^{-j{4\pi\over N}kn}
}{}
 چنانچه بخواهیم ضرایب فوریه‌ی این سیگنال را با در نظر گرفتن دوره‌ی تناوب $N$ محاسبه و آن را $c_k$ نام گذاری کنیم، در این صورت:
\qn{
c_k={1\over N}\sum_{n=0}^{{N}-1}\{x[n]+x[n+N/2]\}e^{-j{2\pi\over N}kn}
}{}
پرواضح است که 
$
c_{2k}=b_k
$
 و 
$
c_{2k+1}=0
$
. 
از طرفی
$$
c_k=a_k[1+(-1)^k]=\begin{cases}2a_k&,\quad \text{k زوج}
\\
0&,\quad \text{k فرد}
\end{cases}
$$
در این صورت:
$$
b_k=2a_{2k}
$$
پ) بر طبق خواص
$$
b_k=a_{k-{N\over 2}}
$$
ت) اگر ضرایب $x[n]$ را با $a_k$ نشان دهیم، $a_k$ با دوره‌ی $N$ متناوب است. اکنون فرض کنید ضرایب $x[n]$ را با دوره‌ی تناوب $2N$ محاسبه کرده و آن را $c_k$ نامیده ایم. در این صورت:
\qn{
&a_k={1\over N}\sum_{n=0}^{{N}-1}x[n]e^{-j{2\pi\over N}kn}
\\&c_k={1\over 2N}\sum_{n=0}^{{2N}-1}x[n]e^{-j{\pi\over N}kn}
}{}
بنابراین
$
c_{2k}=a_k
$
و 
$
c_{2k+1}=0
$
.
از طرفی $(-1)^n$، شیفتی به اندازه‌ی نصف دوره تناوب یعنی 
$
{2N\over 2}=N
$
 در حوزه‌ی فوریه تحمیل می کند؛ پس:
$$
b_k=\begin{cases}
0&,\quad \text{k زوج}
\\
a_{k-N\over 2}&,\quad \text{k فرد}
\end{cases}
$$
\nl
ث)
$$
y[n]=x[n]{1+(-1)^n\over 2}
$$
بنابراین طبق خواص اگر $N$ زوج باشد:
$$
b_k={a_k+a_{k-{N\over 2}}\over 2}
$$
و اگر $N$ فرد باشد:
$$
b_k=\begin{cases}
a_k&,\quad \text{k زوج}
\\
{a_k+a_{k-N\over 2}\over 2}&,\quad \text{k فرد}
\end{cases}
$$
\Q

در هر خاصیت، ضرایب سری فوریه‌ی سیگنال را با $c_k$ نشان می دهیم.

الف)
\qn{
c_k&={1\over N}\sum_{n=<N>}\sum_{r=<N>}x[r]y[n-r]e^{-jkn{2\pi\over N}}
\\&={1\over N}\sum_{r=<N>}x[r]\sum_{n=<N>}y[n-r]e^{-jkn{2\pi\over N}}
\\&={1\over N}\sum_{r=<N>}x[r]\sum_{n=<N>}y[n-r]e^{-jk(n-r){2\pi\over N}}e^{-jkr{2\pi\over N}}
\\&={1\over N}\sum_{r=<N>}x[r]\sum_{m=<N>}y[m]e^{-jkm{2\pi\over N}}e^{-jkr{2\pi\over N}}
\\&=\sum_{r=<N>}x[r]b_ke^{-jkr{2\pi\over N}}
\\&=Na_kb_k
}{}
ب) 
\qn{
c_k&={1\over N}\sum_{n=<N>}x[n]y[n]e^{-jk{2\pi\over N}n}
\\&={1\over N}\sum_{n=<N>}\sum_{p=<N>,q=<N>}a_pb_qe^{jk(p+q){2\pi\over N}n}e^{-jk{2\pi\over N}n}
\\&={1\over N}\sum_{p=<N>,q=<N>}a_pb_q\sum_{n=<N>}e^{jk(p+q){2\pi\over N}n}e^{-jk{2\pi\over N}n}
\\&=\sum_{p=<N>}a_pb_{k-p}
}{}
پ)

از خواص سری فوریه فورا نتیجه می شود.
\nl
ت) با تعریف 
$
y[n]=\sum_{k=-\infty}^n x[k]
$
 خواهیم داشت:
\qn{
y[n]-y[n-1]=x[n]
}{}
در نتیجه
$$
b_k(1-e^{-jk{2\pi\over N}})=a_k
$$
به ازای $k=0$ باید الزاما داشته باشیم $
a_0=0
$
 و به ازای 
$
k\ne 0
$:
$$
b_k={a_k\over 1-e^{-jk{2\pi\over N}}}
$$
ث)
$$
x[n]=\sum_{n=<N>}a_ke^{jk{2\pi\over N}n}
$$
\qn{
x^*[n]&=\sum_{n=<N>}a_k^*e^{-jk{2\pi\over N}n}
\\&=\sum_{n=<N>}a_{-k}^*e^{jk{2\pi\over N}n}
}{}
در نتیجه
$$
x[n]=x^*[n]\iff a_k=a^*_{-k}
$$
از این گزاره، سایر گزاره ها نتیجه می شوند.
\nl
ج) می دانیم
$$
x[-n]\implies a_{-k}
$$
بنابراین به کمک تقارن هرمیتیک:
\qn{
&x_e[n]={x[n]+x[-n]\over 2}\implies {a_k+a_{-k}\over 2}={a_k+a^*_{k}\over 2}=\Re\{a_k\}
\\&x_o[n]={x[n]-x[-n]\over 2}\implies {a_k-a_{-k}\over 2}={a_k-a^*_{k}\over 2}=j\Im\{a_k\}
}{}
\nl
چ) طبق خاصیت قسمت ب، سیگنال 
$
x[n]x^*[n]
$
 دارای ضرایب سری فوریه‌ی 
$
\sum_{p=<N>}a_pa^*_{p-k}
$
است؛ از طرفی محاسبه کردن ضریب سری فوریه‌ی سیگنال در فرکانس صفر، معادل با 
$
1\over N
$
 حاصل جمع سیگنال روی یک دوره‌ی تناوب آن است؛ بنابراین با قرار دادن $k=0$ در ضرایب بالا و جمع بستن کل سیگنال خواهیم داشت:
$$
{1\over N}\sum_{n=<N>}|x[n]|^2={1\over N}\sum_{n=<N>}x[n]x^*[n]
=\sum_{p=<N>}a_pa^*_{p}=\sum_{p=<N>}|a_p|^2
$$
\end{document}