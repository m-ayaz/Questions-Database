\documentclass[50pt]{article}
\RequirePackage{pdfpages}
\renewcommand{\baselinestretch}{1.4}
\RequirePackage{amsthm,amssymb,amsmath,graphicx}
\RequirePackage{color}
\RequirePackage[top=2cm, bottom=2cm, left=2.5cm, right=3cm]{geometry}
\RequirePackage[pagebackref=false,colorlinks,linkcolor=blue,citecolor=magenta]{hyperref}
\RequirePackage{xepersian}
\RequirePackage{MnSymbol}
\RequirePackage{graphicx}
\newcommand{\wid}{1.8in}
\newtheorem{theorem}{Theorem}
\newcommand{\hl}{
\begin{center}
\line(1,0){450}
\end{center}}
\newenvironment{amatrix}[1]{%
\left[\begin{array}{@{}*{#1}{c}|c@{}}
}{%
\end{array}\right]
}
\settextfont{B Nazanin}
\setlatintextfont{Times New Roman}

\begin{document}
\setLTR 




\begin{RTL}
\Large{








\begin{center}
به نام خدا

تمرینات سری ششم درس سیگنالها و سیستمها

دکتر لطف الله بیگی

مهلت تحویل: 98/2/1
\end{center}

\hl
\begin{latin}
Chapter 4:

$21\_\{a,e,h,j\} , 22\_\{b,d\} , 23 ,24 ,25 , 26\_\Big\{a\_\{II\}\Big\} , 27 $
\end{latin}

امتیازی (اختیاری)

\begin{latin}
Chapter 4:

$31,40$
\end{latin}
 
همچنین سوالات زیر را به صورت اختیاری حل کنید:

(1) به کمک تساوی پارسوال، مقدار انتگرال 
$\int_{-\infty}^{\infty}{\sin^2\pi x\over x^2}dx$
 را  به دست آورید.

(2) به کمک رابطه ی دوگانی تبدیل فوریه و استفاده از کانولوشن فرکانسی، تبدیل فوریه ی سیگنال زیر را به دست آورید:

$$x(t)=e^{-t}\cdot{\sin \pi t\over \pi t}u(t)$$




}





\end{RTL}



\end{document}


