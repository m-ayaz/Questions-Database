\documentclass{article}


\usepackage{amsmath,amssymb,graphicx,geometry,xepersian}

\setlength{\parindent}{0mm}
\setlength{\parskip}{3mm}


\begin{document}

\begin{center}
\large
به نام او

شبیه سازی 1 سیگنال ها و سیستم ها

\hrulefill
\end{center}
\large

سوال 1) سیگنال 
$
x(t)=\sin t
$
را در بازه‌ی 
$
[-\pi,\pi]
$
ایجاد کنید و نمایش دهید. سیگنال های زیر با ایجاد تغییر در متغیر مستقل ایجاد نمایید و هر کدام را در شکل جداگانه نمایش دهید. سپس، تفاوت این نمایش ها را با نمایش سیگنال اصلی توجیه کنید.

الف) 
$
x(2t)
$

ب)
$
x(-t)
$

پ)
$
x(-2t+1)
$

ت)
$
4x(t)
$

ث)
$
x_o(t)
$

سوال 2) سیستم 
$
y(t)=x^2(t)
$
را در نظر بگیرید. مراحل زیر را انجام دهید:
\begin{itemize}
\item
 دو سیگنال 
$
x_1(t)=\sin t
$
و
$
x_2(t)=\cos t
$
را در بازه‌ی 
$
[-\pi,\pi]
$
ایجاد کرده و رسم کنید.
\item
این دو سیگنال را به عنوان ورودی به سیستم داده، خروجی متناظر آن ها را 
$
y_1(t)
$
و 
$
y_2(t)
$
نامیده و آنها را رسم کنید. 
\item
سیگنال 
$
2x_1(t)+3x_2(t)
$
را به ورودی سیستم دهید و خروجی متناظر با آن را رسم کنید.
\item
خروجی مرحله‌ی قبل را با سیگنال
$
2y_1(t)+3y_2(t)
$
مقایسه کنید.
\end{itemize}
از مقایسه‌ی این دو سیگنال، چه نتیجه ای در مورد خاصیت خطی بودن این سیستم می‌گیرید؟

سوال 3) مراحل سوال پیش را برای سیستم
$
y(t)=2x(t)
$
اجرا کنید و مقایسه را انجام دهید. چه نتیجه ای در مورد خاصیت خطی بودن سیستم می‌گیرید؟

سوال 4) سیستم زیر را در نظر بگیرید. پاسخ ضربه‌ی این سیستم را به روش بازگشتی محاسبه و 10 نمونه‌ی اول پاسخ ضربه را ترسیم کنید.
$
y[n]=x[n]-x[n-1]+3x[n-2]
$

\end{document}