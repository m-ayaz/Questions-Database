\documentclass[50pt]{article}
\RequirePackage{pdfpages}
\renewcommand{\baselinestretch}{1.4}
\RequirePackage{amsthm,amssymb,amsmath,graphicx}
\RequirePackage{color}
\RequirePackage[top=2cm, bottom=2cm, left=2.5cm, right=3cm]{geometry}
\RequirePackage[pagebackref=false,colorlinks,linkcolor=blue,citecolor=magenta]{hyperref}
\RequirePackage{xepersian}
\RequirePackage{MnSymbol}
\RequirePackage{graphicx}
\newcommand{\wid}{1.8in}
\newtheorem{theorem}{Theorem}
\newcommand{\hl}{
\begin{center}
\line(1,0){450}
\end{center}}
\newenvironment{amatrix}[1]{%
\left[\begin{array}{@{}*{#1}{c}|c@{}}
}{%
\end{array}\right]
}
\settextfont{B Nazanin}
\setlatintextfont{Times New Roman}

\begin{document}
\setLTR 




\begin{RTL}
\Large{








\begin{center}
به نام خدا

پاسخ سوالات 1 و 2 تمرینات سری ششم درس سیگنالها و سیستمها

دکتر لطف الله بیگی
\end{center}

\hl
\textbf{(1) به کمک تساوی پارسوال، مقدار انتگرال 
$\int_{-\infty}^{\infty}{\sin^2\pi x\over x^2}dx$
 را  به دست آورید.}

پاسخ) برای حل این سوال از رابطه‌ی زیر موسوم به اتحاد پارسوال استفاده می کنیم:
$$
\int_{-\infty}^{\infty}|x(t)|^2 dt={1\over 2\pi}\int_{-\infty}^{\infty}|X(j\omega)|^2 d\omega
$$
فرض کنیم 
$x(t)={\sin \pi t\over \pi t}$. 
می دانیم
$$
\Pi(t)\overset{F}{\iff}{\sin {\omega\over 2}\over {\omega\over 2}}
$$
در اینصورت
$$
{\sin {t\over 2}\over {t\over 2}}\overset{F}{\iff}2\pi\Pi(-\omega)=2\pi\Pi(\omega)
$$
به کمک خاصیت مقیاس در حوزه زمان تبدیل فوریه، می توان نوشت:
$$
{\sin {\pi t}\over {\pi t}}\overset{F}{\iff}\Pi({\omega\over 2\pi})
$$
اکنون با اعمال رابطه‌ی پارسوال می توان نوشت:
\[
\int_{-\infty}^{\infty}{\sin^2\pi x\over \pi^2x^2}dx={1\over 2\pi}\int_{-\infty}^{\infty}|\Pi({\omega\over 2\pi})|^2 d\omega={1\over 2\pi}\cdot 2\pi =1
\]
بنابراین:
\[
\int_{-\infty}^{\infty}{\sin^2\pi x\over x^2}dx=\pi^2
\]






\textbf{(2) به کمک رابطه ی دوگانی تبدیل فوریه و استفاده از کانولوشن فرکانسی، تبدیل فوریه ی سیگنال زیر را به دست آورید:
}$$x(t)=e^{-t}\cdot{\sin \pi t\over \pi t}u(t)$$

پاسخ) ابتدا $x(t)$ را به دو جزء $e^{-t}u(t)$ و ${\sin \pi t\over \pi t}$ می شکنیم؛ به عبارت دیگر:
$$
x(t)=e^{-t}\cdot{\sin \pi t\over \pi t}u(t)=e^{-t}u(t)\cdot {\sin \pi t\over \pi t}
$$
هم چنین:
$$
e^{-t}u(t)\overset{F}{\iff}{1\over 1+j\omega}
$$
$$
{\sin \pi t\over \pi t}\overset{F}{\iff}\Pi({\omega\over 2\pi})
$$
در نتیجه:
\[
\begin{split}
X(j\omega)&={1\over 2\pi}\cdot{1\over 1+j\omega}*\Pi({\omega\over 2\pi})
\\&={1\over 2\pi}\int_{-\infty}^{\infty}{1\over 1+j(\omega-u)}\Pi({u\over 2\pi})du
\\&={1\over 2\pi}\int_{-\pi}^{\pi}{1\over 1+j(\omega-u)}du
\\&={j\over 2\pi}\ln({1+j\omega-ju})\Big|_{-\pi}^\pi
\\&={j\over 2\pi}\ln\left({1+j(\omega-\pi)\over 1+j(\omega+\pi)}\right)
\end{split}
\]

}





\end{RTL}



\end{document}


