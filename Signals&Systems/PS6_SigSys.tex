\documentclass[10pt,letterpaper]{article} 
\usepackage{tikz}
\usepackage{toolsper}
%\usepackage{graphicx}‎‎
%\usefonttheme{serif}‎
%\usepackage{ptext}‎
%\usepackage{xepersian}
%\settextfont{B Nazanin}
\usepackage{lipsum}
\setlength{\parindent}{0pt}
%\usepackage{enumitem}
%\setlist[enumerate,1]{label=(\arabic*)}
\newcommand{\pf}{$\blacksquare$}
\newcommand{\EX}{\Bbb E}
\newcommand{\nl}{\newline\newline}

\newcounter{QuestionNumber}
\setcounter{QuestionNumber}{1}

\newcommand{\Q}{
\textbf{
سوال \theQuestionNumber)
}
\stepcounter{QuestionNumber}
}

%\newcommand{\pic}[2]{
%\begin{center}
%\includegraphics[width=#2]{#1}
%\end{center}
%}
\begin{document}
\Large
\begin{center}
به نام زیبایی

تمرینات سری ششم سیگنال ها و سیستم ها
\hl
\end{center}
\Q

اگر $x(t)$، سیگنالی با دوره تناوب $T$ و ضرایب سری فوریه‌ی $a_k$ باشد، ضرایب سری فوریه‌ی هر یک از سیگنال های زیر را به دست آورید.

الف) 
$
x(t-t_0)+x(t+t_0)
$

ب) 
$
{d^n\over dt^n}x(t)
$

پ) 
$
x(at+b)
$
 (به دوره تناوب این سیگنال دقت کنید)

ت) سیگنال متناوب $y(t)$ با دوره تناوب $T$ که 
$$
y(t)=\begin{cases}
x(t)&,\quad 0\le t<T_1
\\0&,\quad T_1\le t<T
\end{cases}
$$
 و $T_1<T$.
\nl
\Q

الف) نشان دهید اگر دو سیگنال متناوب $x(t)$ و $y(t)$ با دوره تناوب $T$، به ترتیب دارای ضرایب سری فوریه‌ی $a_k$ و $b_k$ باشند، آنگاه سیگنال $x(t)y(t)$ دارای ضرایب سری فوریه‌ی 
$
\sum_{l=-\infty}^{\infty}a_lb_{k-l}
$
 است.

(راهنمایی: هر دو سیگنال را بر حسب ضرایب سری فوریه‌ی آنها بسط دهید. سپس با جایگذاری آن ها در رابطه‌ی انتگرالی
$
{1\over T}\int_0^T x(t)y(t)e^{-jk{2\pi\over T}t}dt
$
 و ساده سازی، عبارت
$
\sum_{l=-\infty}^{\infty}a_lb_{k-l}
$
 را بسازید.
)
\nl
ب) از قسمت قبل نتیجه بگیرید: 
$$
{1\over T}\int_0^T |x(t)|^2\sin {2\pi\over T}tdt=\sum_{k=-\infty}^{\infty} \Im \left\{a_ka^*_{k+1}\right\}
$$
که عملگر $\Im$، قسمت موهومی یک عدد مختلط را می دهد.
%\nl
%\Q
% \textbf{(امتیازی)}
%
%سیگنال دوبعدی $x(t_1,t_2)$ متناوب با دوره ی $T_1$ نسبت به $t_1$ و متناوب با دوره ی $T_2$ نسبت به $t_2$ گفته می شود هرگاه:
%$$
%x(t_1+T_1,t_2+T_2)=x(t_1,t_2)
%$$
%برای این سیگنال می توان ضرایب سری فوریه‌ی دوبعدی را به صورت زیر تعریف کرد:
%$$
%a_{mn}={1\over T_1T_2}\int_0^{T_1}\int_0^{T_2}x(t_1,t_2) e^{jm{2\pi\over T_1}t_1}e^{jn{2\pi\over T_2}t_2}dt_1dt_2
%$$
%در اینصورت ضرایب سری فوریه‌ی سیگنال های زیر را بیابید:
%
%الف) 
%$
%\cos (2\pi t_1+2t_2)
%$
%
%ب) 
%$
%\sin \pi t_1\cos t_2
%$
%
%پ) $x(t)$ سیگنالی متناوب شامل ضربه هایی با مساحت 1 و در مکان های 
%$
%(mT_1,nT_2)
%$
% است.
\end{document}