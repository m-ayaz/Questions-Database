\documentclass[50pt]{article}
\RequirePackage{pdfpages}
\usepackage{listings}
\usepackage{color} %red, green, blue, yellow, cyan, magenta, black, white
\definecolor{mygreen}{RGB}{28,172,0} % color values Red, Green, Blue
\definecolor{mylilas}{RGB}{170,55,241}
\renewcommand{\baselinestretch}{1.4}
\RequirePackage{amsthm,amssymb,amsmath,graphicx}
\RequirePackage{color}
\RequirePackage[top=2cm, bottom=2cm, left=2.5cm, right=3cm]{geometry}
\RequirePackage[pagebackref=false,colorlinks,linkcolor=blue,citecolor=magenta]{hyperref}
\RequirePackage{xepersian}
\RequirePackage{MnSymbol}
\RequirePackage{graphicx}
\newcommand{\wid}{1.8in}
\newtheorem{theorem}{Theorem}
\newcommand{\hl}{
\begin{center}
\line(1,0){450}
\end{center}}
\newenvironment{amatrix}[1]{%
\left[\begin{array}{@{}*{#1}{c}|c@{}}
}{%
\end{array}\right]
}
\settextfont{B Nazanin}
\setlatintextfont{Times New Roman}

\begin{document}

\lstset{language=Matlab,%
    %basicstyle=\color{red},
    breaklines=true,%
    morekeywords={matlab2tikz},
    keywordstyle=\color{blue},%
    morekeywords=[2]{1}, keywordstyle=[2]{\color{black}},
    identifierstyle=\color{black},%
    stringstyle=\color{mylilas},
    commentstyle=\color{mygreen},%
    showstringspaces=false,%without this there will be a symbol in the places where there is a space
    numbers=left,%
    numberstyle={\tiny \color{black}},% size of the numbers
    numbersep=9pt, % this defines how far the numbers are from the text
    emph=[1]{for,end,break},emphstyle=[1]\color{red}, %some words to emphasise
    %emph=[2]{word1,word2}, emphstyle=[2]{style},    
}





\setLTR 




\begin{RTL}
\Large{








\begin{center}
به نام خدا


\end{center}
\hl
\[
\begin{split}
H(s)&={1\over s^n+1}
\\&= \prod_{k=1}^{n}{1\over s-s_k}
\end{split}
\]
که در آن 
$
s_k=e^{j{2k-1\over n}\pi}
$
  و همگی ریشه های ساده‌ی $s^n+1=0$ هستند؛ در نتیجه:
\[
\begin{split}
\prod_{k=1}^{n}{1\over s-s_k}=\sum_{k=1}^n {b_k\over s-s_k}
\end{split}
\]
و
\[
\begin{split}
b_k&=\lim_{s\to s_k}{s-s_k\over s^n+1}
\\&=\lim_{s\to s_k}{1\over ns^{n-1}}
\\&=\lim_{s\to s_k}{s\over ns^{n}}=-{s_k\over n}
\end{split}
\]
بنابراین:
\[
\sum_{k=1}^n {b_k\over s-s_k}=-{1\over n}\sum_{k=1}^{n} {s_k\over s-s_k}
\]
و عکس تبدیل لاپلاس، به صورت زیر است:
\[
h(t)=-{1\over n}\left(\sum_{k=1}^{n} s_k e^{s_k t}\right)u(t)
\]







}





\end{RTL}



\end{document}


