\documentclass[10pt,letterpaper]{article} 
\usepackage{tikz}
\usepackage{toolsper}
%\usepackage{graphicx}‎‎
%\usefonttheme{serif}‎
%\usepackage{ptext}‎
%\usepackage{xepersian}
%\settextfont{B Nazanin}
\usepackage{lipsum}
\setlength{\parindent}{0pt}
\newcommand{\pf}{$\blacksquare$}
\newcommand{\EX}{\Bbb E}
\newcommand{\nl}{\newline\newline}
\newcommand{\Q}[1]{\textbf{
سوال #1)
}}
\newcommand{\pic}[2]{
\begin{center}
\includegraphics[width=#2]{#1}
\end{center}
}
\begin{document}
\Large
\begin{center}
به نام زیبایی

پاسخ تمرینات سری اول سیگنال ها و سیستم ها
\hl
\end{center}
\Q1

ابتدا به رابطه‌ی توان متوسل می شویم؛ در این صورت:

\qn{
P_y&=\lim_{L\to \infty}{1\over L}\int_{-{L\over 2}}^{L\over 2} |y(t)|^2dt
\\&=\lim_{L\to \infty}{1\over L}\int_{-{L\over 2}}^{L\over 2} y(t)y^*(t)dt
\\&=\lim_{L\to \infty}{1\over L}\int_{-{L\over 2}}^{L\over 2} \sum_m\sum_n x(t-mT)x^*(t-nT)dt
\\&=\lim_{L\to \infty}{1\over L}\int_{-{L\over 2}}^{L\over 2} \sum_{m,n} x(t-mT)x^*(t-nT)dt
\\&=\lim_{L\to \infty}{1\over L}\int_{-{L\over 2}}^{L\over 2} \sum_{\substack{m,n\\m=n}} x(t-mT)x^*(t-nT)dt
\\&=\lim_{L\to \infty}{1\over L}\int_{-{L\over 2}}^{L\over 2} \sum_n x(t-nT)x^*(t-nT)dt
\\&=\lim_{L\to \infty}{1\over L}\int_{-{L\over 2}}^{L\over 2} \sum_n |x(t-nT)|^2dt
\\&=\lim_{k\to \infty}{1\over kT}\int_{-{kT\over 2}}^{kT\over 2} \sum_n |x(t-nT)|^2dt
\\&=\lim_{k\to \infty}{1\over T}\int_{-{T\over 2}}^{T\over 2} \sum_n |x(t-nT)|^2dt
\\&=\lim_{k\to \infty}{1\over T}\int_{-{T\over 2}}^{T\over 2} |x(t)|^2dt
\\&={1\over T}\int_{-{T\over 2}}^{T\over 2} |x(t)|^2dt
\\&={E_{x(t)}\over T}
}{}
ب) بدیهی است جز در حالت های خاص (شامل ضربه یا صفر بودن سیگنال)، سیگنال $x(t)$ از نوع انرژی و سیگنال $y(t)$ از نوع توان خواهد بود.
% فرض کنید سیگنال $x(t)$، سیگنال زمان محدودی باشد؛ به گونه ای که
%$$
%x(t)=0\quad,\quad |t|>T
%$$
%و
%$$
%y(t)=\sum_{n=-\infty}^{\infty}x(t-nT)
%$$
%در این صورت:
%
%الف) توان $y(t)$ را بیابید.
%
%ب) هر یک از سیگنال های $x(t)$ و $y(t)$ را از نقطه نظر سیگنال توان یا انرژی بودن یا نبودن طبقه‌بندی کنید.
\nl
\Q2
%کدام یک از سیگنال های زیر متناوب اند و در صورت متناوب بودن، دوره‌ی تناوب اساسی آنها را محاسبه کنید.
الف) متناوب نیست؛ زیرا اگر $T>0$ چنان موجود باشد که:
$$
x(t+T)=\cos {\pi\over 7}(t+T)^2=\cos {\pi\over 7}t^2=x(t)
$$
آنگاه
$$
(t+T)^2=14k\pm t^2\quad,\quad k\in\Bbb Z
$$
که هیچ یک از شاخه های جواب های فوق، $T$ مستقل از زمانی نمی دهد.
\nl
ب) متناوب با دوره‌ی $N=7$ است. به طریقی مشابه بالا:
$$
(n+N)^2=14k\pm n^2\quad,\quad k\in\Bbb Z
$$
از شاخه‌ی 
$
(n+N)^2=14k+ n^2\quad,\quad k\in\Bbb Z
$
 خواهیم داشت:
$$
N^2+2nN=14k
$$
که به ازای هر $n$ فقط زمانی درست است که $N=14$.
\nl
پ) متناوب نیست؛ زیرا $\sin n=0$ تنها یک ریشه در اعداد صحیح دارد و آن $n=0$ است؛ بنابراین
$$
x[n]=\delta[\sin n]=\delta[n]
$$
\nl
ت) از آنجا که $\sin {\pi\over 3}n$ با دوره‌ی تناوب اساسی 
$
{2\pi\over{\pi\over 3}}=6
$
 متناوب است، در نتیجه
$$
x[n+6]=\delta[\sin {\pi\over 3}(n+6)]=\delta[\sin {\pi\over 3}n]=x[n]
$$
\nl
ث) داخل آرگومان بیرونی ترین $\cos$، تابع 
$
{\pi\over 3}n+\cos{\pi\over 2}n
$
 وجود دارد که نه زوج است و نه فرد. از آنجا که $\cos{\pi\over 2}n$ با دوره‌ی اساسی 4 متناوب است، در نتیجه باید دوره تناوب سیگنال مضربی از 4 باشد. از طرفی از آنجا که جمله‌ی ${\pi\over 3}n$ نیز در این آرگومان وجود دارد، باید دوره‌ی تناوب سیگنال مضربی از 6 نیز باشد؛ بنابراین دوره تناوب برابر 12 است.
\qn{
x[n+12]&=\cos\left\{{\pi\over 3}(n+12)+\cos{\pi\over 2}(n+12)\right\}
\\&=\cos\left\{{\pi\over 3}n+4\pi+\cos\left({\pi\over 2}n+6\pi\right)\right\}
\\&=\cos\left\{{\pi\over 3}n+\cos\left({\pi\over 2}n\right)\right\}
\\&=x[n]
}{}

ج) متناوب نیست؛ زیرا جمع دو سیگنال متناوب یکی با تناوب 2 و یکی $2\pi$ است و این دو عدد مضرب مشترک صحیح ندارند.
%$
%x(t)=\sin t+\sin \pi t
%$
\nl
\Q3
الف، ب، پ و ت) 
\lr{
\begin{figure}[htb]
\begin{subfigure}{0.5\textwidth}
\pic{PSol1_Q3_a_1}{75mm}
\end{subfigure}
\begin{subfigure}{0.5\textwidth}
\pic{PSol1_Q3_a_2}{75mm}
\end{subfigure}
\begin{subfigure}{0.33\textwidth}
\pic{PSol1_Q3_b_1}{50mm}
\end{subfigure}
\begin{subfigure}{0.33\textwidth}
\pic{PSol1_Q3_b_2}{50mm}
\end{subfigure}
\begin{subfigure}{0.33\textwidth}
\pic{PSol1_Q3_b_3}{50mm}
\end{subfigure}
\begin{subfigure}{0.33\textwidth}
\pic{PSol1_Q3_c_1}{50mm}
\end{subfigure}
\begin{subfigure}{0.33\textwidth}
\pic{PSol1_Q3_c_2}{50mm}
\end{subfigure}
\begin{subfigure}{0.33\textwidth}
\pic{PSol1_Q3_c_3}{50mm}
\end{subfigure}
\begin{subfigure}{0.33\textwidth}
\pic{PSol1_Q3_d_1}{50mm}
\end{subfigure}
\begin{subfigure}{0.33\textwidth}
\pic{PSol1_Q3_d_2}{50mm}
\end{subfigure}
\begin{subfigure}{0.33\textwidth}
\pic{PSol1_Q3_d_3}{50mm}
\end{subfigure}
\end{figure}
}
\end{document}