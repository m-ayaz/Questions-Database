\documentclass[10pt,letterpaper]{article} 
%\usepackage{tikz}
\usepackage{amsmath,amssymb,geometry}
%\usepackage{graphicx}‎‎
%\usefonttheme{serif}‎
%\usepackage{ptext}‎
\usepackage{xepersian}
%\settextfont{B Nazanin}
\usepackage{lipsum}
\setlength{\parindent}{0pt}
%\usepackage{enumitem}
%\setlist[enumerate,1]{label=(\arabic*)}
\newcommand{\pf}{$\blacksquare$}
\newcommand{\EX}{\Bbb E}
\newcommand{\nl}{\newline\newline}
\setlength{\parskip}{1em}

\usepackage{amsmath}
\usepackage{accents}
\newlength{\dhatheight}
\newcommand{\doublehat}[1]{%
    \settoheight{\dhatheight}{\ensuremath{\hat{#1}}}%
    \addtolength{\dhatheight}{-0.35ex}%
    \hat{\vphantom{\rule{1pt}{\dhatheight}}%
    \smash{\hat{#1}}}}

\newcounter{QuestionNumber}
\setcounter{QuestionNumber}{1}

\newcommand{\Q}{
\textbf{
سوال \theQuestionNumber)
}
\stepcounter{QuestionNumber}
}

\newcommand{\fig}[3]{
\begin{figure}[h!]
#1
\caption{#2}
\label{#3}
\end{figure}
}

\newcommand{\subfig}[3]{
\begin{subfigure}{#3}
#1
\caption{#2}
\end{subfigure}
}


\newcommand{\qn}[1]{
\[
\begin{split}
#1
\end{split}
\]
}
%\newcommand{\pic}[2]{
%\begin{center}
%\includegraphics[width=#2]{#1}
%\end{center}
%}
\begin{document}
\Large
\begin{center}
به نام زیبایی

پاسخ تمرینات سری دهم سیگنال ها و سیستم ها

\hrulefill
\end{center}
%\Q
%
%تبدیل فوریه‌ی سیگنال های زمان گسسته‌ی زیر را به دست آورید.
%
%الف)
%$
%x[n]=u[n]-u[n-5]
%$
%
%ب)
%$
%x[n]=({1\over 3})^nu[n]
%$
%
%پ)
%$
%x[n]=-({1\over 3})^nu[-n-1]
%$
%
%ت)
%$
%x[n]=\sin{\pi\over 2}n+\cos n
%$
%
%ث) 
%$
%x[n]=n({1\over 3})^nu[n]
%$
%
%\Q
%
%عکس تبدیل فوریه‌ی سیگنال های زیر را به دست آورید.
%
%الف)
%$
%X(e^{j\omega})=\sum_{k=-\infty}^{\infty}(-1)^k\delta(\omega-{\pi\over 2}k)
%$
%
%ب)
%$
%X(e^{j\omega})={1-{1\over 3}e^{-j\omega}\over 1-{1\over 4}e^{-j\omega}-{1\over 8}e^{-2j\omega}}
%$
%
%پ) 
%$
%X(e^{j\omega})={1\over 1-e^{-4j\omega}}
%$

\large

\Q

الف)
\qn{
&x[n-1]\iff X(e^{j\omega})e^{-j\omega}
\\&
x[n+1]\iff X(e^{j\omega})e^{j\omega}
}
بنابراین
\qn{
&x[-n-1]\iff X(e^{-j\omega})e^{j\omega}
\\&
x[-n+1]\iff X(e^{-j\omega})e^{-j\omega}
}
و
$$
y[n]\iff X(e^{-j\omega})\cos \omega
$$

ب)
$$
x^*[-n]\iff X^*(e^{j\omega})
$$
بنابراین
$$
y[n]\iff \frac{X(e^{j\omega})+X^*(e^{j\omega})}{2}=\Re\left\{X^*(e^{j\omega})\right\}
$$

پ)
$
y[n]=(n^2-2n+1)x[n]
$
در نتیجه
\qn{
Y(e^{j\omega})&=
j^2\frac{d^2}{d\omega^2}X(e^{j\omega})
-2j\frac{d}{d\omega}X(e^{j\omega})
+X(e^{j\omega})
\\&=
-\frac{d^2}{d\omega^2}X(e^{j\omega})
-2j\frac{d}{d\omega}X(e^{j\omega})
+X(e^{j\omega})
}


\Q

الف) سیگنال $x[n]$ حول $n=2$ تقارن فرد دارد؛ در نتیجه فاز آن برابر است با
$
-2\omega\pm\frac{\pi}{2}
$
.

ب)
$$
\int_{-\pi}^{\pi} X(e^{j\omega})d\omega=2\pi x[0]=6\pi
$$

پ)
$$
X(e^{j\pi})=\sum_n (-1)^nx[n]=0
$$

ت)
$$
\int_{-\pi}^{\pi} \left|{dX(e^{j\omega})\over d\omega}\right|^2d\omega
=2\pi\sum_n n^2x^2[n]=516\pi
$$

%چ) سیگنالی که تبدیل فوریه‌ی آن 
%$
%\Re\{X(e^{j\omega})\}
%$
% باشد.


\Q

از گزاره‌ی اول نتیجه می شود که 
$
g[n]=a+b\delta[n-1]
$
به ازای مقادیری از $a,b$. بنابراین پاسخ فرکانسی به فرم زیر است:
\qn{
H(e^{j\omega})=\frac{Y(e^{j\omega})}{X(e^{j\omega})}=\frac{a+be^{-j\omega}}{\frac{1}{1-\frac{1}{4}e^{-j\omega}}}
=(a+be^{-j\omega})(1-\frac{1}{4}e^{-j\omega})
=a-\frac{b}{4}e^{-2j\omega}+(b-\frac{a}{4})e^{-j\omega}
}
از گزاره‌ی سوم نتیجه می‌شود که باید ضریب $e^{-j\omega}$ صفر باشد؛ زیرا تنها این جمله با دوره‌ی $\pi$ متناوب نیست. بنابراین:
$$
a=4b\implies 
H(e^{j\omega})=4b-\frac{b}{4}e^{-2j\omega}
$$
از طرفی 
$$
H(e^{j{\pi\over 2}})=1
$$
بنابراین
$$
b=\frac{4}{17}\implies a=\frac{16}{17}\implies h[n]=\frac{16}{17}+\frac{4}{17}\delta[n-1]
$$

\Q

الف) به ازای 
$
\omega\ge 0
$
، پاسخ فرکانسی دارای اندازه‌ی 
$
\omega
$
 و فاز 
$
0
$
 و به ازای 
$
\omega< 0
$
، پاسخ فرکانسی دارای اندازه‌ی 
$
-\omega
$
 و فاز 
$
\pi
$
است؛ بنابراین
$$
H(e^{j\omega})=\omega\quad ,\quad -\pi<\omega\le \pi
$$
ب)
$$
h[n]={1\over 2\pi}\int_{-\pi}^{\pi} \omega e^{j\omega n}d\omega=\begin{cases}
{2\pi\over jn}(-1)^n&,\quad n\ne0\\
0&,\quad n=0
\end{cases}
$$
\Q

الف)
$$
X(e^{j\omega})=X(e^{j(\omega-1)})\implies x[n]=x[n]e^{jn}
$$
از آنجا که جز در $n=0$ مقدار 
$
e^{jn}
$
 هیچگاه 1 نمی شود، در نتیجه داریم:
$$
x[n]=0\quad,\quad n\ne 0
$$
 و این گزاره درست است.

ب)
$$
X(e^{j\omega})=X(e^{j(\omega-\pi)})\implies x[n]=x[n]e^{j\pi n}=x[n](-1)^n
$$
از آنجا که در 
$
n=2k
$
 مقدار 
$
(-1)^n
$
 برابر 1 می شود، در نتیجه این گزاره نادرست است.

پ) از این تساوی نتیجه می شود که چون
$
X(e^{j\omega})
$
به دوره‌ی 
$
2\pi
$
 متناوب است، در نتیجه
$
X(e^{j{\omega\over 2}})
$
 نیز با دوره‌ی 
$
2\pi
$
متناوب بوده و داریم:
$$
X(e^{j{\omega+2\pi\over 2}})=X(e^{j({\omega\over 2}+\pi)})=X(e^{j{\omega\over 2}})=X(e^{j\omega})
$$
بنابراین
$$
{1\over2}X(e^{j({\omega\over 2})})+{1\over2}X(e^{j({\omega\over 2}+\pi)})=X(e^{j\omega})
$$
از طرفی می دانیم
\qn{
&x[2n]\iff {1\over2}X(e^{j({\omega\over 2})})+{1\over2}X(e^{j({\omega\over 2}+\pi)})
\\&x[2n+1]\iff {1\over2}e^{-j\omega}(X(e^{j({\omega\over 2})})-X(e^{j({\omega\over 2}+\pi)}))
}{}
بنابراین
\qn{
&x[n]=x[2n]
\\&x[2n+1]=0
}{}
که نتیجه می دهد 
$$
x[n]=0\quad,\quad n\ne 0
$$
و این گزاره درست است.

ت) می دانیم
\qn{
\begin{cases}
x[n/2]&,\quad n \text{زوج} \\
0&,\quad n \text{فرد} 
\end{cases}
\iff X(e^{2j\omega})
}{}
بنابراین
$$
x[n]=\begin{cases}
x[n/2]&,\quad n \text{زوج} \\
0&,\quad n \text{فرد} 
\end{cases}
$$
که مشابه همان قسمت قبل است؛ یعنی نمونه های فرد سیگنال صفر هستند و برای نمونه های زوج
$$
x[n]=x[2n]
$$
و به همین دلیل این گزاره نیز درست است.

\Q

الف-1) این سیستم خطی است زیرا عملیات های 
$
e^{-j\omega}X(e^{j\omega})
$
 و
$
{d\over d\omega}X(e^{j\omega})
$
  به ترتیب معادل با 
$
x[n-1]
$
و
$
-jnx[n]
$
 هستند که هر دو خطی اند؛ ولی چون عملیات 
$
-jnx[n]
$
تغییر پذیر با زمان است، در نتیجه کل سیستم خطی و تغییرپذیر با زمان می شود.

الف-2) به وضوح
$
X(e^{j\omega})=1
$
 بنابراین
$$
Y(e^{j\omega})=2+e^{-j\omega}
$$
و
$$
y[n]=2+\delta[n-1]
$$
ب) می توان نوشت:
$$
Y(e^{j\omega})=\int_{\omega-{\pi\over 4}}^{\omega+{\pi\over 4}}X(e^{j\omega})d\omega=\int_{-\pi}^{\pi}X(e^{j\theta})H(e^{j(\omega-\theta)})d\theta=
$$
که در آن:
$$
H(e^{j\omega})=\begin{cases}
1&,\quad |\omega|<{\pi\over 4}\\
0&,\quad {\pi\over 4}\le|\omega|\le \pi
\end{cases}
$$
بنابراین طبق خواص:
$$
y[n]=2 x[n]{\sin{\pi\over 4}n\over n}
$$
\end{document}