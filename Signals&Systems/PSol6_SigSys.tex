\documentclass[10pt,letterpaper]{article} 
\usepackage{tikz}
\usepackage{toolsper}
%\usepackage{graphicx}‎‎
%\usefonttheme{serif}‎
%\usepackage{ptext}‎
%\usepackage{xepersian}
%\settextfont{B Nazanin}
\usepackage{lipsum}
\setlength{\parindent}{0pt}
\newcommand{\pf}{$\blacksquare$}
\newcommand{\EX}{\Bbb E}
\newcommand{\nl}{\newline\newline}
\newcounter{QuestionNumber}
\setcounter{QuestionNumber}{1}

\newcommand{\wid}{40mm}
%\newcommand

\newcommand{\Q}{
\textbf{
سوال \theQuestionNumber)
}
\stepcounter{QuestionNumber}
}
\begin{document}
\Large
\begin{center}
به نام زیبایی

پاسخ تمرینات سری ششم سیگنال ها و سیستم ها
\hl
\end{center}
\Q

الف)
$$
x(t-t_0)+x(t+t_0)\iff a_ke^{-j{2\pi\over T}kt_0}+a_ke^{j{2\pi\over T}kt_0}=2a_k\cos{{2\pi\over T}kt_0}
$$
ب)
$$
x'(t)\iff jk\omega_0a_k
$$
بنابراین
$$
{d^n\over dt^n}x(t)\iff (jk\omega_0)^na_k
$$
پ) سیگنال $x(t+b)$ دارای ضرایب سری فوریه ی 
$
a_k e^{j{2\pi\over T}kb}
$
 است. از آنجا که فشردگی یا گستردگی سیگنال متناوب، ضرایب سری فوریه‌ی آن را تغییر نمی دهد، بنابراین، سیگنال $x(at+b)$ نیز دارای ضرایب سری فوریه ی 
$
a_k e^{j{2\pi\over T}kb}
$
 است.
\nl
ت)
$$
b_k={1\over T}\int_0^{T_1}x(t)e^{jk{2\pi\over T}t}dt
$$
\Q

الف) 
\qn{
c_k&={1\over T}\int_0^T x(t)y(t)e^{-j2\pi k{1\over T}t}dt
\\&={1\over T}\int_0^T \sum_{m=-\infty}^\infty a_me^{j2\pi m{1\over T}t}
\sum_{n=-\infty}^\infty b_{n}e^{j2\pi n{1\over T}t}
 e^{-j2\pi k{1\over T}t}dt
\\&={1\over T}\sum_{m=-\infty}^\infty \sum_{n=-\infty}^\infty a_mb_{n}
\int_0^T e^{j2\pi m{1\over T}t}
e^{j2\pi n{1\over T}t}
 e^{-j2\pi k{1\over T}t}dt
\\&=\sum_{m=-\infty}^\infty a_mb_{k-m}
}{}


ب) چنانچه فرض کنیم ضرایب سری فوریه ی سیگنال های
$
y(t)=x(t)\sin {2\pi\over T}t
$
 و
$
z(t)=x^*(t)
$ 
 به ترتیب برابر 
$
b_n
$
 و 
$
c_n
$
 باشد، در این صورت
$$
b_k={a_{k-1}-a_{k+1}\over 2j}
$$
و
$$
c_k=a^*_{-k}
$$
در این صورت به کمک تساوی اثبات شده در قسمت قبل:
\qn{
{1\over T}\int_0^T |x(t)|^2\sin {2\pi\over T}tdt
&={1\over T}\int_0^T x^*(t)x(t)\sin {2\pi\over T}tdt
\\&={1\over T}\int_0^T y(t)z(t)dt
\\&=\sum_{k=-\infty}^{\infty} {a_{k-1}-a_{k+1}\over 2j}\cdot a^*_k
\\&=\sum_{k=-\infty}^{\infty} {a^*_ka_{k-1}-a^*_ka_{k+1}\over 2j}
\\&=\sum_{k=-\infty}^{\infty} {a^*_ka_{k-1}\over 2j}-\sum_{k=-\infty}^{\infty}{a^*_ka_{k+1}\over 2j}
\\&=\sum_{k=-\infty}^{\infty} {a^*_{k+1}a_{k}\over 2j}-\sum_{k=-\infty}^{\infty}{a^*_ka_{k+1}\over 2j}
\\&=\sum_{k=-\infty}^{\infty} {a^*_{k+1}a_{k}\over 2j}-\sum_{k=-\infty}^{\infty}{\left[a_ka^*_{k+1}\right]^*\over 2j}
\\&=\sum_{k=-\infty}^{\infty} {a^*_{k+1}a_{k}\over 2j}-\sum_{k=-\infty}^{\infty}{\left[a_ka^*_{k+1}\right]^*\over 2j}
\\&=\sum_{k=-\infty}^{\infty} \Im \left\{a_ka^*_{k+1}\right\}
}{}
%الف)
%\qn{
%a_k&={1\over 2}\int_0^2 x(t)e^{-jk\pi t}dt
%\\&={1\over 2}\int_0^2 x(t)\cos k\pi tdt
%\\&-j{1\over 2}\int_0^2 x(t)\sin k\pi tdt
%\\&={1\over 2}\int_0^2 x(t)\cos k\pi tdt
%\\&=\int_0^1 x(t)\cos k\pi tdt
%\\&=\int_0^1 (1-t)\cos k\pi tdt
%\\&={(-1)^k-1\over k^2\pi^2}
%}{}
%ب)
%\qn{
%a_k&={1\over 3}\int_0^3 x(t)e^{-jk{2\pi\over 3} t}dt
%\\&={1\over 3}\int_0^1 2e^{-jk{2\pi\over 3} t}dt
%\\&+{1\over 3}\int_1^2 e^{-jk{2\pi\over 3} t}dt
%\\&=-{1\over \pi k}(e^{-jk{2\pi\over 3}}-1)
%-{1\over 2\pi k}(e^{-jk{4\pi\over 3}}-e^{-jk{2\pi\over 3}})
%}{}
%
%\Q
%
%الف) از آنجا که سیستم LTI، فرکانس جدیدی به ورودی اضافه نمی کند، باید خروجی به صورت 
%$
%A\cos 2\pi t+B\sin 2\pi t
%$
% باشد. با جایگذاری خواهیم داشت:
%$$
%(4A+2\pi B)\cos 2\pi t+(4B-2\pi A)\sin 2\pi t=\cos 2\pi t
%$$
%که نتیجه می دهد:
%\qn{
%&A={2\over 2\pi^2+8}
%\\&B={\pi\over 2\pi^2+8}
%}{}
%بنابراین ضرایب سری فوریه عبارتند از:
%\qn{
%&a_1={A\over 2}+{B\over 2j}
%\\&a_{-1}={A\over 2}-{B\over 2j}
%}{}
%
%\Q
%
%این سیگنال با دوره ی 1 متناوب است؛ بنابراین
%$$
%x(t)=\sum_{n=-\infty}^\infty e^{-|t-n|}
%$$
%\qn{
%a_k&=\int_0^1 x(t)e^{j2\pi kt}dt
%\\&=
%\int_0^1 \sum_{n=-\infty}^\infty e^{-|t-n|} e^{j2\pi kt}dt
%\\&=
%\sum_{n=-\infty}^\infty\int_0^1 e^{-|t-n|} e^{j2\pi kt}dt
%\\&=
%\sum_{n=-\infty}^\infty\int_0^1 e^{-|t-n|} e^{j2\pi k(t-n)}dt
%\\&=
%\sum_{n=-\infty}^\infty\int_n^{n+1} e^{-|t_1|} e^{j2\pi kt_1}dt_1
%\\&=
%\int_{-\infty}^\infty e^{-|t_1|} e^{j2\pi kt_1}dt_1
%\\&=
%\int_{0}^\infty e^{-t_1} e^{j2\pi kt_1}dt_1
%\\&+
%\int_{-\infty}^0 e^{t_1} e^{j2\pi kt_1}dt_1
%\\&={1\over 1-j2\pi k}+{1\over 1+j2\pi k}
%\\&={2\over 1+4\pi^2 k^2}
%}{}
%
%\Q
%
%شرط اول بیان می دارد:
%$$
%x(t)=x(t)e^{-j{2\pi\over 3}2t}
%$$
%به عبارت دیگر، در نقاط 
%$
%t={3\over 2}k
%$
% که مقدار 
%$
%e^{-j{2\pi\over 3}2t}
%$
% برابر 1 است، سیگنال مقدار دارد و در غیر این نقاط الزاما داریم $x(t)=0$. بنابراین چنین سیگنالی فقط می تواند شامل ضربه در نقاط 
%$
%t={3\over 2}k
%$
% باشد. چنین سیگنالی، دارای فرم کلی زیر است:
%$$
%x(t)=\sum_{n=-\infty}^{\infty}A\delta(t-3n)
%+\sum_{n=-\infty}^{\infty}B\delta(t-3n-{3\over 2})
%$$
%شرط 3، الزام می دارد $A=1$؛ زیرا سطح زیر ضربه برابر 1 است و به طریق مشابه $B=2$؛ در نتیجه
%$$
%x(t)=\sum_{n=-\infty}^{\infty}\delta(t-3n)
%+2\cdot\sum_{n=-\infty}^{\infty}\delta(t-3n-{3\over 2})
%$$
%\Q
%
%الف)
%\qn{
%&x(t)=\sum_{k=-\infty}^{\infty}a_ke^{jkt{2\pi\over t}}\implies
%\\& x'(t)=\sum_{k=-\infty}^{\infty}jk{2\pi\over T}a_ke^{jkt{2\pi\over T}}
%}{}
%
%ب)
%\qn{
%&x(t)=\sum_{k=-\infty}^{\infty}a_ke^{jkt{2\pi\over t}}\implies
%\\& \int_{-\infty}^tx(u)du=\sum_{k=-\infty}^{\infty}a_k\int_{-\infty}^te^{jku{2\pi\over T}}du
%}{}
%از آنجا که انتگرال تابع ثابت روی بازه‌ی نامحدود، بینهایت است باید داشته باشیم $a_0=0$. همچنین 
%$$
%e^{-j\infty \omega_0}=0
%$$
%بنابراین
%$$
%\int_{-\infty}^tx(u)du=\sum_{k=-\infty}^{\infty}a_k\int_{-\infty}^te^{jku{2\pi\over T}}du
%=\int_{-\infty}^tx(u)du=\sum_{k=-\infty}^{\infty}a_k{T\over j2\pi k}e^{jkt{2\pi\over T}}
%$$
%
%پ) اگر $x(t)$ حقیقی باشد، آنگاه با توجه به
%$
%x(t)=x^*(t)
%$
% داریم:
%\qn{
%&x(t)=\sum_{k=-\infty}^{\infty}a_ke^{jkt{2\pi\over t}}
%\\&x^*(t)=\sum_{k=-\infty}^{\infty}a^*_ke^{-jkt{2\pi\over t}}
%\\&=\sum_{k=-\infty}^{\infty}a^*_{-k}e^{-j(-k)t{2\pi\over t}}
%\\&=\sum_{k=-\infty}^{\infty}a^*_{-k}e^{jkt{2\pi\over t}}
%\\&=\sum_{k=-\infty}^{\infty}a_ke^{jkt{2\pi\over t}}
%}{}
%بنابراین 
%$
%a_k=a^*_{-k}
%$
%. این گزاره، معادل با تمام گزاره های این بخش است.
%
%ت) طبق تعریف
%$$
%x_e(t)={x(t)+x(-t)\over 2}={x(t)+x^*(-t)\over 2}=\iff {a_k+a^{*}_k\over 2}=\Re\{a_k\}
%$$
%به طریق مشابه
%$$
%x_o(t)={x(t)-x(-t)\over 2}={x(t)-x^*(-t)\over 2}=\iff {a_k-a^{*}_k\over 2}=j\Im\{a_k\}
%$$
%
%\Q
%
%اگر 
%$
%c_k
%$
% ضرایب سری فوریه ی سیگنال $x(t)y(t)$ باشد، در این صورت:
%\qn{
%c_k&={1\over T}\int_0^T x(t)y^*(t)e^{-j2\pi k{1\over T}t}dt
%\\&={1\over T}\int_0^T \sum_{m=-\infty}^\infty a_me^{j2\pi m{1\over T}t}
%\sum_{n=-\infty}^\infty b^*_{-n}e^{j2\pi n{1\over T}t}
% e^{-j2\pi k{1\over T}t}dt
%\\&={1\over T}\sum_{m=-\infty}^\infty \sum_{n=-\infty}^\infty a_mb^*_{-n}
%\int_0^T e^{j2\pi m{1\over T}t}
%e^{j2\pi n{1\over T}t}
% e^{-j2\pi k{1\over T}t}dt
%\\&=\sum_{m=-\infty}^\infty a_mb^*_{m-k}
%}{}
%از آنجا که برای سیگنال $z(t)$ با ضرایب سری فوریه‌ی $d_k$ داریم
%$$
%d_0={1\over T}\int _0^T z(t)dt
%$$
%در اینصورت 
%$$
%c_0=\sum_{m=-\infty}^\infty a_mb^*_{m}={1\over T}\int_0^T x(t)y^*(t)dt=0
%$$
%$
%\blacksquare
%$
\end{document}