\documentclass[10pt,letterpaper]{article} 
\usepackage{tikz}
\usepackage{toolsper}
%\usepackage{graphicx}‎‎
%\usefonttheme{serif}‎
%\usepackage{ptext}‎
%\usepackage{xepersian}
%\settextfont{B Nazanin}
\usepackage{lipsum}
\setlength{\parindent}{0pt}
%\usepackage{enumitem}
%\setlist[enumerate,1]{label=(\arabic*)}
\newcommand{\pf}{$\blacksquare$}
\newcommand{\EX}{\Bbb E}
\newcommand{\nl}{\newline\newline}

\usepackage{amsmath}
\usepackage{accents}
\newlength{\dhatheight}
\newcommand{\doublehat}[1]{%
    \settoheight{\dhatheight}{\ensuremath{\hat{#1}}}%
    \addtolength{\dhatheight}{-0.35ex}%
    \hat{\vphantom{\rule{1pt}{\dhatheight}}%
    \smash{\hat{#1}}}}

\newcounter{QuestionNumber}
\setcounter{QuestionNumber}{1}

\newcommand{\Q}{
\textbf{
سوال \theQuestionNumber)
}
\stepcounter{QuestionNumber}
}

%\newcommand{\pic}[2]{
%\begin{center}
%\includegraphics[width=#2]{#1}
%\end{center}
%}
\begin{document}
\Large
\begin{center}
به نام زیبایی

تمرینات سری هفتم سیگنال ها و سیستم ها
\hl
\end{center}
\Q

تبدیل فوریه‌ی پیوسته ی هریک از سیگنال های زیر را به دست آورید.

الف)
$
e^{\alpha t}\cos \omega_0t u(t)
$
 که در آن $\alpha<0$. 

ب) 
$
te^{\alpha t}\cos \omega_0t u(t)
$

پ) 
$
{\sin\pi t\over \pi t}\cdot {\sin\pi (t-1)\over \pi (t-1)}
$

ت) 
$
\sum_{n=-\infty}^{\infty}e^{-|t-2n|}
$

ث) 
\picnocapt{PS6_Q1_4}{80mm}


\Q

عکس تبدیل فوریه ی هر یک از موارد زیر را به دست آورید.

الف) 
$
X(\omega)=\cos\left(4\omega+{\pi\over 3}\right)
$

ب) 
$
X(\omega)=2[\delta(\omega-1)-\delta(\omega+1)]
+
3[\delta(\omega-2\pi)+\delta(\omega+2\pi)]
$

پ) 
$
X(\omega)={2\sin 3(\omega-2\pi)\over \omega-2\pi}
$

ت)
\picnocapt{PS6_Q2_4}{80mm}
\newpage
\Q

فرض کنید 
$
X(j\omega)
$
 تبدیل فوریه ی سیگنال 
$
x(t)
$
و به شکل زیر باشد. 
\picnocapt{PS6_Q3_4}{80mm}

در این صورت موارد زیر را محاسبه کنید.

الف) 
$
\measuredangle X(j\omega) 
$

ب) 
$
X(j0)
$

پ) 
$
\int_{-\infty}^\infty X(j\omega) d\omega
$

ت)
$
\int_{-\infty}^\infty X(j\omega) {2\sin\omega\over \omega}e^{j2\omega}d\omega
$

ث)
$
\int_{-\infty}^\infty |X(j\omega)|^2 d\omega
$

ج) تبدیل فوریه معکوس
$
\Re\left\{X(j\omega)\right\}
$

(دقت کنید که تمام موارد بالا را باید بدون محاسبه مستقیم $X(j\omega)$ انجام دهید)
\nl
\Q

الف) سه سیستم با پاسخ ضربه‌ی زیر داده شده اند.
\qn{
&h_1(t)=u(t)
\\& h_2(t)=-2\delta(t)+5e^{-2t}u(t)
\\& h_3(t)=2te^{-t}u(t)
}{}
به کمک تبدیل فوریه، نشان دهید هر سه ی این سیستم ها به ورودی
$
x(t)=\cos t
$
، خروجی یکسان می دهند.

ب) سیستم دیگری را بیابید که پاسخ یکسانی مانند سه سیستم قبل به ورودی 
$
x(t)=\cos t
$
 بدهد.

\textit{
از این تمرین می توان نتیجه گرفت که نمیتوان از روی ورودی تک فرکانس، مشخصه ی سیستم LTI را به طور یکتا یافت.
}
\nl
\Q

فرض کنید یک سیستم LTI با معادله دیفرانسیل زیر داده شده باشد:
$$
{d^2y(t)\over dt^2}+6{dy(t)\over dt}+8y(t)=2x(t)
$$
الف) پاسخ ضربه ی این سیستم را بیابید.

ب) پاسخ این سیستم به ورودی 
$
x(t)=te^{-2t}u(t)
$
 چیست؟
\nl
\Q

سیستمی را با پاسخ فرکانسی زیر در نظر بگیرید:
$$
H(j\omega)={a-j\omega\over a+j\omega}
$$
الف) نمودار دامنه و فاز این پاسخ فرکانسی را رسم کنید و پاسخ ضربه ی این سیستم را به دست آورید.

ب) به ازای $a=1$، پاسخ این سیستم را به ورودی
$
x(t)=\cos{t\over \sqrt 3}+\cos{t}+\cos{t\sqrt 3}
$
 به دست آورید و به طور تقریبی، هردوی ورودی و خروجی را ترسیم کنید.

\textit{
به چنین سیستمی، فیلتر تمام گذر می گویند؛ زیرا دامنه‌ی تمام فرکانس های ورودی با خروجی یکسان است.
}
\nl
\Q

الف) نمودار بلوکی سیستم زیر را با ورودی
$
x(t)
$
 و خروجی
 $
y(t)
$
 در نظر بگیرید. فرض کنید 
$
\omega_1>\omega_0
$
.
\picnocapt{PS7_Q7_b1}{130mm}

اگر نمودار
$
X(j\omega)
$
 و 
$
H(j\omega)
$
 به شکل زیر باشد، نمودار 
$
Z_1(j\omega)
$
،
$
Z_2(j\omega)
$
 و 
$
Y(j\omega)
$
 را رسم‌کنید.
\begin{figure}[htb]
\begin{subfigure}{0.5\textwidth}
\includegraphics[width=70mm]{PS7_Q7_1}
\end{subfigure}
\begin{subfigure}{0.5\textwidth}
\includegraphics[width=70mm]{PS7_Q7_2}
\end{subfigure}
\end{figure}
\newpage
ب) نمودار بلوکی سیستم زیر را با دو ورودی 
$
x_1(t)
$
 و 
$
x_2(t)
$
 و دو خروجی
 $
y_1(t)
$
 و 
$
y_2(t)
$
 در نظر بگیرید. فرض کنید 
$
\omega_1>\omega_0
$
.
\picnocapt{PS7_Q7_b2}{130mm}

اگر نمودارهای 
$
X_1(j\omega)
$
 و 
 $
X_2(j\omega)
$
 به شکل زیر باشد، قسمت حقیقی و موهومی 
$
Z_1(j\omega)
$
، 
$
Z_2(j\omega)
$
، 
$
Z_3(j\omega)
$
، 
$
Y_1(j\omega)
$
 و 
$
Y_2(j\omega)
$
 را رسم کنید.
\begin{figure}[htb]
\begin{subfigure}{0.5\textwidth}
\includegraphics[width=70mm]{PS7_Q7_3}
\end{subfigure}
\begin{subfigure}{0.5\textwidth}
\includegraphics[width=70mm]{PS7_Q7_4}
\end{subfigure}
\end{figure}
نمودار $H(j\omega)$ مانند قسمت قبل است.
\nl
\Q

فرض کنید ورودی یک سیستم LTI با پاسخ فرکانسی $H(j\omega)$، برابر $x(t)$، متناوب با دوره ی $T$ و دارای ضرایب سری فوریه ی $a_k$ باشد.

الف) نشان دهید خروجی $y(t)$ این سیستم متناوب است و دوره تناوب آن را بیابید.

ب) نشان دهید اگر ضرایب سری فوریه ی $y(t)$ برابر $b_k$ باشد، آنگاه:
$$
b_k=a_kH(jk\omega_0)
$$
که در آن 
$
\omega_0={2\pi\over T}
$
.

(راهنمایی: ابتدا سیگنال $y(t)$ را بر حسب $h(t)$ و $x(t)$ نوشته و سپس ضرایب سری فوریه ی $x(t)$ را در حاصل جای گذاری کنید. با ساده سازی، می توانید رابطه ی مورد نظر را اثبات کنید.)
%\nl
\newpage
\Q

سیستم LTI ای با معادله ورودی-خروجی زیر توصیف می شود:
$$
y(t)={1\over \pi}\int_{-\infty}^{\infty} {x(\tau)\over t-\tau}d\tau
$$
الف) پاسخ ضربه‌ی این سیستم را به دست آورید.

ب) نشان دهید 
$$
H(j\omega)=\begin{cases}
-j&,\quad \omega>0\\
j&,\quad \omega<0
\end{cases}
$$

پ) در چنین سیستمی، اصطلاحأ گفته می شود که $y(t)$ تبدیل هیلبرت $x(t)$ است. بسیاری از اوقات نیز نماد زیر استفاده می شود:
$$
y(t)=\hat x(t)
$$
تبدیل هیلبرت $\cos 3t$ چیست؟

ت) (امتیازی) نشان دهید 
$$
\doublehat x(t)=-x(t)
$$
به عبارت دیگر، با دو بار تبدیل هیلبرت گرفتن از یک سیگنال، به قرینه ی آن می رسیم.
\Q

نشان دهید اگر برای دو سیگنال $x(t)$ و $y(t)$ داشته باشیم:
$$
\int_{-\infty}^\infty x(t)y^*(t)dt=0
$$
در این صورت
$$
\int_{-\infty}^\infty X(j\omega)Y^*(j\omega)d\omega=0
$$
(راهنمایی: مشابه سری فوریه، تبدیل فوریه ی دو سیگنال $x(t)$ و $y(t)$ را در رابطه ی انتگرالی حاصلضرب آنها جایگذاری کرده و نتیجه را ساده کنید.)
\nl
\Q

الف) نشان دهید تبدیل فوریه ی 
$
e^{-at}u(t)
$
 برابر 
$
1\over a+j\omega
$
 است.

ب) به کمک دوگانی و با استفاده از خواص تبدیل فوریه، تبدیل فوریه ی 
$$
x(t)={1\over (a+jt)^n}
$$
 را بیابید.

(راهنمایی: می توانید از جدول 4-2 در کتاب اوپنهایم استفاده کنید.)
%\nl
%\Q
\end{document}