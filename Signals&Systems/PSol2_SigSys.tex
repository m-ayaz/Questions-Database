\documentclass[10pt,letterpaper]{article} 
\usepackage{tikz}
\usepackage{toolsper}
%\usepackage{graphicx}‎‎
%\usefonttheme{serif}‎
%\usepackage{ptext}‎
%\usepackage{xepersian}
%\settextfont{B Nazanin}
\usepackage{lipsum}
\setlength{\parindent}{0pt}
\newcommand{\pf}{$\blacksquare$}
\newcommand{\EX}{\Bbb E}
\newcommand{\nl}{\newline\newline}
\newcommand{\Q}[1]{\textbf{
سوال #1)
}}
\newcommand{\pic}[2]{
\begin{center}
\includegraphics[width=#2]{#1}
\end{center}
}
\begin{document}
\Large
\begin{center}
به نام زیبایی

پاسخ تمرینات سری دوم سیگنال ها و سیستم ها
\hl
\end{center}
(یک قرارداد مهم)

تعریف میکنیم
$$
x(\infty)=\lim_{t\to \infty} x(t)
$$
 و 
$$
x(-\infty)=\lim_{t\to -\infty} x(t)
$$


\Q1

الف) ابتدا رابطه‌ی 
$
\phi_{yx}(t)
$
 را می نویسیم:
\qn{
\phi_{yx}(t)&=\int_{-\infty}^\infty y(t+\tau) x^*(\tau)d\tau
\\&=\int_{-\infty}^\infty y(\tau) x^*(\tau-t)d\tau
%\\&=\left[\int_{-\infty}^\infty x(\tau-t) y^*(\tau)d\tau\right]^*
\\&=\left[\underbrace{\int_{-\infty}^\infty x(\tau-t) y^*(\tau)d\tau}_{\phi_{xy}(-t)}\right]^*
\\&=\phi^*_{xy}(-t)
}{}

ب) اگر فرض کنیم 
$
\phi_{yx}(t)
$
 با دوره‌ی $T$ متناوب است در اینصورت باید داشته باشیم:
\qn{
\phi_{yx}(t+T)=\int_{-\infty}^\infty y(t+T+\tau) x^*(\tau)d\tau
=\phi_{yx}(t)=\int_{-\infty}^\infty y(t+\tau) x^*(\tau)d\tau
}{}
در نتیجه باید $y(t)$ نیز متناوب با $T$ باشد. چون $y(t)$ با دوره‌ی $T_2$ متناوب است، در نتیجه باید $T$ مضرب صحیحی از $T_2$ باشد.

پ) با جایگذاری خواهیم داشت:
\qn{
\phi_{xy}(t)&=\int_{-\infty}^\infty x(t+\tau) y^*(\tau)d\tau
\\&=\int_{-\infty}^\infty x(t+\tau) x^*(\tau+T)d\tau
\\&=\int_{-\infty}^\infty x(t+\tau-T) x^*(\tau)d\tau
\\&=\phi_{xx}(t-T)
}{}
\qn{
\phi_{yy}(t)&=\int_{-\infty}^\infty y(t+\tau) y^*(\tau)d\tau
\\&=\int_{-\infty}^\infty x(t+\tau+T) x^*(\tau+T)d\tau
\\&=\int_{-\infty}^\infty x(t+\tau) x^*(\tau)d\tau
\\&=\phi_{xx}(t)
}{}
ت) $\sin \pi t$ دارای ریشه های $2k$ و $2k+1$ به ازای $k$ های صحیح است. از طرفی می دانیم 
\qn{
\sin\pi t=\begin{cases}
\pi(t-2k)&,\quad t=2k
\\
-\pi(t-2k-1)&,\quad t=2k+1
\end{cases}
}
\qn{
\delta(\sin\pi t)&=\begin{cases}
\delta(\pi(t-2k))&,\quad t=2k
\\
\delta(-\pi(t-2k-1))&,\quad t=2k+1
\end{cases}
\\&=\begin{cases}
\delta(\pi(t-2k))&,\quad t=2k
\\
\delta(\pi(t-2k-1))&,\quad t=2k+1
\end{cases}
\\&=\begin{cases}
{1\over \pi}\delta(t-2k)&,\quad t=2k
\\
{1\over \pi}\delta(t-2k-1)&,\quad t=2k+1
\end{cases}
\\&=\begin{cases}
{1\over \pi}\delta(t-k)&,\quad t=k
\end{cases}
}
همچنین 
\qn{
\phi_{xy}(t)&=\int_{-\infty}^\infty x(t+\tau) \delta(\sin \pi \tau)d\tau
\\&=\int_{-\infty}^\infty x(t+\tau) \delta(\sin \pi \tau)d\tau
\\&=\sum_{k}{1\over \pi}\int_{-\infty}^\infty x(t+\tau) \delta(\tau-k)d\tau
\\&=\sum_{k}{1\over \pi}\int_{-\infty}^\infty x(t+k) \delta(\tau-k)d\tau
\\&={1\over \pi}\sum_{k}x(t+k) \int_{-\infty}^\infty \delta(\tau-k)d\tau
\\&={1\over \pi}\sum_{k}x(t+k)
}
\newpage
\Q2

الف)
%رابطه‌ی ورودی خروجی سیستم $S_1$ به صورت زیر است:
%$$
%y_1[n]-3y_1[n-1]+2y_1[n-2]=x[n]
%$$
سیستم $S_1$:
%\begin{figure}[h]1
%\centering
\pic{DR1.pdf}{130mm}
%\end{figure}
سیستم $S_2$:
%\begin{figure}[h]1
%\centering
\pic{DR1_1.pdf}{130mm}
%\end{figure}
ب) سیستم $S$:
%\begin{figure}[h]1
%\centering
\pic{DR2.pdf}{130mm}
%\end{figure}
\Q3

الف) نادرست. سیستم ناپایدار 
$
y(t)=t\cdot x(t)
$
 و سیستم پایدار 
$
y(t)={x(t)\over 1+t^2}
$
 را در نظر بگیرید. ترکیب سری این دو سیستم معادل سیستم زیر است:
$$
y(t)={t\over 1+t^2}x(t)
$$
که یک سیستم پایدار است.
\nl
ب) درست. سیستم علی 
$
y(t)=x(t-1)
$
 و غیرعلی 
$
y(t)=x(t+1)
$
 را در نظر بگیرید. ترکیب سری این دو سیستم، سیستم 
$
y(t)=x(t)
$
 را می دهد که علی است (سیستم اتصال کوتاه).
\nl
پ) درست. فرض کنید پاسخ سیستم پایدار 1 به ورودی $x(t)$ برابر $
y_1(t)
$
 و پاسخ سیستم ناپایدار 2 به این ورودی برابر 
$
y_2(t)
$
 باشد. در اینصورت پاسخ ترکیب موازی این دو سیستم به این ورودی برابر
$
y_1(t)+y_2(t)
$
خواهد بود. فرض کنید سیگنال ورودی $x(t)$ سیگنال کرانداری باشد به گونه ای که 
$
y_2(t)
$ 
در حداقل یک لحظه بیکران شود (چون سیستم 2 ناپایدار است، چنین ورودی ای وجود دارد). به عبارت دیگر
$$
\exists t_0\in\Bbb R \text{\fa{ یا }}t_0=\infty\text{\fa{ یا }}t_0=-\infty\quad,\quad y_2(t_0)=\pm\infty
$$
در این صورت خروجی سیستم موازی معادل در لحظه ی $t_0$ برابر 
$
y_1(t_0)+y_2(t_0)
$
 خواهد بود که جمع یک کراندار و یک بیکران است؛ بنابراین مقدار آن بیکران شده و سیستم ناپایدار می شود.
\nl
ت) از آنجا که در یک سیستم معکوس ناپذیر نمی توان از خروجی به طور یکتا ورودی رسید، سیستم سری معادل، معکوس ناپذیر می شود زیرا حداقل یکی از سیستم های سری شده، این یکتایی بین ورودی و خروجی را از بین می برد. بنابراین حتی معکوس ناپذیر بودن یک سیستم برای از بین بردن یکتایی کافی است.

حالت مهم: اگر دامنه‌ی تعریف سیگنال ها محدود باشد، می توان دو سیستم یافت که یکی معکوس ناپذیر باشد و ترکیب سری آنها معکوس پذیر شود؛ به طور مثال در ترکیب سری دو سیستم زیر:
\pic{pic1.jpg}{130mm}
سیستم
$
y(t)=\sin x(t)
$
 به وضوح معکوس ناپذیر است ( به
$
x(t)=k\pi
$
پاسخ صفر می دهد
)؛ با این حال ترکیب سری این دو سیستم معکوس پذیر می شود (سیستم اتصال کوتاه) زیرا فقط سیگنال های حقیقی مدنظر هستند و باید داشته باشیم 
$
|x(t)|\le 1
$
.
 (دقت کنید که اگر جای دو سیستم را در ترکیب سری عوض کنیم، سیستم حاصل معکوس ناپذیر می شود؛ زیرا دیگر قیدی روی ورودی نداریم)
\nl
ث) نادرست. سیستم های 
$
y_1(t)=x^2(t)
$
 و 
$
y_2(t)=x(t)-x^2(t)
$
 را در نظر بگیرید. سیستم 
$
y_1(t)=x^2(t)
$
 به یک ورودی و قرینه ی آن پاسخ یکسان و سیستم
$
y_2(t)=x(t)-x^2(t)
$
 به ورودی های 
$
x(t)=0
$
 و 
$
x(t)=1
$
 پاسخ تمام صفر می دهد. بنابراین هر دو معکوس ناپذیرند؛ با این حال در ترکیب موازی داریم
$$
y(t)=y_1(t)+y_2(t)=x(t)
$$
 که معکوس پذیر است.
\nl
\Q4

الف)

\qn{
y(t)&=\int_{-\infty}^t x(\tau)-x(\tau-T)d\tau
\\&=\int_{-\infty}^t x(\tau)d\tau -\int_{-\infty}^tx(\tau-T)d\tau
\\&=\int_{-\infty}^t x(\tau)d\tau -\int_{-\infty}^{t-T}x(\tau)d\tau
\\&=\int_{{t-T}}^t x(\tau)d\tau
}
خروجی این سیستم در لحظه ی $t$، به ورودی در لحظات بین $t$ و $t-T$ بستگی دارد؛ در نتیجه این سیستم برای $T\ge 0$ علی و برای $T<0$ غیرعلی است.

این سیستم پایدار نیز هست؛ زیرا برای هر ورودی کراندار، انتگرال زمان محدود ورودی است که خود کراندار می شود:
\qn{
|y(t)|&=\left|\int_{{t-T}}^t x(\tau)d\tau\right|
\\&\le \int_{{t-T}}^t |x(\tau)|d\tau
\\&= \int_{{t-T}}^t Bd\tau
\\&= B\cdot T
}
این سیستم خطی است؛ زیرا ترکیب دو عملگر خطی (یکی کم شدن سیگنال شیفت یافته از سیگنال اصلی و دیگری عملگر انتگرال) است.
\nl
این سیستم مستقل از زمان است. ورودی $x(t-t_0)$ را در نظر بگیرید. در این صورت
\qn{
\hat y(t)&=\int_{-\infty}^{t} x(\underbrace{\tau-t_0}_{u})-x(\tau-t_0-T)d\tau
\\&=\int_{-\infty}^{t-t_0} x(u)-x(u-T)du
\\&=y(t-t_0)
}
که نشان می دهد هر میزان شیفت در ورودی، در خروجی نیز ظاهر می شود و در نتیجه سیستم مستقل از زمان است.
\nl
این سیستم معکوس پذیر نیست؛ زیرا به ورودی های ثابت خروجی تمام صفر می دهد.
\nl
ب) این سیستم علی (و لحظه ای) است؛ زیرا خروجی در لحظه ی $t$، به ورودی فقط در لحظه ی $t$ بستگی دارد.
\nl
پایدار است زیرا 
$
\cos(\cdot)
$
 یک تابع کراندار است.
\nl
خطی نیست زیرا تابع
$
\cos(\cdot)
$
 یک عملگر خطی نیست (به عبارت دیگر
$
\cos(a+b)\ne \cos a+\cos b
$
)
 \nl
تغییر ناپذیر با زمان است؛ هر شیفت در ورودی در خروجی هم ظاهر می شود.
$$
\hat y(t)=\cos x(t-t_0)=y(t-t_0)
$$
معکوس پذیر نیست؛ زیرا به ورودی های 
$
x(t)=k\pi +{\pi\over 2}
$
 پاسخ تمام صفر می دهد.
\nl
پ) علی است؛ طبق تعریف مشتق
$$
y(t)=\lim_{h\to 0^+} {x(t)-x(t-h)\over h}
$$
در نتیجه خروجی در لحظه ی $t$، به ورودی فقط در لحظات 
$
[t-h,t]
$
 بستگی دارد.

دقیق تر آن است که بگوییم سیستم مشتق از چپ سیستم علی است زیر می توان برای هر تابع پیوسته در هر نقطه ی آن همزمان هم مشتق راست و هم مشتق چپ تعریف کرد (از توابع بدرفتار چشم پوشی کنید!). با این حال چون ناپیوستگی ها در تحلیل سیگنال اهمیت چندانی نسبت به تحلیل توابع ندارند، همواره سیستم مشتق گیر به مشتق چپ اطلاق می شود.
\nl
پایدار نیست؛ زیرا به ورودی پله پاسخ ضربه می دهد.
\nl
خطی است؛ طبق تعریف مشتق، هم حد گیری و هم تفاضل سیگنال شیفت یافته از شیگنال اصلی هر دو عملگرهای خطی اند.
\nl
مستقل از زمان است؛ زیرا هر میزان شیفت در سیگنال در مشتق آن هم ظاهر می شود.
\nl
معکوس پذیر نیست؛ زیرا به هر ورودی ثابت خروجی صفر می دهد.
\nl
ت) علی نیست؛ زیرا هنگامی که سیگنال مثبت است، خروجی در لحظه‌ی $n$، به ورودی در لحظه‌ی $n+2$ بستگی دارد(
$
n+2\notin (-\infty ,n]
$
).
\nl
پایدار است زیرا هر دو سیستم 
$
y[n]=x^2[n]
$
 و 
$
y[n]=x[n+2]
$
 پایدار اند.
\qn{
|x[n]|<B\implies |y[n]|\le \max\{|x^2[n]|,|x[n+2]|\}\le \max\{B^2,B\}
}
خطی نیست؛ زیرا هنگامی که سیگنال منفی است، آن را به توان 2 می رساند که می دانیم عملگر مجذور خطی نیست
(
$(a+b)^2\ne a^2+b^2$
).
\nl
تغییر ناپذیر با زمان است؛ زیرا
\qn{
\hat y[n]&=\begin{cases}
x^2[n-n_0]&,\quad x[n-n_0]<0
\\
x[n-n_0+2]&,\quad x[n-n_0]\ge0
\end{cases}
\\&=y[n-n_0]
}
این سیستم معکوس ناپذیر است زیرا به ورودی های 
$
x[n]=-1
$
 و 
$
x[n]=1
$
 پاسخ 
$
y[n]=1
$
 می‌دهد.
\nl
ث) علی نیست؛ زیرا در لحظات مثبت، خروجی در لحظه ی $n$ به ورودی در لحظه ی $n+2$ بستگی دارد.
\nl
با استدلالی مشابه قسمت قبل، پایدار و غیرخطی است.
\nl
وابسته به زمان است؛ زیرا
\qn{
\hat y[n]&=\begin{cases}
x^2[n-n_0]&,\quad n<0
\\
x[n-n_0+2]&,\quad n\ge0
\end{cases}
}
از طرفی 
\qn{
y[n-n_0]&=\begin{cases}
x^2[n-n_0]&,\quad n-n_0<0
\\
x[n-n_0+2]&,\quad n-n_0\ge0
\end{cases}
\\&=\begin{cases}
x^2[n-n_0]&,\quad n<n_0
\\
x[n-n_0+2]&,\quad n\ge n_0
\end{cases}
}
بنابراین
$$
\hat y[n]=y[n-n_0]
$$
همچنین معکوس ناپذیر است زیرا به ورودی های 
$
x[n]=u[-n-1]
$
 و 
$
x[n]=-u[-n-1]
$
 خروجی 
$
y[n]=u[-n-1]
$
 می دهد.
\nl
ج) علی نیست؛ برای زمانهای منفی، خروجی در لحظه ی $t$، به ورودی در لحظه ی $t^2$ بستگی دارد و می دانیم
$$
t<t^2\quad,\quad \forall t<0
$$
خطی است؛ زیرا اگر
\qn{
y_1(t)=
\begin{cases}
x_1(t^2)&,\quad t<0
\\
x_1(t^3)&,\quad t\ge 0
\end{cases}
}
و
\qn{
y_2(t)=
\begin{cases}
x_2(t^2)&,\quad t<0
\\
x_2(t^3)&,\quad t\ge 0
\end{cases}
}
در این صورت پاسخ سیستم به ورودی 
$
ax_1(t)+bx_2(t)
$
 خواهد بود:
\qn{
\hat y(t)&=
\begin{cases}
ax_1(t^2)+bx_2(t^2)&,\quad t<0
\\
ax_1(t^3)+bx_2(t^3)&,\quad t\ge 0
\end{cases}
\\&=
\begin{cases}
ax_1(t^2)&,\quad t<0
\\
ax_1(t^3)&,\quad t\ge 0
\end{cases}
\\&+
\begin{cases}
bx_2(t^2)&,\quad t<0
\\
bx_2(t^3)&,\quad t\ge 0
\end{cases}
\\&=
a\cdot\begin{cases}
x_1(t^2)&,\quad t<0
\\
x_1(t^3)&,\quad t\ge 0
\end{cases}
\\&+
b\cdot
\begin{cases}
x_2(t^2)&,\quad t<0
\\
x_2(t^3)&,\quad t\ge 0
\end{cases}
\\&=
ay_1(t)+by_2(t)
}
این سیستم تغییر پذیر با زمان است؛ زیرا پاسخ آن به ورودی 
$
x(t-t_0)
$
 عبارتست از
\qn{
\hat y(t)=\begin{cases}
x_2(t^2-t_0)&,\quad t<0
\\
x_2(t^3-t_0)&,\quad t\ge 0
\end{cases}
}
همچنین
\qn{
y(t-t_0)=\begin{cases}
x_2\left((t-t_0)^2\right)&,\quad t<0
\\
x_2\left((t-t_0)^3-t_0\right)&,\quad t\ge 0
\end{cases}
}
بنابراین
$$
\hat y(t)=y(t-t_0)
$$
برای بررسی معکوس پذیری، توجه داشته باشید که خروجی این سیستم فقط به لحظات نامنفی سیگنال بستگی دارد؛ زیرا 
$
t^2
$
 برای 
$
t
$
 های منفی و 
$
t^3
$
 برای 
$
t
$ های نامنفی همواره نامنفی اند؛ بنابراین این سیستم معکوس ناپذیر است. به طور مثال، پاسخ آن به دو ورودی 
$
x(t)=u(-t-1)
$
 و
 $
x(t)=-u(-t-1)
$ یکسان است.
\end{document}