\documentclass[10pt,letterpaper]{article} 
\usepackage{tikz}
\usepackage{toolsper}
%\usepackage{graphicx}‎‎
%\usefonttheme{serif}‎
%\usepackage{ptext}‎
%\usepackage{xepersian}
%\settextfont{B Nazanin}
\usepackage{lipsum}
\setlength{\parindent}{0pt}
\newcommand{\pf}{$\blacksquare$}
\newcommand{\EX}{\Bbb E}
\newcommand{\nl}{\newline\newline}
\newcounter{QuestionNumber}
\setcounter{QuestionNumber}{1}

\newcommand{\wid}{40mm}
%\newcommand

\newcommand{\Q}{
\textbf{
سوال \theQuestionNumber)
}
\stepcounter{QuestionNumber}
}
\begin{document}
\Large
\begin{center}
به نام زیبایی

پاسخ تمرینات سری هفتم سیگنال ها و سیستم ها
\hl
\end{center}
\Q

الف) 
\qn{
x(t)&=e^{\alpha t}\cos\omega_0 t u(t)
\\&={1\over 2}(e^{(\alpha+j\omega_0) t}+e^{(\alpha-j\omega_0) t})u(t)
}{}
از طرفی
\qn{
\int_{-\infty}^\infty e^{(\alpha+j\omega) t} e^{-j\omega t}u(t)dt
&=
\int_{0}^\infty e^{(\alpha+j\omega_0-j\omega) t} dt
\\&={1\over \alpha+j\omega_0-j\omega}e^{(\alpha+j\omega_0-j\omega) t}\Big|_0^\infty
\\&={-1\over \alpha+j\omega_0-j\omega}
}{}
بنابراین، تبدیل فوریه برابر است با
\qn{
X(j\omega)={-{1\over 2}\over \alpha+j\omega_0-j\omega}
+{-{1\over 2}\over \alpha-j\omega_0-j\omega}
}{}
\nl
ب) ضرب در $t$ در حوزه‌ی زمان، معادل با مشتق در حوزه‌ی فرکانس است؛ بنابراین کافی است از تبدیل فوریه ی بالا مشتق گرفته و در j ضرب کنیم. با این کار:
\qn{
X(j\omega)={{1\over 2}\over (\alpha+j\omega_0-j\omega)^2}
+{{1\over 2}\over (\alpha-j\omega_0-j\omega)^2}
}{}
\nl
پ) 
$
\text{sinc}(t)
$
 دارای تبدیل فوریه ی 
$
\Pi({\omega\over 2\pi})
$
 و 
$
\text{sinc}(t)
$
 دارای تبدیل فوریه‌ی 
$
\Pi({\omega\over 2\pi})e^{-j\omega}
$
 است. از طرفی ضرب دو سیگنال در زمان، معادل با کانولوشن تبدیل های فوریه‌ی آنها در فرکانس است؛ بنابراین
\qn{
X(j\omega)&={1\over 2\pi}\Pi({\omega\over 2\pi})*\Pi({\omega\over 2\pi})e^{-j\omega}
\\&={1\over 2\pi}\int_{-\infty}^\infty \Pi({\omega-u\over 2\pi})\Pi({u\over 2\pi})e^{-ju} du
\\&={1\over 2\pi}\int_{-\pi}^\pi \Pi({\omega-u\over 2\pi})e^{-ju} du
}{}
تابع $\Pi({\omega-u\over 2\pi})$ در بازه ای که
$
\omega-\pi<u<\omega+\pi
$
 مقدار ثابت 1 دارد و در غیر این صورت برابر 0 است. به ازای 
$
0\le \omega\le \pi
$

\qn{
X(j\omega)&={1\over 2\pi}\int_{-\pi}^\pi \Pi({\omega-u\over 2\pi})e^{-ju} du
\\&={1\over 2\pi}\int_{\omega-\pi}^\pi e^{-ju} du
\\&={j\over 2\pi} e^{-ju} \Big|_{\omega-\pi}^\pi
\\&={j\over 2\pi}[e^{-j\omega}-1]
}{}
و به ازای 
$
-\pi\le \omega\le 0
$

\qn{
X(j\omega)&={1\over 2\pi}\int_{-\pi}^\pi \Pi({\omega-u\over 2\pi})e^{-ju} du
\\&={1\over 2\pi}\int_{-\pi}^{\omega+\pi} e^{-ju} du
\\&={j\over 2\pi} e^{-ju} \Big|_{-\pi}^{\omega+\pi}
\\&={j\over 2\pi}[1+e^{-j(\pi+\omega)}]
\\&={j\over 2\pi}[1-e^{-j\omega}]
}{}
بنابراین
\qn{
X(j\omega)&=\begin{cases}
{j\over 2\pi}[e^{-j\omega}-1]&,\quad 0\le\omega\le \pi
\\
{j\over 2\pi}[1-e^{-j\omega}]&,\quad -\pi\le\omega\le 0
\end{cases}
}{}
ت) این سیگنال عبارت است از خروجی سیستمی با پاسخ ضربه‌ی 
$
e^{-|t|}
$
 و پاسخ فرکانسی
$
H(j\omega)={2\over 1+\omega^2}
$
 که ورودی متناوب
$
x(t)=\sum_{k=-\infty}^{\infty}\delta(t-2k)
$
 به آن اعمال شده است. ضرایب سری فوریه‌ی ورودی بسادگی برابر 
$
a_k={1\over 2}
$
است. پس از آنجا که خروجی نیز متناوب می شود، ضرایب سری فوریه‎ی آن عبارتست از (اثبات در سوال 8):
$$
b_k=a_kH(jk\omega_0)={1\over 2}H(jk\pi)={1\over 1+k^2\pi^2}
$$
در نتیجه تبدیل فوریه‌ی خروجی و در نتیجه سیگنال مورد نظر ما عبارتست از:
$$
Y(j\omega)=\sum_{k=-\infty}^{\infty}{2\pi\over 1+k^2\pi^2}\delta(\omega-k\pi)
$$
ث) سیگنال $x(t)$ فرد است؛ در نتیجه:
\qn{
X(j\omega)&=\int_{-\infty}^\infty x(t)e^{-j\omega t}dt
\\&=\int_{-2}^2 x(t)e^{-j\omega t}dt
\\&=\int_{-2}^2 x(t)[\cos\omega t-j\sin\omega t]dt
\\&=-j\int_{-2}^2 x(t)\sin\omega tdt
\\&=-2j\int_{0}^2 x(t)\sin\omega tdt
\\&=-2j\int_{0}^1 x(t)\sin\omega tdt
\\&-2j\int_{1}^2 x(t)\sin\omega tdt
\\&=-2j\int_{0}^1 t\sin\omega tdt
\\&-2j\int_{1}^2 \sin\omega tdt
\\&=-2j{\sin\omega-\omega\cos2\omega\over\omega^2}
}{}
\Q

الف)
\qn{
&\cos (4\omega+{\pi\over 3})={1\over 2}(e^{j4\omega+{j\pi\over 3}}+e^{-j4\omega-{j\pi\over 3}})\iff\\& {1\over 2}[e^{j{\pi\over 3}}\delta(t+4)
+e^{-j{\pi\over 3}}\delta(t-4)
]
}{}
ب) 
\qn{
x(t)&={1\over \pi}(e^{jt}-e^{-jt})+{3\over 2\pi}(e^{j2\pi t}+e^{-j2\pi t})
\\&={2j\over \pi}\sin t+{3\over \pi}\cos2\pi t
}{}
پ) می خواهیم عکس تبدیل فوریه‌ی 
$
6\text{sinc}({3\over \pi}(\omega-2\pi))
$
 را پیدا کنیم. برای حل این سوال، از مجموعه ای از خواص تبدیل فوریه استفاده می کنیم. از آنجا که
$$
\Pi(t)\iff \text{sinc}({\omega\over 2\pi})
$$
بنابراین
\qn{
&{1\over 6}\Pi({t\over 6})\iff\text{sinc}({3\omega\over \pi})
\\&\Pi({t\over 6})\iff6\text{sinc}({3\omega\over \pi})
\\&\Pi({t\over 6})e^{j2\pi t}\iff6\text{sinc}({3(\omega-2\pi)\over \pi})
}{}
پس سیگنال مورد نظر ما، 
$
\Pi({t\over 6})e^{j2\pi t}
$
 است.
\nl
ت) مشابه قسمت ث سوال پیش، $X(j\omega)$ فرد است؛ در نتیجه:
\qn{
x(t)&={1\over 2\pi}\int_{-\infty}^\infty X(j\omega)e^{j\omega t}d\omega
\\&={1\over 2\pi}\int_{-3}^3 X(j\omega)e^{j\omega t}d\omega
\\&={1\over 2\pi}\int_{-3}^3 X(j\omega)[\cos\omega t+j\sin\omega t]d\omega
\\&={j\over 2\pi}\int_{-3}^3 X(j\omega)\sin\omega td\omega
\\&={j\over \pi}\int_0^3 X(j\omega)\sin\omega td\omega
\\&={j\over \pi}\int_1^2 X(j\omega)\sin\omega td\omega
\\&+{j\over \pi}\int_2^3 X(j\omega)\sin\omega td\omega
\\&={j\over \pi}\int_1^2 (\omega-1)\sin\omega td\omega
\\&+{j\over \pi}\int_2^3 \sin\omega td\omega
\\&={\sin 2t-\sin t -t\cos 3t\over t^2}
}{}


\Q

الف) سیگنال $x(t)$، شیفت یافته‌ی یک سیگنال زوج به اندازه ی 1 واحد به راست است. از آنجا که تبدیل فوریه ی سیگنال حقیقی و زوج، خود حقیقی و زوج است، بنابراین دارای فاز صفر بوده و با شیفت 1 واحد به راست، تبدیل فوریه در عبارت 
$
e^{-j\omega}
$
 ضرب می شود که باعث می شود فاز خالص 
$
-\omega
$
 داشته باشیم.
\nl
ب)
$$
X(j0)=\int_{-\infty}^\infty x(t)dt=7
$$
پ)
$$
x(0)={1\over 2\pi} \int_{-\infty}^\infty X(j\omega)d\omega=2\implies
\int_{-\infty}^\infty x(j\omega)d\omega=4\pi
$$
\nl
ت) عکس تبدیل فوریه‌ی 
$
2{\sin\omega\over \omega}e^{2j\omega}
$
 برابر 
$
y(t)=\Pi\left({t\over 2}+1\right)
$
 است. طبق تساوی
$$
\int_{-\infty}^{\infty} x(t)y(-t)dt={1\over 2\pi}\int_{-\infty}^{\infty} X(j\omega)Y(j\omega)d\omega
$$
خواهیم داشت
$$
\int_{-\infty}^{\infty} X(j\omega)Y(j\omega)d\omega=7\pi
$$
ث) طبق اتحاد پارسوال:
$$
\int_{-\infty}^\infty |X(j\omega)|^2d\omega=
2\pi\int_{-\infty}^\infty |x(t)|^2dt={76\pi\over 3}
$$
\nl
ج)

از آنجا که سیگنال حقیقی است، $
\Re\{X(j\omega)\}
$
 برابر تبدیل فوریه ی قسمت زوج سیگنال است؛ یعنی
$$
\Re\{X(j\omega)\}\iff x_e(t)={x(t)+x(-t)\over 2}
$$
این سیگنال دارای شکل زیر است:
\picnocapt{PSol7_Q3_f.eps}{120mm}
\nl
\Q

پاسخ فرکانسی این سه سیستم برابر است با
\qn{
&H_1(j\omega)={1\over j\omega}+\pi\delta(\omega)
\\&H_2(j\omega)=-2+{5\over 2+j\omega}
\\&H_3(j\omega)={2\over (1+j\omega)^2}
}{}
از آنجا که سیستم LTI فرکانس جدیدی به سیستم اضافه نمی کند، پاسخ به یک سیگنال کسینوسی معادل با سیگنال کسینوسی ای با همان فرکانس است؛ پس با قرار دادن
$
\omega=1
$
 در تمام پاسخ فرکانسی ها بالا، به مقدار یکسان
$$
H_1(j1)=H_2(j1)=H_3(j1)=-j
$$
می رسیم که معادل با پاسخ 
$
\cos(t-{\pi\over 2})
$
است
. 
\nl
ب) سیستمی با پاسخ فرکانسی
$
H(j\omega)=-j{a^2+1\over a^2+\omega^2}\omega
$
 نیز به ازای $a\ne 0$ دارای چنین خاصیتی است؛ زیرا:
$$
H(j1)=-j
$$
\Q

برای این سیستم:
$$
[(j\omega)^2+6j\omega+8]Y(j\omega)=2X(j\omega)
$$
بنابراین
$$
H(j\omega)={Y(j\omega)\over X(j\omega)}={2\over (j\omega)^2+(j\omega)^2+8}
={2\over (j\omega+2)(j\omega+4)}
={1\over j\omega+2}-{1\over j\omega+4}
$$
که معادل با
$$
h(t)=(e^{-2t}-e^{-4t})u(t)
$$
%است (این سیستم ضد علی است). 

ب) تبدیل فوریه‌ی ورودی عبارت است از:
$$
X(j\omega)=j{d\over d\omega}{1\over 2+j\omega}={1\over (2+j\omega)^2}
$$
بنابراین
\qn{
Y(j\omega)&=X(j\omega)H(j\omega)
\\&={1\over (2+j\omega)^3}-{1\over (2+j\omega)^2(4+j\omega)}
\\&={1\over (2+j\omega)^3}+{1/4\over 4+j\omega}
-{1/4\over 2+j\omega}+{1/8\over (2+j\omega)^2}
}{}
و در نتیجه
\qn{
y(t)=\left[{1\over 2}t^2e^{-2t}-{1\over 4}e^{-2t}+
{1\over 8}te^{-2t}+{1\over 4}e^{-4t}
\right]u(t)
}{}
\Q

بسادگی دیده می شود:
$$
|H(j\omega)|={\sqrt{a^2+\omega^2}\over \sqrt{a^2+\omega^2}}=1
$$
و
$$
\angle H(j\omega)=\tan^{-1}-{\omega\over a}-\tan^{-1}{\omega\over a}=-2\tan^{-1}{\omega\over a}
$$
\begin{figure}[h!]
\begin{subfigure}{0.5\textwidth}
\includegraphics[width=80mm]{PSol7_Q6_2.eps}
\end{subfigure}
\begin{subfigure}{0.5\textwidth}
\includegraphics[width=80mm]{PSol7_Q6_1.eps}
\end{subfigure}
\end{figure}

به ازای 
$
\omega={1\over \sqrt 3}
$
،
$
\omega=1
$
 و 
$
\omega=\sqrt 3
$
، فاز این سیستم برابر خواهد بود با
$
-{\pi\over 3}
$
،
$
-{\pi\over 2}
$
 و 
$
-{2\pi\over 3}
$
. در نتیجه
$$
y(t)=\cos({t\over \sqrt 3}-{\pi\over 3})+\cos(t-{\pi\over 2})
+\cos({t\sqrt 3}-{2\pi\over 3})
$$
شکل ورودی و خروجی به صورت زیر است:
\begin{figure}[h!]
\centering
\includegraphics[width=100mm]{PSol7_Q6_4.eps}
\end{figure}
%\newpage
\nl
\Q
 الف)
$$
z_1(t)=x(t)\cos \omega_1 t\implies Z_1(j\omega)={X(j(\omega-\omega_1))+X(j(\omega+\omega_1))\over 2}
$$
\qn{
&z_2(t)=2z_1(t)\cos \omega_1 t\implies
\\&
Z_2(j\omega)={Z_1(j(\omega-\omega_1))+Z_1(j(\omega+\omega_1))}
\\&=X(j\omega)+{X(j(\omega-2\omega_1))+X(j(\omega+2\omega_1))\over 2}
}{}
$$
Y(j\omega)=X(j\omega)
$$
\begin{figure}[h!]
\begin{subfigure}{0.33\textwidth}
\includegraphics[width=50mm]{PSol7_Q7_3.eps}
\end{subfigure}
\begin{subfigure}{0.33\textwidth}
\includegraphics[width=50mm]{PSol7_Q7_2.eps}
\end{subfigure}
\begin{subfigure}{0.33\textwidth}
\includegraphics[width=50mm]{PSol7_Q7_1.eps}
\end{subfigure}
\end{figure}

ب)
\qn{
Z_1(j\omega)&={1\over 2}[X_1(j(\omega-\omega_1))+X_1(j(\omega+\omega_1))]
\\&+
{1\over 2j}[X_2(j(\omega-\omega_1))-X_2(j(\omega+\omega_1))]
}{}
\qn{
Z_2(j\omega)&=Z_1(j(\omega-\omega_1))+Z_1(j(\omega+\omega_1))
\\&=
{1\over 2}X_1(j(\omega-2\omega_1))+X_1(j\omega)+{1\over 2}X_1(j(\omega+2\omega_1))
\\&+
{1\over 2j}[X_2(j(\omega-2\omega_1))-X_2(j(\omega+2\omega_1))]
}{}
%%%%%%%%%
\qn{
Z_3(j\omega)&=-{1\over j}[Z_1(j(\omega-\omega_1))-Z_1(j(\omega+\omega_1))]
\\&=
-{1\over 2j}X_1(j(\omega-2\omega_1))-{1\over 2j}X_1(j(\omega+2\omega_1))
\\&-
{1\over 2}X_2(j(\omega-2\omega_1))-X_2(j\omega)-{1\over 2}X_2(j(\omega+2\omega_1))
}{}
بنابراین
\qn{
&Y_1(j\omega)=X_1(j\omega)
\\&Y_2(j\omega)=X_2(j\omega)
}{}
\begin{figure}[h!]
\centering
\begin{subfigure}{0.49\textwidth}
\includegraphics[width=80mm]{PSol7_Q7_5.eps}
\end{subfigure}
\begin{subfigure}{0.49\textwidth}
\includegraphics[width=80mm]{PSol7_Q7_4.eps}
\end{subfigure}
\begin{subfigure}{0.49\textwidth}
\includegraphics[width=80mm]{PSol7_Q7_7.eps}
\end{subfigure}
\begin{subfigure}{0.49\textwidth}
\includegraphics[width=80mm]{PSol7_Q7_6.eps}
\end{subfigure}
\begin{subfigure}{0.49\textwidth}
\includegraphics[width=80mm]{PSol7_Q7_8.eps}
\end{subfigure}
\begin{subfigure}{0.49\textwidth}
\includegraphics[width=80mm]{PSol7_Q7_9.eps}
\end{subfigure}
\begin{subfigure}{0.49\textwidth}
\includegraphics[width=80mm]{PSol7_Q7_11.eps}
\end{subfigure}
\begin{subfigure}{0.49\textwidth}
\includegraphics[width=80mm]{PSol7_Q7_10.eps}
\end{subfigure}
\end{figure}
\Q

الف) چون
$$
y(t)=x(t)*h(t)
$$
و 
$$
y(t+T)=x(t+T)*h(t)
$$
در نتیجه اگر دوره‌ی تناوب $x(t)$ برابر $T$ باشد، خواهیم داشت:
$$
x(t+T)=x(t)\implies y(t+T)=y(t)
$$
ب)
\qn{
y(t)&=x(t)*h(t)
=\sum_{k=-\infty}^\infty a_ke^{jk\omega_0 t}*h(t)
\\&\iff
2\pi \sum_{k=-\infty}^\infty a_k\delta(\omega-k\omega_0)H(j\omega)
\\&=
2\pi \sum_{k=-\infty}^\infty a_kH(jk\omega_0)\delta(\omega-k\omega_0)
}{}
از آنجا که تبدیل فوریه ی $y(t)$ شامل ضربه هایی با اندازه‌ی $a_kH(jk\omega_0)$ است، پس ضرایب سری فوریه‌ی $y(t)$ برابر $a_kH(jk\omega_0)$ بوده و اثبات کامل است.
\nl
\Q

الف)
$$
h(t)={1\over \pi t}
$$
ب)
از آنجا که
$$
u(t)\iff {1\over j\omega}+\pi\delta(\omega)
$$
طبق دوگانی
$$
{1\over jt}+\pi\delta(t)\iff 2\pi u(-\omega)
$$
بنابراین
$$
{1\over jt}\iff 2\pi u(-\omega)-\pi
$$
$$
{1\over \pi t}\iff 2j u(-\omega)-j
$$
که با شکستن آن روی مقادیر منفی و مثبت $\omega$ می توان به نتیجه دلخواه رسید.

پ)
$$
\cos 3t\iff \pi[\delta(\omega-3)+\delta(\omega+3)]
$$
بنابراین تبدیل هیلبرت آن در حوزه‌ی فرکانس برابر است با
$$
\pi[-j\delta(\omega-3)+j\delta(\omega+3)]={\pi\over j}
\pi[\delta(\omega-3)-\delta(\omega+3)]
$$
که معادل با سیگنال $\sin 3t$ است.
\nl
ت)
$$
h(t)*h(t)\iff H(j\omega)\times H(j\omega)=H^2(j\omega)=-1
$$
\Q

تعریف کنید 
$
z(t)=x(t)y^*(t)
$
. در این صورت طبق خاصیت ضرب دو سیگنال در حوزه‌ی زمان
$$
x(t)y^*(t)\iff {1\over 2\pi}X(j\omega)*Y^*(-j\omega)
$$
از آنجا که
$$
Z(j0)=\int_{-\infty}^\infty z(t)dt
$$
در نتیجه
\qn{
\int_{-\infty}^\infty z(t)dt
&=
\int_{-\infty}^\infty x(t)y^*(t)dt
\\&=
{1\over 2\pi}\int_{-\infty}^\infty X(ju)Y^*(j[u-\omega])du\Bigg|_{\omega=0}
\\&=
{1\over 2\pi}\int_{-\infty}^\infty X(ju)Y^*(ju)du
\\&=0
}{}
در نتیجه 
$
\int_{-\infty}^\infty X(ju)Y^*(ju)du=0
$
 و اثبات کامل است.
\nl
\Q

الف) به ازای $a>0$
\qn{
X(j\omega)&=\int_{-\infty}^\infty e^{-at}e^{-j\omega t}u(t)dt
\\&=\int_{0}^\infty e^{-at}e^{-j\omega t}dt
\\&={1\over a+j\omega}
}{}
ب) با $n$ بار ضرب کردن سیگنال حوزه‌ی زمان فوق در $t$ خواهیم داشت:
$$
t^ne^{-at}u(t)\iff j^n {d^n\over d\omega^n}{1\over a+j\omega}
={j^n\times j^n \times (-1)^n (n-1)!\over (a+j\omega)^n}
={(n-1)!\over (a+j\omega)^n}
$$
در نتیجه طبق دوگانی
$$
(n-1)!{1\over (a+jt)^n}\iff 2\pi (-\omega)^n e^{a\omega} u(-\omega)
$$
 و خواهیم داشت
$$
{1\over (a+jt)^n}\iff {2\pi(-1)^n\over (n-1)!} \omega^n e^{a\omega} u(-\omega)
$$
\end{document}