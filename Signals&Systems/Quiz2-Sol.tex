\documentclass{article}


\usepackage{amsmath,amssymb,graphicx,xepersian}


\begin{document}

%{
%\centering
%به نام زیبایی
%
%کوئیز 2 درس سیگنال ها و سیستم ها
%}

\large

برای سیستم های LTI با تابع ضربه های زیر:

1.
$h(t)=e^{t}u(-t+1)$

2.
$h(t)=e^{-t}u(t+1)$

3.
$h(t)=\frac{1}{t^2+1}u(t)$

4.
$h[n]=\frac{n}{n^3+0.5n}$

5.
$h[n]=(\frac{1}{2})^nu[n-3]$



الف) سیستم علی سیستمی است که خروجی آن در یک لحظه، به ورودی تنها در آن لحظه و لحظه‌های پیش از آن بستگی داشته باشد. اگر سیستم LTI باشد، پاسخ ضربه‌ی آن باید در لحظات منفی برابر صفر باشد.

سیستم های 3 و 5 علی و سایرین غیرعلی هستند.

ب) سیستم پایدار سیستمی است که به ورودی کراندار، خروجی کراندار بدهد. اگر سیستم LTI باشد، باید پاسخ ضربه‌ی آن مطلقأ انتگرال پذیر (جمع پذیر) باشد؛ به عبارت دیگر
$$
\int_{-\infty}^\infty |h(t)|dt<\infty
$$
یا
$$
\sum_{n=-\infty}^\infty |h(n)|<\infty
$$

در این صورت، تمام این سیستم های پایدارند؛ زیرا انتگرال (جمع) محدود دارند.




\end{document}