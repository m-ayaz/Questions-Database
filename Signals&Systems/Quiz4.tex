\documentclass{article}



\usepackage{amsmath,amssymb,graphicx,geometry,caption}
\usepackage{xepersian}

\setlength{\parindent}{0mm}
\setlength{\parskip}{3mm}




\begin{document}

\large

\begin{center}
به نام او

\hrulefill
\end{center}

سوال) تبدیل فوریه‌ی سیگنال 
$$
x(t)=(-\frac{1}{2})^tu(t)+(\frac{1}{3})^tu(t)
$$
را به دست آورید (
$
(-\frac{1}{2})^t=(\frac{1}{2})^te^{j\pi t}
$
).

پاسخ) می دانیم
$$
(-\frac{1}{2})^t=(\frac{1}{2})^te^{j\pi t}
=e^{t(j\pi-\ln 2)}
$$
و
$$
(\frac{1}{3})^t=e^{-t\ln 3}
.
$$
بنابراین، طبق تعریف تبدیل فوریه داریم:
\[\begin{split}
X(j\omega)
&=
\int_{-\infty}^{\infty} x(t)e^{-j\omega t}dt
\\&=
\int_{-\infty}^{\infty} \left[
e^{t(j\pi-\ln 2)}+e^{-t\ln 3}
\right]u(t)e^{-j\omega t}dt
\\&=
\int_{0}^{\infty} \left[
e^{t(j\pi-\ln 2)}+e^{-t\ln 3}
\right]e^{-j\omega t}dt
\\&=
\int_{0}^{\infty} \left[
e^{t(j\pi-j\omega-\ln 2)}+e^{-(\ln 3+j\omega) t}
\right]dt
\\&=
\frac{1}{j\pi-j\omega-\ln 2}e^{t(j\pi-j\omega-\ln 2)}\Big|_{0}^{\infty}
\\&-\frac{1}{\ln 3+j\omega}e^{-(\ln 3+j\omega) t}\Big|_{0}^{\infty}
\\&=
\frac{1}{j(\omega-\pi)+\ln 2}
+\frac{1}{j\omega+\ln 3}.
\end{split}\]


\end{document}