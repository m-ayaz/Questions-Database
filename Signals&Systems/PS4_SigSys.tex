\documentclass[10pt,letterpaper]{article} 
\usepackage{tikz}
\usepackage{toolsper}
%\usepackage{graphicx}‎‎
%\usefonttheme{serif}‎
%\usepackage{ptext}‎
%\usepackage{xepersian}
%\settextfont{B Nazanin}
\usepackage{lipsum}
\setlength{\parindent}{0pt}
\newcommand{\pf}{$\blacksquare$}
\newcommand{\EX}{\Bbb E}
\newcommand{\nl}{\newline\newline}
\newcommand{\Q}[1]{\textbf{
سوال #1)
}}
%\newcommand{\pic}[2]{
%\begin{center}
%\includegraphics[width=#2]{#1}
%\end{center}
%}
\begin{document}
\Large
\begin{center}
به نام زیبایی

تمرینات سری چهارم سیگنال ها و سیستم ها
\hl
\end{center}
\Q1

کانولوشن را برای هر یک از جفت سیگنال های (پیوسته یا گسسته‌ی) زیر به صورت نموداری (به کمک شیفت های متوالی نمودار سیگنال ها) به دست آورید.

الف) 
\[
\includegraphics[width=60mm]{PS4_Q1_a_1.eps}
%\longrightarrow
\includegraphics[width=60mm]{PS4_Q1_a_2.eps}
\]

ب) 
\[
\includegraphics[width=60mm]{PS4_Q1_b_1.eps}
%\longrightarrow
\includegraphics[width=60mm]{PS4_Q1_b_2.eps}
\]
پ) 
\[
\includegraphics[width=60mm]{PS4_Q2_b_1.eps}
%\longrightarrow
\includegraphics[width=60mm]{PS4_Q2_b_2.eps}
\]

ت) (
$x(t)$
 یک دوره‌ی تناوب از سیگنال 
$\sin \pi t$
 است
)
\[
\includegraphics[width=60mm]{PS4_Q2_a_1.eps}
%\longrightarrow
\includegraphics[width=60mm]{PS4_Q2_a_2.eps}
\]
\nl
\Q3

سیستم زمان گسسته‌ی زیر را در نظر بگیرید:

$$
y[n]+2y[n-1]=x[n]
$$

فرض کنید این سیستم در شرایط اولیه‌ی صفر قرار دارد. در این صورت پاسخ آن را به ورودی زیر
\[
\includegraphics[width=60mm]{PS4_Q3.eps}
\]
با حل معادله تفاضلی فوق به صورت بازگشتی ترسیم کنید.
\nl
\Q4

یک سیستم زمان گسسته با پاسخ ضربه‌ی 
$$
h[n]=(n+1)\alpha^nu[n]
$$
را در نظر بگیرید که در آن 
$
|\alpha|<1
$
. نشان دهید که پاسخ پله‌ی آن به صورت 
$$
s[n]=\left[{1\over (\alpha-1)^2}-{\alpha^{n+1}\over (\alpha-1)^2}+{(n+1)\alpha^{n+1}\over \alpha-1}\right]u[n]
$$
 است.

(راهنمایی:
$
\sum_{k=0}^{N} (k+1)\alpha^k={d\over d\alpha}\sum_{k=0}^{N+1} \alpha^k
$
)
\nl
\Q5
\textbf{
(امتیازی)
}

فرض کنید 
$$
u_1[n]=\delta[n]-\delta[n-1]
$$
الف) سیستم گسسته با پاسخ ضربه $u_1[n]$ معادل چه عملی را در یک سیستم پیوسته انجام می دهد؟ (در شکل 3-24 از کتاب اوپنهایم می‌توان نمونه ای از کارکرد این سیستم ها را بر روی تصاویر مشاهده کرد. به لبه‌های تصاویر ورودی و خروجی سیستم دقت کنید.)
\nl
ب) اکنون تعریف کنید
$$
u_k[n]=\underbrace{u_1[n]*u_1[n]*\cdots *u_1[n]}_{\text{\rl{ بار}} k}
$$
به عبارت دیگر سیگنال $u_k[n]$ را از $k$ با کانولوشن سیگنال $u_1[n]$ با خودش تولید کرده‌ایم که در آن $k$ یک عدد طبیعی است. نشان دهید
$$
u_k[n]={(-1)^n k!\over n! (k-n)!}\Big(u[n]-u[n-k-1]\Big)
$$
که در آن $u[n]$، سیگنال پله است.


(راهنمایی: از استقرا استفاده کنید؛ یعنی ابتدا نشان دهید این رابطه برای $k=2$ صحیح است. می‌توانید برای نشان دادن آن، $u_2[n]$ را ترسیم کنید. سپس فرض کنید این رابطه‌ی برای $k$ صحیح است و تلاش کنید از روی آن، به
 $u_{k+1}[n]$
 برسید؛ به عبارت دیگر نشان دهید 
$
u_{k+1}[n]=u_k[n]*u_1[n]
$)
\nl
پ) سیستم گسسته با پاسخ ضربه $u_k[n]$ معادل چه عملی را در یک سیستم پیوسته انجام می دهد؟
%\newpage

%\qn{
%s[n]&=\sum_{k=-\infty}^n h[k]
%\\&=\sum_{k=-\infty}^n (k+1)\alpha^ku[k]
%\\&=u[n]\sum_{k=0}^n (k+1)\alpha^k
%\\&=u[n]\sum_{k=0}^n {d\over d\alpha}\alpha^{k+1}
%\\&=u[n]{d\over d\alpha}\sum_{k=0}^n\alpha^{k+1}
%\\&=u[n]{d\over d\alpha}{\alpha-1+1-\alpha^{n+2}\over 1-\alpha}
%}
\end{document}