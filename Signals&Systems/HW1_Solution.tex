\documentclass[50pt]{article}
\RequirePackage{pdfpages}
\renewcommand{\baselinestretch}{1.4}
\RequirePackage{amsthm,amssymb,amsmath,graphicx}
\RequirePackage{color}
\RequirePackage[top=2cm, bottom=2cm, left=2.5cm, right=3cm]{geometry}
\RequirePackage[pagebackref=false,colorlinks,linkcolor=blue,citecolor=magenta]{hyperref}
\RequirePackage{xepersian}
\RequirePackage{MnSymbol}
\RequirePackage{graphicx}
\newcommand{\wid}{1.8in}
\newtheorem{theorem}{Theorem}
\newcommand{\hl}{
\begin{center}
\line(1,0){450}
\end{center}}
\newenvironment{amatrix}[1]{%
\left[\begin{array}{@{}*{#1}{c}|c@{}}
}{%
\end{array}\right]
}
\settextfont{B Nazanin}
\setlatintextfont{Times New Roman}

\begin{document}
\setLTR 




\begin{RTL}
\Large{








\begin{center}
به نام خدا

پاسخ سوالات 1 و 2 تمرینات سری اول درس سیگنالها و سیستمها

دکتر لطف الله بیگی
\end{center}

\hl

1)

الف)

%\begin{latin}
\[
\begin{split}
\text{\rl{انرژی}}&=\int_{-\infty}^\infty |x(t)|^2dt\\&=\int_0^\infty e^{-2t}\sin^2 tdt\\&={1\over 2}\int_0^\infty e^{-2t}(1-\cos 2t)dt\\&={1\over 4}-{1\over 2}\int_0^\infty e^{-2t}\cos 2tdt
\end{split}
\]
%\end{latin}


همچنین برای محاسبه ی $\int_0^\infty e^{-2t}\cos 2tdt$ از رابطه ی 
$\cos u={e^{ju}+e^{-ju}\over 2}$
جایگذاری می کنیم و خواهیم داشت:

\[
\begin{split}
\int_0^\infty e^{-2t}\cos 2tdt&={1\over 2}\int_{0}^\infty e^{(-2+2j)t}+e^{(-2-2j)t}dt\\&=-{1\over 2}\left({{1\over -2+2j}+{1\over -2-2j}}\right)\\&=-{1\over 2}\cdot{-4\over 8}\\&={1\over 4}
\end{split}
\]
بنابراین کل انرژی سیگنال برابر $1\over 8$ خواهد بود. چون انرژی سیگنال محدود است، لذا توان آن برابر  صفر است.

ب)

برای سیگنال های متناوب $x[n]$ با دوره تناوب $N$، توان از رابطه ی زیر به دست می آید:

\[
P={1\over N}\sum_{n=0}^{N-1}|x[n]|^2
\]

از طرفی سیگنال $x[n]=\cos{\pi \over 4} n^2$ با دوره ی $4$ متناوب است؛ بنابراین در این حالت خواهیم داشت:

\[
\begin{split}
P&={1\over 4}\Big(x^2[0]+x^2[1]+x^2[2]+x^2[3]\Big)\\&={1\over 4}\Big(1+{1\over 2}+1+{1\over 2}\Big)\\&={3\over 4}
\end{split}
\]

چون توان سیگنال عددی مثبت است لذا انرژی آن بینهایت است.


پ)
 
از آنجا که این سیگنال دارای بینهایت نمونه ی برابر $1$ است، انرژی آن نامحدود ($\infty$) است که این از تعریف انرژی سیگنال مستقیما نتیجه می شود. از طرفی طبق تعریف توان سیگنال:
\[
\begin{split}
P&=\lim_{N\to \infty}{1\over 2N+1}\sum_{n=-N}^{N}|x[n]|^2\\
\end{split}
\]
با جایگذاری 
$N={k(k+1)\over 2}$
در رابطه‌ی حد بالا نتیجه می گیریم:

\[
\begin{split}
P&=\lim_{{k(k+1)\over 2}\to \infty}{1\over 2N+1}\sum_{n=-N}^{N}|x[n]|^2\\&=\lim_{k\to \infty}{k\over k^2+k+1}\\&=0
\end{split}
\]
 این نتیجه قابل انتظار است؛ زیرا فاصله‌ی نمونه‌های غیر صفر سیگنال با افزایش $n$ دائما افزایش می یابد (به نوعی میزان اطلاعات با ارزش سیگنال با گذشت زمان کم می شود).

این مثال سیگنالی را نشان می دهد که انرژی آن بینهایت و توان آن صفر است.



\hl
2) سیگنال 
$x[n]=\cos {\pi \over 8} n^k$
  به ازای $k<0$ و برای $n$ های بزرگ به طور مجانبی به سمت $1$ میل می‌کند و لذا متناوب نیست. برای $k=0$ ،  سیگنال مقدار ثابت  $\cos{\pi\over 8}$ را داراست و لذا با دوره ی $1 $ متناوب است. همچنین برای $k>0$ فرض کنیم سیگنال با دوره تناوب $N$ متناوب باشد. در اینصورت 
\[
x[n+N]=\cos{\pi \over 8}(n+N)^k
\]
با استفاده از رابطه‌ی بسط دوجمله ای:
\[
\cos{\pi \over 8}(n+N)^k=\cos{\pi \over 8}(n^k+k\cdot n^{k-1}\cdot N+\cdots +N^k)
\]
برای $k$ های فرد دوره تناوب برابر $16$ است (به جمله‌ی $k\cdot n^{k-1}\cdot N$ دقت کنید) و برای $k$ زوج حالت های زیر را داریم:

\[
\begin{split}
&k=8l\quad,\quad l\in\Bbb N\Longrightarrow N=2\\&
k=8l+4\quad,\quad l\in\Bbb N\Longrightarrow N=4\\&
k=4l+2\quad,\quad l\in\Bbb N\Longrightarrow N=8\\
\end{split}
\]

















}





\end{RTL}



\end{document}


