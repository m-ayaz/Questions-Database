\documentclass[10pt,letterpaper]{article} 
\usepackage{tikz}
\usepackage{toolsper}
%\usepackage{graphicx}‎‎
%\usefonttheme{serif}‎
%\usepackage{ptext}‎
%\usepackage{xepersian}
%\settextfont{B Nazanin}
\usepackage{lipsum}
\setlength{\parindent}{0pt}
\newcommand{\pf}{$\blacksquare$}
\newcommand{\EX}{\Bbb E}
\newcommand{\nl}{\newline\newline}
\newcommand{\Q}[1]{\textbf{
سوال #1)
}}
\newcommand{\pic}[2]{
\begin{center}
\includegraphics[width=#2]{#1}
\end{center}
}
\begin{document}
\Large
\begin{center}
به نام زیبایی

تمرینات سری سوم سیگنال ها و سیستم ها
\hl
\end{center}
\Q1

یک سیستم LTI و سیگنال 
$
x(t)=e^{-5t}u(t-2)
$
 مفروض است. اگر
$$
x(t)\to y(t)
$$
 و 
$$
{dx(t)\over dt}\to -5y(t)+{1\over 1+t^2}u(t)
$$
در اینصورت پاسخ ضربه‌ی این سیستم را بیابید.
\nl
\Q2

یک سیستم LTI دارای جفت ورودی-خروجی زیر است:
\[
\includegraphics[width=60mm]{PS3_Q2_1.eps}
\longrightarrow
\includegraphics[width=60mm]{PS3_Q2_2.eps}
\]
پاسخ این سیستم به ورودی زیر
\pic{PS3_Q2_3.eps}{60mm}
چیست؟
\nl
\Q3

الف) اگر تابع $f(t)$، تابعی یک مرتبه مشتق پذیر با ریشه های 
$
r_1,r_2,\cdots,r_n
$
 باشد به گونه ای که 
$
f'(r_i)\ne 0
$
. در این صورت نشان دهید
$$
\delta(f(t))=\sum_{i=1}^{n} {1\over |f'(r_i)|}\delta(t-r_i)
$$
ب) مقادیر انتگرال های زیر را محاسبه کنید.

1) 
$
\int_{-\infty}^\infty \delta(t^2)dt
$

2)  
$
\int_0^\infty \delta\left(\sin{1\over t}\right)dt
$

3)
 $
\int_{-1}^3 \sin \pi t\delta(t^2-t)dt
$

4)
 $
\int_{-1}^3 \delta^2(t)dt
$

(راهنمایی: برای اثبات الف، سیگنال $u(f(t))$ را در نظر گرفته و با مشتق گیری از آن، به تابع $\delta(f(t))$ برسید. این مشتق را در نزدیکی هر $r_i$ تحلیل کنید.)
\nl
\Q4

الف) هر یک از گزاره‌های زیر را برای یک سیستم LTI تعیین درستی کنید.

\begin{enumerate}
\item
معکوس هر سیستم علی و LTI ، همواره علی است.
\item
اگر یک سیستم LTI زمان گسسته دارای پاسخ ضربه ای با دوره‌ی زمانی محدود باشد، آنگاه پایدار است.
\end{enumerate}

ب) فرض کنید یک سیستم LTI گسسته، دارای پاسخ ضربه‌ی $h[n]$ باشد؛ به گونه ای که 
$$
\sum_{n=-\infty}^\infty |h[n]|=\infty
$$
ورودی کرانداری بیابید که پاسخ این سیستم به چنین ورودی ای، بیکران شود و از آنجا نتیجه بگیرید که سیستم ناپایدار است.
\end{document}