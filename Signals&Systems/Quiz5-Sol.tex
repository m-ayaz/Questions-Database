\documentclass{article}


\usepackage{amsmath,amssymb,graphicx,xcolor,xepersian}

\newcommand{\qn}[1]{
\[
\begin{split}
#1
\end{split}
\]
}

\begin{document}

%{
%\centering
%به نام زیبایی
%
%کوئیز 2 درس سیگنال ها و سیستم ها
%}

\large

سوال 1) اگر ورودی گسسته‌ و متناوب
$
x[n]
$
با ضرایب سری فوریه‌ی 
$
a_k
$
و فرکانس اساسی $\omega_0$، وارد سیستم LTI با پاسخ فرکانسی
$
H(e^{j\omega})
$
شود، خروجی متناوب و دارای ضرایب سری فوریه‌ی 
$
a_kH(e^{jk\omega_0})
$
خواهد بود. بر این مبنا، پاسخ فرکانسی سیستم برابر است با
$$
H(e^{j\omega})=\frac{3}{5-4\cos\omega}.
$$
همچنین، برای ورودی داریم
$
\omega_0=\frac{\pi}{2}
$
و ضرایب سری فوریه‌ی آن عبارتست از
\qn{
a_k=\frac{1}{4}،
}
بنابراین ضرایب سری فوریه‌ی خروجی به صورت زیرند:
$$
b_k=a_kH(e^{jk\omega_0})=\frac{3}{20-16\cos k\frac{\pi}{2}}
$$

سوال 2)

اگر سیگنال 
$
x[n]
$
دارای تبدیل فوریه‌ی 
$
X(e^{j\omega})
$
باشد، آنگاه
$$
x[n](-1)^n=x[n]e^{j\pi n}\iff X(e^{j(\omega-\pi)}).
$$
اگر سیگنال فوق، از فیلتر 
$
H(e^{j\omega})
$
عبور کند، دارای تبدیل فوریه‌ی 
$
X(e^{j(\omega-\pi)})H(e^{j\omega})
$
و اگر پس از آن، از بلوک مدولاتور دیجیتال (ضرب در $(-1)^n$) عبور کند، دارای تبدیل فوریه‌ی
$
X(e^{j(\omega-2\pi)})H(e^{j(\omega-\pi)})
$
خواهد بود. پس از تفریق کردن شاخه های بالایی و پایینی فیلتر خواهیم داشت
$$
Y(e^{j\omega})=X(e^{j\omega})-X(e^{j\omega})H(e^{j(\omega-\pi)})
$$
بنابراین پاسخ فرکانسی فیلتر کلی برابر 
$
1-H(e^{j(\omega-\pi)})
$
می شود که پس از رسم آن، از نوع پایین نگذر خواهد بود (بالاگذر هم می تواند قلمداد شود).
\end{document}