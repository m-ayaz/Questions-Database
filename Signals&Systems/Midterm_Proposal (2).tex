\documentclass[10pt,letterpaper]{article} 
\usepackage{tabularx}
\usepackage{tikz}
\usepackage{toolsper}
%\usepackage{graphicx}‎‎
%\usefonttheme{serif}‎
%\usepackage{ptext}‎
%\usepackage{xepersian}
%\settextfont{B Nazanin}
\usepackage{lipsum}
\setlength{\parindent}{0pt}
%\usepackage{enumitem}
%\setlist[enumerate,1]{label=(\arabic*)}
\newcommand{\pf}{$\blacksquare$}
\newcommand{\EX}{\Bbb E}
\newcommand{\nl}{\newline\newline}

\usepackage{amsmath}
\usepackage{accents}
\newlength{\dhatheight}
\newcommand{\doublehat}[1]{%
    \settoheight{\dhatheight}{\ensuremath{\hat{#1}}}%
    \addtolength{\dhatheight}{-0.35ex}%
    \hat{\vphantom{\rule{1pt}{\dhatheight}}%
    \smash{\hat{#1}}}}

\newcounter{QuestionNumber}
\setcounter{QuestionNumber}{1}

\newcommand{\Q}{
\textbf{
سوال \theQuestionNumber-
}
\stepcounter{QuestionNumber}
}

\newcommand{\fig}[3]{
\begin{figure}[h!]
#1
\caption{#2}
\label{#3}
\end{figure}
}

\newcommand{\subfig}[3]{
\begin{subfigure}{#3}
#1
\caption{#2}
\end{subfigure}
}

\newcommand{\figno}[1]{
\begin{figure}[h!]
#1
\end{figure}
}

\newcommand{\subfigno}[2]{
\begin{subfigure}{#2}
#1
\end{subfigure}
}
%\newcommand{\pic}[2]{
%\begin{center}
%\includegraphics[width=#2]{#1}
\newcommand{\testo}[5]{
\Q #1
\nl
1) #2
\nl
2) #3
\nl
3) #4
\nl
4) #5
\nl
}

\newcommand{\test}[5]{
\Q #1
\nl
{
\centering
%\begin{table}
%\begin{tabularx}{\linewidth}{r l X}
%\toprule
%1) #2\qquad\qquad & 2) #3\\ 
%3) #4\qquad\qquad & 4) #5\\
%\bottomrule
%\end{tabularx}
%\end{table}
\begin{tabularx}{0.8\textwidth} { 
   >{\raggedleft\arraybackslash}X 
%   >{\centering\arraybackslash}X 
   >{\raggedleft\arraybackslash}X  }
% \hline
1) #2\qquad\qquad\qquad\qquad & 2) #3\\ 
3) #4\qquad\qquad\qquad\qquad & 4) #5\\
%\hline
\end{tabularx}
%\begin{tabular}{\linewidth}{r c c}
%\end{tabular}
}
\nl
}




%\newcommand{\testo}[5]{
%\Q #1
%\nl
%{
%\centering
%\begin{tabular}{r c c}
%1) #2\qquad\qquad & 2) #3\\ 
%3) #4\qquad\qquad & 4) #5
%\end{tabular}
%}
%\nl
%}



%\newcommand{\testk}[5]{
%\Q #1
%\nl
%{
%\centering
%\begin{tabular}{r c c}
%1) #2\qquad\qquad\qquad\qquad & 2) #3\\ 
%3) #4\qquad\qquad\qquad\qquad & 4) #5
%\end{tabular}
%}
%\nl
%}
%\end{center}
%}
\begin{document}
\Large
\begin{center}
به نام زیبایی

سوالات پیشنهادی میان ترم
\hl
\end{center}

\test{
$
\delta(t^2-1)
$
برابر کدام گزینه است؟
}
{
$
\delta(t-1)+\delta(t+1)
$
}
{
$
\delta(t-1)-\delta(t+1)
$
}
{
$
{1\over 2}\delta(t-1)-{1\over 2}\delta(t+1)
$
}
{
$
{1\over 2}\delta(t-1)+{1\over 2}\delta(t+1)
$
}
\testo{
سیستم کلی با ورودی x[n] و خروجی y[n] را به صورت شکل زیر در نظر بگیرید که در آن، رابطه‌ی ورودی-خروجی هر سیستم به صورت زیر است:
\qn{
&\text{سیستم 1}:\quad y[n]=\begin{cases}x[n/2]&,\quad \text{n زوج}\\0&,\quad \text{n فرد}\end{cases}
\\&\text{سیستم 2}:\quad y[n]=x[n]+{1\over 2}x[n-1]+{1\over 4}x[n-2]
\\&\text{سیستم 3}:\quad y[n]=x[2n]
}{}
کدام گزینه رابطه‌ی ورودی خروجی سیستم زیر را بدرستی نشان می دهد؟
\begin{figure}[h!]
\centering
\includegraphics[width=120mm]{_2Q.pdf}
\end{figure}
}
{
$y[n]=\begin{cases}x[n]+{1\over 2}x[n-1]+{1\over 4}x[n-2]&,\quad \text{n زوج}\\0&,\quad \text{n فرد}\end{cases}$
}
{
$y[n]=\begin{cases}x[n]+{1\over 2}x[n-2]+{1\over 4}x[n-4]&,\quad \text{n زوج}\\0&,\quad \text{n فرد}\end{cases}$
}
{$y[n]=x[n]+{1\over 4}x[n-1]$}
{$y[n]=x[n]+{1\over 2}x[n-1]+{1\over 4}x[n-2]$}
%%%%%%%%%%%%%%%%%%%%%%%%
\test{
مقدار انتگرال 
$
I=\int_{-\infty}^\infty X^2(j\omega)d\omega
$
 برای سیگنال زیر کدام است؟
\begin{figure}[h!]
\centering
\includegraphics[width=70mm]{_3Q.eps}
\end{figure}
}
{$\pi$}
{$2\pi$}
{$1\over 2$}
{1}
%%%%%%%%%%%%%%%%%%%%%%%
\newpage
\test{
یک سیستم با رابطه‌ی ورودی-خروجی 
$
y[n]=\cos \left({\pi\over 2}x[n]\right)
$
داده شده است. کدام گزینه در مورد متناوب بودن خروجی به ازای 
$
x[n]={n^2\over 4}
$
صحیح است؟
}
{متناوب با فرکانس اصلی $\pi\over 3$ است.}
{متناوب با فرکانس اصلی $\pi\over 4$ است.}
{متناوب با فرکانس اصلی $\pi\over 8$ است.}
{متناوب نیست.}
%%%%%%%%%%%%%%%%%%%%%%%%
\test{
در یک سیستم LTI زمان گسسته با پاسخ ضربه‌ی 
$
h[n]=({1\over 2})^{|n|}
$
، 
پاسخ به ورودی 
$
x[n]=u[n]+u[2-n]
$
 در لحظه‌ی 
$
n=-2
$
کدام است؟ ($u[n]$ تابع پله‌ی واحد است)
}
{$31\over 8$}
{$55\over 16$}
{$19\over 4$}
{4}
%%%%%%%%%%%%%%%%%%%%%%%%%%%
\test{
تبدیل فوریه‌ی سیگنال $x(t)$ به شکل زیر است. کدام مورد در مورد این سیگنال صحیح است؟
\begin{figure}[h!]
\centering
\includegraphics[width=70mm]{_6Q.eps}
\end{figure}
}
{$\angle x(t)=t$}
{$x(t)$ حقیقی است.}
{$\angle x(t)=-t$}
{$x(t)$ موهومی و زوج است.}
%%%%%%%%%%%%%%%%%%%%%%%%%%%%%
\test{
سیگنال $x(t)$ را با تبدیل فوریه‌ی 
$
X(j\omega)
$
 در نظر بگیرید. فرض کنید اطلاعات زیر را در مورد سیگنال 
$
x(t)
$
 داریم:
\nl
$\blacksquare$
 سیگنال $x(t)$ حقیقی است.

$\blacksquare$
$x(t)=0\quad,\quad t\le 0$

$\blacksquare$
$\int_{-\infty}^\infty \Re\{X(j\omega)\}e^{j\omega t}d\omega=2\pi |t|e^{-|t|}$
\nl
$x(t)$
 برابر کدام است؟
}
{$2\pi te^{-t}u(t)$}
{$2\pi te^{t}u(t)$}
{$2te^{t}u(t)$}
{$2te^{-t}u(t)$}
%%%%%%%%%%%%%%%%%%%%%%%%%%%%%
\test{
حاصل کانولوشن 
$
{\sin^2\pi t\over \pi^2t^2}*\cos^2{\pi t\over 2}
$
 برابر کدام است؟

(یادآوری:
$
\Pi\left({t\over 2T}\right)\iff 2T\text{sinc}\left({\omega T\over \pi}\right)
$
)
}
{$0.25+0.5\cos^2{\pi t\over 2}$}
{$-0.25+0.5\cos^2{\pi t\over 2}$}
{$\cos^2{\pi t\over 2}$}
{$0.5\cos^2{\pi t\over 2}$}
%%%%%%%%%%%%%%%%%%%%%%%%%%
\test{
رابطه‌ی ورودی-خروجی برای 4 سیستم به صورت زیر است:
\qn{
&\text{سیستم 1}: 
y(t)=\begin{cases}
0&,\quad x(t)<0\\
x(t)+x(t-2)&,\quad x(t)\ge 0
\end{cases}
\\&\text{سیستم 2}: 
y(t)=\begin{cases}
0&,\quad t<0\\
x(t)+x(t-2)&,\quad t\ge 0
\end{cases}
\\&\text{سیستم 3}: y(t)=\int_{-\infty}^{2t}x(\tau)d\tau
\\&\text{سیستم 4}: y(t)=x(t-2)+x(2-t)
}{}
کدام سیستم در خاصیت تغییرپذیری با زمان از بقیه متفاوت است؟
}
{1}
{2}
{3}
{4}
%%%%%%%%%%%%%%%%%%%%%%%%%
\test{
اگر توصیف ورودی خروجی یک سیستم به صورت 
$
y(t)=x(-t)+2
$
 باشد، رابطه ی ورودی-خروجی وارون آن برابر کدام است؟
}
{$y(t)=x(t)-2$}
{$y(t)=x(-t)-2$}
{$y(t)=x(-t)+2$}
{$y(t)=x(t)+2$}
%%%%%%%%%%%%%%%%%%%%%%%%%%
\test{
سیگنالی داریم که دارای تبدیل فوریه‌ی 
$$
X(j\omega)=j\sqrt\pi \text{sgn}(\omega)[u(\omega+1)-u(\omega-1)]
$$
 است. مقدار مشتق این سیگنال در $t=0$ چقدر است؟
}
{$-{\sqrt\pi\over 2}$}
{$-{1\over\sqrt\pi}$}
{0}
{$-{1\over 2\sqrt\pi}$}
%%%%%%%%%%%%%%%%%%%%%%%%%%%
\testo{
$
x_1(t)
$
 متناوب با پریود اصلی
$
T_1$
 و ضرایب سری فوریه‌ی 
$
a_k
$
 و 
$
x_2(t)
$
 متناوب با پریود اصلی
$
T_2=3T_1
$ و ضرایب سری فوریه‌ی 
$
b_k
$
است. ضرایب سری فوریه‌ی 
$
y(t)=x_1(t)+x_2(t)
$
 کدام است؟
}
{
$
\begin{cases}
a_{n\over 3}+b_n&,\quad \text{اگر n مضرب 3 باشد}
\\b_n&,\quad \text{در غیر این صورت}
\end{cases}
$
}
{
$
\begin{cases}
a_{n\over 3}+b_n&,\quad \text{اگر n مضرب 3 باشد}
\\a_n&,\quad \text{در غیر این صورت}
\end{cases}
$
}
{
$
\begin{cases}
b_n+b_{n\over 3}&,\quad \text{اگر n مضرب 3 باشد}
\\b_n&,\quad \text{در غیر این صورت}
\end{cases}
$
}
{
$
\begin{cases}
a_n+b_{n\over 3}&,\quad \text{اگر n مضرب 3 باشد}
\\a_n&,\quad \text{در غیر این صورت}
\end{cases}
$
}
%%%%%%%%%%%%%%%%%%%%%%
\test{
رابطه‌ی 
$
x(t)*x(t)=3x\left({t\over 2}\right)
$
 برای کدام یک از سیگنال های زیر برقرار است؟ (منظور از * عملگر کانولوشن است)
}
{$3\over \pi^2+t^2$}
{${3\over \pi}{1\over 5-jt}$}
{$6{\sin\pi t\over \pi t}$}
{$3\pi \delta({t\over 2}-1)$}
%%%%%%%%%%%%%%%%%%%%%%%%
\test{
با اعمال کدام یک از ورودی‌های زیر به یک سیستم LTI و مشاهده‌ی خروجی، می توان پاسخ ضربه‌ی آن سیستم را به طور یکتا بدست آورد؟
}
{${\sin^2 \pi t\over \pi t}$}
{$\Pi(t)$}
{$u(t)$}
{$\cos 2t$}
%%%%%%%%%%%%%%%%%%%%%%%%%%%
\test{
سیستم LTI ای با پاسخ ضربه‌ی 
$
h(t)={\sin(4(t-1))\over \pi(t-1)}
$
 را در نظر بگیرید. پاسخ این سیستم به ورودی 
$
x(t)=\left[{\sin (2t)\over \pi t}\right]^2
$
 کدام است؟
}
{
$
{\sin(2(t-1))\over \pi(t-1)}
\times
{\sin(2(t-{1\over 2}))\over \pi(t-{1\over 2})}
$
}
{$\left[{\sin (2(t-1))\over \pi (t-1)}\right]^2$}
{$\left[{\sin (4(t-1))\over \pi (t-1)}\right]^2$}
{$\left[{\sin (2(t-{1\over 2}))\over \pi (t-{1\over 2})}\right]^2$}
%%%%%%%%%%%%%%%%%%%%%%%%%%%%%
\test{
مقدار 
$
\sum_{k=-\infty}^{\infty}{\sin^2({k\pi\over 2})\over k^2}
$
برابر کدام است؟
}
{$\pi^2\over 2$}
{$\pi^2$}
{$\pi\over 2$}
{$1\over 2$}
%%%%%%%%%%%%%%%%%%%%%%%%%%%%%%
\test{
سیگنال متناوب نشان داده شده در شکل زیر، از سیستمی با پاسخ ضربه‌ی 
$
h(t)={\sin({\pi\over 2}t)\over \pi t}
$
 عبور می کند. سیگنال خروجی برابر کدام است؟
\begin{figure}[h!]
\centering
\includegraphics[width=70mm]{_17Q.eps}
\end{figure}
}
{$-{2\over \pi}\sin({\pi\over 3}t)$}
{${2\pi}\cos({\pi\over 3}t)$}
{${2\over \pi}\sin({\pi\over 6}t)$}
{${3\over 2\pi}\cos({\pi\over 3}t)$}
%%%%%%%%%%%%%%%%%%%%%%%%%%%%%%%%
\test{
سیگنال متناوب $x(t)$ با ضرایب سری فوریه‌ی زیر مفروض است:
$$
c_k=\begin{cases}
1&,\quad k=0
\\
-j\left({1\over 3}\right)^{|k|}&,\quad k\ne0
\end{cases}
$$
کدام گزینه در مورد این سیگنال درست است؟
}
{سیگنال $x(t)$ حقیقی است.}
{سیگنال $x(t)$ فرد است.}
{مشتق سیگنال $x(t)$ زوج است.}
{مشتق سیگنال $x(t)$ فرد است.}
%%%%%%%%%%%%%%%%%%%%%%%%%%%%%%%%%%
\test{
رابطه‌ی بین ورودی و خروجی یک سیستم زمان گسسته به صورت زیر است:
$$
y[n]=\begin{cases}
\Re\{x[n-1]\}&,\quad \text{n زوج}
\\\Re\{x[n-1]+x[n-2]\}&,\quad \text{n فرد}
\end{cases}
$$
کدام گزینه در مورد این سیستم درست است؟
}
{خطی و تغییر ناپذیر با زمان}
{خطی و تغییر پذیر با زمان}
{غیرخطی و تغییر ناپذیر با زمان}
{غیرخطی و تغییر پذیر با زمان}
%%%%%%%%%%%%%%%%%%%%%%%%%%%%%%
\test{
در شکل زیر، تبدیل فوریه‌ی سیگنال $x(t)$ را $X(\omega)$ می نامیم. رابطه‌ی ورودی و خروجی این سیستم به صورت زیر است. کدام گزینه در مورد این سیستم \underline{نادرست} است؟
\begin{figure}[h!]
\centering
\includegraphics[width=120mm]{_20Q.pdf}
\end{figure}
}
{حافظه دار است.}
{خطی است.}
{غیرعلی است.}
{تغییر ناپذیر در زمان است.}
%%%%%%%%%%%%%%%%%%%%%%%%%%%%%
\test{
سیگنال $x(t)$ که دارای تبدیل فوریه‌ای با اندازه و فاز زیر است کدام است؟
\begin{figure}[h!]
\centering
\begin{subfigure}{0.49\textwidth}
\includegraphics[width=80mm]{_21Q_ab.eps}
\end{subfigure}
\begin{subfigure}{0.49\textwidth}
\includegraphics[width=80mm]{_21Q_an.eps}
\end{subfigure}
\end{figure}
}
{${3\over \pi t^2}(3\pi t\cos 3\pi t-\sin 3\pi t)$}
{${3\over \pi t}(3\pi t\sin 3\pi t-\cos 3\pi t)$}
{$3\sin 3\pi t\over \pi t^2$}
{$3\cos (3\pi t+{\pi\over 4})\over \pi t^2$}
%%%%%%%%%%%%%%%%%%%%%%%%%%%%%
\test{
ورودی یک سیستم LTI 
$
x(t)=\cos 100\pi t [u(t)-u(t-5)]
$
 و پاسخ ضربه ی آن
$
h(t)=x(5-t)
$
است. مقدار خروجی در لحظه‌ی 
$
t=6
$
($y(6)$) چیست؟
}
{2}
{$5\over 2$}
{$9\over 2$}
{5}
%%%%%%%%%%%%%%%%%%%%%%%%%%%%%
\test{
$x[n]$
 یک سیگنال متناوب با دوره‌ی تناوب $N$ زوج است. اگر 
$
z[n]=x[2n]
$
 و ضرایب سری فوریه‌ی 
$
x[n]
$
 دارای خاصیت 
$
a_k=a_{k+{N\over 2}}
$
 باشند، سیگنال 
$
x[2n+1]
$
کدام است؟
}
{$x[2n+1]=-z[n]$}
{$x[2n+1]=(-1)^nz[n]$}
{$x[2n+1]=0$}
{$x[2n+1]=(-1)^n$}
%%%%%%%%%%%%%%%%%%%%%%%%%%%%%
\test{
رابطه‌ی ورودی و خروجی در یک سیستم توسط رابطه‌ی زیر بیان می شود:
$$
y(t)=\begin{cases}
x(t-1)&,\quad x(t-1)\le 1
\\
x(t-2)&,\quad x(t-1)> 1
\end{cases}
$$
این سیستم کدام خواص زیر را دارد؟
}
{علی و خطی}
{علی و غیرخطی}
{غیرعلی و خطی}
{غیرعلی و غیرخطی}
%%%%%%%%%%%%%%%%%%%%%%%%%%%
\testo{
در هر مورد، سیگنال زمانی به همراه نرخ نمونه برداری متناظر آن داده شده است. در کدام گزینه، شرط نایکوئیست رعایت \underline{نمی شود}؟
}
{$x(t)={\sin \pi t\over \pi t}\ \ , \ \ F_s=1.2\text{ Hz}$}
{$x(t)={\sin^2 \pi t\over (\pi t)^2}\ \ ,\ \ F_s=1.2\text{ Hz}$}
{$x(t)=\sin 3t\ \,\ \ F_s={4\over \pi}\text{ Hz}$}
{$x(t)={\sin \pi t\over \pi t}*e^{-|t|}\ \ ,\ \ F_s=3\text{ Hz}$
که منظور از $*$ عملگر کانولوشن است.
}
%%%%%%%%%%%%%%%%%%%%%%%%%%%
\test{
تبدیل فوریه‌ی کدام یک از سیگنال های داده شده‌، دارای همه‌ی ویژگی‌های زیر است؟

الف) 
$
\Re\{X(j\omega)\}=0
$

ب)
$
\int_{-\infty}^\infty \omega X(j\omega) d\omega=0
$

پ)
$
\int_{-\infty}^\infty X(j\omega) d\omega=0
$
}
{$x(t)=e^{-t^2}-1$}
{$x(t)=t^2e^{-|t|}$}
{$x(t)=t^3e^{-|t|}$}
{$x(t)=te^{-|t|}$}
%%%%%%%%%%%%%%%%%%%%%%%%%%%%%
\test{
اگر سیگنال 
$
x(t)
$
 مانند شکل زیر باشد،
\picnocapt{m_1Q.eps}{60mm}
 سیگنال 
$
x(-{t\over 2}-3)
$
 کدام است؟
}
{\includegraphics[width=60mm]{m_1Q_opt1.eps}}
{\includegraphics[width=60mm]{m_1Q_opt2.eps}}
{\includegraphics[width=60mm]{m_1Q_opt3.eps}}
{\includegraphics[width=60mm]{m_1Q_opt4.eps}}
%%%%%%%%%%%%%%%%%%%%%%%%%%%%%%
\test{
ضرایب سری فوریه‌ی سیگنال متناوب 
$
x[n]
$
 با دوره تناوب 6 را با 
$
\alpha_k
$
 نمایش می‌دهیم. از روی سیگنال 
$
x[n]
$
، سیگنال 
$
s(t)
$
 را به صورت 
$
s(t)=\sum_{k=-\infty}^{\infty}x[k]\delta(t-2k)
$
 می‌سازیم. ضرایب سری فوریه‌ی 
$
s(t)
$
 کدام است؟
}
{${1\over 2}\alpha_k$}
{${1\over 6}\alpha_k$}
{$6\alpha_k$}
{$2\alpha_k$}
%%%%%%%%%%%%%%%%%%%%%%%%%%%%%%
\test{
اگر سیگنال 
$
x(4-2t)
$
 مانند شکل زیر باشد،
\picnocapt{m_2Q.eps}{60mm}
 سیگنال 
$
x(t)
$
 کدام است؟
}
{\includegraphics[width=60mm]{m_2Q_opt1.eps}}
{\includegraphics[width=60mm]{m_2Q_opt2.eps}}
{\includegraphics[width=60mm]{m_2Q_opt3.eps}}
{\includegraphics[width=60mm]{m_2Q_opt4.eps}}
\end{document}