\documentclass[10pt,letterpaper]{article} 
\usepackage{tikz}
\usepackage{toolsper}
%\usepackage{graphicx}‎‎
%\usefonttheme{serif}‎
%\usepackage{ptext}‎
%\usepackage{xepersian}
%\settextfont{B Nazanin}
\usepackage{lipsum}
\setlength{\parindent}{0pt}
%\usepackage{enumitem}
%\setlist[enumerate,1]{label=(\arabic*)}
\newcommand{\pf}{$\blacksquare$}
\newcommand{\EX}{\Bbb E}
\newcommand{\nl}{\newline\newline}

\newcounter{QuestionNumber}
\setcounter{QuestionNumber}{1}

\newcommand{\Q}{
\textbf{
سوال \theQuestionNumber)
}
\stepcounter{QuestionNumber}
}

\newcommand{\pic}[2]{
\begin{center}
\includegraphics[width=#2]{#1}
\end{center}
}
\begin{document}
\Large
\begin{center}
به نام زیبایی

تمرینات سری پنجم سیگنال ها و سیستم ها
\hl
\end{center}
\Q

ضرایب سری فوریه‌ی پیوسته‌ی هر یک از سیگنال های زیر را به دست آورید.
\nl
الف)
\pic{PS5_Q1_1.eps}{80mm}

ب)

\pic{PS5_Q1_2.eps}{80mm}

پ)
$
x(t)=e^{j2\pi{m\over n}t}
$
 که $m$ و $n$ دو عدد طبیعی و نسبت به هم اول هستند.
\nl
ت) 
$
x(t)
$
 با دوره‌ی 4 متناوب است و داریم
$$
x(t)=\begin{cases}
\sin \pi t&,\quad 0\le t\le 2
\\
0&,\quad 2< t\le 4
\end{cases}
$$
\nl
ث) 
$
{d\over dt}x(t)
$
 که 
$
x(t)
$
 سیگنال قسمت ب) است.
\nl
\Q

یک سیستم پیوسته و علی، دارای معادله‌ی دیفرانسیل ورودی-خروجی زیر است:
$$
{d\over dt}y(t)+4y(t)=x(t)
$$
ضرایب سری فوریه‌ی خروجی را برای ورودی های زیر به دست آورید.

الف)
 $
x(t)=\cos 2\pi t
$

ب)
$
x(t)=\sin 4\pi t+\cos\left(6\pi t+{\pi\over 4}\right)
$
\nl
آیا برای محاسبه‌ی خروجی از روی ورودی، شرایط اولیه مورد نیاز است؟
\nl
\Q

یک سیستم LTI دارای پاسخ ضربه‌ی 
$
h(t)=e^{-|t|}
$
 است. ضرایب سری فوریه‌ی خروجی را با اعمال ورودی های زیر به دست آورید.

الف) 
$
x(t)=\sum_{n=-\infty}^{\infty} \delta(t-n)
$

ب)
 $
x(t)=\sum_{n=-\infty}^{\infty} (-1)^n\delta(t-n)
$
\nl
\Q

فرض کنید اطلاعات زیر برای یک سیگنال پیوسته‌ی $x(t)$ و متناوب با دوره‌ی 3 داده شده است:
\begin{enumerate}
\item
$a_k=a_{k+2}$
\item
$a_k=a_{-k}$
\item
$\int_{-0.5}^{0.5}x(t)dt=1$
\item
$\int_{1}^{2}x(t)dt=2$
\end{enumerate}
در این صورت $x(t)$ را بیابید.
\nl
\Q

خواص معروف سری فوریه‌ی پیوسته را اثبات کنید!

نشان دهید اگر $x(t)$ دارای دوره تناوب $T$ و ضرایب سری فوریه $a_k$ باشد، در اینصورت
\nl
%الف) $x(\alpha t)$ دارای ضرایب $a_k$ برای $\alpha>0$ و $a_{-k}$ برای $\alpha<0$ است.
%\nl
%ب) 
%$
%x(t)e^{jM{2\pi\over T}t}
%$
% دارای ضرایب 
%$
%a_{k-M}
%$
% است که $M$ عدد صحیحی است.
%\nl
الف) 
$
{d\over dt}x(t)
$
 دارای ضرایب 
$
j{2\pi\over T}ka_k
$
 است
\nl
ب) 
$
\int_{-\infty}^t x(\tau)d\tau
$
 دارای ضرایب 
$
{T\over j2\pi}\cdot{a_k\over k}
$
 است با این شرط که انتگرال، محدود باشد و $a_0=0$. (چرا شرط $a_0=0$ مهم است؟)
\nl
پ) چنانچه $x(t)$ حقیقی باشد، آنگاه داریم
\qn{
&a_k=a^*_{-k}
\\&
\Re(a_k)=\Re(a_{-k})
\\&
\Im(a_k)=-\Im(a_{-k})
\\&
|a_k|=|a_{-k}|
\\&
\angle a_k=-\angle a_{-k}
}
عملگرهای $\Re$ و $\Im$، به ترتیب قسمت های حقیقی و موهومی یک عدد مختلط را به دست می دهند.
\nl
ت) چنانچه $x(t)$ حقیقی باشد، آنگاه $x_e(t)$ و $x_o(t)$ به ترتیب دارای ضرایب فوریه‌ی 
$
\Re (a_k)
$
 و 
$
j\Im (a_k)
$
 هستند.
\nl
(راهنمایی: برای اثبات هر یک از خواص فوق، ابتدا $x(t)$ را به صورت حاصل جمع نمایی هایی با ضرایب سری فوریه اش بنویسید. در هر مورد، سیگنال داده شده را بر حسب $x(t)$ ابتدا به دست آورده، تاثیر آن را بر روی ضرایب سری فوریه‌ی سیگنال بررسی کنید.)
\nl
\Q
\textbf{(امتیازی)}

نشان دهید اگر برای دو سیگنال متناوب 
$
x(t)
$
 و 
$
y(t)
$
 با دوره‌ی $T$ داشته باشیم
$$
\int_{0}^T x(t)y^*(t)dt=0
$$
آنگاه
$$
\sum_{n=-\infty}^{\infty} a_kb_k^*=0
$$
که $a_k$ ضرایب سری فوریه‌ی $x(t)$ و $b_k$ ضرایب سری فوریه‌ی $y(t)$ است.
\nl
(راهنمایی: رابطه‌ی هر یک از سیگنال های فوق را بر حسب ضرایب آن بنویسید و در انتگرال جایگذاری کنید. تابع تحت انتگرال به صورت حاصلضرب دو $\sum$ ظاهر می شود که با ساده کردن آن، می توانید اثبات را کامل کنید.)
%\nl
%\Q
% \textbf{
%(امتیازی ویژه)
%}
%
%نشان دهید ضرایب سری فوریه، بهینه ترین ضرایب برای تخمین سیگنال متناوب $x(t)$ هستند.
%\nl
%فرض کنید $x(t)$ دارای تناوب $T$ باشد. سیگنال جدید $\hat x(t)$ را به صورت زیر می سازیم:
%$$
%\hat x(t)=\sum_{k=-\infty}^\infty a_ke^{jk{2\pi\over T}t}
%$$
%هدف آن است که ضرایب $a_k$ را به گونه ای بیابیم که خطا به مفهوم زیر
%$$
%\fa{خطا}=\int_{-\infty}^{\infty} |x(t)-\hat x(t)|^2dt
%$$
%مینیمم گردد. ابتدا با جایگذاری خواهیم داشت:
%\qn
%\fa{خطا}&=\int_{-\infty}^{\infty} \left|x(t)-\sum_{k=-\infty}^\infty a_ke^{jk{2\pi\over T}t}\right|^2dt
%\\&=
%\int_{-\infty}^{\infty} \left|x(t)-\sum_{k=-\infty}^\infty a_ke^{jk{2\pi\over T}t}\right|^2dt
%}
\end{document}