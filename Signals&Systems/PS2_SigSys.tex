\documentclass[10pt,letterpaper]{article} 
\usepackage{tikz}
\usepackage{toolsper}
%\usepackage{graphicx}‎‎
%\usefonttheme{serif}‎
%\usepackage{ptext}‎
%\usepackage{xepersian}
%\settextfont{B Nazanin}
\usepackage{lipsum}
\setlength{\parindent}{0pt}
\newcommand{\pf}{$\blacksquare$}
\newcommand{\EX}{\Bbb E}
\newcommand{\nl}{\newline\newline}
\newcommand{\Q}[1]{\textbf{
سوال #1)
}}
\newcommand{\pic}[2]{
\begin{center}
\includegraphics[width=#2]{#1}
\end{center}
}
\begin{document}
\Large
\begin{center}
به نام زیبایی

تمرینات سری دوم سیگنال ها و سیستم ها
\hl
\end{center}
\Q1

برای دو سیگنال $x(t)$ و $y(t)$، تابع همبستگی به صورت زیر تعریف می شود:
$$
\phi_{xy}(t)=\int_{-\infty}^\infty x(t+\tau) y^*(\tau)d\tau
$$
الف) چه رابطه‌ای بین 
$
\phi_{xy}(t)
$
 و
 $
\phi_{yx}(t)
$
 وجود دارد؟

ب) اگر $x(t)$ و $y(t)$ هر دو متناوب با دوره‌ی تناوب اساسی به ترتیب $T_1$ و $T_2$ باشند، در چه شرایطی $\phi_{yx}(t)$ متناوب است و در این صورت دوره‌ی تناوب اساسی آن را بیابید.

پ) اگر 
$
y(t)=x(t+T)
$
، رابطه‌ی بین 
$
\phi_{yy}(t)
$
 و 
$
\phi_{xy}(t)
$
 را با
$
\phi_{xx}(t)
$
 بیابید.

ت) تابع همبستگی $
\phi_{xy}(t)
$
 را به ازای سیگنال دلخواه $x(t)$ و 
$
y(t)=\delta(\sin \pi t)
$
 محاسبه کنید.
\nl
\Q2

سیستم LTI زیر را در نظر بگیرید:
$$
y[n]-3y[n-1]+2y[n-2]=3x[n-4]+4x[n-5]
$$
الف) تحقق مستقیم نوع 1
\footnote{
\lr{
Direct Form I realization
}}
 را برای سیستم های زیر ترسیم کنید.
$$
\begin{cases}
S_1:
y_1[n]=3y_1[n-1]-2y_1[n-2]+x_1[n]
\\
S_2:
y_2[n]=3x_2[n-4]+4x_2[n-5]
\end{cases}
$$
ب) تحقق مستقیم نوع 1 را به صورت ترکیب سری تحقق‌های مستقیم نوع 1 سیستم‌های $S_1$ و $S_2$ (ابتدا $S_1$ و سپس $S_2$) پیاده سازی کنید.

پ) از روی تحقق مستقیم نوع 1 در قسمت ب، تحقق مستقیم نوع 2
\footnote{
\lr{
Direct Form II realization
}}
 سیستم $S$ را پیاده سازی کنید.
\nl
\Q3

گزاره های زیر را تعیین درستی کنید و در صورت نادرست بودن، مثال نقض بیاورید.

الف) ترکیب سری یک سیستم ناپایدار و یک سیستم پایدار، همواره ناپایدار است.

ب) ترکیب سری یک سیستم غیرعلی و یک سیستم علی، می‌تواند علی باشد.

پ) ترکیب موازی یک سیستم ناپایدار و یک سیستم پایدار، همواره ناپایدار است.

ت) ترکیب سری دو سیستم معکوس ناپذیر، همواره معکوس ناپذیر است.

ث) ترکیب موازی دو سیستم معکوس ناپذیر، همواره معکوس ناپذیر است.
\nl
\Q4

خواص علی بودن، پایدار بودن، خطی بودن، تغییر ناپذیر با زمان بودن و معکوس پذیر بودن را برای سیستم های زیر بررسی کنید.

الف) 
$
y(t)=\int_{-\infty}^{t} x(\tau)-x(\tau-T)d\tau
$

ب) 
$
y(t)=\cos [x(t)]
$

پ) 
$
y(t)={d\over dt}x(t)
$

ت)
$
y[n]=\begin{cases}
x^2[n]&,\quad x[n]<0
\\
x[n+2]&,\quad x[n]\ge 0
\end{cases}
$

ث) 
$
y[n]=\begin{cases}
x^2[n]&,\quad n<0
\\
x[n+2]&,\quad n\ge 0
\end{cases}
$

ج)
$
y(t)=\begin{cases}
x(t^2)&,\quad t<0
\\
x(t^3)&,\quad t\ge 0
\end{cases}
$
\end{document}