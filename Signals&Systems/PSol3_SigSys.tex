\documentclass[10pt,letterpaper]{article} 
\usepackage{tikz}
\usepackage{toolsper}
%\usepackage{graphicx}‎‎
%\usefonttheme{serif}‎
%\usepackage{ptext}‎
%\usepackage{xepersian}
%\settextfont{B Nazanin}
\usepackage{lipsum}
\setlength{\parindent}{0pt}
\newcommand{\pf}{$\blacksquare$}
\newcommand{\EX}{\Bbb E}
\newcommand{\nl}{\newline\newline}
\newcounter{QuestionNumber}
\setcounter{QuestionNumber}{1}

\newcommand{\Q}{
\textbf{
سوال \theQuestionNumber)
}
\stepcounter{QuestionNumber}
}
\newcommand{\pic}[2]{
\begin{center}
\includegraphics[width=#2]{#1}
\end{center}
}
\begin{document}
\Large
\begin{center}
به نام زیبایی

پاسخ تمرینات سری سوم سیگنال ها و سیستم ها
\hl
\end{center}
\Q

با مشتق گیری از سیگنال ورودی خواهیم داشت:
\qn{
{d\over dt}x(t)&=-5e^{-5t}u(t-2)+e^{-5t}\delta(t-2)
\\&=-5e^{-5t}u(t-2)+e^{-10}\delta(t-2)
}
از آنجا که سیستم LTI است، پاسخ آن به ورودی فوق برابر است با
$$
-5y(t)+e^{-10}h(t-2)
$$
که $h(t)$ پاسخ ضربه است. از طرفی
$$
-5y(t)+e^{-10}h(t-2)=-5y(t)+{1\over 1+t^2}u(t)
$$
درنتیجه
$$
h(t)={e^{10}\over 1+(t+2)^2}u(t+2)
$$
\Q

برای حل این دسته از سوالات، یک راه حل مناسب ساختن ورودی دوم از روی ورودی اول به کمک شیفت های متوالی و مقیاس های دامنه است. از آنجا که سیستم LTI است، این دو عمل به همان ترتیب، روی خروجی نیز اعمال می شوند.

همچنین می توان ورودی اول را از روی ورودی دوم به کمک شیفت های متوالی و مقیاس های دامنه ساخت و همان عملیات را روی خروجی اعمال کرد. در این سوال، با کمی دقت دیده می شود
$$
x_1(t)=x_2(t)+x_2(t-1)
$$
به دلیل LTI بودن سیستم
$$
y_1(t)=y_2(t)+y_2(t-1)
$$
باید شکل سیگنالی یافت که جمع آن با شیفت یافته اش، خروجی اول را بدهد. می توان شکل سیگنال زیر را در نظر گرفت:
\pic{PSol3_Q2.eps}{80mm}
که جمع شیفت یافته‌ی آن با خودش، $y_1(t)$ را می دهد.
\nl
\Q

الف)

با مشتق گیری از 
$
u(f(t))
$
 به کمک قاعده‌ی زنجیری خواهیم داشت :
\qn{
{d\over dt}u(f(t))&=f'(t)\delta(f(t))
}
در نزدیکی هر ریشه‌ی $r_i$ از $f(t)$، طبق قضیه‌ی تیلور می توان تابع $f(t)$ را به صورت زیر نوشت:
$$
f(t)=f'(r_i)(t-r_i)
$$
بنابراین در نزدیکی این ریشه:
$$
\delta(f(t))=\delta(f'(r_i)(t-r_i))={1\over |f'(r_i)|}\delta(t-r_i)
$$
چون تعداد $n$ ریشه به این ترتیب موجود است، در نهایت می توان نوشت:
\qn{
\delta(f(t))=\sum_{i=1}^{n}{1\over |f'(r_i)|}\delta(t-r_i)
}
%از طرفی در نزدیکی این ریشه، تابع $u(f(t))$ بین دو مقدار 0 و 1 تغییر می کند. اگر   $f(t)$ در اطراف $r_i$ نزولی باشد، این تغییر از 1 به 0 و اگر صعودی باشد، از 0 به 1 خواهد بود. بنا
ب) 
\begin{enumerate}[label=(\roman*)]
\item
برای این انتگرال، 0 ریشه‌ی مضاعف $t^2$ است. بنابراین از رابطه‌ی فوق نمی توان بهره گرفت. با این حال، به کمک تعریف $\delta$:
$$
\delta_\Delta(t)=\begin{cases}
{1\over 2\Delta}&,\quad -\Delta<t<\Delta\\
0&,\quad \fa{در غیر این صورت}
\end{cases}
$$
بنابراین
\qn{
\delta_\Delta(t^2)&=\begin{cases}
{1\over 2\Delta}&,\quad -\Delta<t^2<\Delta\\
0&,\quad \fa{در غیر این صورت}
\end{cases}
\\&=\begin{cases}
{1\over 2\Delta}&,\quad -\sqrt\Delta<t<\sqrt\Delta\\
0&,\quad \fa{در غیر این صورت}
\end{cases}
\\&={1\over \sqrt\Delta}\delta_{\sqrt\Delta}(t)
}
از آنجا که 
$
\lim_{\Delta\to 0^+}\delta_{\sqrt\Delta}(t)\to\delta(t)
$
، در این صورت 
$
{1\over \sqrt\Delta}\delta_{\sqrt\Delta}(t)
$
 به ضربه ای با سطح زیر $\infty$ میل می کند و حاصل انتگرال، $\infty$ است.
\item
ریشه های ساده‌ی $\sin{1\over t}$ در $[0,\infty)$ عبارتند از:
$$
r_k={1\over k\pi}\quad,\quad k\in\Bbb N
$$
در این صورت
\qn{
\delta\left(\sin{1\over t}\right)&=\sum_{k=1}^\infty {r_k^2\over |\cos{1\over r_k}|}\delta\left(t-{1\over k\pi}\right)
\\&=\sum_{k=1}^\infty {1\over k^2\pi^2}\delta\left(t-{1\over k\pi}\right)
}
\qn{
\int_0^\infty\delta\left(\sin{1\over t}\right)dt&=\int_0^\infty\sum_{k=1}^\infty {1\over k^2\pi^2}\delta\left(t-{1\over k\pi}\right)dt
\\&=\sum_{k=1}^\infty {1\over k^2\pi^2}\int_0^\infty\delta\left(t-{1\over k\pi}\right)dt
\\&=\sum_{k=1}^\infty {1\over k^2\pi^2}
\\&={1\over 6}
}
\item
\qn{
\delta(t^2-t)=\delta(t)+\delta(t-1)\implies
}
\qn{
\int_{-1}^3\sin \pi t\delta(t^2-t) dt &=
\int_{-1}^3\sin \pi t[\delta(t)+\delta(t-1)] dt\\&=
\int_{-1}^3\sin \pi t\delta(t) dt+\int_{-1}^3\sin \pi t\delta(t-1) dt\\&=
\int_{-1}^3\sin 0\pi \delta(t) dt+\int_{-1}^3\sin 1\pi \delta(t-1) dt\\&=0
}
\item
برای تحلیل $\delta^2(t)$، از 
$
\delta_\Delta(t)
$
 کمک می گیریم. در این صورت:
$$
\delta^2_\Delta(t)=\begin{cases}
{1\over 4\Delta^2}&,\quad -\Delta<t<\Delta\\
0&,\quad \fa{در غیر این صورت}
\end{cases}
=
{1\over 2\Delta}\delta_\Delta(t)
$$
که به سمت ضربه ای با مساحت $\infty$ میل می کند و حاصل انتگرال، $\infty$ است.
\end{enumerate}
%\nl
\Q

الف)
\begin{enumerate}[label=(\roman*)]
\item
نادرست. سیستم 
$
y(t)=x(t-1)
$
، LTI و علی است؛ ولی معکوس آن
$
y(t)=x(t+1)
$
 علی نیست.
\item
درست. شرط پایداری سیستم LTI با پاسخ ضربه‌ی $h[n]$، 
$
\sum_{n} |h[n]|<\infty
$
 است که اگر تعداد نقاط پاسخ ضربه محدود باشد، جمع فوق همواره محدود است. به چنین سیستمی 
FIR\footnote{
\en{Finite Impulse Response}
}
 می‌گویند.
\end{enumerate}
ب) 

می توان نوشت
\qn{
\sum_n |h[n]|&=\sum_{h[n]\ge0} h[n]+\sum_{h[n]<0} -h[n]
\\&=\sum_n x[-n]h[n]
}
که در آن 
$$
x[-n]=\begin{cases}
1&,\quad h[n]\ge0\\
-1&,\quad h[n]<0
\end{cases}
$$
در این صورت
$$
x[n]=\begin{cases}
1&,\quad h[-n]\ge0\\
-1&,\quad h[-n]<0
\end{cases}
$$
از طرفی می توان جمع فوق را برابر خروجی سیستم LTI با پاسخ ضربه‌ی $h[n]$ به ورودی $x[n]$ در لحظه ی 0 دانست؛ زیرا:
\qn{
y[n]=\sum_{k}h[k]x[n-k]
}
بنابراین
\qn{
y[0]=\sum_{k}h[k]x[-k]
}
که طبق استدلال بالا، مقدار آن برابر $\sum_n |h[n]|$ و نامحدود است؛ بنابراین سیستم ناپایدار است.
\end{document}