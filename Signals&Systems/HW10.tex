\documentclass[50pt]{article}
\RequirePackage{pdfpages}
\renewcommand{\baselinestretch}{1.4}
\RequirePackage{amsthm,amssymb,amsmath,graphicx}
\RequirePackage{color}
\RequirePackage[top=2cm, bottom=2cm, left=2.5cm, right=3cm]{geometry}
\RequirePackage[pagebackref=false,colorlinks,linkcolor=blue,citecolor=magenta]{hyperref}
\RequirePackage{xepersian}
\RequirePackage{MnSymbol}
\RequirePackage{graphicx}
\newcommand{\wid}{1.8in}
\newtheorem{theorem}{Theorem}
\newcommand{\hl}{
\begin{center}
\line(1,0){450}
\end{center}}
\newenvironment{amatrix}[1]{%
\left[\begin{array}{@{}*{#1}{c}|c@{}}
}{%
\end{array}\right]
}
\settextfont{B Nazanin}
\setlatintextfont{Times New Roman}

\begin{document}
\setLTR 




\begin{RTL}
\Large{








\begin{center}
به نام خدا

تمرینات سری دهم درس سیگنالها و سیستمها

دکتر لطف الله بیگی

مهلت تحویل: 98/3/12
\end{center}

\hl
\begin{latin}
Chapter 10:

$23\_\{\text{\rl{فقط از روش تجزیه به کسرهای جزئی}}\}$
$, 29\_\{c,e\}$
\end{latin}
به علاوه سوالات زیر را حل کنید:

سوال 1) به کمک دوگانی بین تبدیل فوریه‌ی سیگنال های زمان گسسته و سری فوریه سیگنالهای متناوب پیوسته، تبدیل فوریه‌ی سیگنال 
$x[n]=\cos \omega_0 n$
 را به دست آورید و از روی آن تحقیق کنید
$$
\cos (\omega_0+2\pi) n=\cos \omega_0 n
$$
سوال 2) به کمک خاصیت دوگانی سری فوریه‌ی سیگنال های متناوب گسسته، سری فوریه‌ی سیگنال زیر را بیابید.
$$
x[n]=e^{j2\pi{r\over N}n}\quad,\quad r,N\in\Bbb N \ \ ,\ \ \gcd(r,N)=1
$$

}





\end{RTL}



\end{document}


