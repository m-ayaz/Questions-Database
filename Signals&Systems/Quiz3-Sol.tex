\documentclass{article}


\usepackage{amsmath,amssymb,graphicx,xcolor,xepersian}

\newcommand{\qn}[1]{
\[
\begin{split}
#1
\end{split}
\]
}

\begin{document}

%{
%\centering
%به نام زیبایی
%
%کوئیز 2 درس سیگنال ها و سیستم ها
%}

\large

{{\color{red}سوال باید با بهره گیری از خواص تبدیل فوریه حل شود؛ بنابراین اگر دانشجویی با خواص حل نکرد و مستقیم تبدیل فوریه را حساب کرد، لطفا کسر نمره جزئی منظور فرمایید.
}}

الف)

\qn{
\int_{-\infty}^\infty |X(j\omega)|^2d\omega
&=2\pi\int_{-\infty}^\infty |x(t)|^2dt
\\&=2\pi\int_{-\infty}^\infty |x(t)|^2dt
\\&=2\pi\int_{-\infty}^\infty |x(t+5)|^2dt
\\&=4\pi\int_{0}^2 |x(t+5)|^2dt
\\&=4\pi\int_{0}^2 (4-t)^2dt
\\&=4\pi\frac{(t-4)^3}{3}\Big|_0^2
\\&=\frac{224\pi}{3}
}

ب) سیگنال حول $t=-5$ تقارن زوج دارد؛ پس فاز آن
$
-\omega(-5)=5\omega
$
است.

پ) تبدیل فوریه‌ی 
$
x(t)*x(t)
$
برابر 
$
X^2(j\omega)
$
است؛ پس،
\qn{
\int_{-\infty}^\infty X^2(j\omega)d\omega
&=2\pi x(t)*x(t)\Big|_{t=0}
\\&=2\pi \int_{-\infty}^\infty x(t)x(-t)dt=0
}


ت) می دانیم تبدیل فوریه‌ی 
$
x'(t)
$
برابر 
$
j\omega X(j\omega)
$
است؛ پس تبدیل فوریه‌ی 
$
x'(t-6)
$
برابر 
$
j\omega X(j\omega) \exp(-j6\omega)
$
خواهد بود.
\qn{
\int_{-\infty}^\infty \omega X(j\omega) \exp(-j6\omega)
&=\frac{1}{j}x'(t-6)\Big|_{t=0}=\frac{1}{j}x'(-6)=-j
}
\end{document}