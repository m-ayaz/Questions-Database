\documentclass[50pt]{article}
\RequirePackage{pdfpages}
\renewcommand{\baselinestretch}{1.4}
\RequirePackage{amsthm,amssymb,amsmath,graphicx}
\RequirePackage{color}
\RequirePackage[top=2cm, bottom=2cm, left=2.5cm, right=3cm]{geometry}
\RequirePackage[pagebackref=false,colorlinks,linkcolor=blue,citecolor=magenta]{hyperref}
\RequirePackage{xepersian}
\RequirePackage{MnSymbol}
\RequirePackage{graphicx}
\newcommand{\wid}{1.8in}
\newtheorem{theorem}{Theorem}
\newcommand{\hl}{
\begin{center}
\line(1,0){450}
\end{center}}
\newenvironment{amatrix}[1]{%
\left[\begin{array}{@{}*{#1}{c}|c@{}}
}{%
\end{array}\right]
}
\settextfont{B Nazanin}
\setlatintextfont{Times New Roman}

\begin{document}
\setLTR 




\begin{RTL}
\Large{








\begin{center}
به نام خدا

تمرینات سری یازدهم (پایانی) درس سیگنالها و سیستمها

دکتر لطف الله بیگی
\end{center}

\hl
\begin{latin}
Chapter 1:

$45,46$

Chapter 2:

$66$

Chapter 3:

$65$

Chapter 4:

$53\_\Big\{a,b,c\_\{i,v\},d,e\Big\}$

Chapter 5:

$53$

Chapter 6:

$65$

Chapter 7:

$52$

Chapter 9:

$63$

Chapter 10:

$63,64$


\end{latin}
به علاوه سوالات زیر را حل کنید:
\newline
(1) یک سیستم \text{\lr{FIR}} (زمان گسسته) به سیستم \text{\lr{LTI}} ای گفته می شود که پاسخ ضربه‌ی آن ($h[n]$)، دارای دوره‌ی زمانی محدود باشد؛ به عبارت دیگر:
\[
h[n]=0\quad,\quad N_1<n<N_2
\]
ثابت کنید تمامی قطب های این سیستم (در صورت وجود)، در صفر قرار دارند؛ اگر و تنها اگر سیستم علی باشد.
\newline
(2) سیستم حقیقی،زمان گسسته و پایدار $h[n]$ با تبدیل $z$ زیر توصیف می شود:
\[
H(z)={1-az^{-1}\over (1-2z^{-1})\left(1-{1\over 2}z^{-1}\right)}
\]
الف) آیا می توان با تغییر مقادیر $a$، سیستم را علی نمود؟ ضد-علی چطور؟
\newline
ب) محدوده‌ی مقادیری از $a$ را بیابید که سیستم، وارون علی و پایدار داشته باشد.
\newline
ج) نمودار صفر و قطب سیستم وارون را رسم کنید.



}





\end{RTL}



\end{document}


