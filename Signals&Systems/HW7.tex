\documentclass[50pt]{article}
\RequirePackage{pdfpages}
\renewcommand{\baselinestretch}{1.4}
\RequirePackage{amsthm,amssymb,amsmath,graphicx}
\RequirePackage{color}
\RequirePackage[top=2cm, bottom=2cm, left=2.5cm, right=3cm]{geometry}
\RequirePackage[pagebackref=false,colorlinks,linkcolor=blue,citecolor=magenta]{hyperref}
\RequirePackage{xepersian}
\RequirePackage{MnSymbol}
\RequirePackage{graphicx}
\newcommand{\wid}{1.8in}
\newtheorem{theorem}{Theorem}
\newcommand{\hl}{
\begin{center}
\line(1,0){450}
\end{center}}
\newenvironment{amatrix}[1]{%
\left[\begin{array}{@{}*{#1}{c}|c@{}}
}{%
\end{array}\right]
}
\settextfont{B Nazanin}
\setlatintextfont{Times New Roman}

\begin{document}
\setLTR 




\begin{RTL}
\Large{








\begin{center}
به نام خدا

تمرینات سری هفتم درس سیگنالها و سیستمها

دکتر لطف الله بیگی

مهلت تحویل: 98/2/22
\end{center}

\hl
\begin{latin}
Chapter 9:

$21\_\{b,c,d,g,i\} , 22\_\{b,c,e,g\} , 23 ,25, 26 ,27,29,30, 44 $
\end{latin}
(1) عکس تبدیل لاپلاس 
$$H(s)={1\over s^n+1}$$
را به ازای $n$ های طبیعی به دست آورید.

(2) (امتیازی) فرض کنید تبدیل لاپلاس سیگنال $x(t)$ برابر 
$X(s)$ 
باشد به گونه ای که 
$$x(t)=0\quad,\quad |t|>{T\over 2}$$
سیگنال $y(t)$ را با متناوب کردن $x(t)$ با دوره تناوب $T$ می سازیم؛ به عبارت دیگر:
$$
y(t)=\sum_{n=-\infty}^{\infty}x(t-nT)
$$
تبدیل لاپلاس $y(t)$ را بر حسب $X(s)$ به دست آورید.

(3) (امتیازی) عکس تبدیل لاپلاس 
$\sin {1\over s+1}$ 
را از روی سری تیلور $\sin(\cdot)$ به دست آورید.



}





\end{RTL}



\end{document}


