\documentclass[10pt,letterpaper]{article} 
\usepackage{tikz}
\usepackage{toolsper}
%\usepackage{graphicx}‎‎
%\usefonttheme{serif}‎
%\usepackage{ptext}‎
%\usepackage{xepersian}
%\settextfont{B Nazanin}
\usepackage{lipsum}
\setlength{\parindent}{0pt}
%\usepackage{enumitem}
%\setlist[enumerate,1]{label=(\arabic*)}
\newcommand{\pf}{$\blacksquare$}
\newcommand{\EX}{\Bbb E}
\newcommand{\nl}{\newline\newline}

\usepackage{amsmath}
\usepackage{accents}
\newlength{\dhatheight}
\newcommand{\doublehat}[1]{%
    \settoheight{\dhatheight}{\ensuremath{\hat{#1}}}%
    \addtolength{\dhatheight}{-0.35ex}%
    \hat{\vphantom{\rule{1pt}{\dhatheight}}%
    \smash{\hat{#1}}}}

\newcounter{QuestionNumber}
\setcounter{QuestionNumber}{1}

\newcommand{\Q}{
\textbf{
سوال \theQuestionNumber)
}
\stepcounter{QuestionNumber}
}

\newcommand{\fig}[3]{
\begin{figure}[h!]
#1
\caption{#2}
\label{#3}
\end{figure}
}

\newcommand{\subfig}[3]{
\begin{subfigure}{#3}
#1
\caption{#2}
\end{subfigure}
}

\newcommand{\figno}[1]{
\begin{figure}[h!]
#1
\end{figure}
}

\newcommand{\subfigno}[2]{
\begin{subfigure}{#2}
#1
\end{subfigure}
}
%\newcommand{\pic}[2]{
%\begin{center}
%\includegraphics[width=#2]{#1}
%\end{center}
%}
\begin{document}
\Large
\begin{center}
به نام زیبایی

یک سوال از نمونه برداری
\hl
\end{center}
%دو سیگنال متناوب زیر را در نظر بگیرید:
%\fig{
%\subfig{
%\includegraphics[width=80mm]{PS8_Q1_2}
%}{}{0.5\textwidth}
%\subfig{
%\includegraphics[width=80mm]{PS8_Q1_1}
%}{}{0.5\textwidth}
%}{}{}
%
%برای این دو سیگنال، کانولوشن دایروی آنها را به کمک محاسبه ی کانولوشن هر تناوب از دو سیگنال به دست آورید.
%%\nl
%%\Q
%%
%%فرض کنید ورودی متناوب 
%%$
%%x[n]
%%$
%% به سیستم LTI با پاسخ ضربه‌ی زمان محدود 
%%$
%%h[n]
%%$
%% داده شده است.
%%\fig{
%%\subfig{
%%\includegraphics[width=80mm]{PS8_Q2_2}
%%}{}{0.5\textwidth}
%%\subfig{
%%\includegraphics[width=80mm]{PS8_Q2_1}
%%}{}{0.5\textwidth}
%%}{}{}
%%خروجی این سیستم، $y[n]$، چیست؟
%%\nl
%%\Q
%%
%%الف) فرض کنید اطلاعات زیر در مورد سیگنال متناوب $x[n]$ با دوره‌ی اساسی 8 و ضرایب فوریه‌ی $a_k$ داده شده است:
%%\nl
%%$
%%a_k=-a_{k-4}
%%$
%%
%%$
%%x[2n+1]=(-1)^n
%%$
%%\nl
%%در این صورت، یک تناوب از $x[n]$ را رسم کنید.
%%\nl
%%ب) فرض کنید سیگنال متناوب $x[n]$ با دوره‌ی اساسی 8 و ضرایب فوریه‌ی $a_k$ داده شده است به گونه ای که
%%$
%%a_k=-a_{k-4}
%%$
%% و داریم
%%$$
%%y[n]=\left({1+(-1)^n\over 2}\right)x[n-1]
%%$$
%%اگر ضرایب فوریه‌ی $y[n]$ را با $b_k$ نمایش دهیم، تابع گسسته‌ی $f[k]$ را به گونه ای بیابید که
%%$$
%%b_k=f[k]a_k
%%$$
%%%\nl
%%%\Q
%%%
%%%برای هر یک از جفت ورودی-خروجی های داده شده‌ی زیر، تعیین کنید آیا سیستم LTIای وجود دارد که $x[n]$ را به $y[n]$ نگاشت دهد. در هر مورد که چنین سیستمی وجود داشت، تعیین کنید آیا این سیستم یکتاست؟
%%%
%%%الف) 
%%%$
%%%x[n]=\left({1\over 2}\right)^n\quad,\quad
%%%y[n]=\left({1\over 4}\right)^n
%%%$
%%%
%%%ب) 
%%%$
%%%x[n]=\left({1\over 2}\right)^n\quad,\quad
%%%y[n]=\left({1\over 4}\right)^n
%%%$
%%%\nl
%%\Q
%%
%%اگر سیگنال $x[n]$، متناوب با دوره‌ی تناوب اساسی $N$ و دارای ضرایب سری فوریه‌ی $a_k$ باشد، در این صورت ضرایب سری فوریه‌ی سیگنال های زیر را به دست آورید.
%%\nl
%%الف)
%%$
%%x[n]-x[n-{N\over 2}]
%%$
%% (فرض کنید $N$ زوج است)
%%
%%ب)
%%$
%%x[n]+x[n+{N\over 2}]
%%$
%% (فرض کنید $N$ زوج است؛ دقت کنید که این سیگنال با دوره‌ی $N\over 2$ متناوب است)
%%
%%پ)
%%$
%%(-1)^nx[n]
%%$
%% (فرض کنید $N$ زوج است)
%%
%%ت)
%%$
%%(-1)^nx[n]
%%$
%% (فرض کنید $N$ فرد است؛ دقت کنید که این سیگنال با دوره‌ی $2N$ متناوب است)
%%
%%ث)
%%$
%%y[n]=\begin{cases} x[n] &,\quad \text{زوج}n\\0 &,\quad \text{فرد}n\end{cases}
%%$
%%\nl
%%\Q
%%
%%(خواص سری فوریه‌ی گسسته)
%%
%%فرض کنید ضرایب سری فوریه‌ی سیگنال های متناوب  $x[n]$ و $y[n]$ با دوره تناوب $N$، به ترتیب برابر $a_k$ و $b_k$ باشد؛ یعنی
%%\qn{
%%&x[n]\iff a_k
%%\\&y[n]\iff b_k
%%}{}
%% در این صورت، نشان دهید
%%
%%الف)
%%$$
%%\sum_{r=0}^{N-1}x[r]y[n-r]\iff Na_kb_k
%%$$
%%ب)
%%$$
%%x[n]y[n]\iff \sum_{l=0}^{N-1}a_lb_{k-l}
%%$$
%%پ) 
%%$$
%%x[n]-x[n-1]\iff \left(1-e^{-jk{2\pi\over n}}\right)a_k
%%$$
%%ت)
%%$$
%%\sum_{k=-\infty}^n x[k]\iff \left({1\over 1-e^{-jk{2\pi\over n}}}\right)a_k
%%$$
%%ث) اگر 
%%$
%%x[n]
%%$
%% حقیقی باشد، آنگاه
%%\qn{
%%&a_k=a^*_{-k}
%%\\& \Re\{a_k\}=\Re\{a_{-k}\}
%%\\& \Im\{a_k\}=-\Im\{a_{-k}\}
%%\\& |a_k|=|a_{-k}|
%%\\& \measuredangle a_k=-\measuredangle a_{-k}
%%}{}
%%و نتیجه بگیرید اگر $x[n]$ حقیقی و زوج باشد، 
%%$
%%a_k
%%$
%% حقیقی و زوج و اگر 
%%$
%%x[n]
%%$
%% حقیقی و فرد باشد، 
%%$
%%a_k
%%$
%% موهومی محض و فرد خواهد بود.
%%\nl
%%ج) اگر 
%%$
%%x[n]
%%$
%% حقیقی باشد، آنگاه
%%\qn{
%%&x_e[n]={x[n]+x[-n]\over 2}\iff \Re\{a_k\}
%%\\&x_o[n]={x[n]-x[-n]\over 2}\iff j\Im\{a_k\}
%%}{}
%%چ) (رابطه‌ی پارسوال)
%%$$
%%{1\over N}\sum_{n=0}^{N-1} |x[n]|^2=\sum_{k=0}^{N-1} |a_k|^2
%%$$
%\Q
%
%یک سیگنال پیوسته‌ی $x(t)$ با تبدیل فوریه ی $X(j\omega)$ از داخل سیستم نمونه برداری ضربه ای با دوره‌ی نمونه برداری $T$ عبور می کند و سیگنال زیر را به دست می دهد:
%$$
%x_p(t)=\sum_{n=-\infty}^\infty x(nT)\delta(t-nT)
%$$
%که 
%$
%T=10^{-4}
%$
%. تحقیق کنید با کدام یک از شرایط زیر (قیدهایی روی $x(t)$ یا $X(j\omega)$)، طبق قضیه‌ی تمونه برداری نایکوییست می توان سیگنال $x(t)$ را از روی $x_p(t)$ به طور کامل بازیابی کرد.
%\nl
%الف) 
%$
%X(j\omega)=0\quad,\quad |\omega|>5000\pi
%$
%
%ب)
%$
%X(j\omega)=0\quad,\quad |\omega|>15000\pi
%$
%
%پ)
%$
%\Re\{X(j\omega)\}=0\quad,\quad |\omega|>5000\pi
%$
%
%ت) $x(t)$ حقیقی است و 
%$
%X(j\omega)=0\quad,\quad \omega>5000\pi
%$
%
%ت) $x(t)$ حقیقی است و 
%$
%X(j\omega)=0\quad,\quad \omega<-15000\pi
%$
%
%ث)
%$
%X(j\omega)*X(j\omega)=0\quad,\quad |\omega|>15000\pi
%$
%
%ج)
%$
%|X(j\omega)|=0\quad,\quad |\omega|>5000\pi
%$
%\nl
%\Q
%
%فرض کنید دو سیگنال 
%$
%s_1(t)
%$
% و 
%$
%s_2(t)
%$
%، زمان پیوسته، باند محدود و دارای تبدیل فوریه‌های زیر باشند:
%\figno{
%\subfigno{
%\includegraphics[width=80mm]{PS8_Q4_2}
%}{0.5\textwidth}
%\subfigno{
%\includegraphics[width=80mm]{PS8_Q4_1}
%}{0.5\textwidth}
%}
%
%سیگنال 
%$
%y(t)=s_1(t)e^{j2\pi (2R)t}+s_2(t)e^{j2\pi (3R)t}
%$
% را در نظر بگیرید.
%\nl
%الف) تبدیل فوریه ی 
%$
%y(t)
%$
% را رسم کنید.
%\nl
%ب) حداقل نرخ نمونه برداری را برای برآوردن شرط نایکوییست در مورد سیگنال $y(t)$ به دست آورده و نشان دهید اگر سیگنال 
%$
%y(t)
%$
% را با نرخ 
%$
%R
%$
% نمونه برداری کنیم و سیگنال 
%$
%\hat y[n]=y\left({n\over R}\right)
%$
% را بسازیم، همپوشانی (aliasing) رخ می دهد؛ به گونه ای که نمی توان از روی 
%$
%\hat y[n]
%$
%، 
%$
%y(t)
%$
% ،
%$
%s_1(t)
%$
%و
% $
%s_2(t)
%$
% را بازسازی کرد. برای این کار، تبدیل فوریه ی 
%$
%y\left({n\over R}\right)
%$
% را رسم کنید.
%\nl
%پ) اگر فرض کنیم نرخ نمونه برداری به جای $R$، برابر $2R$ است، تبدیل فوریه‌ی 
%$
%y\left({n\over 2R}\right)
%$
% را رسم کنید.
%\nl
%ت) نشان دهید اگر سیگنال 
%$
%z(t)=s_1(t)e^{j2\pi (4R)t}+s_2(t)e^{j2\pi (5R)t}
%$
% را با نرخ $2R$ نمونه برداری کنیم و سیگنال 
%$\hat z[n]=z\left({n\over 2R}\right)$
% را بسازیم، خواهیم داشت:
%$$
%\hat z[n]=\hat y[n]
%$$
%بنابراین حتی با برآوردن نرخ نایکوییست، الزاما نمی توان سیگنال زمان پیوسته را بازسازی کرد.
%\nl
%\Q 
%(امتیازی)
%%مانند سوال پیش،
%
%سیگنال باند محدود $x(t)$ با تبدیل فوریه‌ی $X(j\omega)$ مفروض است؛ به گونه ای که
%$$
%X(j\omega)=0\quad,\quad |\omega|>R
%$$
%% از داخل سیستم نمونه برداری ضربه ای با دوره‌ی نمونه برداری $T$ عبور داده ایم و سیگنال زیر ساخته شده است:
%%$$
%%x_p(t)=\sum_{n=-\infty}^\infty x(nT)\delta(t-nT)
%%$$
%%این سیگنال از داخل بلوکی که ضربه‌‌های پیوسته را به ضربه‌های گسسته با فاصله های 1 از هم تبدیل می کند، عبور می کند و سیگنال $\hat x[n]$ را می سازد؛
%این سیگنال با نرخ نمونه برداری $R_s$ که $R_s>2R$، نمونه برداری می شود و سیگنال $\hat x[n]$ را می سازد؛ به عبارت دیگر:
%$$
%\hat x[n]=x\left({n\over R_s}\right)
%$$
%رابطه‌ی بین انرژی این دو سیگنال، یعنی 
%$$
%E_1=\sum_{n=-\infty}^{\infty}|x[n]|^2
%$$
%و
%$$
%E_2=\int_{-\infty}^{\infty}|x(t)|^2dt
%$$
% را بیابید.
%\nl
%\newpage
%\Q
%(امتیازی)
%
%اگر 
%$
%x[n]=\cos {\pi\over 4}n+\phi_0
%$
% که
%$
%0\le \phi_0<2\pi
%$
%و
%$
%g[n]=x[n]\sum_{k=-\infty}^{\infty}\delta[n-4k]
%$
%، چه شرط اضافه ای باید روی $\phi_0$ داشته باشیم تا
%$$
%g[n]*\left({\sin {\pi\over 4}n\over {\pi\over 4}n}\right)=x[n]
%$$
%؟
%\nl
%%\newpage
\Q

فرض کنید میخواهیم سیستم پیوسته ای را با پاسخ ضربه ی $h(t)$، ورودی $x(t)$ و خروجی $y(t)$ شبیه سازی کنیم. می دانیم
$$
y(t)=x(t)*h(t)
$$
 از آنجا که نمی توان سیستم پیوسته را در دنیای واقعی شبیه سازی کرد (زیرا پیوسته بودن سیگنال ها معادل با اطلاعات بینهایت است)، ناگزیریم شبیه سازی را در حوزه‌ی گسسته انجام دهیم؛ یعنی سیگنال‌های $x(t)$ و $h(t)$ را با نرخ نمونه برداری مناسب $R_s$ نمونه برداری کرده، سیگنال های 
\qn{
&\hat x[n]=x\left({n\over R_s}\right)
\\&\hat h[n]=h\left({n\over R_s}\right)
}{}
را بسازیم و سپس سیگنال 
$$
\hat y[n]=\hat x[n]*\hat h[n]
$$
را تولید کنیم. در مرحله‌ی آخر، ادعا می کنیم 
$$
\hat y[n]=y\left({n\over R_s}\right)
$$
برای بررسی این ادعا، تبدیل فوریه های سیگنال های $x(t)$ و $h(t)$ را به صورت زیر بگیرید:
\figno{
\subfigno{
\includegraphics[width=80mm]{PS8_Q5_2}
}{0.5\textwidth}
\subfigno{
\includegraphics[width=80mm]{PS8_Q5_1}
}{0.5\textwidth}
}

الف) تبدیل فوریه های $\hat x[n]$ و $\hat h[n]$ را ترسیم کنید.

ب) تبدیل فوریه های (گسسته‌ی)
$
\hat y[n]
$
 و (پیوسته‌ی) $y(t)$ را رسم کنید.

پ) سیگنال آنالوگ $y_c(t)$ را چنان بیابید که تساوی $
\hat y[n]=y\left({n\over R_s}\right)
$ محقق شود. برای این کار، یک تناوب از تبدیل فوریه‌ی 
$
\hat y[n]
$
 را برگزیده، آن را با ضریب $R_s$  بسط دهید؛ به گونه ای که فرکانس های $\pi$ و $-\pi$ به ترتیب به 
$
\pi R_s
$
 و 
$
-\pi R_s
$
نگاشت شوند. نشان دهید تبدیل فوریه ی $y(t)$ با تبدیل فوریه ی  $y_c(t)$ برابر است.

ت) آیا می توان این استدلال را برای حالاتی که فقط یکی از $x(t)$ و $h(t)$ باند محدود باشند به کار برد؟ چرا؟
%فرض کنید بخواهیم سیستم پیوسته ای را در دامنه ی گسسته پیاده سازی کنیم. پاسخ فرکانسی این سیستم عبارتست از
%$$
%H(j\omega)=e^{j{\omega^2\over 2}\alpha}
%$$
%که $\alpha$ ثابت است. اگر ورودی این سیستم، سیگنال $x(t)$ باشد به گونه ای که 
%$$
%X(j\omega)=0\quad,\quad \omega>{R\over 2}
%$$
%نشان دهید با نمونه برداری هر دوی 
%$
%x(t)
%$
% و 
%$
%h(t)
%$
% با نرخ $R_s$ که $R_s\ge R$ و ساختن سیگنالهای 
%$
%\hat x[n]
%$
%و
%$
%\hat h[n]
%$
%که 
%\qn{
%&\hat x[n]\iff \hat X(e^{j\omega})
%\\& \hat h[n]\iff \hat H(e^{j\omega})
%}{}
%می توان خروجی 
%$
%\hat y[n]
%$
% را چنان ساخت که 
%$
%\hat Y(e^{j\omega})=\hat X(e^{j\omega})\hat H(e^{j\omega})
%$
% و سپس با بازسازی سیگنال $y(t)$ از روی $\hat y[n]$ خواهیم داشت
%$$
%Y(j\omega)=X(j\omega)H(j\omega)
%$$
%(می توانید برای اثبات، از روش ترسیمی استفاده کنید؛ یعنی برای طیف فرکانسی $x(t)$ شکل دلخواهی بکشید و سپس تبدیل های ذکر شده در سوال را روی آن اعمال کنید)
%\nl
%\Q


\end{document}