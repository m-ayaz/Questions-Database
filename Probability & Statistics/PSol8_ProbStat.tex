\documentclass{article}

\usepackage{amsmath,amssymb,geometry,tikz}
\usepackage{xepersian}

\setlength{\parindent}{0pt}
\setlength{\parskip}{3mm}

\newcounter{questionnumber}
\setcounter{questionnumber}{1}

\newcommand{\Q}{
\textbf{سوال \thequestionnumber)}
\stepcounter{questionnumber}
}

\newcommand{\eqn}[1]{
\begin{equation}\begin{split}
#1
\end{split}\end{equation}
}

\begin{document}
\LARGE
\begin{center}
\settextfont{IranNastaliq}

به نام زیبایی

%\begin{figure}[h]
%\centering
%\includegraphics[width=30mm]{kntu_logo.eps}
%\end{figure}

تمرینات سری هشتم درس احتمال مهندسی

\end{center}
\hrulefill
\large

\Q

احتمال آبی بودن توپ در هر آزمایش، برابر
$
0.7
$
است. چون توپ در هر مرحله به کیسه باز می‌گردد، آزمایش ها مستقلند و در نتیجه، احتمال مطلوب برابر است با
$$
\binom{11}{7}(0.7)^7(0.3)^4
$$

\Q

چنانچه حداکثر 10 کاربر بخواهند به طور همزمان از کانال استفاده کنند، کمبود پهنای باند نخواهیم داشت. بنابراین احتمال مطلوب برابر است با
\eqn{
\sum_{k=0}^{10} \binom{12}{k}(0.6)^k(0.4)^{12-k}&=
1-\sum_{k=11}^{12} \binom{12}{k}(0.6)^k(0.4)^{12-k}
\\&=
1-12(0.6)^{11}(0.4)-(0.6)^{12}
\\&\approx0.9804
}

\Q

مطلوب آن است که مقدار 
$
P\left\{\frac{97}{300}<\frac{k}{n}<\frac{103}{300}\right\}
$،
حداقل $99\%$ باشد. با تقریب دموآور لاپلاس خواهیم داشت:
$$
2G(\epsilon\sqrt{\frac{n}{p(1-p)}})-1\ge 0.99
$$
که در آن
$
\epsilon=0.01
$
و
$
p=\frac{1}{3}
$.
در نتیجه
\eqn{
&\quad\quad\quad G(0.01\sqrt{\frac{n}{\frac{2}{9}}})\ge 0.995
\\&\implies
0.01\sqrt{\frac{n}{\frac{2}{9}}}\ge G^{-1}(0.995)
\\&\implies
0.01\sqrt{\frac{n}{\frac{2}{9}}}\gtrapprox 2.5758
\\&\implies
n\ge 14745
}
\end{document}