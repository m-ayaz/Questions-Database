\documentclass{article}

\usepackage{amsmath,amssymb,geometry,tikz}
\usepackage{xepersian}

\setlength{\parindent}{0pt}
\setlength{\parskip}{3mm}

\newcounter{questionnumber}
\setcounter{questionnumber}{1}

\newcommand{\Q}{
\textbf{پرسش \thequestionnumber)}
\stepcounter{questionnumber}
}

\newcommand{\eqn}[1]{
\begin{equation}\begin{split}
#1
\end{split}\end{equation}
}

\begin{document}
\LARGE
\begin{center}
\settextfont{IranNastaliq}

به نام زیبایی

%\begin{figure}[h]
%\centering
%\includegraphics[width=30mm]{kntu_logo.eps}
%\end{figure}

پاسخ تمرینات سری چهاردهم درس احتمال مهندسی

\end{center}
\hrulefill
\large

\Q

ابتدا باید 
$
f_X(x|X\ge 1)
$
را محاسبه کنیم. برای این منظور داریم:
\eqn{
\Pr\{X\le x|X\ge 1\}&=\frac{\Pr\{1\le X\le x\}}{\Pr\{X\ge 1\}}
\\&=\frac{\Pr\{1\le X\le x\}}{\Pr\{X=1\}}
\\&=\frac{\Pr\{1\le X\le x\}}{\frac{1}{2}}
\\&=\begin{cases}
1&,\quad x\ge 1\\
0&,\quad x< 1
\end{cases}
}
در این صورت
$$
f_X(x|X\ge 1)=\delta(x-1)
$$
و خواهیم داشت
\eqn{
E\{X|X\ge 1\}&=\int_{-\infty}^\infty xf_X(x|X\ge 1)dx
\\&=\int_{-\infty}^\infty x\delta(x-1)dx=1
}
\eqn{
E\{X^2|X\ge 1\}&=\int_{-\infty}^\infty x^2f_X(x|X\ge 1)dx
\\&=\int_{-\infty}^\infty x^2\delta(x-1)dx=1
}
بنابراین
\eqn{
E\{X|X\ge 1\}=1\quad,\quad
\sigma^2(X|X\ge 1)=E\{X^2|X\ge 1\}-E^2\{X|X\ge 1\}=0
}

برای چگالی احتمال زیر، مقادیر
$
\mathbb{E}\{X|X>1\}
$
و
$
\sigma_X^2(X|X>1)
$
برای محاسبه‌ی
$
f(x|X\ne 1)
$
داریم
\eqn{
\Pr\{X\le x|X\ne 1\}&=\frac{\Pr\{X\le x,X\ne 1\}}{\Pr\{X\ne 1\}}
\\&=\frac{\Pr\{X\le x,X\ne 1\}}{\frac{1}{2}}
\\&=
\begin{cases}
\frac{\Pr\{X<1\}}{\frac{1}{2}}&,\quad x\ge 1\\
\frac{\Pr\{X\le x\}}{\frac{1}{2}}&,\quad x<1
\end{cases}
\\&=
\begin{cases}
1&,\quad x\ge 1\\
3x^2-2x^3&,\quad 0\le x<1\\
0&,\quad x<0
\end{cases}
}
در نتیجه
\eqn{
f_X(x|X\ne 1)=
\begin{cases}
6x-6x^2&,\quad 0<x< 1\\
0&,\quad \text{سایر جاها}
\end{cases}
}

در نهایت، برای محاسبه‌ی 
$
f(x|X<\frac{1}{2})
$
چنین می‌نویسیم:
\eqn{
\Pr\{X\le x|X<\frac{1}{2}\}&=
\frac{\Pr\{X\le x,X<\frac{1}{2}\}}{\Pr\{X<\frac{1}{2}\}}
\\&=
\begin{cases}
\frac{\Pr\{X<\frac{1}{2}\}}{\Pr\{X<\frac{1}{2}\}}&,\quad x\ge \frac{1}{2}\\
\frac{\Pr\{X\le x\}}{\Pr\{X<\frac{1}{2}\}}&,\quad x<\frac{1}{2}
\end{cases}
\\&=
\begin{cases}
1&,\quad x\ge \frac{1}{2}\\
6x^2-4x^3&,\quad 0\le x<\frac{1}{2}\\
0&,\quad x<0
\end{cases}
}
پس:
$$
f_X(x|X<\frac{1}{2})=
\begin{cases}
12x-12x^2&,\quad 0< x<\frac{1}{2}\\
0&,\quad \text{سایر جاها}
\end{cases}.
$$

\Q

الف) 

\eqn{
\Pr\{\max\{X,Y\}\le u|X\le \frac{1}{2}\}&=
\Pr\{X\le u,Y\le u|X\le \frac{1}{2}\}
\\&=
\frac{\Pr\{X\le u,Y\le u,X\le \frac{1}{2}\}}{\Pr\{X\le \frac{1}{2}\}}
\\&=
\begin{cases}
\frac{\Pr\{Y\le u,X\le \frac{1}{2}\}}{\Pr\{X\le \frac{1}{2}\}}&,\quad u\ge \frac{1}{2}\\
\frac{\Pr\{X\le u,Y\le u\}}{\Pr\{X\le \frac{1}{2}\}}&,\quad u< \frac{1}{2}
\end{cases}
}
از طرفی
\eqn{
\Pr\{X\le \frac{1}{2}\}&=\int_{x^2+y^2\le 1,x\le \frac{1}{2}}
\frac{1}{\pi}dxdy
\\&=
\frac{1}{\pi}\int_{x^2+y^2\le 1,x\le \frac{1}{2}}dxdy
\\&=\frac{\sqrt 3}{4\pi}+\frac{2}{3}.
}

همچنین
\eqn{
\Pr\{X\le \frac{1}{2},Y\le u\}&=
\int_{-1}^{\frac{1}{2}}
\int_{-\sqrt{1-x^2}}^{\min\{\sqrt{1-x^2},u\}}\frac{1}{\pi}dydx
\\&=
\frac{1}{\pi}
\int_{x\in(-1,\frac{1}{2}),u\ge -\sqrt{1-x^2}}
\min\{\sqrt{1-x^2},u\}+\sqrt{1-x^2}dx
\\&=
C_1+\frac{1}{\pi}
\int_{x\in(-1,\frac{1}{2}),u\ge -\sqrt{1-x^2}}
\min\{\sqrt{1-x^2},u\}dx
}
که در رابطه بالا، 
$
C_1=\frac{1}{\pi}
\int_{-1}^{\frac{1}{2}}
\sqrt{1-x^2}dx
$
(تعیین مقدار دقیق آن اهمیت ندارد؛ زیرا همانطور که بعدا دیده می‌شود، در مشتق گیری حذف خواهد شد.)

با مشتق گیری خواهیم داشت:
\eqn{
\frac{d}{du}
\Pr\{X\le \frac{1}{2},Y\le u\}
&=
\frac{1}{\pi}\int_{x\in(-1,\frac{1}{2}),-\sqrt{1-x^2}\le u\le \sqrt{1-x^2}}dx
\\&=
\begin{cases}
\frac{1}{\pi}\int_{x\in(-1,\frac{1}{2}),-\sqrt{1-x^2}\le u}dx&,\quad u\le 0\\
\frac{1}{\pi}\int_{x\in(-1,\frac{1}{2}),u\le \sqrt{1-x^2}}dx&,\quad u\ge 0
\end{cases}
\\&=
\frac{1}{\pi}\int_{x\in(-1,\frac{1}{2}),-\sqrt{1-u^2}<x<\sqrt{1-u^2}}dx
\\&=
\frac{1}{\pi}\min\left\{\sqrt{1-u^2},\frac{1}{2}\right\}+\frac{1}{\pi}\min\{\sqrt{1-u^2},1\}
}

به طریق مشابه

\eqn{
\frac{d}{du}
\Pr\{X\le u,Y\le u\}
&=
\frac{1}{\pi}\int_{x\in(-1,u),-\sqrt{1-x^2}\le u\le \sqrt{1-x^2}}dx
\\&=
\begin{cases}
\frac{1}{\pi}\int_{x\in(-1,u),-\sqrt{1-x^2}\le u}dx&,\quad u\le 0\\
\frac{1}{\pi}\int_{x\in(-1,u),u\le \sqrt{1-x^2}}dx&,\quad u\ge 0
\end{cases}
\\&=
\frac{1}{\pi}\int_{x\in(-1,u),-\sqrt{1-u^2}<x<\sqrt{1-u^2}}dx
\\&=
\frac{1}{\pi}\min\left\{\sqrt{1-u^2},u\right\}+\frac{1}{\pi}\min\{\sqrt{1-u^2},1\}
}
بنابراین
%\eqn{
%\Pr\{X\le u,Y\le u\}&=
%\int_{-1}^{u}
%\int_{-\sqrt{1-x^2}}^{\min\{\sqrt{1-x^2},u\}}\frac{1}{\pi}dydx
%\\&=
%\frac{1}{\pi}
%\int_{x\in(-1,u),u\ge -\sqrt{1-x^2}}
%\min\{\sqrt{1-x^2},u\}+\sqrt{1-x^2}dx
%\\&=
%C_1+\frac{1}{\pi}
%\int_{x\in(-1,u),u\ge -\sqrt{1-x^2}}
%\min\{\sqrt{1-x^2},u\}dx
%}

%\eqn{
%\Pr\{Y\le u\}=
%\begin{cases}
%1-\frac{1}{\pi}\cos^{-1}u+\frac{u}{\pi}\sqrt{1-u^2}&,\quad 0\le u\le 1\\
%\frac{1}{\pi}\cos^{-1}(-u)+\frac{u}{\pi}\sqrt{1-u^2}&,\quad -1\le u\le 0\\
%0&,\quad\text{سایر جاها}
%\end{cases}
%}
%و
%\eqn{
%\Pr\{Y\le u,X\ge \frac{1}{2}\}=
%\begin{cases}
%1-\frac{1}{\pi}\cos^{-1}u+\frac{u}{\pi}\sqrt{1-u^2}&,\quad u\ge 0\\
%\frac{1}{\pi}\cos^{-1}u+\frac{u}{\pi}\sqrt{1-u^2}&,\quad u<0\\
%\end{cases}
%}

\eqn{
f_{\max\{X,Y\}}(u|X\le \frac{1}{2})
&=
\begin{cases}
\frac{1}{\pi}\min\left\{\sqrt{1-u^2},\frac{1}{2}\right\}\\+
\frac{1}{\pi}\min\left\{\sqrt{1-u^2},u\right\}\\+
\frac{2}{\pi}\min\{\sqrt{1-u^2},1\}&,\quad |u|\le 1\\
0&,\quad |u|>1
\end{cases}
}

ب)

\eqn{
&\Pr\{\sqrt{X^2+Y^2}\le u|X+Y\le 1\}=
\frac{\Pr\{\sqrt{X^2+Y^2}\le u,X+Y\le 1\}}{\Pr\{X+Y\le 1\}}
\\&=
\frac{1}{\frac{3}{4}+\frac{1}{2\pi}}
\Pr\{\sqrt{X^2+Y^2}\le u,X+Y\le 1\}
\\&=
\begin{cases}
1&,\quad u\ge 1\\
\frac{1}{\frac{3}{4}+\frac{1}{2\pi}}
\left[(1-\frac{1}{\pi}\cos^{-1}\frac{1}{u\sqrt 2})u^2+\frac{1}{\pi}\sqrt{\frac{u^2}{2}-\frac{1}{4}}\right]&,\quad \frac{1}{\sqrt 2}\le u< 1\\
\frac{u^2}{\frac{3}{4}+\frac{1}{2\pi}}&,\quad 0\le u< \frac{1}{\sqrt 2}\\
0&,\quad u\le 0\\
\end{cases}
}
در نتیجه
\eqn{
&f_{\sqrt{X^2+Y^2}|X+Y\le 1}(u)=
\begin{cases}
\frac{2u}{\frac{3}{4}+\frac{1}{2\pi}}
(1-\frac{1}{\pi}\cos^{-1}\frac{1}{u\sqrt 2})&,\quad \frac{1}{\sqrt 2}\le u< 1\\
\frac{2u}{\frac{3}{4}+\frac{1}{2\pi}}&,\quad 0\le u< \frac{1}{\sqrt 2}\\
0&,\quad \text{سایر جاها}\\
\end{cases}
}
در نتیجه
\eqn{
\mathbb{E}\{\sqrt{X^2+Y^2}|X+Y\le 1\}
&=\int_0^1 uf_{\sqrt{X^2+Y^2}|X+Y\le 1}(u)du
\\&=
\int_0^{\frac{1}{\sqrt2}} \frac{2u^2}{\frac{3}{4}+\frac{1}{2\pi}}du
\\&+
\int_{\frac{1}{\sqrt2}}^1 \frac{2u^2}{\frac{3}{4}+\frac{1}{2\pi}}
(1-\frac{1}{\pi}\cos^{-1}\frac{1}{u\sqrt 2})du
\\&=
\int_0^1 \frac{2u^2}{\frac{3}{4}+\frac{1}{2\pi}}du
\\&-
\frac{2}{\frac{3\pi}{4}+\frac{1}{2}}\int_{\frac{1}{\sqrt2}}^1 u^2\cos^{-1}\frac{1}{u\sqrt 2}du
\\&=
\frac{2}{3}+\frac{\sqrt2}{9\pi+6}\ln(1+\sqrt 2)
%\\&=
%\frac{1}{\frac{3}{4}+\frac{1}{2\pi}}
%\left[
%\frac{1}{2}+\frac{1}{3\pi}+\frac{\sqrt 2}{12\pi}\ln(1+\sqrt 2)
%\right]
}
%
%
%
% مقدار 
%$
%\mathbb{E}\{\sqrt{X^2+Y^2}|X+Y\le 1\}
%$
%را بیابید.

\Q

الف)
\eqn{
\Pr\{Y\le 0|X=1\}&=
\int_{-\infty}^0 f(y|X=1)dy
\\&=
\int_{-\infty}^0 \frac{a}{2}\exp(-a|y-1|)dy
\\&=
\int_{-\infty}^0 \frac{a}{2}\exp(a(y-1))dy
=\frac{e^{-a}}{2}
}
و
\eqn{
\Pr\{Y\ge 0|X=-1\}&=
\int_{-\infty}^0 f(y|X=1)dy
\\&=
\int_0^\infty \frac{a}{2}\exp(-a|y+1|)dy
\\&=
\int_0^\infty \frac{a}{2}\exp(-a(y+1))dy
=\frac{e^{-a}}{2}
}
مشاهده می شود که هر دو احتمال، با افزایش مقدار $a$ افت می‌کنند.

ب) صورت سوال، یک مسئله‌ی مخابراتی را نشان می دهد که در آن، مقادیر $-1$ و $1$ روی کانال ارسال می‌شوند و نویزی با چگالی احتمال نمایی دوطرفه، با سیگنال ارسالی جمع می‌شود. هر چه $a$ بیشتر باشد، واریانس (توان) نویز کاهش می‌یابد و انتظار می‌رود که احتمال خطا در آشکارسازی سمبلهای ارسالی در گیرنده نیز کاهش یابد. این امر، با محاسبه‌ی احتمالهای 
$
\Pr\{Y\le 0|X=1\}
$
و
$
\Pr\{Y\ge 0|X=-1\}
$
تحقیق شد.

\end{document}