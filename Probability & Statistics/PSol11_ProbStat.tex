\documentclass{article}

\usepackage{amsmath,amssymb,geometry,tikz}
\usepackage{xepersian}

\setlength{\parindent}{0pt}
\setlength{\parskip}{3mm}

\newcounter{questionnumber}
\setcounter{questionnumber}{1}

\newcommand{\Q}{
\textbf{سوال \thequestionnumber)}
\stepcounter{questionnumber}
}

\newcommand{\eqn}[1]{
\[\begin{split}
#1
\end{split}\]
}

\begin{document}
\LARGE
\begin{center}
\settextfont{IranNastaliq}

به نام زیبایی

%\begin{figure}[h]
%\centering
%\includegraphics[width=30mm]{kntu_logo.eps}
%\end{figure}

تمرینات سری یازدهم درس احتمال مهندسی

\end{center}
\hrulefill
\large

\Q

الف)
\eqn{
&\Pr\{X\ge \alpha\}=\int_\alpha^\infty \exp(-x)dx=\exp(-\alpha)
\\&\text{\lr{Markov's Bound}}=\frac{\mathbb{E}\{X\}}{\alpha}=\frac{1}{\alpha}
}

ب)
%$
%f(x)=\frac{1}{\ln 2}\frac{1}{1+e^x}\quad,\quad x>0
%$
\eqn{
&\Pr\{X\ge \alpha\}=
\int_\alpha^\infty \frac{1}{\ln 2}\frac{1}{1+e^x}dx=\log_2(1+e^{-\alpha})
\\&\text{\lr{Markov's Bound}}=\frac{\mathbb{E}\{X\}}{\alpha}\approx\frac{1.1866}{\alpha}
}

پ)
\eqn{
&\Pr\{X\ge \alpha\}
=\int_\alpha^\infty x\exp(-x)dx=(\alpha+1)\exp(-\alpha)
\\&\text{\lr{Markov's Bound}}=\frac{\mathbb{E}\{X\}}{\alpha}=\frac{2}{\alpha}
}

\Q

الف)
\eqn{
&\mathbb{E}\{X\}=\int_0^\infty xe^{-x}dx=-(x+1)e^{-x}|_0^\infty=1
\\&\mathbb{E}\{X^2\}=\int_0^\infty x^2e^{-x}dx=-(x^2+2x+2)e^{-x}|_0^\infty=2
\\&\implies\sigma_X^2=1
}

پ)
\eqn{
&\mathbb{E}\{X\}=\int_0^\infty x^2e^{-x}dx=-(x^2+2x+2)e^{-x}|_0^\infty=2
\\&\mathbb{E}\{X^2\}=\int_0^\infty x^3e^{-x}dx=-(x^3+3x^2+6x+6)e^{-x}|_0^\infty=6
\\&\implies\sigma_X^2=2
}

\Q

الف)
\eqn{
&\phi_X(s)
=\mathbb{E}\{e^{sX}\}
=\int_{-\infty}^\infty e^{sx}f_X(x)dx
=\int_{0}^1 (1-x)e^{sx}dx
+
\int_{1^-}^{1^+} \frac{1}{2}\delta(x-1)e^{sx}dx
\\&=
e^{sx}(\frac{1-x}{s}+\frac{1}{s^2})|_0^1
+
\frac{e^s}{2}
=
\frac{e^s-s-1}{s^2}
+
\frac{e^s}{2}
=
\frac{e^s}{2}+\sum_{n=0}^\infty\frac{s^n}{(n+2)!}
\implies
\\&
\mathbb{E}\{X^2\}=\frac{d^2\phi(s)}{ds^2}\Big|_{s=0}
=\frac{1}{2}+\frac{1}{12}=\frac{7}{12}
}

ب)
\eqn{
&\phi_X(s)
=\mathbb{E}\{e^{sX}\}
=\int_0^\frac{\pi}{2} e^{sx}\cos xdx=\frac{e^{\frac{\pi}{2}s}}{s^2+1}
\implies
\\&
\mathbb{E}\{X^2\}=\frac{d^2\phi(s)}{ds^2}\Big|_{s=0}
=\frac{\pi^2}{4}-2
}

پ)
\eqn{
&\phi_X(s)=\sum_x e^{sx}P(X=x)
=\sum_{x=1}^\infty \frac{x}{2}(\frac{e^s}{2})^x
=\frac{e^s}{(e^s-2)^2}
\implies
\\&
\frac{d\phi(s)}{ds}=-e^s\frac{e^s+2}{(e^s-2)^3}
\implies
\frac{d^2\phi(s)}{ds^2}=\frac{e^{3s}+8e^{2s}+4e^s}{(e^s-2)^4}
\implies
\\&
\mathbb{E}\{X^2\}=13
}{}

ت) ابتدا مقادیر تابع جرم احتمال $X$ را می نویسیم:
\eqn{
&\Pr\{X=1\}=\frac{1}{36}
\\&\Pr\{X=2\}=\frac{2}{36}
\\&\Pr\{X=3\}=\frac{2}{36}
\\&\Pr\{X=4\}=\frac{3}{36}
\\&\Pr\{X=5\}=\frac{2}{36}
\\&\Pr\{X=6\}=\frac{4}{36}
\\&\Pr\{X=8\}=\frac{2}{36}
\\&\Pr\{X=9\}=\frac{1}{36}
\\&\Pr\{X=10\}=\frac{2}{36}
\\&\Pr\{X=12\}=\frac{4}{36}
\\&\Pr\{X=15\}=\frac{2}{36}
\\&\Pr\{X=16\}=\frac{1}{36}
\\&\Pr\{X=18\}=\frac{2}{36}
\\&\Pr\{X=20\}=\frac{2}{36}
\\&\Pr\{X=24\}=\frac{2}{36}
\\&\Pr\{X=25\}=\frac{1}{36}
\\&\Pr\{X=30\}=\frac{2}{36}
\\&\Pr\{X=36\}=\frac{1}{36}
}{}
بنابراین
\eqn{
\phi_X(s)&=
\frac{1}{36}[
e^s+
2e^{2s}+
2e^{3s}+
3e^{4s}+
2e^{5s}+
4e^{6s}+
2e^{8s}+
e^{9s}+
2e^{10s}
\\&+
4e^{12s}+
2e^{15s}+
e^{16s}+
2e^{18s}+
2e^{20s}+
2e^{24s}+
e^{25s}+
2e^{30s}+
e^{36s}
]
}{}
بنابراین
\eqn{
\mathbb{E}\{X^2\}&=
\frac{1}{36}[1+8+18+48+50+144+128+81+200+576\\&+450+256+648+800+1152+625+1800+1296]
=\frac{8281}{36}
}{}
\end{document}