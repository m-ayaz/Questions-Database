\documentclass[10pt,letterpaper]{article} 
\usepackage{toolsper}
%\usepackage{graphicx}‎‎
%\usefonttheme{serif}‎
%\usepackage{ptext}‎
\usepackage{xepersian}
\settextfont{B Nazanin}
\usepackage{lipsum}
\setlength{\parindent}{0pt}
\newcommand{\pf}{$\blacksquare$}
\newcommand{\pic}[2]{
\begin{center}
\includegraphics[width=#2]{#1}
\end{center}
}
\begin{document}
\Large
\begin{center}
به نام خدا

پاسخ تمرینات سری هشتم درس آمار و احتمال
\hl
\end{center}
سوال 1) الف) از آنجا که مجموع اعداد دو تاس هرگز نمی تواند 1 شود، در نتیجه احتمال موردنظر برابر صفر است.

\textbf{
اگر 
$
\Pr\{X=7,Y=1\}
$
 مورد نظر بود، مقدار این احتمال برابر با 
$$
\Pr\{X=7\}={1\over 18}
$$
 است.
}

ب) ابتدا احتمالات غیر صفر را برای حالاتی که $X,Y\ne 0$ به دست می آوریم:
\eqn{
&\Pr\{X=12,Y=2\}={1\over 36}
\\&\Pr\{X=x,Y=1\}={1\over 18}\quad,\quad x\in\{7,8,9,10,11\}
%\\&\Pr\{X=2,Y=0\}={1\over 36}
%\\&\Pr\{X=3,Y=0\}={1\over 18}
%\\&\Pr\{X=4,Y=0\}={1\over 12}
%\\&\Pr\{X=5,Y=0\}={1\over 9}
%\\&\Pr\{X=6,Y=0\}={5\over 36}
%\\&\Pr\{X=7,Y=0\}={1\over 9}
%\\&\Pr\{X=8,Y=0\}={1\over 12}
%\\&\Pr\{X=9,Y=0\}={1\over 18}
%\\&\Pr\{X=10,Y=0\}={1\over 36}
}{}
بنابراین
\eqn{
&E\{XY\}=\sum_{x=7}^{11}\sum_{y=1}xy\Pr\{X=x,Y=1\}
\\&+12\times 2\times\Pr\{X=12,Y=2\}
\\&={45\over 18}+{24\over 36}
\\&={5\over 2}+{2\over 3}
\\&={19\over 6}
}{}

پ) مشاهده می کنیم که 
$$
\Pr\{X=x\}=\Pr\{X=14-x\}\quad,\quad x\in\{2,3,4,5,6,7\}
$$
بنابراین چون توزیع $X$ حول $x=7$ متقارن است، خواهیم داشت 
$$
E\{X\}=7
$$
%\eqn{
%&E\{X\}=\sum_{x=2}^{12}x\cdot\Pr\{X=x\}
%\\&=(2+12)\cdot{1\over 36}+(3+11)\cdot{2\over 36}
%\\&+(4+10)\cdot{3\over 36}+(5+9)\cdot{4\over 36}
%\\&+(6+8)\cdot{5\over 36}+7\cdot{6\over 36}
%\\&=14\cdot{1+2+3+4+5\over 36}+{42\over 36}
%\\&={252\over 36}=7
%}{}

به علاوه
\eqn{
&\Pr\{Y=2\}={1\over 36}
\\&\Pr\{Y=1\}={11\over 36}
}{}
در نتیجه
$$
E\{Y\}={13\over 36}
$$
به وضوح $X$ و $Y$ ناهمبسته نیستند؛ زیرا 
$$
E\{XY\}\ne E\{X\}E\{Y\}
$$
\newline\newline
سوال 2) الف) به وضوح 
\eqn{
&\mu_X=E\{X\}=p_3+p_4
\\&\mu_Y=E\{Y\}=p_2+p_4
}{}
در نتیجه
\eqn{
&\text{\lr{cov}}(X,Y)=E\{(X-\mu_X)(Y-\mu_Y)\}
\\&=E\{XY\}-\mu_X\mu_Y
\\&=p_4-(p_2+p_4)(p_3+p_4)
}{}
در نتیجه برای صفر بودن کوواریانس باید داشته باشیم 
$
p_4=(p_2+p_4)(p_3+p_4)
$
.

ب) شرط همبستگی که در قسمت قبلی به دست آمد. برای استقلال باید داشته باشیم:
\eqn{
&p_1=(p_1+p_3)(p_1+p_2)
\\&p_2=(p_1+p_2)(p_2+p_4)
\\&p_3=(p_1+p_3)(p_3+p_4)
\\&p_4=(p_2+p_4)(p_3+p_4)
}{}
%از جمع هر چهار معادله‌ی فوق نتیجه می شود:
%\eqn{
%&p_1+p_2+p_3+p_4=
%p_1^2+p_2^2+p_3^2+p_4^2
%\\&+2p_1p_2+2p_1p_3+2p_1p_4
%\\&+2p_2p_3+2p_2p_4+2p_3p_4
%\\&\implies 1=1
%}{}
%بنابراین این 4 معادله از هم مستقل نیستند و می توان هر یک از این 4 معادله را حذف کرد. با حذف معادله‌ی 1، به معادلات زیر می رسیم:
%\eqn{
%&p_2=(1-p_3-p_4)(p_2+p_4)
%\\&p_3=(1-p_2-p_4)(p_3+p_4)
%\\&p_4=(p_2+p_4)(p_3+p_4)
%}{}
%از دو معادله‌ی اوب 
نکته اینجاست که
\begin{enumerate}
\item
از معادله‌ی 4، با کم کردن 
$
p_2+p_4
$
 به معادله‌ی 2 می رسیم.
\item
از معادله‌ی 4، با کم کردن 
$
p_3+p_4
$
 به معادله‌ی 3 می رسیم.
\item
از جمع معادله‌های 2، 3 و 4 به معادله‌ی 1 می رسیم.
\end{enumerate}
بنابراین در این سوال، 
\textbf{
ناهمبستگی و استقلال معادلند.
}
\newline\newline
سوال 3) الف) به سادگی و با انتگرال گیری می توان نتیجه گرفت:
\eqn{
&f_X(x)=\begin{cases}
1&,\quad 0<x<1
\\0&,\quad \text{در غیر این صورت}
\end{cases}
\\&f_Y(y)=\begin{cases}
1&,\quad 0<y<1
\\0&,\quad \text{در غیر این صورت}
\end{cases}
}{}
در نتیجه
\eqn{
E\{X\}=E\{Y\}={1\over 2}
}{}
هم چنین می دانیم
\eqn{
E\{XY\}&=\int_0^1\int_0^1 xy+\alpha xy\sin[2\pi(x+y)]dxdy
\\&={1\over 4}+\alpha\int_0^1\int_0^1 xy\sin[2\pi(x+y)]dxdy
\\&={1\over 4}+\alpha\int_0^1y\int_0^1 x\sin[2\pi(x+y)]dxdy
}{}
هم چنین
\eqn{
\int_0^1 x\sin[2\pi(x+y)]dx&=\left[-{x\over 2\pi}\cos[2\pi(x+y)]+{1\over 4\pi^2}\sin[2\pi(x+y)]\right]_{x=0}^{x=1}
\\&=-{\cos 2\pi y\over 2\pi}
}{}
در نتیجه
\eqn{
E\{XY\}&=
{1\over 4}-{\alpha\over 2\pi}\int_0^1y{\cos 2\pi y}dy
\\&={1\over 4}-{\alpha\over 2\pi}\left[{y\sin 2\pi y\over 2\pi}+{\cos 2\pi y\over 4\pi^2}\right]_{y=0}^{y=1}
\\&={1\over 4}
}{}
از آنجا که همواره 
$
E\{XY\}=E\{X\}E\{Y\}
$
، در نتیجه همواره دو متغیر تصادفی $X$ و $Y$ ناهمبسته هستند.

ب) به سادگی دیده می شود که 
$$
f_{X,Y}(x,y)=f_X(x)f_Y(y)\iff \alpha=0
$$
\newline\newline
سوال 4) الف)
\eqn{
&f_X(x)=\int_{-\infty}^{\infty} f(x,y)dy
\\&=
{1\over 2\pi \sqrt{1-\rho^2}}\int_{-\infty}^{\infty}\exp\left[-{1\over 2}\cdot{1\over 1-\rho^2}(x^2+y^2-2\rho xy)\right]dy
\\&=
{1\over 2\pi \sqrt{1-\rho^2}}\int_{-\infty}^{\infty}\exp\left[-{1\over 2}\cdot{1\over 1-\rho^2}(\rho^2x^2+y^2-2\rho xy+(1-\rho^2)x^2)\right]dy
\\&=
{1\over 2\pi \sqrt{1-\rho^2}}\int_{-\infty}^{\infty}\exp\left[-{1\over 2}\cdot{1\over 1-\rho^2}([\rho x-y]^2+(1-\rho^2)x^2)\right]dy
\\&=
{1\over 2\pi \sqrt{1-\rho^2}}\exp\left[-{1\over 2}x^2\right]\int_{-\infty}^{\infty}\exp\left[-{1\over 2}\cdot{1\over 1-\rho^2}[\rho x-y]^2\right]dy
\\&=
{1\over \sqrt{2\pi}}\exp\left[-{1\over 2}x^2\right]\cdot
\underbrace{
{1\over \sqrt{2\pi} \sqrt{1-\rho^2}}\int_{-\infty}^{\infty}\exp\left[-{1\over 2}\cdot{1\over 1-\rho^2}y^2\right]dy}_{1}
\\&=
{1\over \sqrt{2\pi}}\exp\left[-{1\over 2}x^2\right]
}{}

ب) با 
$
\rho=0
$
 خواهیم داشت:
\eqn{
f(x,y)&={1\over 2\pi}\exp\left[-{1\over 2}x^2-{1\over 2}y^2\right]
\\&
={1\over \sqrt{2\pi}}\exp\left[-{1\over 2}x^2\right]\cdot
{1\over \sqrt{2\pi}}\exp\left[-{1\over 2}y^2\right]
\\&=f_X(x)f_Y(y)
}{}

پ) باید داشته باشیم
\eqn{
f(x)f(y)&={1\over 2\pi}\exp\left[-{1\over 2}x^2-{1\over 2}y^2\right]
\\&
=f(x,y)
\\&=
{1\over 2\pi \sqrt{1-\rho^2}}\exp\left[-{1\over 2}\cdot{1\over 1-\rho^2}(x^2+y^2-2\rho xy)\right]
}{}
به طور مثال به ازای $x=y=0$ نتیجه می شود $\rho=0$ که پاسخ درستی است و اثبات را کامل می کند.

ت) (!)
\end{document}