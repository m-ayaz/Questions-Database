\documentclass{article}

\usepackage{amsmath,amssymb,geometry,tikz}
\usepackage{xepersian}

\setlength{\parindent}{0pt}
\setlength{\parskip}{3mm}

\newcounter{questionnumber}
\setcounter{questionnumber}{1}

\newcommand{\Q}{
\textbf{سوال \thequestionnumber)}
\stepcounter{questionnumber}
}

\newcommand{\eqn}[1]{
\begin{equation}\begin{split}
#1
\end{split}\end{equation}
}

\begin{document}
\LARGE
\begin{center}
\settextfont{IranNastaliq}

به نام زیبایی

%\begin{figure}[h]
%\centering
%\includegraphics[width=30mm]{kntu_logo.eps}
%\end{figure}

تمرینات سری هشتم درس احتمال مهندسی

\end{center}
\hrulefill
\large

\Q

از کیسه‌ای که شامل 7 توپ آبی و 3 توپ سفید است، 1 توپ به تصادف برداشته، رنگ آن را یادداشت کرده و دوباره به کیسه بر می‌گردانیم. اگر این کار را 11 بار انجام دهیم، احتمال آن که از این 11 بار دقیقأ در 7 مرتبه، توپ آبی بیرون آمده باشد چقدر است؟

\Q

یک کانال مخابراتی دارای ظرفیت 25 گیگابیت بر ثانیه است. در مجموع، 12 کاربر قصد استفاده از این کانال برای ارسال داده‌ی خود را دارند که هر کاربر، $2.5$ گیگابیت بر ثانیه از کانال را اشغال می‌کند و احتمال فعال بودن او، مستقل از سایرین برابر $p=0.6$ است. با چه احتمالی، برای تخصیص کانال به کاربران فعال، دچار کمبود ظرفیت کانال نخواهیم شد؟

\Q

یک آزمایش برنولی را که احتمال موفقیت در آن برابر 
$
\frac{1}{3}
$
است، $n$ بار تکرار می‌کنیم. اگر $k$ تعداد موفقیت ها در $n$ آزمایش باشد، $n$ حداقل چقدر باشد تا احتمال رخداد 
$
\{\frac{97}{300}<\frac{k}{n}<\frac{103}{300}\}
$
برابر $99\%$ باشد؟
\end{document}