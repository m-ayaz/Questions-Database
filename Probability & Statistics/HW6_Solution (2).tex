\documentclass[10pt,letterpaper]{article}
\usepackage{toolsper}
%\settextfont{B Nazanin}
\usepackage{lipsum}
\setlength{\parindent}{0mm}
\setlength{\parskip}{3mm}
\newcommand{\pic}[2]{
\begin{center}
\includegraphics[width=#2]{#1}
\end{center}
}
\begin{document}
\Large
\begin{center}
به نام او

پاسخ تمرینات سری ششم درس احتمال مهندسی
\hl
\end{center}
%\color{red}
سوال 1) الف) 
$$
\binom{n}{k}p^k(1-p)^{n-k}=0.2668\quad,\quad e^{-np}{(np)^k\over k!}=0.1490\quad,\quad \text{Rel. Err.}= 44.16\%
$$

ب)
$$
\binom{n}{k}p^k(1-p)^{n-k}=0.1573\quad,\quad e^{-np}{(np)^k\over k!}=0.1318\quad,\quad \text{Rel. Err.}= 16.23\%
$$

پ)
$$
\binom{n}{k}p^k(1-p)^{n-k}=0.3716\quad,\quad e^{-np}{(np)^k\over k!}=0.3679\quad,\quad \text{Rel. Err.}= 1\%
$$

مشاهده می شود که در حالت سوم، خطای تقریب از همه کم تر است؛ زیرا شرایطی که باعث افزایش دقت تقریب می شوند ($n>>1$ و $k\approx np$)، در این حالت به خوبی مراعات شده اند.

سوال 2) در پرتاب دو تاس سالم، اگر متغیر تصادفی $X$ را برابر تعداد اعداد زوج رو آمده در هر دو تاس در نظر بگیریم:

الف) 
$$
\Omega=\{0,1,2\}
$$

ب) با توجه به فضای شدنی X، داریم:
$$
\Pr\{X\le1.5\}=\Pr\{X=0\text{ یا } X=1\}=\Pr\{X=0\}+\Pr\{X=1\}
$$
و
$$
\Pr\{X\le0.5\}=\Pr\{X=0\}
$$
بنابراین 
$$
\Pr\{X\le1.5\}-\Pr\{X\le0.5\}=\Pr\{X=1\}=\binom{2}{1}\times {3\over 6}\times {3\over 6}={1\over 2}
$$

پ)
$$
p_X(x)=\Pr\{X=x\}=
\begin{cases}
{1\over 4}&,\quad x=0,2\\
{1\over 2}&,\quad x=1
\end{cases}
$$

سوال 3) فرض کنید یک سکه سالم را n بار پرتاب کرده ایم. در اینصورت pmf متغیر تصادفی X را در حالت های زیر بیابید.

الف) متغیر تصادفی X برابر تعداد روها در پرتاب های زوج است. 

$$
\Pr\{X=x\}=\binom{\lfloor{n\over 2}\rfloor}{x}\left({1\over 2}\right)^{\lfloor{n\over 2}\rfloor}
$$

ب) متغیر تصادفی X فقط 5 مقدار 
$
\{0,1,2,3,4\}
$
را با احتمال غیرصفر اختیار می کند؛ بنابراین
$$
p_X(x)=\Pr\{X=x\}=
\binom{4}{x}{1\over 16}
$$

پ) اگر n فرد باشد، تعداد روها و پشت ها هرگز برابر نمی شوند؛ پس 
$$
\Pr\{X=x\}=\begin{cases}
1&,\quad x=0\\
0&,\quad x=1
\end{cases}
$$
و برای n زوج
$$
\Pr\{X=x\}=\begin{cases}
1-\binom{n}{{n\over 2}}\left({1\over 2}\right)^n&,\quad x=0\\
\binom{n}{{n\over 2}}\left({1\over 2}\right)^n&,\quad x=1
\end{cases}
$$
\end{document}