\documentclass{article}

\usepackage{amsmath,amssymb,geometry}
\usepackage{xepersian}

\setlength{\parindent}{0pt}
\setlength{\parskip}{3mm}

\newcounter{questionnumber}
\setcounter{questionnumber}{1}

\newcommand{\Q}{
\textbf{سوال \thequestionnumber)}
\stepcounter{questionnumber}
}

\newcommand{\eqn}[1]{
\begin{equation}\begin{split}
#1
\end{split}\end{equation}
}

\begin{document}
\LARGE
\begin{center}
\settextfont{IranNastaliq}

به نام زیبایی

%\begin{figure}[h]
%\centering
%\includegraphics[width=30mm]{kntu_logo.eps}
%\end{figure}

تمرینات سری دوم درس احتمال مهندسی

\end{center}
\hrulefill
\large

\Q

در یک جامعه، احتمال اینکه فردی به کرونا مبتلا باشد $0.07$ و احتمال آن که به آنفلوآنزا مبتلا باشد $0.19$ است. اگر 20 درصد افراد این جامعه مبتلا به حداقل یکی از این دو بیماری باشند،

الف) چند درصد افراد به هر دو بیماری مبتلا هستند؟

ب) چند درصد افراد \underline{فقط} به کرونا مبتلا هستند؟

\Q

یک عدد از مجموعه‌ی 
$
\{1,2,3,\cdots,10\}
$
به تصادف بر می‌گزینیم. اگر تمام وقایع ساده هم شانس باشند و تعریف کنیم
\eqn{
&A=\{2,3,5,7\}
\\&B=\{1,3,5,7,9\}
}

الف) مقدار
$
P(A)
$
را بیابید.

ب) مجموعه‌های 
$
A-B
$
و
$
A\cap B
$
را به دست آورده و مقادیر 
$
P(A-B)
$
و
$
P(A\cap B)
$
را بیابید.

پ) تحقیق کنید
$$
P(A-B)=P(A)-P(A\cap B).
$$
چه توجیهی برای پاسخ شما وجود دارد؟



\Q

(امتیازی) برای هر دو مجموعه‌ی A و B ثابت کنید:
$$
P(A)+P(B)-\frac{1}{4\max\{1-P(A),1-P(B)\}}
\le 
P(A\cup B)
\le 
P(A)+P(B)
.
$$


\end{document}