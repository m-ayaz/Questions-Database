\documentclass[10pt,letterpaper]{article}
\usepackage{toolsper}
%\settextfont{B Nazanin}
\usepackage{lipsum}
\setlength{\parindent}{0mm}
\setlength{\parskip}{3mm}
\newcommand{\pic}[2]{
\begin{center}
\includegraphics[width=#2]{#1}
\end{center}
}
\begin{document}
\Large
\begin{center}
به نام او

پاسخ تمرینات سری پنجم درس احتمال مهندسی
\hl
\end{center}
%\color{red}
سوال 1) الف) به دلیل استقلال پرتاب ها، می توان تنها دو پرتاب اول و آخر را در نظر گرفت. بنابراین احتمال اینکه در هر دو پرتاب سکه رو یا پشت بیاید برابر $0.5$ است.

ب)
\qn{
\{
\text{
پیشامد حداقل 2 رو و 3 پشت
}\}
&=
\{
\text{
پیشامد 2 رو و 5 پشت
}\}
\\&\cup
\{
\text{
پیشامد 3 رو و 4 پشت
}\}
\\&\cup
\{
\text{
پیشامد 4 رو و 3 پشت
}\}
\\&=
\binom{7}{2}\left({1\over 2}\right)^7
+\binom{7}{3}\left({1\over 2}\right)^7
+\binom{7}{4}\left({1\over 2}\right)^7
\\&={91\over 128}\approx 0.71
}{}

پ) تعریف می کنیم:
\qn{
&
A=
\text{
پیشامد رو آمدن در سه پرتاب اول
}
\\&
B=
\text{
پیشامد پشت آمدن در سه پرتاب اول
}
\\&
C=\text{
دقیقا 4 بار رو آمدن سکه در 7 پرتاب
}
}{}
با تعاریف فوق مطلوب است:
$$
\Pr(
C|A\cup B
)
$$
بنابراین
\qn{
\Pr(
C|A\cup B
)
&={p(C\cap[A\cup B])\over p(A\cup B)}
\\&={p(C\cap A)+p(C\cap B)\over p(A)+p(B)}
\\&={p(C|A)p(A)+\left({1\over 2}\right)^7\over {1\over 8}+{1\over 8}}
\\&={\binom{4}{1}\left({1\over 2}\right)^4\cdot{1\over 8}+\left({1\over 2}\right)^7\over {1\over 4}}
\\&={5\over 32}\approx 0.16
}{}

سوال 2) الف) پیشامد مطلوب برابر است با:
\qn{
\text{
پیشامد جمع 7
}
&=
\text{
یک بار رو آمدن عدد 3 و چهار بار رو آمدن عدد 1
}
\\&\cup
\text{
دو بار رو آمدن عدد 2 و سه بار رو آمدن عدد 1
}
\\&=
\binom{5}{1}\left({1\over 6}\right)^1\left({1\over 6}\right)^4
\\&+
\binom{5}{2}\left({1\over 6}\right)^2\left({1\over 6}\right)^3
\\&={5\over 2592}\approx 0.002
}{}

ب) 
\qn{
\text{
پیشامد مطلوب
}
&=
\text{
رو آمدن 4 در پرتاب آخر و 1 در چهار پرتاب اول
}
\\&\cup
\text{
رو آمدن 5 در پرتاب آخر، سه تا 1 و یک 2 در چهار پرتاب اول
}
\\&\cup
\text{
رو آمدن 6 در پرتاب آخر و سه تا 1 و یک 3 در چهار پرتاب اول
}
\\&\cup
\text{
رو آمدن 6 در پرتاب آخر و دو تا 2 و دو تا 1 در چهار پرتاب اول
}
\\&=
\left({1\over 6}\right)^5
+
\binom{4}{3}\left({1\over 6}\right)^4\times {1\over 6}
+
\binom{4}{3}\left({1\over 6}\right)^4\times {1\over 6}
+
\binom{4}{2}\left({1\over 6}\right)^4\times {1\over 6}
\\&={5\over 2592}\approx 0.002
}{}

سوال 3) الف) این اتفاق تنها زمان می افتد که حداکثر 8 ماشین وارد بزرگراه شوند. مکمل این پیشامد حالتی است که هر 9 ماشین همزمان وارد بزرگراه شوند که این رخداد دارای احتمال 
$
p^9
$
است. پس احتمال مطلوب، 
$
1-p^9
$
خواهد بود.

ب) $p$ چقدر باشد تا احتمال قسمت الف بیشتر از $0.99$ باشد؟
$$
1-p^9>0.99\iff p^9<0.01\iff p<0.6
$$

سوال 4) الف) زمانی تیم A پس از 6 دست موفق به بردن بازی می شود که در دست ششم، برد پنجم خود را کسب کند. در این صورت باید در 4 دست از 5 دست پیش پیروز شده باشد. این رخداد با احتمال
$$
\binom{5}{4}p^4(1-p)\times p=5p^5(1-p)
$$
رخ می دهد.

ب) با توجه به رابطه‌ی 
$
P(A'|B)+P(A|B)=1
$
داریم:
\qn{
&\Pr\{
\text{\rl{باخت در حداقل یک دست به تیم $B$}}|\text{\rl{برد تیم $A$}}
\}
\\&=
1-\Pr\{
\text{\rl{باخت در هیچ دستی به تیم $B$}}|\text{\rl{برد تیم $A$}}
\}
\\&=
1-{
\Pr\{
\text{\rl{باخت در هیچ دستی به تیم $B$}}\cap\text{\rl{برد تیم $A$}}
\}
\over
\Pr\{
\text{\rl{برد تیم $A$}}
\}
}
}{}
احتمال آن که تیم \lr{A} بازی را ببرد و هیچ دستی را به تیم \lr{B} نبازد برابر $p^9$ است. هم چنین احتمال برد تیم $A$ نیز برابر 
$
\sum_{k=5}^{9}\binom{9}{k}p^k(1-p)^{9-k}
$
 خواهد بود. در نتیجه احتمال مطلوب به شکل زیر محاسبه می‌شود:
$$
1-{p^9\over \sum_{k=5}^{9}\binom{9}{k}p^k(1-p)^{9-k}}
$$
ج) تیم $A$ در صورتی بازی را می برد که حداقل 4 دست از 8 دست باقی مانده را ببرد. این احتمال برابرست با:
$$
{\sum_{k=4}^{8}\binom{8}{k}\over 2^8}\approx 0.6367
$$
سوال 5) قضیه ی دوموآو-لاپلاس بیان می دارد:
\qn{
\binom{n}{k}p^k(1-p)^{n-k}\approx G\left({k-np\over\sqrt{np(1-p)}}\right)
}{}
زمانی که $k$ بسیار به $n$ نزدیک و $n$ بسیار بزرگ باشد. از جمله نتایجی که می توان از این قضیه گرفت، عبارتست از:
\qn{
\sum_{k=k_1}^{k_2}\binom{n}{k}p^k(1-p)^{n-k}\approx
G\left({k_2-np\over\sqrt{np(1-p)}}\right)
-
G\left({k_1-np\over\sqrt{np(1-p)}}\right)
}{}
در این سوال:
$$
p={1\over 2}
$$
بنابراین
\qn{
\Pr\left\{0.49\le{k\over n}\le0.51\right\}&=
\sum_{0.49n}^{0.51n}\binom{n}{k}p^k(1-p)^{n-k}
\\&\approx
G\left({0.01n\over\sqrt{n\over 4}}\right)
-
G\left({-0.01n\over\sqrt{n\over 4}}\right)
\\&=
2G\left({0.02\sqrt n}\right)-1
>0.95
}{}
در نتیجه
$$
G\left({0.02\sqrt n}\right)>0.975
$$
{\color{red}
(پیدا کردن دقیق کران $n$ اختیاری است)
}
$$
G\left({0.02\sqrt n}\right)>0.975\iff {0.02\sqrt n}>G^{-1}(0.975)\iff n\ge 9604\quad(!!)
$$
\end{document}