\documentclass[10pt,letterpaper]{article} 
\usepackage{toolsper}
%\usepackage{graphicx}‎‎
%\usefonttheme{serif}‎
%\usepackage{ptext}‎
\usepackage{xepersian}
\settextfont{B Nazanin}
\usepackage{lipsum}
\setlength{\parindent}{0pt}
\newcommand{\pf}{$\blacksquare$}
\newcommand{\pic}[2]{
\begin{center}
\includegraphics[width=#2]{#1}
\end{center}
}
%\newcommand{\eqn}[2]{
%\begin{equation}
%\begin{split}
%#1
%\end{split}
%\label{#2}
%\end{equation}
%}
\begin{document}
\Large
\begin{center}
به نام خدا

پاسخ تمرینات سری هفتم درس آمار و احتمال
\hl
\end{center}
سوال 1) الف)
\eqn{
&\int_{-\infty}^{\infty}\int_{-\infty}^{\infty}f(x,y)dxdy=
\int_{-\infty}^{\infty}\int_{-\infty}^{\infty}{k\over 1+x^2+y^2}dxdy
\\&=
\int_{0}^{2\pi}\int_{0}^{\infty}{k\over 1+r^2}rdrd\phi
\\&=
\int_{0}^{2\pi}\int_{0}^{\infty}k\ln 1+r^2|_0^\infty d\phi
\\&=\infty\times k
}{}
واضح است که مقدار فوق هرگز نمی تواند برابر 1 باشد؛ در نتیجه تابع این سوال هرگز یک \lr{PDF} نیست.

ب) به وضوح مقدار $k$ باید منفی باشد؛ زیرا در غیر اینصورت انتگرال توزیع احتمال نامحدود خواهد شد.
\eqn{
&\int_{-\infty}^{\infty}\int_{-\infty}^{\infty}f(x,y)dxdy=
\int_{-\infty}^{\infty}\int_{-\infty}^{\infty}e^{k(x^2+y^2)}dxdy
\\&=
\int_{0}^{2\pi}\int_{0}^{\infty}re^{kr^2}drd\phi
\\&=
\int_{0}^{2\pi}\int_{0}^{\infty}{1\over 2k}e^{kr^2}|_0^\infty d\phi
\\&=
\int_{0}^{2\pi}\int_{0}^{\infty}-{1\over 2k} d\phi
\\&=-{\pi\over k}
}{}
بنابراین بوضوح $k=-\pi$.

پ) تابع توزیع این بخش، یک استوانه با مساحت قاعده‌ی $\pi$ و ارتفاع $k$ را نشان می دهد. از آنجا که حجم این استوانه (یا همان انتگرال توزیع احتمال) برابر $k\pi$ است، باید داشته باشیم $k={1\over\pi}$.

ت) تابع توزیع این بخش، یک مخروط با مساحت قاعده‌ی $\pi$ و ارتفاع $k$ را نشان می دهد. از آنجا که حجم این مخروط برابر $k\pi\over 3$ است، باید داشته باشیم $k={3\over\pi}$.

ث) 
\eqn{
&\int_{-\infty}^{\infty}\int_{-\infty}^{\infty}f(x,y)dxdy=
\int_{0}^{k}\int_{0}^{k}xydxdy
\\&=
\int_{0}^{k}xdx\cdot \int_{0}^{k}ydy
\\&=
{k^4\over 4}
=1
}{}
بنابراین $k=\sqrt 2$.

ج) تابع توزیع احتمال این بخش، یک منشور قائم با قاعده‌ی مثلثی (به صورت قائم الزاویه و با مساحت $k^2\over 2$) و ارتفاع 1 را نشان می دهد. از آنجا که حجم این مخروط برابر $k^2\over 2$ است، باید داشته باشیم $k=\sqrt 2$.
\newline
\newline
سوال 2) (سوال 1 بخش های ب، پ و ت) با توجه به راهنمایی سوال و با توجه به اینکه \lr{PDF} داده شده، یک تابع دایروی-متقارن است، بنابراین نسبت به هردو محور $X=0$ و $X+Y=0$ تقارن خطی داشته و لذا احتمال نواحی مزبور، برابر $1\over 2$ است.

(سوال 1 بخش ث و ج) برای این دو بخش، مقدار $X+Y$ همواره با احتمال 1 مثبت است و در نتیجه مقدار این دو احتمال، همواره برابر یک خواهد بود.
\newline
\newline
سوال 3) (سوال 2 قسمت ث)
\eqn{
&f_X(x)=\int_{-\infty}^\infty f(x,y)dy=\int_{0}^{\sqrt 2} xydy
\\&=x
}{}
در نتیجه
\eqn{
f_X(x)=\begin{cases}
x&,\quad 0<x<\sqrt2
\\0&,\quad \text{در غیر این صورت}
\end{cases}
}{}

(سوال 2 قسمت ج)
\eqn{
&f_X(x)=\int_{-\infty}^\infty f(x,y)dy=\int_{0}^{\sqrt 2-x} dy
\\&=\sqrt 2-x
}{}
در نتیجه
\eqn{
f_X(x)=\begin{cases}
\sqrt2-x&,\quad 0<x<\sqrt2
\\0&,\quad \text{در غیر این صورت}
\end{cases}
}{}
\newline\newline
سوال 4) الف)
\eqn{
\Pr\{X=Y\}=\Pr\{X=Y=0\}+\Pr\{X=Y=1\}=2\theta=0
}{}
در نتیجه $\theta=0$

ب) با جمع سطری و ستونی نتیجه می شود:
\eqn{
&\Pr\{X=x\}={1\over 2}
\\&\Pr\{Y=y\}={1\over 2}
\\& x,y\in\{0,1\}
}{}
 در نتیجه:
\eqn{
\theta={1\over 4}&,\quad x\ne y
\\{1\over 2}-\theta={1\over 4}&,\quad x=y
}{}
از هر دو حالت خواهیم داشت $\theta={1\over 4}$.
\end{document}