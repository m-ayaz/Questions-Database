\chapter{متغیرهای تصادفی توأم}

\Q
فرض کنید $X$ و $Y$ دو متغیر تصادفی مستقل برنولی به ترتیب با پارامترهای $1\over 2$ و $p$ باشند. ثابت کنید
$Z=X\oplus Y \mod 2$
و
$Z=XY$
 دارای توزیع برنولی با پارامتر ${1\over 2}p$ هستند.

%%%%%%%%%%%%%%%%%%%%%%%%%%%%%%%%%%%%%%

\Q
اگر $X$ و $Y$ دو توزیع چندجمله ای به ترتیب با پارامترهای 
$
(n_1,p)
$
 و 
$
(n_2,p)
$
 باشند، ثابت کنید توزیع $X+Y$ دوجمله ای با پارامترهای $(n_1+n_2,p)$ است.

%%%%%%%%%%%%%%%%%%%%%%%%%%%%%%%%%%%%%%

\Q
تابع چگالی احتمال توام دو متغیر تصادفی $X$ و $Y$ به صورت زیر است:
$$
f(x,y)=
\begin{cases}
k(4-x-y) &,\quad 1 < x < 2\ \ ,\ \   0 < y < 2 \\
0 &,\quad \text{\rl{در غیر این صورت}}
\end{cases}
$$
الف) مقدار مناسب $k$ را بیابید.

ب) با مقدار $k$ به دست آمده در قسمت قبل، مقدار 
$
\EX \{XY\}
$
 را به دست آورید.

%%%%%%%%%%%%%%%%%%%%%%%%%%%%%%%%%%%%%%

\Q
تابع چگالی احتمال توام دو متغیر تصادفی $X$ و $Y$ به صورت زیر است:
$$
f(x,y)=
\begin{cases}
1 &,\quad |x|+2|y|<1\\
0 &,\quad |x|+2|y|\ge 1
\end{cases}
$$
الف)  چگالی های احتمال حاشیه ای $X$ و $Y$ را به دست آورید. همچنین ناهمبستگی، استقلال و تعامد این دو متغیر تصادفی را تحقیق کنید.

ب) چگالی احتمال $X+Y$ را به دست آورید.

%%%%%%%%%%%%%%%%%%%%%%%%%%%%%%%%%%%%%%

\Q
یک قطار و اتوبوس به طور تصادفی و مستقل از هم بین ساعات 5 تا 6 وارد یک ایستگاه می‌شوند. فردی نیز به طور تصادفی بین ساعت 5 تا $5:30$ وارد همان ایستگاه می‌شود.

الف) احتمال آن که فرد بیش از 10 دقیقه منتظر قطار و اتوبوس شود چقدر است؟

ب) اگر قطار و اتوبوس هر یک 10 دقیقه در ایستگاه تاخیر داشته باشند، احتمال با هم بودن آنها در ایستگاه چفدر است؟

پ) اگر فرد پس از ساعت $5:15$ به ایستگاه برسد، با چه احتمالی به هیچ یک نمی‌رسد؟

%%%%%%%%%%%%%%%%%%%%%%%%%%%%%%%%%%%%%%

\Q
یک قطار و اتوبوس به طور تصادفی و مستقل از هم بین ساعات 6 تا 7 صبح وارد ایستگاهی می شوند. فردي نیز به طور تصادفی بین ساعات 5:50 تا 6:50 وارد همان ایستگاه می شود.

الف) احتمال اینکه فرد بیش از 10 دقیقه منتظر قطار ویا اتوبوس بماند چقدر است؟

ب) احتمال اینکه این فرد به هیچ یک از قطار یا اتوبوس نرسد چقدر است؟

%%%%%%%%%%%%%%%%%%%%%%%%%%%%%%%%%%%%%%

\Q
اگر $X$ و $Y$، دو متغیر تصادفی نرمال با میانگین 0 و واریانس 1 باشند به گونه ای که 
$
\text{\lr{cov}}(X,Y)=0.5
$
، در این صورت مقدار $a$  را به گونه ای بیابید که $X+aY$ و $X+2Y$ مستقل از هم باشند و در این صورت، واریانس هر یک را بیابید.

%%%%%%%%%%%%%%%%%%%%%%%%%%%%%%%%%%%%%%

\Q
تابع چگالی احتمال توام دو متغیر تصادفی $X$ و $Y$ به صورت زیر است:
$$
f(x,y)=
\begin{cases}
k(4-x-y) &,\quad 1 < x < 2\ \ ,\ \   0 < y < 2 \\
0 &,\quad \text{\rl{در غیر این صورت}}
\end{cases}
$$
الف) مقدار مناسب $k$ را بیابید.

ب) با مقدار $k$ به دست آمده در قسمت قبل، مقدار 
$
\EX \{XY\}
$
 را به دست آورید.

%%%%%%%%%%%%%%%%%%%%%%%%%%%%%%%%%%%%%%

\Q
تابع چگالی احتمال توأم دو متغیر تصادفی $X$ و $Y$ به صورت زیر است:
$$
f_{X,Y}(x,y)=\begin{cases}
k&,\quad x-1<y<x\ \ ,\ \ 0<x<2\ \ ,\ \ 0<y<1
\\0&,\quad \text{در غیر این صورت}
\end{cases}
$$
که $k$ ثابت است.

الف) مقدار $k$ را به دست آورید.

ب) نشان دهید $Y$ و $X-Y$ از هم مستقل هستند.

%%%%%%%%%%%%%%%%%%%%%%%%%%%%%%%%%%%%%%

\Q
یک قطار و یک اتوبوس بین ساعت 9 و 10 در زمانی تصادفی وارد ایستگاه می‌شوند. قطار 10 دقیقه و اتوبوس $x$ دقیقه توقف دارند. $x$ را طوری تعیین کنید که احتمال با هم بودن قطار و اتوبوس برابر $0.5$ باشد.

%%%%%%%%%%%%%%%%%%%%%%%%%%%%%%%%%%%%%%

\Q
دایره‌ی واحد را با مرکز مبدا مختصات در نظر بگیرید.

الف) نقطه ای به تصادف از داخل این دایره انتخاب می شود. اعداد 
$
0<r_0<1
$
 و 
$
0<\phi_0<2\pi
$ 
را در نظر بگیرید.  اگر مختصات قطبی این نقطه را با 
$
(r,\phi)
$
 نشان دهیم، با چه احتمالی داریم 
$
r_0<r<r+\Delta r_0
$
 و 
$
\phi_0<\phi<\phi_0+\Delta\phi_0
$
؟

ب) ابتدا قطری از دایره را به تصادف انتخاب کرده و سپس نقطه ای از این قطر را به تصادف بر می‌گزینیم. اگر مختصات قطبی این نقطه را با 
$
(r,\phi)
$
 نشان دهیم، با چه احتمالی داریم 
$
r_0<r<r+\Delta r_0
$
 و 
$
\phi_0<\phi<\phi_0+\Delta\phi_0
$
؟

پ) تابع چگالی احتمال نقطه را در هر دو حالت قسمت های الف و ب به دست آورید.

%%%%%%%%%%%%%%%%%%%%%%%%%%%%%%%%%%%%%%

\Q
(ناوردایی متغیرهای تصادفی گوسی تحت عمل جمع)

الف) فرض کنید $X$ و $Y$ دو متغیر تصادفی با توابع چگالی احتمال زیر باشند:
\[
f_X(x)={1\over \sqrt{2\pi\sigma_X^2}}\exp\left(-{x^2\over 2\sigma_X^2}\right)
\]
\[
f_Y(y)={1\over \sqrt{2\pi\sigma_Y^2}}\exp\left(-{y^2\over 2\sigma_Y^2}\right)
\]
ثابت کنید متغیر تصادفی $X+Y$ از توزیع زیر پیروی می کند:
\[
f_{X+Y}(u)={1\over \sqrt{2\pi[\sigma_X^2+\sigma_Y^2]}}\exp\left(-{u^2\over 2\pi[\sigma_X^2+\sigma_Y^2]}\right)
\]
ب) رابطه‌ی کلی تری را که می‌توان از تعمیم قسمت الف استنتاج کرد، بنویسید.

%%%%%%%%%%%%%%%%%%%%%%%%%%%%%%%%%%%%%%

\Q
برای هر کدام از توابع زیر که می‌توانند چگالی احتمال مشترک دو متغیر تصادفی باشند، ثابت مناسب $k$ و مقادیر
$\Pr\{X>0\}$
و
$\Pr\{X+Y>0\}$
 را بیابید. همچنین برای قسمت های ث) و ج)، چگالی احتمال متغیر تصادفی $X$ را به دست آورید.

الف) 
$
f(x,y)={k\over 1+x^2+y^2}
$

ب)
$
f(x,y)=e^{a(x^2+y^2)}
$

پ)
$
f(x,y)=\begin{cases}
k&,\quad x^2+y^2<1
\\
0&,\quad \text{\rl{در غیر این صورت}}
\end{cases}
$

ت)
$
f(x,y)=\begin{cases}
k-k\sqrt{x^2+y^2}&,\quad x^2+y^2<1
\\
0&,\quad \text{\rl{در غیر این صورت}}
\end{cases}
$

ث)
$
f(x,y)=\begin{cases}
xy&,\quad 0<x<k\quad,\quad 0<y<k
\\
0&,\quad \text{\rl{در غیر این صورت}}
\end{cases}
$

ج)
$
f(x,y)=\begin{cases}
1&,\quad x>0\ \ ,\ \ y>0\ \ ,\ \ x+y<a
\\
0&,\quad \text{\rl{در غیر این صورت}}
\end{cases}
$

(راهنمایی: ابتدا تحقیق کنید اگر $f(x,y)$ تابعی از $x^2+y^2$ باشد، داریم 
$$\Pr\{aX+bY>0\}=\Pr\{X>0\}\quad,\quad a^2+b^2\ne 0$$)

%%%%%%%%%%%%%%%%%%%%%%%%%%%%%%%%%%%%%%

\Q
دو متغیر تصادفی $X$ و $Y$ با چگالی احتمال توأم زیر مفروضند. در این صورت، مقدار $\alpha$ را به گونه ای بیابید که این دو متغیر تصادفی ناهمبسته شوند.
$$
f_{XY}(x,y)=\begin{cases}
\frac{x^2}{2}+y^2+\alpha xy&,\quad -1<x<1\ \ ,\ \ -1<y<1\\
0&,\quad \text{سایر جاها}
\end{cases}
$$

%%%%%%%%%%%%%%%%%%%%%%%%%%%%%%%%%%%%%%

\Q
دو متغیر تصادفی $X$ و $Y$ با چگالی احتمال توأم زیر مفروضند. در این صورت، تابع توزیع تجمعی توأم این دو متغیر تصادفی را بیابید.
$$
f_{XY}(x,y)=\begin{cases}
ye^{1-xy}&,\quad x>1\ \ ,\ \ y>1\\
0&,\quad \text{سایر جاها}
\end{cases}
$$

%%%%%%%%%%%%%%%%%%%%%%%%%%%%%%%%%%%%%%

\Q
دو نفر به طور مستقل از هم و کاملا تصادفی بین ساعت 4 و 5 وارد فروشگاهی می شوند و هر یک پس از 10 دقیقه، از آن خارج می شوند. احتمال آن که این دو نفر یکدیگر را در فروشگاه ملاقات کنند چقدر است؟

%%%%%%%%%%%%%%%%%%%%%%%%%%%%%%%%%%%%%%

\Q
جدول زیر را برای متغیرهای تصادفی $X$ و $Y$ در نظر بگیرید:
\begin{table}[h]
\centering
\Large
\lr{
\begin{tabular}{|c|c|c|}
\hline
\backslashbox{$X$}{$Y$}&0&1\\\hline
0&$\frac{1}{2}-\theta$&$\theta$\\\hline
1&$\theta$&$\frac{1}{2}-\theta$\\\hline
\end{tabular}
}
\end{table}

الف) توابع توزیع احتمال حاشیه‌ای متغیرهای $X$ و $Y$ را به دست آورید.

ب) به ازای چه مقدار $\theta$ داریم
$
P(X=Y)=1
$
؟

پ) به ازای چه مقدار $\theta$ داریم
$
P(X=x,Y=y)=P(X=x)P(Y=y)
$
؟

%%%%%%%%%%%%%%%%%%%%%%%%%%%%%%%%%%%%%%

\Q
در پرتاب دو تاس سالم و متمایز، متغیر تصادفی $X$ را مجموع اعداد رو آمده و $Y$ را تعداد 6 های رو آمده در نظر بگیرید.

الف) مقادیر 
$
\Pr\{X=1,Y=7\}
$
و
$
\mathbb{E}\{XY\}
$
 چقدر است؟

ب) آیا این دو متغیر تصادفی ناهمبسته اند؟

%%%%%%%%%%%%%%%%%%%%%%%%%%%%%%%%%%%%%%

\Q
در جدول زیر که توزیع احتمال را برای متغیر های تصادفی $X$ و $Y$ نشان می‌دهد،
\begin{table}[h]
\centering
\Large
\lr{
\begin{tabular}{|c|c|c|}
\hline
\backslashbox{$X$}{$Y$}&0&1\\\hline
0&$p_1$&$p_2$\\\hline
1&$p_3$&$p_4$\\\hline
\end{tabular}
}
\end{table}

الف) مقدار 
$
\text{\lr{cov}}(X,Y)
$
 را به دست آورید و تحقیق کنید چه زمانی این کمیت صفر است.

ب) آیا برای این دو متغیر تصادفی، ناهمبستگی، استقلال را نتیجه می دهد؟ اگر چنین است، نشان دهید و اگر چنین نیست، مثالی برای مقادیر 
$
p_1,p_2,p_3,p_4
$
 بزنید که ناهمبستگی، استقلال را نتیجه نمی‌دهد (دقت داشته باشید که جمع احتمالات برابر یک است و احتمالات نامنفی اند).

%%%%%%%%%%%%%%%%%%%%%%%%%%%%%%%%%%%%%%

\Q
چگالی احتمال زیر را در نظر بگیرید:
$$
f_{X,Y}(x,y)=\begin{cases}
1+\alpha\sin[2\pi(x+y)]&,\quad 0\le x\le1,0\le y\le1
\\0&,\quad \text{در غیر این صورت}
\end{cases}
$$
که $\alpha$ مقدار مناسبی است.

الف) کوواریانس این دو متغیر تصادفی را به دست آورید. آیا این دو متغیر تصادفی ناهمبسته هستند؟

ب) مقادیری از $\alpha$ را بیابید که این دو متغیر تصادفی مستقل باشند.

%%%%%%%%%%%%%%%%%%%%%%%%%%%%%%%%%%%%%%

\Q
تابع چگالی احتمال توام زیر را در نظر بگیرید:
$$
f_{X,Y}(x,y)={1\over 2\pi \sqrt{1-\rho^2}}\exp\left[-{1\over 2}\cdot{1\over 1-\rho^2}(x^2+y^2-2\rho xy)\right]
$$
الف) ثابت کنید $X$ (و مشابها همچنین $Y$) دارای توزیع نرمال با میانگین صفر و واریانس $1$ است.

ب) ثابت کنید اگر $\rho=0$، در این صورت متغیرهای تصادفی $X$ و $Y$ مستقل هستند.

پ) ثابت کنید اگر متغیرهای تصادفی $X$ و $Y$ مستقل باشند آنگاه $\rho=0$.

ت) تابع چگالی احتمالی که در صورت این سوال تعریف شد، حالت خاصی از چگالی احتمال چند متغیره‌ی نرمال است.

ضریب همبستگی $\rho$ در حالت دو متغیره، میزان همبستگی دو متغیر تصادفی را نشان می دهد. ابتدا تحقیق کنید به ازای چه مقداری از $\rho$، این چگالی احتمال، دایروی-متقارن خواهد بود. چگالی احتمال دو متغیره را به ازای مقادیر 
$\rho=-0.5,\rho=0,\rho=0.5$
 ترسیم کنید. به طور شهودی چگونه می توان از روی نمودارها، به میزان همبستگی این دو متغیر تصادفی پی برد؟

این تابع چگالی را به صورت دیگری نیز می توان نوشت:
$$
f(x,y)={1\over \sqrt{(2\pi)^2}\det(\Sigma)}\exp\left[-{1\over 2}\cdot([x,y]\Sigma^{-1}[x,y]^T)\right]
$$
که بردار $[x,y]$ یک بردار سطری دوتایی است و 
$
\Sigma=\begin{bmatrix}
1&\rho\\
\rho&1
\end{bmatrix}
$.
ماتریس $\Sigma$ در متغیرهای تصادفی نرمال توأم، مفهوم مهمی است و ماتریس کوواریانس نام دارد.

به ازای هر یک از مقادیر 
$\rho=-0.5,\rho=0,\rho=0.5$
 و به کمک دستور 
\lr{mvnrnd()}
 در متلب، 1000 جفت داده‌ی تصادفی تولید و آنها را در یک نمودار پراکندگی ترسیم کنید (پس از اجرای دستور فوق در متلب به شیوه ی مناسب، 1000 داده‌ی تصادفی برای $X$ و 1000 داده‌ی تصادفی برای $Y$ خواهید داشت. کافی است $Y$ را برحسب $X$ رسم کنید تا به نمودار پراکندگی برسید. همچنین می توانید از \lr{Help} متلب برای توضیحات بیشتر در مورد \lr{mvnrnd()} بهره ببرید). چگونه از روی نمودار پراکندگی می توان میزان همبستگی دو متغیر تصادفی را نشان داد؟ چه شهودی در آن نهفته است؟ (بسیار مهم است که در این تحقیق، تحلیل و دیدگاه خود را نیز ذکر بفرمایید.)

هنگامی که $\rho=1$، توضیح دهید چه اتفاقی می افتد؟ تفاوت آن با حالت $\rho=-1$ چیست؟ آیا همچنان می‌توان از چگالی احتمال داده شده استفاده کرد؟ چرا؟

%%%%%%%%%%%%%%%%%%%%%%%%%%%%%%%%%%%%%%

\Q
برای هر یک از چگالی احتمال های توام داده شده‌ی زیر، موارد 
$
f_X(x)
$
،
$
\mathbb{E}\{X\}
$
و
$
\mathbb{E}\{XY\}
$
را به دست آورید.

الف) 
$
f_{X,Y}(x,y)=\frac{1}{\pi}e^{-x^2-y^2}
$

ب) 
$
f_{X,Y}(x,y)=\begin{cases}
\frac{3}{2}(1-|x-1|-|y-1|)&,\quad |x-1|+|y-1|<1\\
0&,\quad \text{سایر جاها}
\end{cases}
$

پ)
$
f_{X,Y}(x,y)=\begin{cases}
e^{1-x}&,\quad 0<x<y<1\\
0&,\quad \text{سایر جاها}
\end{cases}
$

ت) X و Y ، دو متغیر تصادفی گسسته (با مقادیر صحیح) اند و تابع جرم احتمال آنها به صورت زیر است،
$$
\Pr\{X=x,Y=y\}=\begin{cases}
\frac{1}{16}&,\quad x^2+y^2\le 10 \ \ ,\ \ x\ge y\\
0&,\quad \text{سایر جاها}
\end{cases}
.
$$

%%%%%%%%%%%%%%%%%%%%%%%%%%%%%%%%%%%%%%

\Q
ابتدا فرض کنید متغیرهای تصادفی $X$ و $Y$ دارای توزیع یکنواخت در بازه‌ی $[0,1]$ و مستقل هستند. توزیع احتمال متغیرهای تصادفی 

الف) $XY$
\quad,\quad
ب) $X+Y$
\quad,\quad
پ) $X\over Y$
\quad,\quad
ت) $\max\{X,Y\}$

ث) $\min\{X,Y\}$

را به دست آورید. سپس فرض کنید $X$ و $Y$ دو متغیر تصادفی نمایی و مستقل با پارامتر 1 باشند. توزیع احتمال هر یک از متغیرهای تصادفی قسمت ب و پ را بیابید.

\Q
تابع چگالی احتمال توام دو متغیر تصادفی $X$ و $Y$ به صورت زیر است:
$$
f_{XY}(x,y)=\begin{cases}
k&,\quad |x|+|y|<1\\
0&,\quad \text{\rl{در غیر این صورت}}
\end{cases}
$$
الف) مقدار مناسب $k$ را بیابید.
\quad,\quad
ب) کوواریانس و ضریب همبستگی $X$ و $Y$ را بیابید.

پ) ثابت کنید متغیرهای تصادفی $X+Y$ و $X-Y$ مستقل هستند و توزیع توام آنها را به دست آورید.
\quad,\quad
ت) توزیع $X$ و میانگین و واریانس آن را به دست آورید.

\Q
برای متغیر تصادفی $X$ که دارای توزیع زیر است
$$
f_X(x)=\begin{cases}
1&,\quad |x|<{1\over 2}\\
0&,\quad \text{\rl{در غیر این صورت}}
\end{cases}
$$
تابع مولد گشتاور را به دست آورده و از روی آن، $\mathbb{E}\{X^4\}$ را محاسبه نمایید.

\Q
توزیع مشترک دو متغیرتصادفی به صورت زیر است،
$$
f_{X,Y}(x,y)={1\over 2\pi \sqrt{1-\rho^2}}\exp\left[-{1\over 2}\cdot{1\over 1-\rho^2}(x^2+y^2-2\rho xy)\right]
$$
الف) ثابت کنید متغیر تصادفی $X+Y$ یک متغیر تصادفی نرمال است و سپس واریانس آن را به دست آورید. چه زمانی این واریانس بیشینه است و چرا؟ در شرایطی که واریانس بیشینه باشد، متغیرهای تصادفی $X$ و $Y$ چه رابطه‌ای دارند؟

ب) ثابت کنید به ازای $\rho=0$، متغیر تصادفی $\tan^{-1}{Y\over X}$ دارای توزیع یکنواخت در بازه‌ی 
$\left[-\frac{\pi}{2},\frac{\pi}{2}\right]$
خواهد بود.

\Q
برای هر یک از توابع دومتغیره‌ی زیر، محدوده مقادیر $k$ را به گونه ای بیابید که تابع مورد نظر، چگالی احتمال توأم دو متغیر تصادفی باشد و سپس، توزیع تجمعی توأم و مقدار $\Pr\left\{X+3Y<{1\over 3}\right\}$ را (در صورت وجود) بیابید. به ازای هر تابع توزیع تجمعی، آیا $X$ و $Y$ مستقلند؟

الف)
$
f(x,y)=
\begin{cases}
xy+kx+ky&,\quad 0<x<1,0<y<1\\
0&,\quad \text{سایر جاها}
\end{cases}
$

ب)
$
f(x,y)=
\begin{cases}
k\sin(x+3y)&,\quad 0<x<\frac{\pi}{2},0<y<\frac{\pi}{6}\\
0&,\quad \text{سایر جاها}
\end{cases}
$

پ)
$
f(x,y)=\begin{cases}
kxy(1-y)&,\quad 0<x<1,0<y<1\\
0&,\quad \text{سایر جاها}
\end{cases}
$

%پ) (امتیازی)
%
%$
%f(x,y)=
%\begin{cases}
%\frac{1}{2}&,\quad |x|^k+|y|^k<1\\
%0&,\quad \text{سایر جاها}
%\end{cases}
%$

\Q
برای هر یک از چگالی های احتمال زیر، مقادیر
$\Pr\{X\le 4,Y\le -2\}$
،
$\Pr\{X+Y\le 2\}$
و
$\Pr\{X=4Y\}$
را بیابید.

الف)
$
f_{XY}(x,y)=
\begin{cases}
\frac{1}{2}\sin (x+y)&,\quad 0<x<\frac{\pi}{2},0<y<\frac{\pi}{2}\\
0&,\quad \text{سایر جاها}
\end{cases}
$

ب)
$
f_{XY}(x,y)=
\begin{cases}
\frac{1}{2}\delta\left(\sqrt{(x+4)^2+(y+1)^2}\right)&,\quad x=-4,y=-1\\
\frac{1}{2}&,\quad 0<x<1,0<y<1\\
0&,\quad \text{سایر جاها}
\end{cases}
$

(دقت شود که همانگونه که 
$
\delta(x-x_0)
$
نشان دهنده‌ی ضربه ای در 
$
x=x_0
$
است، 
$
\delta(\sqrt{(x-x_0)^2+(y-y_0)^2})
$
نیز نشان دهنده‌ی ضربه ای در 
$
x=x_0,y=y_0
$
در دو بعد و دارای سطح زیر یک است.)

\Q
برای چگالی احتمال توأم زیر، مقادیر
$
\sigma_X^2
$،
$
\sigma_Y^2
$،
$
\Phi_X(s)
$،
$
\Phi_Y(s)
$
و چگالی احتمال متغیرهای تصادفی $XY$ و 
$
\max\{X,Y\}
$
 را محاسبه کنید.
$$
f_{X,Y}(x,y)=\begin{cases}
(xy-1)e^{1-xy}&,\quad x\ge 1,y\ge 1\\
0&,\quad \text{سایر جاها}
\end{cases}
$$

\Q
چگالی احتمال توأم زیر برای دو متغیر تصادفی $X$ و $Y$ داده شده است:
$$
f_{X,Y}(x,y)=\begin{cases}
\alpha+2(\frac{1}{\pi}-\alpha)(x^2+y^2)&,\quad x^2+y^2\le 1\\
0&,\quad \text{سایر جاها}
\end{cases}
$$

الف) محدوده‌ی مقادیر مجاز 
$
\alpha
$
را بیابید.

ب) به ازای چه مقدار از 
$
\alpha
$
، دو متغیر تصادفی 
$
X
$
و
$
Y
$
مستقل اند؟ ناهمبسته اند؟

پ) احتمال های
$
\Pr\{aX+bY\ge 0\}
$
و
$
\Pr\{XY\ge 0\}
$
را بیابید.

\Q
اگر چگالی احتمال مشترک دو متغیر تصادفی $X$ و $Y$ به صورت زیر باشد
\[
f(x,y)=\begin{cases}
12x^2&,\quad 0<x<y<1\\
0&,\quad \text{سایر جاها}
\end{cases}
\]
در این صورت مقدار 
$
\text{cov}(X,Y)
$
را بیابید.

\Q
سکه ای را 10 بار پرتاب می‌کنیم. متغیر تصادفی 
$X$،
تعداد دفعات رو آمدن سکه در 5 پرتاب اول و متغیر تصادفی $Y$، تعداد دفعات رو آمدن سکه در پرتاب های زوج است. احتمال 
$\Pr\{X=5Y\}$
را بیابید.

\Q
متغیر تصادفی $X$، دارای تابع جرم احتمال زیر است:
\rl{
\begin{table}[h]
\centering
\Large
\begin{tabular}{|c|c|c|c|c|}
\hline
$x$&$-1$&$0$&$1$&$2$\\\hline
$\Pr\{X=x\}$&$0.1$&$0.2$&$0.3$&$0.4$\\\hline
\end{tabular}
\end{table}
}

اگر داشته باشیم
$Y=X^2-1$،
در اینصورت مقدار
$\text{cov}(X,Y)$
را بیابید.

\Q
برای متغیرهای تصادفی $X$ و $Y$ با چگالی‌های احتمال توأم زیر،

$$
f_{X,Y}(x,y)=\begin{cases}
x+ky&,\quad 0<x<1\ \ ,\ \ 0<y<1\\
0&,\quad \text{سایر جاها}
\end{cases}
$$

$$
f_{X,Y}(x,y)=\begin{cases}
kx&,\quad 0<x<1\ \ ,\ \ 0<x<y<1\\
0&,\quad \text{سایر جاها}
\end{cases}
$$

$$
f_{X,Y}(x,y)=\begin{cases}
kx&,\quad 0<y<1\ \ ,\ \ 0<y<x<1\\
0&,\quad \text{سایر جاها}
\end{cases}
$$

$$
f_{X,Y}(x,y)=\begin{cases}
kx&,\quad x+y<1 \ \ ,\ \ x>0 \ \ , \ \ y>0\\
0&,\quad \text{سایر جاها}
\end{cases}
$$

$$
f_{X,Y}(x,y)=\begin{cases}
k(1+x+y)&,\quad 1<x<2\ \ , \ \ 0<y<1\\
0&,\quad \text{سایر جاها}
\end{cases}
$$

الف) مقدار مناسب $k$ را بیابید.

ب) توزیع حاشیه ای 
$
f_X(x)
$
را پیدا کنید.

پ) مقدار 
$
\text{cov}(X,Y)
$
را محاسبه کنید.

ت) استقلال دو متغیر تصادفی را بررسی کنید.

ث) چگالی احتمال $X+Y$ را بیابید.

\Q
سکه سالمی را 5 بار می اندازیم. تابع مولد گشتاور مشترک $X$ و $Y$ و مقدار
$\mathbb{E}\{X+Y|X=1\}$
را محاسبه کنید، اگر:

الف) $X$ تعداد رو آمدن ها در سه پرتاب اول و $Y$ تعداد پشت آمدن ها در سه پرتاب آخر باشد.

ب) $X$ تعداد روها در پرتاب های فرد و $Y$ تعداد پشت ها در سه پرتاب آخر باشد.

پ) $X$ تعداد پشت ها در پرتاب های زوج و $Y$ تعداد پشت ها در سه پرتاب آخر باشد.

ت) $X$ تعداد پشت ها و $Y$ تعداد رو ها در سه پرتاب آخر باشد.

ث) $X$ تعداد پشت ها و $Y$ تعداد رو ها در دو پرتاب اول باشد.

\Q
یک امتحان احتمال مهندسی به صورت آنلاین (و با رعایت پروتکل های بهداشتی!) به مدت 2 ساعت برگزار می‌شود. فرهاد و آرش، هر یک مستقل از دیگری و به تصادف در 10 دقیقه‌ی اول (با توزیع یکنواخت) وارد جلسه امتحان می‌شوند. اگر این دو نفر مستقل از هم در بازه‌ی 1 تا $1.5$ ساعت از شروع و با توزیع یکنواخت، امتحان خود را به پایان رسانده و از جلسه خارج شوند،

الف) با چه احتمالی فرهاد زودتر از آرش از جلسه امتحان خارج می شود؟

ب) اگر آرش زودتر از فرهاد به جلسه آمده باشد، با چه احتمالی حداکثر 15 دقیقه دیرتر از از جلسه خارج می‌شود؟

پ) با چه احتمالی آرش حداکثر 10 دقیقه پس از فرهاد از جلسه خارج می شود؟

ت) احتمال آن که فرهاد زودتر از آرش به جلسه آمده و زودتر از او خارج شود چقدر است؟

ث) اگر آرش از جلسه امتحان خارج شده باشد، با چه احتمالی فرهاد حداکثر 20 دقیقه پس از او از جلسه خارج می‌شود؟

ج) اگر فرهاد دیرتر از آرش از جلسه خارج شود، با چه احتمالی زودتر از او وارد جلسه شده است؟

چ) احتمال آن که فرهاد زودتر از آرش به جلسه آمده ولی دیرتر از او خارج شود چقدر است؟

ح) اگر آرش از جلسه امتحان خارج شده باشد، با چه احتمالی فرهاد حداقل 20 دقیقه پیش از او از جلسه خارج شده است؟

خ) اگر آرش زودتر از فرهاد به جلسه آمده باشد، با چه احتمالی دیرتر از او از جلسه خارج می‌شود؟

د) احتمال آن که زمان مورد نیاز آرش برای حل سوالات، 10 دقیقه بیشتر از فرهاد باشد چقدر است؟