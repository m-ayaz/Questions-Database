\documentclass{article}

\usepackage{amsmath,amssymb,geometry}
\usepackage{xepersian}

\setlength{\parindent}{0pt}
\setlength{\parskip}{3mm}

\newcounter{questionnumber}
\setcounter{questionnumber}{1}

\newcommand{\Q}{
\textbf{سوال \thequestionnumber)}
\stepcounter{questionnumber}
}

\newcommand{\eqn}[1]{
\[\begin{split}
#1
\end{split}\]
}

\begin{document}
\LARGE
\begin{center}
\settextfont{IranNastaliq}

به نام زیبایی

%\begin{figure}[h]
%\centering
%\includegraphics[width=30mm]{kntu_logo.eps}
%\end{figure}

کوئیز 4 درس احتمال مهندسی

\end{center}
\hrulefill
\large

\Q

کشوری شامل دو استان 1 و 2 است. استان 1 ، شامل 60 مرد و 40 زن و استان 2 شامل 650 زن و 350 مرد است. در استان 1 ، 10 مرد و 10 زن چشم آبی و در استان 2 ، 30 مرد و 20 زن چشم آبی هستند. فردی را به تصادف از این کشور انتخاب می کنیم.

الف) اگر این فرد چشم آبی باشد، با چه احتمالی از استان 1 انتخاب شده است؟

ب) اگر این فرد زن باشد، با چه احتمالی از استان 2 انتخاب شده و چشم آبی نیست؟

پاسخ:

اصولأ در پرسشهای احتمالاتی، باید فضای نمونه و پیشامدها را در ابتدا به درستی تعریف کرد.  اینجا نیز چنین قاعده‌ای را پی می‌گیریم.

از آنجا که یک فرد خاص می‌تواند زن یا مرد باشد یا چشم آبی باشد یا نباشد، چهار پیشامد ممکن وجود دارد:

\eqn{
&
M=\text{
پیشامد مرد بودن
}
\\&
F=\text{
پیشامد زن بودن
}
\\&
B=\text{
پیشامد چشم آبی بودن
}
\\&
N=\text{
پیشامد چشم آبی نبودن
}
\\&
S_1=\text{
پیشامد اهل استان 1 بودن
}
\\&
S_2=\text{
پیشامد اهل استان 2 بودن
}
}

صورت سوال، اطلاعات احتمالاتی زیر را به ما می‌دهد:
\eqn{
&
P(S_1)=\frac{100}{1100}
\\&
P(S_2)=\frac{1000}{1100}
\\&
P(B|S_1)=\frac{20}{100}
\\&
P(B|S_2)=\frac{50}{1000}
\\&
P(M|S_1)=\frac{60}{100}
\\&
P(M|S_2)=\frac{350}{1000}
}

الف) احتمال مطلوب ما،
$
P(S_1|B)
$
است که به صورت زیر به دست می‌آید:
\eqn{
P(S_1|B)&=
\frac{P(S_1\cap B)}{P(B)}
\\&=
\underbrace{\frac{P(S_1)P(B|S_1)}{P(B)}}_{\text{قاعده‌ی بیز}}
\\&=
\frac{P(S_1)P(B|S_1)}{P(S_1)P(B|S_1)+P(S_2)P(B|S_2)}
\\&=
\frac{
\frac{100}{1100}\times \frac{20}{100}
}{
\frac{100}{1100}\times \frac{20}{100}+\frac{1000}{1100}\times \frac{50}{1000}
}
=\frac{2}{7}
}

ب) برای این بخش داریم:

\eqn{
P(S_2\cap N|F)&=
\frac{P(S_2\cap N\cap F)}{P(F)}
}
پیشامد $S_2\cap N\cap F$، پیشامد حالتی است که فرد انتخاب شده، زن بوده، از استان 2 انتخاب شود و چشم آبی نباشد. از آنجا که از جامعه‌ی 1100 نفری، 630 نفر چنین ویژگی‌ای دارند در نتیجه:
$$
P(S_2\cap N\cap F)=\frac{630}{1100}
$$
و می‌توان نوشت
\eqn{
P(S_2\cap N|F)&=
\frac{P(S_2\cap N\cap F)}{P(F)}
\\&=
\frac{P(S_2\cap N\cap F)}{P(S_1)P(F|S_1)+P(S_2)P(F|S_2)}
\\&=
\frac{P(S_2\cap N\cap F)}{
\frac{100}{1100}\times \frac{40}{100}+\frac{1000}{1100}\times \frac{650}{1000}
}
\\&=
\frac{\frac{630}{1100}}{
\frac{100}{1100}\times \frac{40}{100}+\frac{1000}{1100}\times \frac{650}{1000}
}
\\&=
\frac{21}{23}
}


\end{document}