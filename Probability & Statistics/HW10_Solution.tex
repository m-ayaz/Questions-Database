\documentclass[10pt,letterpaper]{article} 
\usepackage{toolsper}
%\usepackage{graphicx}‎‎
%\usefonttheme{serif}‎
%\usepackage{ptext}‎
%\usepackage{xepersian}
%\settextfont{B Nazanin}
\usepackage{lipsum}
\setlength{\parindent}{0pt}
\newcommand{\pf}{$\blacksquare$}
\newcommand{\pic}[2]{
\begin{center}
\includegraphics[width=#2]{#1}
\end{center}
}
\begin{document}
\Large
\begin{center}
به نام خدا

پاسخ تمرینات سری دهم درس آمار و احتمال
\hl
\end{center}
سوال 1) پیشامد های $A$ و $B$ را به صورت زیر در نظر می گیریم:
\qn{
&A=\text{\rl{
پیشامد رو آمدن دو خط در سه پرتاب اول (و شیر یا خط آمدن پرتاب دیگر)
}}
\\&
B=\text{\rl{
پیشامد رو آمدن سه خط در سه پرتاب اول
}}
}
الف) اگر متغیر تصادفی مورد نظر $X$ باشد، مطلوب است
\qn{
p(X=x|B)={p(X=x,B)\over p(B)}
}
می دانیم
$$
p(B)=\left({1\over 2}\right)^3
$$
هم چنین
\qn{
p(X=x,B)&=\Pr\{\text{\rl{
رو آمدن $x$ شیر در 10 پرتاب و سه خط در سه پرتاب اول
}}\}
\\&=\binom{7}{x}\left({1\over 2}\right)^{10}\quad , \quad 0\le x \le 7
}
بنابراین
\qn{
p(X=x|B)=\binom{7}{x}\left({1\over 2}\right)^{7}\quad , \quad 0\le x \le 7
}
ب) مشابه قسمت قبل، مطلوب است
\qn{
p(X=x|A\cup B)&={p\Big([X=x]\cap [A\cup B]\Big)\over p(A\cup B)}
\\&={p\Big(\left\{[X=x]\cap A\right\}\cup\left\{[X=x]\cap B\right\}\Big)\over p(A)+p(B)}
\\&={p\Big([X=x]\cap A\Big)+p\Big([X=x]\cap B\Big)\over p(A)+p(B)}
}
با توجه به بخش قبل
\qn{
p(X=x,B)=\binom{7}{x}\left({1\over 2}\right)^{10}\quad , \quad 0\le x \le 7
}
همچنین
$$
p(A)=\binom{3}{2}\left({1\over 2}\right)^{3}
$$
و
\qn{
p(X=x,A)&=\Pr\{\text{\rl{
 رو آمدن دو خط و یک شیر در سه پرتاب اول و رو آمدن $x-1$ شیر در 7 پرتاب بعدی
}}\}
\\&=\binom{3}{1}\binom{7}{x-1}\left({1\over 2}\right)^{10}
\quad,\quad 1\le x\le 8
}
بنابراین
\qn{
p(X=x|A\cup B)=
\begin{cases}
\left({1\over 2}\right)^{9}&,\quad x=0\\
\left\{\binom{7}{x}+
\binom{3}{1}\binom{7}{x-1}\right\}\cdot\left({1\over 2}\right)^{9}
&,\quad 1\le x\le 7\\
3\cdot\left({1\over 2}\right)^{9}&,\quad x=8
\end{cases}
}
\newline\newline
سوال 2) الف) می دانیم
\qn{
p(X=x,Y=y)=p(X=x)p(Y=y|X=x)
}
چون احتمالاتی از جنس $p(Y=y|X=x)$ را می توان از روی کانال استخراج کرد، در نتیجه خواهیم داشت
\qn{
&p(X=0,Y=0)=q(1-p)
\\&p(X=0,Y=1)=qp
\\&p(X=1,Y=0)=(1-q)p
\\&p(X=1,Y=1)=(1-q)(1-p)
}
ب) 
$$
p(X\ne Y)=p(X=0,Y=1)+p(X=1,Y=0)=p
$$
بنابراین احتمال خطا بر حسب توزیع ورودی، مقدار ثابتی است و این امر شهودا به علت تقارن خطای کانال است. به این کانال در تئوری اطلاعات، کانال باینری متقارن (\lr{Binary Symmetric Channel}) گفته می شود.
\newline\newline
سوال 3) به سادگی می توان به کمک انتگرال گیری جزء به جزء نتیجه گرفت
$$
E\{X\}=\int_0^\infty \lambda x e^{-\lambda x}dx={1\over \lambda}
$$
همچنین به ازای $x>a$
\qn{
\Pr\{X|X>a\}&=\Pr\{X<x|X>a\}
\\&={\Pr\{a<X<x\}\over \Pr\{X>a\}}
\\&={e^{-\lambda a}-e^{-\lambda x}\over e^{-\lambda a}}
\\&={1-e^{\lambda a-\lambda x}}
}
بنابراین
\qn{
p(X|X>a)=\lambda e^{-\lambda (x-a)}
}
(با ترسیم، )دیده می شود که این توزیع، انتقال یافته‌ی توزیع نمایی به اندازه‌ی $a$ به راست است؛ در نتیجه مقدار میانگین آن هم به همین اندازه افزایش خواهد داشت و اثبات کامل است (می توان مشابها از روی انتگرال جزء به جزء نیز به این نتیجه رسید).
\newline\newline
سوال 4) طبق تعریف
$$
\Pr\{X=0\}=F(0)-F(0^-)={1\over 2}
$$
$$
\Pr\{X=1\}=F(1)-F(1^-)={1\over 4}
$$
تابع توزیع داده شده در سوال، دارای دو ضربه در 0 و 1 و یک مقدار یکنواخت $1\over 4$ در بازه‌ی $(0,1)$ خواهد بود؛ در نتیجه با حذف این دو ضربه، توزیع شرطی، یکنواخت بین 0 و 1 خواهد بود.
\end{document}