\documentclass{article}

\usepackage{amsmath,amssymb,geometry,tikz}
\usepackage{xepersian}

\setlength{\parindent}{0pt}
\setlength{\parskip}{3mm}

\newcounter{questionnumber}
\setcounter{questionnumber}{1}

\newcommand{\Q}{
\textbf{سوال \thequestionnumber)}
\stepcounter{questionnumber}
}

\newcommand{\eqn}[1]{
\begin{equation}\begin{split}
#1
\end{split}\end{equation}
}

\begin{document}
\LARGE
\begin{center}
\settextfont{IranNastaliq}

به نام زیبایی

%\begin{figure}[h]
%\centering
%\includegraphics[width=30mm]{kntu_logo.eps}
%\end{figure}

پاسخ تمرینات سری دهم درس احتمال مهندسی

\end{center}
\hrulefill
\large

\Q

الف)
$
Y=e^X
$
به ازای $y>1$،
\eqn{
\Pr\{Y\le y\}&=\Pr\{e^X\le y\}
=\Pr\{X\le \ln y\}
=1-e^{-\ln y}=1-\frac{1}{y}
}
و به ازای $y\le1$،
\eqn{
\Pr\{Y\le y\}&=\Pr\{e^X\le y\}
=0
}
در این صورت
$$
f_Y(y)=\begin{cases}
\frac{1}{y^2}&,\quad y>1\\
0&,\quad y\le 1
\end{cases}
$$

ب)
به ازای $y>0$،
\eqn{
\Pr\{Y\le y\}&=\Pr\{X^\alpha \le y\}
=\Pr\{X\le \sqrt[\alpha]{y}\}
=1-e^{-\sqrt[\alpha]{y}}
}
و به ازای $y\le0$،
\eqn{
\Pr\{Y\le y\}=0
}
در این صورت
$$
f_Y(y)=\begin{cases}
\frac{1}{\alpha} y^{\frac{1}{\alpha}-1}e^{-\sqrt[\alpha]{y}}&,\quad y>0\\
0&,\quad y\le 0
\end{cases}
$$
البته به ازای $\alpha=0$ داریم
$
\Pr\{Y=1\}=1
$.

پ) به ازای $y>0$،
\eqn{
\Pr\{Y\le y\}&=\Pr\{\lfloor X\rfloor \le y\}
=\Pr\{\lfloor X\rfloor \le \lfloor y\rfloor\}
=\Pr\{X<\lfloor y\rfloor+1\}
=1-e^{-\lfloor y\rfloor-1}
}
و به ازای $y\le0$،
\eqn{
\Pr\{Y\le y\}=0
}
در نتیجه، توزیع تجمعی $Y$ دارای پرشهایی در نقاط صحیح نامنفی $Y$ بوده و چگالی احتمال آن عبارتست از
\eqn{
f_Y(y)=\sum_{k=0}^\infty e^{-k}(1-e^{-1})\delta(y-k)
}

\Q

الف)
\eqn{
&\int_{-\infty}^\infty f_X(x)dx=1\implies
\\&\int_0^1 kxdx+\int_{1^-}^{1^+} \frac{1}{2}\delta(x-1)dx=1\implies
\\&\frac{k}{2}+\frac{1}{2}=1\implies k=1
}

ب)
\eqn{
&\mathbb{E}\{X\}=\int_{-\infty}^\infty xf_X(x)dx
\\&=\int_0^1 x^2dx+\int_{1^-}^{1^+} \frac{1}{2}x\delta(x-1)dx
\\&=\frac{1}{3}+\frac{1}{2}=\frac{5}{6}
}

پ)
\eqn{
&\mathbb{E}\{e^{aX}\}=
\int_{-\infty}^\infty e^{ax}f_X(x)dx
\\&=\int_0^1 xe^{ax}dx+\int_{1^-}^{1^+} \frac{1}{2}e^{ax}\delta(x-1)dx
\\&=(x-\frac{1}{a})e^{ax}\Big|_0^1+\int_{1^-}^{1^+} \frac{1}{2}e^a\delta(x-1)dx
\\&=(\frac{3}{2}-\frac{1}{a})e^a+\frac{1}{a}
}

\Q

%متغیر تصادفی $X$ از توزیع زیر پیروی می‌کند:
%$$
%f_X(x)=\begin{cases}
%2xe^{-x^2}&,\quad x>0\\
%0&,\quad \text{جاهای دیگر}\\
%\end{cases}
%$$
%متغیر تصادفی
%$
%Y=X^2
%$
%مفروض است.

الف) به ازای $y>0$
\eqn{
\Pr\{Y\le y\}&=\Pr\{-\sqrt y\le X\le\sqrt y\}
=\Pr\{X\le\sqrt{y}\}
\\&=\int_0^{\sqrt y}2x\exp(-x^2)dx=1-\exp(-y)
}
در نتیجه
\eqn{
f_Y(y)=\exp(-y)\quad,\quad y>0
}

ب)
\eqn{
&\mathbb{E}\{X\}=\int_0^\infty 2x^2\exp(-x^2)dx=-x\exp(-x^2)\Big|_0^\infty+\int_0^\infty \exp(-x^2)dx
\\&=\int_0^\infty \exp(-x^2)dx
=\frac{1}{2}\int_{-\infty}^\infty \exp(-x^2)dx
=\frac{1}{2}\sqrt\pi\frac{1}{\sqrt\pi}\int_{-\infty}^\infty \exp(-x^2)dx
=\frac{\sqrt\pi}{2}
}

پ)
\eqn{
&\mathbb{E}\{X^2\}=\int_0^\infty 2x^3\exp(-x^2)dx
=\int_0^\infty u\exp(-u)du=1
\\&\mathbb{E}\{Y\}=\int_0^\infty y\exp(-y)dy=1\implies
\mathbb{E}\{X^2\}=\mathbb{E}\{Y\}
}
همانگونه که انتظار می‌رفت، $Y=X^2$ تساوی $\mathbb{E}\{X^2\}=\mathbb{E}\{Y\}$ را نتیجه می‌دهد.

ت)
\eqn{
&\Pr\{X<\frac{1}{2}\}=\int_0^{\frac{1}{2}}2x\exp(-x^2)dx
\\&=-\exp(-x^2)|_{0}^{\frac{1}{2}}=1-\exp(-\frac{1}{4})
}

\eqn{
&\Pr\{Y<\frac{1}{4}\}=\int_0^{\frac{1}{4}}\exp(-y)dx
\\&=-\exp(-y)|_{0}^{\frac{1}{4}}=1-\exp(-\frac{1}{4})
}

همانگونه که انتظار می‌رفت، $Y=X^2$ به نتیجه‌ی زیر منجر می‌شود:
$$
\{X\le\frac{1}{2}\}\equiv \{X^2\le\frac{1}{4}\}\equiv \{Y\le\frac{1}{4}\}.
$$

\end{document}