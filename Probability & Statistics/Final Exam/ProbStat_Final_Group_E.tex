\documentclass{article}
\usepackage{amsmath,graphicx,geometry,tcolorbox,xcolor,amssymb}
\usepackage{xepersian}

%\newcommand{\q}{\newpage\question}
\setlength{\parskip}{3mm}
\setlength{\parindent}{0mm}
\newcommand{\red}[1]{{\color{red}#1}}
\begin{document}
{
\large 
\centering
به نام او

امتحان پایان ترم درس احتمال مهندسی

مدت امتحان: 120 دقیقه

}

\hrulefill

\Large

%{\color{red}شسشس}

سوال 1) برای متغیر تصادفی $X$ با چگالی احتمال زیر، ابتدا تابع مولد گشتاور را یافته و سپس از روی آن، مقادیر میانگین و واریانس را بیابید.





$$
f_X(x)=\begin{cases}
\frac{3}{7}x^2&,\quad 1<x<2\\
0&,\quad \text{سایر جاها}
\end{cases}
$$




\newpage
سوال 2) برای متغیر های تصادفی $X$ و $Y$ با چگالی احتمال توام زیر،





$$
f_{X,Y}(x,y)=\begin{cases}
kx&,\quad x+y<1 \ \ ,\ \ x>0 \ \ , \ \ y>0\\
0&,\quad \text{سایر جاها}
\end{cases}
$$

الف) مقدار مناسب $k$ را بیابید.

ب) توزیع حاشیه ای 
$
f_X(x)
$
را پیدا کنید.

ت) مقدار 
$
\text{cov}(X,Y)
$
را محاسبه کنید.



\newpage
سوال 3) برای متغیر های تصادفی $X$ و $Y$ با چگالی احتمال توام زیر،

$$
f_{X,Y}(x,y)=\begin{cases}
x+ky&,\quad 0<x<1\ \ ,\ \ 0<y<1\\
0&,\quad \text{سایر جاها}
\end{cases}
$$

الف) مقدار مناسب k را بیابید.

ب) استقلال دو متغیر تصادفی را بررسی کنید.

پ) چگالی احتمال $X+Y$ را بیابید.

\newpage
سوال 4) اگر $X$ یک متغیر تصادفی با چگالی احتمال زیر باشد،

$$
f_X(x)=\begin{cases}
{1\over 2}\sin x&,\quad 0<x<\pi\\
0&,\quad \text{سایر جاها}
\end{cases}
$$

مقدار 
$
\text{var}\{X|X>{\pi\over 2}\}
$
را به دست آورید.

\newpage
سوال 5) سکه سالمی را 5 بار می اندازیم. اگر $X$ تعداد پشت آمدن ها در تمام پرتاب ها و $Y$ تعداد رو آمدن ها در دو پرتاب اول باشد،



الف) تابع مولد گشتاور مشترک $X$ و $Y$ را محاسبه کنید.

ب) مقدار 
$
\mathbb{E}\{X+Y|X=1\}
$
را به دست آورید.


\newpage
سوال 6) یک امتحان احتمال مهندسی به صورت آنلاین (و با رعایت پروتکل ها!) به مدت 2 ساعت برگزار می‌شود. فرهاد و آرش، هر یک مستقل از دیگری و به تصادف در 10 دقیقه‌ی اول (با توزیع یکنواخت) وارد جلسه امتحان می‌شوند. اگر این دو نفر مستقل از هم در بازه‌ی 1 تا $1.5$ ساعت از شروع و با توزیع یکنواخت، امتحان خود را به پایان رسانده و از جلسه خارج شوند،


الف) اگر آرش زودتر از فرهاد به جلسه آمده باشد، با چه احتمالی دیرتر از او از جلسه خارج می‌شود؟

ب) احتمال آن که زمان مورد نیاز آرش برای حل سوالات، 10 دقیقه بیشتر از فرهاد باشد چقدر است؟


(یادآوری: اگر متغیر تصادفی $X$ دارای توزیع یکنواخت در بازه $(a,b)$ باشد آنگاه:
$$
f_X(x)=\begin{cases}
{1\over b-a}&,\quad a<x<b\\
0&,\quad \text{سایر جاها}
\end{cases}
$$
)







%سوال 6) متغیرهای تصادفی $X$ و $Y$، متغیرهای نرمال مستقل، هر یک با میانگین صفر و واریانس 1 هستند. مقدار $a$ را به گونه ای بیابید که 
%\[
%\begin{split}
%&X+aY,X-Y
%\\&aX+Y,X-Y
%\\&X+aY,X+2Y
%\\&X+aY,2aX-Y
%\\&aX+Y,aX-Y
%\end{split}
%\]
%مستقل شوند.


\vspace{3cm}

\hspace{3cm}{
موفق باشید!
}







\end{document}