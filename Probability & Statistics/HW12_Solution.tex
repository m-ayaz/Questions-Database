\documentclass[10pt,letterpaper]{article}
%\usepackage{toolsper}
\usepackage{amsmath,geometry,amssymb,xepersian}
\newcommand{\eqn}[1]{
\begin{equation}
\begin{split}
#1
%\label{#2}
\end{split}
\end{equation}
}
%%%%%%%%%%%%%

%       \eqn{
%       x=x^2
%       }{label}

%%%%%%%%%%%%%
\newcommand{\feqn}[2]{
\begin{tcolorbox}[width=7in, colback=white]
\begin{equation}
\begin{split}
#1
\label{#2}
\end{split}
\end{equation}
\end{tcolorbox}
}
%%%%%%%%%%%%%%%
\newcommand{\hl}{
\begin{center}
\line(1,0){450}
\end{center}}
%%%%%%%%%%%%%%%
\newcommand{\qn}[1]{
\[
\begin{split}
#1
\end{split}
\]
}
%\settextfont{B Nazanin}
\usepackage{lipsum}
\setlength{\parindent}{0mm}
\setlength{\parskip}{3mm}
\newcommand{\pic}[2]{
\begin{center}
\includegraphics[width=#2]{#1}
\end{center}
}
\begin{document}
\Large
\begin{center}
به نام او

تمرینات سری دوازدهم درس احتمال مهندسی

\hrulefill
\end{center}

سوال 1)
\eqn{
F_X(x|X<1)&=\Pr\{X\le x|X<1\}
\\&={\Pr\{X\le x,x<1\}\over \Pr\{X<1\}}
\\&=\begin{cases}
{\Pr\{X\le x\}\over \Pr\{X<1\}}&,\quad 0<x<1\\
{\Pr\{X<1\}\over \Pr\{X<1\}}&,\quad x\ge1\\
\end{cases}
\\&=\begin{cases}
{{x\over 2}\over {1\over 2}}&,\quad 0<x<1\\
1&,\quad x\ge1\\
\end{cases}
\\&=\begin{cases}
x&,\quad 0<x<1\\
1&,\quad x\ge1\\
\end{cases}
}
همچنین
\eqn{
F_X(x|X>1)&=\Pr\{X\le x|X>1\}
\\&={\Pr\{X\le x,x>1\}\over \Pr\{X>1\}}
\\&=\begin{cases}
0&,\quad 0<x<1\\
{\Pr\{1<X< x\}\over \Pr\{X>1\}}&,\quad x\ge1\\
\end{cases}
\\&=\begin{cases}
0&,\quad 0<x<1\\
{{x-1\over 2}\over {1\over 2}}&,\quad x\ge1\\
\end{cases}
\\&=\begin{cases}
0&,\quad 0<x<1\\
x-1&,\quad 1<x<2\\
\end{cases}
}
بنابراین
\eqn{
f_X(x|X>1)&={d\over dx}F_X(x|X>1)
\\&=
\begin{cases}
1&,\quad 1<x<2\\
0&,\quad\text{سایر جاها}
\end{cases}
}

بسادگی از محاسبات (و همچنین شهود) می توان نتیجه گرفت که چون $X$ دارای توزیع یکنواخت بین 0 و 2 است، 
$
X|0.5<X<1.5
$
دارای توزیع یکنواخت بین 
$
0.5
$
و
$
1.5
$
خواهد بود. بنابراین واضح است که مقدار متوسط این متغیر تصادفی برابر 1 خواهد بود.

سوال 2) توزیع $X|X>a$ عبارتست از
\eqn{
F(x|X>a)&=\Pr\{X\le x|X>a\}
\\&={\Pr\{X\le x,X>a\}\over \Pr\{X>a\}}
\\&=
\begin{cases}
{\Pr\{X\le x,X>a\}\over \Pr\{X>a\}} &,\quad x>a\\
0&,\quad x<a
\end{cases}
\\&=
\begin{cases}
{e^{-\lambda a}-e^{-\lambda x}\over e^{-\lambda a}} &,\quad x>a\\
0&,\quad x<a
\end{cases}
\\&=
\begin{cases}
{1-e^{-\lambda (x-a)}} &,\quad x>a\\
0&,\quad x<a
\end{cases}
}
بنابراین
\eqn{
f(x|X>a)=
\begin{cases}
{\lambda e^{-\lambda (x-a)}} &,\quad x>a\\
0&,\quad x<a
\end{cases}
}
بنابراین
\eqn{
E\{X|X>a\}&=\int xf(X|X>a)dx
\\&=\int_a^\infty {\lambda xe^{-\lambda (x-a)}}dx
\\&=\int_a^\infty {\lambda (x-a+a)e^{-\lambda (x-a)}}dx
\\&=\int_0^\infty {\lambda (u+a)e^{-\lambda u}}du
\\&=\int_0^\infty {(\lambda u+\lambda a)e^{-\lambda u}}du
\\&=\int_0^\infty {\lambda ue^{-\lambda u}}du
+\int_0^\infty \lambda ae^{-\lambda u}du
\\&={1\over\lambda}
+a
}

همچنین
\eqn{
E\{X\}&=\int xf(x)dx
\\&=\int_0^\infty {\lambda xe^{-\lambda x}}dx
\\&={1\over\lambda}
}

این نشان می دهد
$$
E\{X|X>a\}=E\{X\}+a
$$
که همان خاصیت بی حافظه بودن متغیرهای نمایی است؛ به این معنا که تفاوتی نمیکند این متغیر از چه لحظه ای به بعد مشاهده شود. از هر لحظه ای به بعد، معادل با مشاهده آن در لحظه صفر خواهد بود.

سوال 3) الف) ابتدا توزیع 
$
X|X\ge 4
$
را می یابیم. بدین منظور:
\eqn{
p(x|X\ge4)&=\Pr\{X= x|X\ge 4\}
\\&={\Pr\{X= x,X\ge 4\}\over \Pr\{X\ge 4\}}
\\&=
\begin{cases}
{\Pr\{X= x\}\over \Pr\{X\ge 4\}}&,\quad x\ge 4\\
0&,\quad x< 4
\end{cases}
\\&=
\begin{cases}
{\Pr\{X=x\}\over (1-p)^4}&,\quad x\ge 4\\
0&,\quad x< 4
\end{cases}
\\&=
\begin{cases}
{(1-p)^xp\over (1-p)^4}&,\quad x\ge 4\\
0&,\quad x< 4
\end{cases}
\\&=
\begin{cases}
{(1-p)^{x-4}p}&,\quad x\ge 4\\
0&,\quad x<4
\end{cases}
}
بنابراین داریم

\eqn{
E\{X|X\ge 4\}&=\sum_{x=4}^\infty x(1-p)^{x-4}p
\\&=\sum_{u+4=4}^\infty (u+4)(1-p)^up
\\&=4+\sum_{u=0}^\infty u(1-p)^up
\\&={1\over p}+3
}

و

\eqn{
E\{X^2|X\ge 4\}&=\sum_{x=4}^\infty x^2(1-p)^{x-4}p
\\&=\sum_{u+4=4}^\infty (u+4)^2(1-p)^up
\\&=\sum_{u=0}^\infty (u^2+8u+16)(1-p)^up
\\&=16+8({1\over p}-1)+{(1-p)(2-p)\over p^2}
\\&={16p^2+8p(1-p)+(1-p)(2-p)\over p^2}
\\&={16p^2+8p-8p^2+p^2-3p+2\over p^2}
\\&={9p^2+5p+2\over p^2}
}
و می توان نوشت

\eqn{
\text{var}(X|X\ge 4)&=E\{X^2|X\ge 4\}-E^2\{X|X\ge 4\}
\\&={9p^2+5p+2\over p^2}-{9p^2+6p+1\over p^2}\\&={1-p\over p^2}
}

ب)
\eqn{
\Pr\{X=x|\text{X زوج است}\}&=
{\Pr\{X=x , X=0,2,4,\cdots\}\over \Pr\{X=0,2,4,\cdots\}}
\\&
=\begin{cases}
{\Pr\{X=x\}\over \Pr\{X=0,2,4,\cdots\}}&,\quad \text{x زوج}\\
0&,\quad \text{x فرد}
\end{cases}
\\&
=\begin{cases}
{(1-p)^xp\over \sum_{u=0}^\infty(1-p)^{2u}p}&,\quad \text{x زوج}\\
0&,\quad \text{x فرد}
\end{cases}
\\&
=\begin{cases}
{(1-p)^x\over {1\over 1-(1-p)^2}}&,\quad \text{x زوج}\\
0&,\quad \text{x فرد}
\end{cases}
\\&
=\begin{cases}
{(1-p)^x-(1-p)^{x+2}}&,\quad \text{x زوج}\\
0&,\quad \text{x فرد}
\end{cases}
\\&
=\begin{cases}
{p(2-p)(1-p)^x}&,\quad \text{x زوج}\\
0&,\quad \text{x فرد}
\end{cases}
}

سوال 4) الف)
\eqn{
\Pr\{Y\ge 3\}&=\int_3^\infty f_Y(y)dy
\\&=\int_3^\infty \sum_{x=1}^6 f_{X,Y}(x,y)dy
\\&=\int_3^\infty \sum_{x=1}^6 f_x(x)f_{Y|X}(x,y)dy
\\&=\int_3^\infty \sum_{x=1}^6 {1\over 6}f_{Y|X}(x,y)dy
\\&={1\over 6}\int_3^\infty f_{Y|X}(1,y)dy
\\&+{1\over 6}\int_3^\infty f_{Y|X}(2,y)dy
\\&+{1\over 6}\int_3^\infty f_{Y|X}(3,y)dy
\\&+{1\over 6}\int_3^\infty f_{Y|X}(4,y)dy
\\&+{1\over 6}\int_3^\infty f_{Y|X}(5,y)dy
\\&+{1\over 6}\int_3^\infty f_{Y|X}(6,y)dy
\\&={1\over 6}({1\over 4}+{2\over 5}+{3\over 6})
\\&={23\over 120}
}

ب)
\eqn{
f_Y(y)&=\sum_x f_{X,Y}(x,y)
\\&=\sum_x f_x(x)f_{Y|X}(x,y)
\\&={1\over 6}\sum_{x=1}^6f_{Y|X}(x,y)
\\&={1\over 6}\times\begin{cases}
1+{1\over 2}+{1\over 3}+{1\over 4}+{1\over 5}+{1\over 6}&,\quad 0<y<1\\
{1\over 2}+{1\over 3}+{1\over 4}+{1\over 5}+{1\over 6}&,\quad 1<y<2\\
{1\over 3}+{1\over 4}+{1\over 5}+{1\over 6}&,\quad 2<y<3\\
{1\over 4}+{1\over 5}+{1\over 6}&,\quad 3<y<4\\
{1\over 5}+{1\over 6}&,\quad 4<y<5\\
{1\over 6}&,\quad 5<y<6\\
\end{cases}
\\&=\begin{cases}
{49\over 120}&,\quad 0<y<1\\
{29\over 120}&,\quad 1<y<2\\
{19\over 120}&,\quad 2<y<3\\
{37\over 360}&,\quad 3<y<4\\
{11\over 180}&,\quad 4<y<5\\
{1\over 36}&,\quad 5<y<6\\
\end{cases}
}

بنابراین
\eqn{
\mathbb{E}\{Y\}&=\int_0^6 yf_Y(y)dy
\\&=\cdots(!!!)
\\&=1.75
}

\eqn{
\mathbb{E}\{Y^2\}&=\int_0^6 y^2f_Y(y)dy
\\&=\cdots(!!!)
\\&\approx5.06
}
بنابراین
$$
\text{var}(Y)\approx2
$$
\end{document}