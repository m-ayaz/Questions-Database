\documentclass{article}

\usepackage{amsmath,amssymb,geometry,tikz}
\usepackage{xepersian}

\setlength{\parindent}{0pt}
\setlength{\parskip}{3mm}

\newcounter{questionnumber}
\setcounter{questionnumber}{1}

\newcommand{\Q}{
\textbf{سوال \thequestionnumber)}
\stepcounter{questionnumber}
}

\newcommand{\eqn}[1]{
\begin{equation}\begin{split}
#1
\end{split}\end{equation}
}

\begin{document}
\LARGE
\begin{center}
\settextfont{IranNastaliq}

به نام زیبایی

%\begin{figure}[h]
%\centering
%\includegraphics[width=30mm]{kntu_logo.eps}
%\end{figure}

تمرینات سری ششم درس احتمال مهندسی

\end{center}
\hrulefill
\large

\Q

از آنجا که هر رخداد سکه (پشت یا رو) معادل با پرتاب یک یا دو تاس است، پیشامد های زیر را تعریف می‌کنیم:
\eqn{
&
A=\text{
پیشامد رو آمدن سکه (پرتاب 1 تاس)
}
\\&
B=\text{
پیشامد پشت آمدن سکه (پرتاب 2 تاس)
}
\\&
C_n=\text{
پیشامد رو آمدن عدد $n$
}
}{}
می‌دانیم
$$
P(A)=P(B)=0.5.
$$
از طرفی
\eqn{
&
P(C_n|A)=\begin{cases}
\frac{1}{6}&,\quad n=1,2,\cdots,6\\
0&,\quad n=7,8,\cdots,12
\end{cases}
}{}
و
\eqn{
&
P(C_n|B)=\begin{cases}
0&,\quad n=1\\
\frac{6-|n-7|}{36}&,\quad n=2,\cdots,12
\end{cases}
}{}
مطلوبست $P(C_n)$. در نتیجه طبق قاعده‌ی احتمال کل
\eqn{
P(C_n)&=P(A)P(C_n|A)+P(B)P(C_n|B)
\\&=\frac{1}{2}P(C_n|A)+\frac{1}{2}P(C_n|B)
\\&=
\begin{cases}
\frac{1}{12}&,\quad n=1,2,\cdots,6\\
0&,\quad n=7,8,\cdots,12
\end{cases}
\\&+
\begin{cases}
0&,\quad n=1\\
\frac{6-|n-7|}{72}&,\quad n=2,\cdots,12
\end{cases}
}{}

با ساده سازی خواهیم داشت
$$
P(C_n)=
\begin{cases}
\frac{1}{12}&,\quad n=1\\
\frac{5+n}{72}&,\quad n=2,\cdots,6\\
\frac{13-n}{72}&,\quad n=7,\cdots,12
\end{cases}
$$
\Q

راه اول)

پیشامدهای زیر را تعریف می‌کنیم:
\eqn{
&
A_n=\text{
پیشامد سیاه بودن $n$ توپ از 3 توپ بیرون آمده در بار اول
}
\\&
B=\text{
پیشامد سفید بودن توپ انتخابی از 3 توپ بیرون آمده در بار اول
}
}{}
در نتیجه
\eqn{
P(A_n)=\frac{\binom{10}{3-n}\binom{7}{n}}{\binom{17}{3}}
}{}
همچنین
\eqn{
P(B|A_n)=\frac{3-n}{3}
}{}

مطلوبست
$
P(B|A_1\cup A_2\cup A_3)
$.
در نتیجه
\eqn{
&
P(B|A_1\cup A_2\cup A_3)
=
\frac{
P(B\cap[A_1\cup A_2\cup A_3])
}{
P(A_1\cup A_2\cup A_3)
}
\\&=
\frac{
P(B\cap A_1)+P(B\cap A_2)+P(B\cap A_3)
}{
P(A_1)+P(A_2)+P(A_3)
}
\\&=
\frac{
P(B|A_1)P(A_1)+P(B|A_2)P(A_2)+P(B|A_3)P(A_3)
}{
1-P(A_0)
}
\\&=
\frac{
\frac{2}{3}\times \frac{\binom{10}{2}\binom{7}{1}}{\binom{17}{3}}+
\frac{1}{3}\times \frac{\binom{10}{1}\binom{7}{2}}{\binom{17}{3}}+0\times P(A_3)
}{
1-\frac{\binom{10}{3}}{\binom{17}{3}}
}
\\&=\frac{1}{2}
}{}

راه دوم)

پیشامدهای زیر را تعریف می‌کنیم:
\eqn{
&
A=\text{
پیشامد سیاه بودن دست کم یک توپ از 3 توپ بیرون آمده در بار اول
}
\\&
B=\text{
پیشامد سفید بودن توپ انتخابی از 3 توپ بیرون آمده در بار اول
}
}{}

در نتیجه
\eqn{
&P(A)=1-P(A')=P(\text{
پیشامد سیاه بودن هیچ توپ از 3 توپ بیرون آمده در بار اول
})
\\&=
P(\text{
پیشامد سفید بودن هر سه توپ بیرون آمده در بار اول
})
\\&=
1-\frac{\binom{10}{3}}{\binom{17}{3}}
\\&=
1-\frac{10\times 9\times 8}{17\times 16\times 15}
\\&=
\frac{14}{17}
}{}
مطلوبست
$
P(B|A)
$
. در نتیجه
\eqn{
P(B|A)&=\frac{P(B\cap A)}{P(A)}
\\&=\frac{P(B\cap A)+P(B\cap A')-P(B\cap A')}{P(A)}
\\&=\frac{P(B)-P(B|A')P(A')}{P(A)}
\\&=\frac{P(B)-P(A')}{P(A)}
}{}
از طرفی، پیشامد $B$ معادل با این است که توپی از کیسه برداریم و سفید باشد (به عبارت دیگر، از 3 توپ برداشته شده‌ی مرحله‌ی قبل، به طور مثال با بستن چشم بی خبر باشیم). در نتیجه
$$
P(B)=\frac{10}{17}
$$
و با جایگذاری خواهیم داشت
\eqn{
P(B|A)=\frac{P(B)-P(A')}{P(A)}
=\frac{\frac{10}{17}-\frac{3}{17}}{\frac{14}{17}}=\frac{1}{2}.
}{}















\Q

پیشامدهای زیر را تعریف می‌کنیم:
\eqn{
&
A=\text{
پیشامد انتخاب کیسه‌ی 1
}
\\&
B=\text{
پیشامد انتخاب کیسه‌ی 2
}
\\&
C=\text{
پیشامد سفید نبودن توپ انتخابی
}
}{}
در نتیجه خواهیم داشت
\eqn{
P(A)=P(B)=0.5
}{}
و
\eqn{
P(C|A)=\frac{7}{17}\quad,\quad P(C|B)=\frac{7}{9}
}{}
مطلوب است 
$P(B|C)$
. در نتیجه طبق قاعده‌ی احتمال بیز
\eqn{
P(B|C)&=\frac{P(C|B)P(B)}{P(C|A)P(A)+P(C|B)P(B)}
\\&=
\frac{
\frac{7}{9}\times \frac{1}{2}
}{
\frac{7}{9}\times \frac{1}{2}
+
\frac{7}{17}\times \frac{1}{2}
}=\frac{17}{26}
}{}
\end{document}