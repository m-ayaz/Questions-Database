\documentclass[10pt,letterpaper]{article} 
\usepackage{toolsper}
%\usepackage{graphicx}‎‎
%\usefonttheme{serif}‎
%\usepackage{ptext}‎
\usepackage{xepersian}
\settextfont{B Nazanin}
\usepackage{lipsum}
\setlength{\parindent}{0pt}
\newcommand{\pf}{$\blacksquare$}
\newcommand{\pic}[2]{
\begin{center}
\includegraphics[width=#2]{#1}
\end{center}
}
\begin{document}
\Large
\begin{center}
به نام خدا

پاسخ تمرینات سری چهارم درس آمار و احتمال
\hl
\end{center}
سوال 1)

الف) از آنجا که برای $n<a$ و $n>b$ به ترتیب داریم
$
F(n)=0
$
 و 
$
F(n)=1
$
 در نتیجه می توان نوشت:
$$
f(n)=\begin{cases}
{1\over b-a+1}&,\quad a\le n\le b
\\0&,\quad\text{\rl{
در غیر این صورت
}} 
\end{cases}
$$
بنابراین
\qn{
\sum_{n=-\infty}^\infty nf(n)&=\sum_{n=a}^b {n\over b-a+1}
\\&=
\sum_{n=1}^b {n\over b-a+1}-
\sum_{n=1}^{a-1} {n\over b-a+1}
\\&={b^2+b-a^2+a\over 2(b-a+1)}
\\&={(b+a)(b-a+1)\over 2(b-a+1)}
\\&={b+a\over 2}
}{}
ب) 
\qn{
f(n)=A^n-A^{n+1}=A^n(1-A)\quad,\quad n\ge 0
}{}
در نتیجه 
\qn{
\sum_{n=-\infty}^\infty nf(n)&=
\sum_{n=0}^\infty nA^n(1-A)
\\&=
\sum_{n=1}^\infty nA^n(1-A)
\\&=
A(1-A)\sum_{n=1}^\infty nA^{n-1}
\\&=
A(1-A){d\over dA}\sum_{n=1}^\infty A^{n}
\\&=
A(1-A){d\over dA}{A\over 1-A}
\\&=
{A\over 1-A}
}{}
سوال 2)

الف)
\qn{
f(x)=
{1\over \lambda}e^{-{1\over \lambda}x}\quad,\quad x>0
}{}
در نتیجه
\qn{
\int_{-\infty}^\infty xf(x)dx&=\int_0^\infty {x\over \lambda}e^{-{1\over \lambda}x}dx
\\&=
-{x}e^{-{1\over \lambda}x}\Big|_{0}^{\infty}+
\int_0^\infty e^{-{1\over \lambda}x}dx
\\&=
\int_0^\infty e^{-{1\over \lambda}x}dx
\\&=
-\lambda e^{-{1\over \lambda}x}\Big|_0^\infty
\\&=\lambda
}{}
ب)
\[
f(x)={1\over b-a}\quad,\quad a<x<b
\]
بنابراین
\qn{
\int_{-\infty}^\infty xf(x)dx&=\int_a^b {x\over b-a}dx
\\&={b^2-a^2\over 2(b-a)}
\\&={b+a\over 2}
}{}
پ)
\qn{
f(x)&={1\over \sqrt{2\pi\sigma^2}}\exp\left({(t-\mu)^2\over 2\sigma^2}\right)\Big|_{t=x}-
{1\over \sqrt{2\pi\sigma^2}}\exp\left({(t-\mu)^2\over 2\sigma^2}\right)\Big|_{t=-\infty}
\\&={1\over \sqrt{2\pi\sigma^2}}\exp\left({(x-\mu)^2\over 2\sigma^2}\right)
}{}
بنابراین
\qn{
\int_{-\infty}^\infty xf(x)dx&=
{1\over \sqrt{2\pi\sigma^2}}
\int_{-\infty}^\infty
x\cdot\exp\left({(x-\mu)^2\over 2\sigma^2}\right)
dx
\\&=
{1\over \sqrt{2\pi\sigma^2}}
\int_{-\infty}^\infty
(x-\mu+\mu)\cdot\exp\left({(x-\mu)^2\over 2\sigma^2}\right)
dx
\\&=
\underbrace{{1\over \sqrt{2\pi\sigma^2}}
\int_{-\infty}^\infty
(x-\mu)\cdot\exp\left({(x-\mu)^2\over 2\sigma^2}\right)
dx}_{\triangleq I_1}
\\&
+\underbrace{{1\over \sqrt{2\pi\sigma^2}}
\int_{-\infty}^\infty
\mu\cdot\exp\left({(x-\mu)^2\over 2\sigma^2}\right)
dx}_{\triangleq I_2}
}{}
انتگرال $I_1$ برابر است با $
{1\over \sqrt{2\pi\sigma^2}}
\int_{-\infty}^\infty
w\cdot\exp\left({w^2\over 2\sigma^2}\right)
dw
$
 که همگرا و برابر صفر است؛ زیرا این انتگرال، انتگرال یک تابع فرد را روی بازه‌ی متقارنی نشان می دهد به علاوه جمله ی نمایی میرا شونده باعث کاهش سریع تابع تحت انتگرال می گردد. برای انتگرال $I_2$ نیز با توجه به تابع $F(x)$ تعریف شده در صورت سوال می توان نوشت:
\qn{
I_2=\mu{1\over \sqrt{2\pi\sigma^2}}
\int_{-\infty}^\infty
\exp\left({(x-\mu)^2\over 2\sigma^2}\right)
dx
=\mu F(\infty)=\mu
}{}
در مجموع خواهیم داشت:
\qn{
\int_{-\infty}^\infty xf(x)dx=\mu
}{}
سوال 3) 

متغیر تصادفی برنولی، یک متغیر تصادفی دو مقداره است که تنها مقادیر صفر و یک را می پذیرد. به طور مثال برای متغیر تصادفی برنولی $X$ داریم
$$
\Pr\{X=1\}=p
$$
از آنجا که عملگر این قسمت یک \text{\lr{xor}} (یا جمع به پیمانه‌ی 2) است، متغیر تصادفی $Z$ نیز دو مقداره و دارای توزیع برنولی خواهد بود.
\qn{
\Pr\{Z=1\}&=\Pr\{X\oplus Y\mod 2=1\}
\\&=\Pr\{X\text{\lr{xor}} Y=1\}
\\&=\Pr\{X=1,Y=0\text{\rl{ یا }}X=0,Y=1\}
\\&=\Pr\{X=1,Y=0\}+\Pr\{X=1,Y=0\}
\\&=\Pr\{X=1\}\Pr\{Y=0\}+\Pr\{X=0\}\Pr\{Y=1\}
\\&={1\over 2}\cdot (1-p)+{1\over 2}\cdot p
\\&={1\over 2}
}{}
سوال 4)

 با توضیحاتی مشابه سوال قبل، به سادگی دیده می شود که متغیر تصادفی $Z$ دو مقداره و دارای توزیع برنولی خواهد بود.
\qn{
\Pr\{Z=1\}&=\Pr\{XY=1\}
\\&=\Pr\{X=1,Y=1\}
\\&=\Pr\{X=1\}\Pr\{Y=1\}
\\&={1\over 2}p
}{}
سوال 5)

متغیر تصادفی $X$ حاصل $n_1$ کوشش برنولی مستقل با پارامتر احتمالاتی $p$ و متغیر تصادفی $Y$ حاصل $n_2$ کوشش برنولی مستقل با پارامتر احتمالاتی $p$ است؛ در نتیجه می توان متغیر های تصادفی $X$ و $Y$ را به مانند تعداد شیرهای رو آمده در پرتاب به ترتیب $n_1$ و $n_2$ بار سکه‌ی ناسالم تعبیر نمود. با این تعبیر، متغیر تصادفی $X+Y$ تعداد شیرها را در $n_1+n_2$ بار پرتاب همان سکه‌ی ناسالم نشان می دهد؛ به عبارت دیگر فرقی نمی کند که ابتدا سکه را $n_1$ بار پرتاب کرده و تعداد شیر ها را یادداشت کنیم، سپس با تعداد شیرها در $n_2$ بار پرتاب جمع بزنیم یا از ابتدا سکه را $n_1+n_2$ بار پرتاب کنیم و سپس تعداد شیرها را یادداشت کنیم. این عدم تفاوت، به دلیل مستقل بودن پرتاب هاست و در صورت مستقل نبودن پرتاب ها، استدلال های فوق معتبر نخواهند بود.
\end{document}