\documentclass{article}

\usepackage{amsmath,amssymb,geometry}
\usepackage{xepersian}

\setlength{\parindent}{0pt}
\setlength{\parskip}{3mm}

\newcounter{questionnumber}
\setcounter{questionnumber}{1}

\newcommand{\Q}{
\textbf{سوال \thequestionnumber)}
\stepcounter{questionnumber}
}

\newcommand{\eqn}[1]{
\[\begin{split}
#1
\end{split}\]
}

\begin{document}
\LARGE
\begin{center}
\settextfont{IranNastaliq}

به نام زیبایی

%\begin{figure}[h]
%\centering
%\includegraphics[width=30mm]{kntu_logo.eps}
%\end{figure}

پاسخ کوئیز 6 درس احتمال مهندسی

\end{center}
\hrulefill
\large


مکمل پیشامد مطلوب آن است که خانواده، دارای هیچ فرزند پسر یا هیچ فرزند دختری نباشد. احتمال پیشامد اخیر برابر است با
$$
(\frac{1}{2})^n+(\frac{1}{2})^n=(\frac{1}{2})^{n-1}؛
$$
در نتیجه، احتمال مطلوب برابر است با
$$
1-(\frac{1}{2})^{n-1}
$$
که باید بیشتر از یا مساوی با $0.95$ باشد. در نتیجه
\eqn{
&1-(\frac{1}{2})^{n-1}\ge 0.95
\\&\implies
(\frac{1}{2})^{n-1}\le 0.05
\\&\implies
2^{n-1}\ge 20
\\&\implies
n-1\ge \log_220
\\&\implies
n\ge 6
}{}

گزینه (ج)








\end{document}