\chapter{دنباله‌ی متغیرهای تصادفی}

\Q
(قدم زدن تصادفی) فردی از نقطه‌ی صفر روی محور اعداد حقیقی با احتمال $p$ یک متر به سمت راست و با احتمال $1-p$ یک متر به سمت چپ می  رود. اگر این فرد این نوع قدم زدن را $n$ بار و هربار از روی نقطه ای که روی آن ایستاده تکرار کند، با چه احتمالی پس از $k$ بار قدم زدن به مبدا باز می گردد؟

\Q
فرض کنید دنباله‌ی متغیرهای تصادفی 
$\{X_n\}$،
از توزیع یکنواخت بین 
$-\frac{1}{2}$
و
$\frac{1}{2}$
به طور مستقل پیروی می‌کند. به کمک قضیه‌ی حد مرکزی، توزیع متغیر تصادفی $Y$ و میانگین و واریانس آن را به دست آورید؛ اگر 
$$Y=\lim_{n\to \infty} {X_1+X_2+\cdots +X_n\over \sqrt n}.$$

\Q
الف) تابع مولد گشتاور متغیر تصادفی پواسون با پارامتر $\lambda$  را به دست آورید.

ب) اگر دنباله‌ی متغیرهای تصادفی مستقل $\{X_n\}$، از نوع پواسون با پارامتر $\lambda$ باشد، نشان دهید متغیر تصادفی 
$$
Y=\sum_{i=1}^NX_i
,
$$
دارای توزیع پواسون با پارامتر $N\lambda$ است.

پ) نشان دهید اگر دنباله‌ی متغیرهای تصادفی مستقل $\{X_n\}$، برنولی با پارامتر $p$ و $N$ از نوع پواسون با پارامتر $\lambda$ باشد، متغیر تصادفی
$Y=\sum_{n=0}^{N-1} X_n$
 دارای توزیع پواسون با پارامتر $\lambda p$ است.



\Q
تحقیق کنید هر یک از دنباله‌ی متغیرهای تصادفی زیر، با چه مفهومی به یک متغیر تصادفی میل می‌کنند. برای هر یک دلیل بیاورید.

الف) 
$
X_n=X+{1\over n}
$
 که $X$، یک متغیر تصادفی یکنواخت در بازه‌ی $[0,1]$ است.

ب) متغیر تصادفی نمایی با پارامتر ${n+1\over n}$

پ) متوسط تعداد شیرها در $n$ بار پرتاب مستقل یک سکه‌ی سالم