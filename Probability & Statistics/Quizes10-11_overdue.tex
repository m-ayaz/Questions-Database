\documentclass{article}

\usepackage{amsmath,amssymb,geometry}
\usepackage{xepersian}

\setlength{\parindent}{0pt}
\setlength{\parskip}{3mm}

\newcounter{questionnumber}
\setcounter{questionnumber}{1}

\newcommand{\Q}{
\textbf{سوال \thequestionnumber)}
\stepcounter{questionnumber}
}

\newcommand{\eqn}[1]{
\[\begin{split}
#1
\end{split}\]
}

\begin{document}
\LARGE
\begin{center}
\settextfont{IranNastaliq}

به نام زیبایی

%\begin{figure}[h]
%\centering
%\includegraphics[width=30mm]{kntu_logo.eps}
%\end{figure}

کوئیزهای 10 و 11 درس احتمال مهندسی (اضافی)

\end{center}
\hrulefill
\large

کوئیز 10)

سکه ای را 10 بار پرتاب می‌کنیم. متغیر تصادفی 
$
X
$،
تعداد دفعات رو آمدن سکه در 5 پرتاب اول و متغیر تصادفی $Y$، تعداد دفعات رو آمدن سکه در پرتاب های زوج است. احتمال 
$
\Pr\{X=5Y\}
$
را بیابید.

کوئیز 11)

متغیر تصادفی $X$، دارای تابع جرم احتمال زیر است:
\rl{
\begin{table}[h]
\centering
\Large
\begin{tabular}{|c|c|c|c|c|}
\hline
$x$&$-1$&$0$&$1$&$2$\\\hline
$\Pr\{X=x\}$&$0.1$&$0.2$&$0.3$&$0.4$\\\hline
\end{tabular}
\end{table}
}

اگر داشته باشیم
$
Y=X^2-1
$
، در اینصورت مقدار
$
\text{cov}(X,Y)
$
را بیابید.


%کوئیز 10)
%$
%X
%$
%می تواند مقادیر 0 و 1 و 
%$
%Y
%$
%می تواند مقادیر 0، 2، 4 و 6 را اختیار کند. در نتیجه
%\eqn{
%\Pr\{X=2Y\}&=\Pr\{X=2Y=0\}+\Pr\{X=2Y=1\}
%\\&=\Pr\{X=0,Y=0\}+\Pr\{X=1,Y=\frac{1}{2}\}
%\\&=\Pr\{i<4,i\in\{1,3,5\}\}+0
%\\&=\Pr\{i\in\{1,3\}\}=\frac{1}{3}
%}

\end{document}