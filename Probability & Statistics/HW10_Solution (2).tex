\documentclass[10pt,letterpaper]{article}
%\usepackage{toolsper}
\usepackage{amsmath,xcolor,geometry,amssymb,xepersian}
\newcommand{\eqn}[2]{
\begin{equation}
\begin{split}
#1
\label{#2}
\end{split}
\end{equation}
}
%%%%%%%%%%%%%

%       \eqn{
%       x=x^2
%       }{label}

%%%%%%%%%%%%%
\newcommand{\feqn}[2]{
\begin{tcolorbox}[width=7in, colback=white]
\begin{equation}
\begin{split}
#1
\label{#2}
\end{split}
\end{equation}
\end{tcolorbox}
}
%%%%%%%%%%%%%%%
\newcommand{\hl}{
\begin{center}
\line(1,0){450}
\end{center}}
%%%%%%%%%%%%%%%
\newcommand{\qn}[2]{
\[
\begin{split}
#1
\label{#2}
\end{split}
\]
}
%\settextfont{B Nazanin}
\usepackage{lipsum}
\setlength{\parindent}{0mm}
\setlength{\parskip}{3mm}
\newcommand{\pic}[2]{
\begin{center}
\includegraphics[width=#2]{#1}
\end{center}
}
\begin{document}
\Large
\begin{center}
به نام او

پاسخ تمرینات سری دهم درس احتمال مهندسی

\hrulefill
\end{center}
سوال 1) طبق تعریف، واریانس برابر است با:
$$
\sigma^2=\mathbb{E}\{X^2\}-\mathbb{E}^2\{X\}
$$
بنابراین برای هر یک از متغیرهای تصادفی زیر، باید مقادیر 
$
\mathbb{E}\{X\}
$
و
$
\mathbb{E}\{X^2\}
$
را بیابیم.

الف) 
{\color{red}
(صورت سوال تصحیح شده و به جای
 $e^{-x}$
،
$e^{1-x}$
قرار گرفته است.
)
}

\begin{equation}
\begin{split}
&
\mathbb{E}\{X\}=\int xf(x)dx=\int_1^\infty xe^{1-x}dx
\\&=
e\int_1^\infty xe^{-x}dx
=e[-(x+1)e^{-x}]|_{1}^\infty=2
\end{split}
\end{equation}
\begin{equation}
\begin{split}
&
\mathbb{E}\{X^2\}=\int x^2f(x)dx=\int_1^\infty x^2e^{1-x}dx
\\&=
e\int_1^\infty x^2e^{-x}dx
=e[-(x^2+2x+2)e^{-x}]|_{1}^\infty=5
\end{split}
\end{equation}
در نتیجه
$$
\sigma^2=1
$$

ب) 
\begin{equation}
\begin{split}
&
\mathbb{E}\{X\}=\int xf(x)dx=\int_0^{\pi\over 2} x\sin xdx
\\&=(\sin x-x\cos x)|_0^{\pi \over 2}=1
\end{split}
\end{equation}
\begin{equation}
\begin{split}
&
\mathbb{E}\{X^2\}=\int x^2f(x)dx=\int_0^{\pi\over 2} x^2\sin xdx
\\&=
(2x\sin x-(x^2-2)\cos x)|_0^{\pi \over 2}=\pi-2
\end{split}
\end{equation}
در نتیجه
$$
\sigma^2=\pi-3
$$

پ)
\begin{equation}
\begin{split}
&
\mathbb{E}\{X\}=\int xf(x)dx=\int_1^\infty x\times {2\over x^3}dx
\\&=
\int_1^\infty {2\over x^2}dx
=-{2\over x}|_{1}^\infty=2
\end{split}
\end{equation}
\begin{equation}
\begin{split}
&
\mathbb{E}\{X^2\}=\int x^2f(x)dx=\int_1^\infty x^2\times {2\over x^3}dx
\\&
=2\ln |x|\Big|_{1}^\infty=\infty
\end{split}
\end{equation}
در نتیجه
$$
\sigma^2=\infty
$$

ت) 
\begin{equation}
\begin{split}
&
\mathbb{E}\{X\}=\sum_{i=1}^\infty i\cdot\Pr\{X=i\}=2\sum_{i=1}^\infty i({1\over 3})^i
\\&
\mathbb{E}\{X^2\}=\sum_{i=1}^\infty i^2\cdot\Pr\{X=i\}=2\sum_{i=1}^\infty i^2({1\over 3})^i
\end{split}
\end{equation}
برای محاسبه‌ی 
$
\sum_{i=1}^\infty iu^i
$
و
$
\sum_{i=1}^\infty i^2u^i
$
، از تساوی
$
\sum_{i=1}^\infty u^i={u\over 1-u}
$
مشتق می گیریم؛ در این صورت
\begin{equation}
\begin{split}
{d\over du}\sum_{i=1}^\infty u^i=\sum_{i=1}^\infty iu^{i-1}={1\over (1-u)^2}\implies \sum_{i=1}^\infty iu^{i}={u\over (1-u)^2}
\end{split}
\end{equation}
\begin{equation}
\begin{split}
&{d\over du}\sum_{i=1}^\infty iu^i=\sum_{i=1}^\infty i^2u^{i-1}=-{1\over (1-u)^2}+{2\over (1-u)^3}
\\&
\implies \sum_{i=1}^\infty i^2u^{i}=-{u\over (1-u)^2}+{2u\over (1-u)^3}
\end{split}
\end{equation}
بنابراین 
$$
\mathbb{E}\{X\}={3\over 2}\quad,\quad \mathbb{E}\{X^2\}=-{3\over 2}+{9\over 2}=3
$$
و
$$
\sigma^2={3\over 4}
$$

سوال 2) الف)
\begin{equation}
\begin{split}
&\phi_X(s)=\mathbb{E}\{e^{sX}\}=\int e^{sx}f(x)dx=\int_1^\infty e^{sx}e^{1-x}dx
\\&=
e\int_1^\infty e^{(s-1)x}dx
=e{e^{(s-1)x}\over s-1}\Big|_1^\infty={e^s\over 1-s}
\end{split}
\end{equation}
بنابراین
$$
\mathbb{E}\{X\}=\phi_X'(0)={e^s(1-s)+e^s\over (1-s)^2}|_{s=0}=2
$$
و
$$
\mathbb{E}\{X^2\}=\phi_X''(0)={e^s(2-s)(1-s)^2-e^s(1-s)^2+2e^s(2-s)(1-s)\over (1-s)^4}|_{s=0}=5
$$
بنابراین
$$
\sigma^2=5-2^2=1
$$
ت)
\begin{equation}
\begin{split}
&\phi_X(s)=\mathbb{E}\{e^{sX}\}=2\sum_{x=1}^\infty e^{sx}({1\over 3})^x
=2\sum_{x=1}^\infty ({e^s\over 3})^x=2{{e^{s}\over 3}\over 1-{e^{s}\over 3}}
\end{split}
\end{equation}
در نتیجه
$$
\phi'_X(s)={2\over (1-{e^s\over 3})^2}{e^s\over 3}
$$
و
$$
\phi''_X(s)={2\over (1-{e^s\over 3})^2}{e^s\over 3}+{4\over (1-{e^s\over 3})^3}{e^{2s}\over 9}
$$
بنابراین
$$
\mathbb{E}\{X\}={3\over 2}\quad,\quad \mathbb{E}\{X^2\}=3
$$
و
$$
\sigma^2={3\over 4}
$$

سوال 3) الف) 
\begin{equation}
\begin{split}
&\mathbb{E}\{e^{-X}\}=\int e^{-x}f(x)dx=\int_1^\infty e^{-x}e^{1-x}dx
=-{e^{1-2x}\over 2}|_1^\infty={1\over 2e}
\end{split}
\end{equation}
ب)
\begin{equation}
\begin{split}
&\mathbb{E}\{e^{-X}\}=\int e^{-x}f(x)dx=\int_0^{\pi\over 2} e^{-x}{e^{ix}-e^{-ix}\over 2i}dx
\\&=
{1\over 2i}{e^{x(i-1)}\over i-1}\Big|_0^{\pi \over 2}+{1\over 2i}{e^{-x(i+1)}\over i+1}\Big|_0^{\pi \over 2}={1\over 2i}{e^{-\pi\over 2}i\over i-1}-{1\over 2i}{e^{-\pi\over 2}i\over i+1}+{1\over 2}={1\over 2}(1-e^{-{\pi\over 2}})
\end{split}
\end{equation}

سوال 4) الف)
\begin{equation}
\begin{split}
&\int_D f(x,y)dxdy=\int_0^{\pi\over 6}\int_0^{\pi\over 2}k\sin (x+3y)dxdy
\\&=\int_0^{\pi\over 6}-k\cos (x+3y)|_{0}^{\pi\over 2}dy
\\&=-k\int_0^{\pi\over 6}\cos ({\pi\over 2}+3y)-\cos (3y)dy
\\&=-{k\over 3}[\sin ({\pi\over 2}+3y)-\sin (3y)]\Big|_0^{\pi\over 6}={2k\over 3}=1
\implies k={3\over 2}
\end{split}
\end{equation}

\begin{equation}
\begin{split}
&\Pr\{X+3Y\le {1\over 3}\}={3\over 2}\int_0^{1\over 9}\int_0^{{1\over 3}-3y}\sin (x+3y)dxdy
\\&=-{3\over 2}\int_0^{1\over 9}\cos (x+3y)|_{0}^{{1\over 3}-3y}dy
\\&=-{3\over 2}\int_0^{1\over 9}\cos ({1\over 3})-\cos (3y)dy
\\&={\sin {1\over 3}\over 2}-{1\over6}\cos{1\over 3}
\end{split}
\end{equation}

ب)
$$
k=12
$$
\begin{equation}
\begin{split}
&\Pr\{X+3Y\le {1\over 3}\}=12\int_0^{1\over 9}\int_0^{{1\over 3}-3y}xy(1-y)dxdy
\\&=6\int_0^{1\over 9}({1\over3}-3y)^2y(1-y)dy={43\over 65610}
\end{split}
\end{equation}

سوال 5) الف) 
$$
\Pr\{X=0\}=\Pr\{X=0,Y=0\}+\Pr\{X=0,Y=1\}={1\over 2}-\theta+\theta={1\over 2}
$$
$$
\Pr\{X=1\}=1-\Pr\{X=0\}={1\over 2}
$$
به طریق مشابه
$$
\Pr\{Y=0\}=\Pr\{Y=1\}={1\over 2}
$$

ب)
$$
P(X=Y)=P(X=Y=0)+P(X=Y=1)=1-2\theta=1\implies \theta=0
$$
پ) باید به ازای هر 
$
x\in\{0,1\}
$
و هر 
$
y\in\{0,1\}
$
داشته باشیم
$$
\Pr\{X=x,Y=y\}=\Pr\{X=x\}\Pr\{Y=y\}
$$
که به چهار معادله زیر منجر می شود
\begin{equation}
\begin{split}
&\Pr\{X=0,Y=0\}=\Pr\{X=0\}\Pr\{Y=0\}={1\over 4}
\\&\Pr\{X=1,Y=0\}=\Pr\{X=1\}\Pr\{Y=0\}={1\over 4}
\\&\Pr\{X=0,Y=1\}=\Pr\{X=0\}\Pr\{Y=1\}={1\over 4}
\\&\Pr\{X=1,Y=1\}=\Pr\{X=1\}\Pr\{Y=1\}={1\over 4}
\end{split}
\end{equation}
از معادله اول داریم
$$
{1\over 2}-\theta={1\over 4}
$$
که نتیجه می دهد
$$
\theta={1\over 4}
$$
سایر معادلات نیز به پاسخ $
\theta={1\over 4}
$
می رسند که نشان می دهد که به ازای این مقدار از 
$
\theta
$
، متغیرهای تصادفی 
$
X
$
و
$
Y
$
مستقل خواهند بود.
\end{document}