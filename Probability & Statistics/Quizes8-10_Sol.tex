\documentclass{article}

\usepackage{amsmath,amssymb,geometry}
\usepackage{xepersian}

\setlength{\parindent}{0pt}
\setlength{\parskip}{3mm}

\newcounter{questionnumber}
\setcounter{questionnumber}{1}

\newcommand{\Q}{
\textbf{سوال \thequestionnumber)}
\stepcounter{questionnumber}
}

\newcommand{\eqn}[1]{
\[\begin{split}
#1
\end{split}\]
}

\begin{document}
\LARGE
\begin{center}
\settextfont{IranNastaliq}

به نام زیبایی

%\begin{figure}[h]
%\centering
%\includegraphics[width=30mm]{kntu_logo.eps}
%\end{figure}

پاسخ کوئیزهای 8 تا 10 درس احتمال مهندسی

\end{center}
\hrulefill
\large

کوئیز 8)

از آنجا که pmf نرمالیزه است، باید داشته باشیم
\eqn{
&\sum_{n=0}^\infty p(a_n)+p(b_n)=1
\implies
\\&\sum_{n=0}^\infty \frac{2k}{(n+2)(n+3)}=1
\implies
\\&2k\sum_{n=0}^\infty \frac{1}{n+2}-\frac{1}{n+3}=1
\implies k=1
}

گزینه 1

کوئیز 9)

pdf به طور خودکار نرمالیزه شده است. پس
\eqn{
&E\{X\}=\int_0^\frac{1}{c}2c^2x^2dx=\frac{2}{3c}
\\&
E\{X^2\}=\int_0^\frac{1}{c}2c^2x^3dx=\frac{1}{2c^2}
\\&
\sigma_X^2=E\{X^2\}-E^2\{X\}=\frac{1}{18c^2}=\frac{1}{2}
}
در نتیجه
$
c=\frac{1}{3}
$
.

سوال امتیازی)
\eqn{
\Pr\{|X|>3\}=\Pr\{X^2>9\}<\frac{\mathbb{E}\{X^2\}}{9}=\frac{4}{9}
}

کوئیز 10)
$
X
$
می تواند مقادیر 0 و 1 و 
$
Y
$
می تواند مقادیر 0، 2، 4 و 6 را اختیار کند. در نتیجه
\eqn{
\Pr\{X=2Y\}&=\Pr\{X=2Y=0\}+\Pr\{X=2Y=1\}
\\&=\Pr\{X=0,Y=0\}+\Pr\{X=1,Y=\frac{1}{2}\}
\\&=\Pr\{i<4,i\in\{1,3,5\}\}+0
\\&=\Pr\{i\in\{1,3\}\}=\frac{1}{3}
}

\end{document}