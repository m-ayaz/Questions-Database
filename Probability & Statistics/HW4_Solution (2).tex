\documentclass[10pt,letterpaper]{article}
\usepackage{toolsper}
%\settextfont{B Nazanin}
\usepackage{lipsum}
\setlength{\parindent}{0mm}
\setlength{\parskip}{3mm}
\newcommand{\pic}[2]{
\begin{center}
\includegraphics[width=#2]{#1}
\end{center}
}
\begin{document}
\Large
\begin{center}
به نام او

تمرینات سری چهارم درس احتمال مهندسی
\hl
\end{center}
%\color{red}
سوال 1)
\[
\begin{split}
&\Pr\{\text{ابتلا به کرونا}|\text{ماسک زدن}\}=0.15
\\&\Pr\{\text{ابتلا به کرونا}|\text{ماسک نزدن}\}=0.7
\\&\Pr\{\text{ماسک زدن}\}=0.05
\end{split}
\]

\[
\begin{split}
\Pr\{\text{ابتلا به کرونا}\}&=
\Pr\{\text{ابتلا به کرونا}|\text{ماسک زدن}\}
\Pr\{\text{ماسک زدن}\}
\\&+
\Pr\{\text{ابتلا به کرونا}|\text{ماسک نزدن}\}
\Pr\{\text{ماسک نزدن}\}
\\&=0.7\times 0.95+0.15\times0.05=67.25\%
\end{split}
\]

سوال 2) 
$$
\Pr\{\text{هردو زوج}\}=\Pr\{\text{هردو زوج}\}={9\over 36}={1\over 4}\implies \Pr\{\text{جمع زوج}\}={1\over2}=0.5
$$

ب)
$$
\Pr\{\text{جمع بیشتر از 8}|\text{جمع زوج}\}=
{\Pr\{\text{جمع بیشتر از 8}\cap\text{جمع زوج}\}\over
\Pr\{\text{جمع زوج}\}
}
=
{{4\over 36}\over {1\over 2}}={2\over 9}\approx0.22
$$

سوال 3) 

\textbf{راه 1}

 اگر عناصر 1 و 2 را که در هر دو زیر مجموعه هستند کنار بگذاریم، سایر اعضا را به 
$
3^{n-2}
%\binom{2^{n-2}}{2}
$
طریق ممکن می توان بین دو زیرمجموعه پخش کرد. از طرفی برای آنکه اشتراک دو زیر مجموعه برابر 
$
\{1,2\}
$
باشد و اعضای 
$
3,4,5
$
داخل یکی از زیرمجموعه ها بیفتند، باید در هنگام انتخاب سایر اعضای زیرمجموعه ها از بین اعضای 
$
\{5,\cdots,n\}
$
ابتدا هر عضو مجموعه‌ی بالا را در نظر بگیریم. هر عضو سه حالت دارد: یا فقط داخل زیرمجموعه‌ی 1 است؛ یا فقط داخل زیر مجموعه‌ی 2 است و یا در هیچ یک از زیر مجموعه ها نیست. بنابراین این کار به 
$
3^{n-5}
$
طریق ممکن امکان پذیر می شود و احتمال مطلوب برابر است با
$$
{3^{n-5}\over 3^{n-2}}={1\over 27}
$$

\textbf{راه 2}

از آنجا که 1 و 2 در هر دو زیرمجموعه هستند، می توان آنها را نادیده گرفت و سایر اعضای هر دو زیرمجموعه را از اعضای 
$
\{3,4,\cdots,n\}
$
برگزید. با این شرط، تعداد کل حالاتی که می توان دو زیر مجموعه را برگزید عبارتست از اینکه ابتدا k عضو برداریم و سپس این k عضو را بین دو زیرمجموعه پخش کنیم؛ یعنی
$$
\sum_{k=0}^{n-2}\binom{n-2}{k}2^{k}=3^{n-2}
$$
تعداد حالاتی که یکی از زیر مجموعه های شامل عناصر $\{3,4,5\}$ باشد، این است که عناصر معلوم الحال را (یعنی $\{1,2,3,4,5\}$) در نظر بگیریم و سپس سایر اعضا را از بین 
$
\{6,\cdots,n\}
$
برگزینیم. این کار به طریق مشابه به 
$$
\sum_{k=0}^{n-5}\binom{n-5}{k}2^{k-1}=3^{n-5}
$$
امکان پذیر است؛ پس احتمال مطلوب می شود:
$$
{\sum_{k=0}^{n-5}\binom{n-5}{k}2^{k-1}\over
\sum_{k=0}^{n-2}\binom{n-2}{k}2^{k-1}
}={1\over 27}
$$

سوال 4) پیشامد معیوب بودن لامپ را با $C$ و انتخاب جعبه‌ی $i$ ام را با $B_i$ نشان می دهیم. در این صورت:

الف)
$$
P(C)=\sum_{i=1}^{3}P(C|B_i)P(B_i)
={1\over 3}\left({3\over 1000}+{3\over 10}+{0\over 3000}\right)=0.101
$$

ب)
$$
P(B_2|C)={P(B_2\cap C)\over P(C)}={P(C|B_2)P(B_2)\over P(C)}
={{3\over 10}\times {1\over 3}\over {101\over 1000}}={100\over 101}\approx0.99
$$

پ)
\[
\begin{split}
P(B_1\cup B_2|C')&=
{P([B_1\cup B_2]\cap C')\over P(C')}
\\&=
{P([B_1\cap C']\cup [B_2\cap C'])\over P(C')}
\\&=
{P(B_1\cap C')+P(B_2\cap C')\over P(C')}
\\&=
{P(C'|B_1)P(B_1)+P(C'|B_2)P(B_2)\over P(C')}
\\&=
{{1\over 3}\left({997\over 1000}+{7\over 10}\right)
\over
1-0.101
}
\approx0.63
\end{split}
\]

سوال 5)

$$
P(P_1|B)={P(P_1\cap B)\over P(B)}={P(B|P_1)P(P_1)\over P(B)}
={{20\over 100}\times{100\over 100+1000}\over {70\over 1100}}={2\over 7}\approx 0.29
$$

$$
P(P_2\cap B'|F)={P(P_2\cap B'\cap F)\over P(F)}
={{630\over 1100}\over {690\over 1100}}={21\over 23}\approx 0.91
$$

$$
P(B|M)={P(B\cap M)\over P(M)}={{40\over 1100}\over {410\over 1100}}
={4\over 41}\approx 0.10
$$
%الف) اگر این فرد چشم آبی باشد، با چه احتمالی از استان 1 انتخاب شده است؟
%
%ب) اگر این فرد زن باشد، با چه احتمالی از استان 2 انتخاب شده و چشم آبی \underline{نیست}؟
%
%پ) اگر فرد انتخاب شده مرد باشد، با چه احتمالی چشم آبی است؟
\end{document}