\documentclass{article}

\usepackage{amsmath,amssymb,geometry}
\usepackage{xepersian}

\setlength{\parindent}{0pt}
\setlength{\parskip}{3mm}

\newcounter{questionnumber}
\setcounter{questionnumber}{1}

\newcommand{\Q}{
\textbf{سوال \thequestionnumber)}
\stepcounter{questionnumber}
}

\newcommand{\eqn}[1]{
\[\begin{split}
#1
\end{split}\]
}

\begin{document}
\LARGE
\begin{center}
\settextfont{IranNastaliq}

به نام زیبایی

%\begin{figure}[h]
%\centering
%\includegraphics[width=30mm]{kntu_logo.eps}
%\end{figure}

کوئیزها

\end{center}
\hrulefill
\large


کوئیز 4)

سه جعبه در اختیار داریم. جعبه‌ی 1 شامل 7 توپ آبی و 3 توپ قرمز، جعبه‌ی 2 شامل 1 توپ آبی، 3 توپ قرمز و 6توپ زرد و جعبه‌ی 3 شامل 7 توپ آبی و 3 توپ زرد هستند. ابتدا یکی از جعبه ها را به تصادف برداشته و سپس توپی از آن جعبه به تصادف بر می‌داریم. اگر توپ بیرون آمده آبی نباشد، با چه احتمالی قرمز است و از جعبه‌ی 1 یا از جعبه‌ی 2 بیرون آمده است؟

کوئیز 5)

سکه ای را پرتاب می‌کنیم. اگر رو آمد، تاسی را 3 بار پرتاب کرده و جمع اعداد رو آمده در 3 پرتاب را در نظر می‌گیریم. اگر پشت آمد، تاسی را 4 بار پرتاب کرده و جمع اعداد رو آمده در 4 پرتاب را در نظر می‌گیریم. اگر جمع اعداد روآمده‌ی تاس برابر 5 باشد، با چه احتمالی سکه پشت آمده است؟

\end{document}