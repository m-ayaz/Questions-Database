\documentclass{article}

\usepackage{amsmath,amssymb,geometry,tikz}
\usepackage{xepersian}

\setlength{\parindent}{0pt}
\setlength{\parskip}{3mm}

\newcounter{questionnumber}
\setcounter{questionnumber}{1}

\newcommand{\Q}{
\textbf{سوال \thequestionnumber)}
\stepcounter{questionnumber}
}

\newcommand{\eqn}[1]{
\begin{equation}\begin{split}
#1
\end{split}\end{equation}
}

\begin{document}
\LARGE
\begin{center}
\settextfont{IranNastaliq}

به نام زیبایی

%\begin{figure}[h]
%\centering
%\includegraphics[width=30mm]{kntu_logo.eps}
%\end{figure}

تمرینات سری یازدهم درس احتمال مهندسی

\end{center}
\hrulefill
\large

\Q

برای هر یک از توزیع های زیر، مقدار 
$
\Pr\{X\ge \alpha\}
$
را به دست آورده و همچنین، یک کران بالا برای این احتمال برای هر توزیع با کمک نامساوی مارکوف به دست آورید. سپس مقدار دقیق احتمال و کران آن را مقایسه کنید.

الف)
$
f(x)=e^{-x}\quad,\quad x>0
$

ب)
$
f(x)=\frac{1}{\ln 2}\frac{1}{1+e^x}\quad,\quad x>0
$

پ)
$
f(x)=xe^{-x}\quad,\quad x>0
$

\Q

برای توزیع‌های بخش های الف و پ سوال پیش، مقدار واریانس را به دست آورید.

\Q

برای هر یک از توزیع های زیر، تابع مولد گشتاور را یافته و سپس از روی آن، مقدار 
$
\mathbb{E}\{X^2\}
$
را بیابید.

الف)
$
f(x)=\begin{cases}
1-x&,\quad 0<x<1\\
\frac{1}{2}\delta(x-1)&,\quad x=1
\end{cases}
$

ب)
$
f(x)=\begin{cases}
\cos x&,\quad 0<x<\frac{\pi}{2}\\
0&,\quad \text{جاهای دیگر}
\end{cases}
$

پ)
$
p(x)=\Pr\{X=x\}=
\begin{cases}
\frac{n}{2^{n+1}}&,\quad n\in\Bbb N\\
0&,\quad \text{جاهای دیگر}
\end{cases}
$

ت)
$
X
$
متغیر تصادفی حاصل ضرب دو عدد رو آمده در پرتاب دو تاس به طور مستقل است.
\end{document}