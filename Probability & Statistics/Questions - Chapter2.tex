\chapter{آزمایش های تکراری}

\Q
تاس سالمی را 3 بار پرتاب می‌کنیم و اعداد رو آمده در سه پرتاب را در نظر می‌گیریم.

الف) احتمال آن که جمع اعداد رو آمده برابر 5 باشد چقدر است؟

ب) اگر عدد رو آمده‌ی اول برابر 4 باشد، احتمال آن که جمع اعداد پرتاب ها برابر 7 باشد چقدر است؟

پ) احتمال آن که جمع اعداد تاس در پرتاب‌های فرد، برابر 5 باشد چقدر است؟

ت) احتمال آنکه از این 3 بار، حداقل 2 بار عدد زوج بیاید چقدر است؟

ث) احتمال رو آمدن مضرب 3 در پرتاب اول چقدر است؟

%%%%%%%%%%%%%%%%%%%%%%%%%%%%%%%%%%%%%%

\Q
سکه ای را پرتاب می‌کنیم. اگر رو آمد، تاسی را 3 بار پرتاب کرده و جمع اعداد رو آمده در 3 پرتاب را در نظر می‌گیریم. اگر پشت آمد، تاسی را 4 بار پرتاب کرده و جمع اعداد رو آمده در 4 پرتاب را در نظر می‌گیریم. اگر جمع اعداد روآمده‌ی تاس برابر 5 باشد، با چه احتمالی سکه پشت آمده است؟

%%%%%%%%%%%%%%%%%%%%%%%%%%%%%%%%%%%%%%

\Q
تاس سالمی را 4 بار پرتاب می کنیم و اعداد رو آمده در چهار پرتاب را در نظر می‌گیریم.

الف) اگر در دو پرتاب این تاس عدد 2 ظاهر شده باشد، احتمال آنکه در دو پرتاب دیگر عدد فردی ظاهر شده باشد چقدر است؟

ب) با چه احتمالی، جمع اعداد در پرتاب های زوج، 5 برابر جمع اعداد در پرتاب‌های فرد است؟

%%%%%%%%%%%%%%%%%%%%%%%%%%%%%%%%%%%%%%
 
\Q
سکه‌ی سالمی را 10 بار پرتاب می کنیم. مطلوبست احتمال آن که

الف) در این 10 پرتاب، حداقل دوبار رو بیاید.

ب) در سه پرتاب اول حداکثر یک بار پشت بیاید.

پ) در پرتاب های زوج، نتیجه یکسان باشد (همگی رو یا همگی پشت باشند).

الف) دقیقا 3 بار شیر بیاید.

ب) دست کم 2 بار خط بیاید.

پ) در مجموع، دقیقا 7 بار خط آمده باشد، اگر بدانیم در 5 پرتاب اول خط آمده است.

%%%%%%%%%%%%%%%%%%%%%%%%%%%%%%%%%%%%%%
 
\Q
الف) اگر یک رشته لامپ متوالی شامل $n$ لامپ که هر لامپ به احتمال $p$ خراب است، به ولتاژ برق وصل شود، با چه احتمالی روشن می شود؟ (در رشته متوالی لامپ ها، لامپ ها به صورت پشت سر هم به یکدیگر وصل شده اند.)

ب) اگر رشته لامپ موازی باشد، مسئله را حل کنید. (در رشته‌ی موازی لامپ ها، یکی از سرهای همه‌ی لامپ ها به یک نقطه و سر دیگر تمام لامپ ها به نقطه‌ی دیگر وصل شده اند.)

%%%%%%%%%%%%%%%%%%%%%%%%%%%%%%%%%%%%%%

\Q
در یک امتحان، احتمال درست پاسخ دادن به یک سوال دو گزینه ای برابر $p$ است. پس از امتحان، $n$ دانشجو پاسخ های خود را با هم مقایسه می کنند و متوجه می‌شوند که همگی به آن سوال پاسخ یکسانی داده اند. با چه احتمالی تمام این $n$ دانشجو به پاسخ درست رسیده اند؟

%%%%%%%%%%%%%%%%%%%%%%%%%%%%%%%%%%%%%%

\Q
یک سکه‌ی سالم را 7بار پرتاب می کنیم.

الف) احتمال اینکه نتیجه‌ی پرتاب اول و آخر برابر باشد چقدر است؟

ب) با چه احتمالی حداقل دو رو و سه پشت در این 7 پرتاب خواهیم داشت؟

پ) اگر نتیجه پرتاب سکه در سه پرتاب اول یکسان باشد، با چه احتمالی در این 7 پرتاب، در مجموع دقیقا 4 بار سکه رو می آید؟

%%%%%%%%%%%%%%%%%%%%%%%%%%%%%%%%%%%%%%

\Q
یک تاس سالم را 5 بار پرتاب می کنیم.

الف) اگر جمع پنج عدد رو آمده در این پنج پرتاب را در نظر بگیریم، با چه احتمالی این مجموع برابر 7 است؟

ب) با چه احتمالی عدد رو آمده در پرتاب پنجم برابر جمع اعداد رو آمده در 4 پرتاب قبلی خواهد بود؟

%%%%%%%%%%%%%%%%%%%%%%%%%%%%%%%%%%%%%%

\Q
بزرگراه 8 بانده ای را در نظر بگیرید که از هر باند آن در هر لحظه حداکثر یک ماشین می‌تواند عبور کند. اگر 9 ماشین هر یک با احتمال $p$ وارد بزرگراه شوند،

الف) با چه احتمالی همه‌ی ماشین های وارد شده به بزرگراه بدون مشکل از آن رد می‌شوند؟

ب) $p$ چقدر باشد تا احتمال قسمت الف بیشتر از $0.99$ باشد؟

%%%%%%%%%%%%%%%%%%%%%%%%%%%%%%%%%%%%%%

\Q
دو تیم ورزشی A و B در یک بازی در 9 دست با هم روبرو می‌شوند و نتیجه‌ی هر دست فقط برد یکی از دو تیم می‌تواند باشد. فرض کنید تیم A با احتمال $p$ در هر دست پیروز می‌شود و نتیجه‌ی دست‌ها مستقل از هم است. برنده‌ی بازی کسی است که بیشتر بازی ها را برده باشد.

الف) با چه احتمالی تیم A پس از 6 دست موفق به بردن بازی می‌شود؟

ب) اگر بدانیم تیم A در نهایت بازی را برده است، با چه احتمالی در حداقل یک دست به تیم B باخته است؟

ج) به ازای $p=0.5$، اگر بدانیم تیم A دست اول را برده، با چه احتمالی بازی را می‌برد؟

%%%%%%%%%%%%%%%%%%%%%%%%%%%%%%%%%%%%%%

\Q
یک سکه‌ی سالم $n$ بار پرتاب شده و $k$ بار رو آمده است. کوچکترین مقدار $n$ را بیابید به گونه‌ای که
$$
P\left\{0.49\le{k\over n}\le0.51\right\}>0.95
$$

%%%%%%%%%%%%%%%%%%%%%%%%%%%%%%%%%%%%%%

\Q
قضیه‌ی دموآو-لاپلاس در چه حالتی برای تکرر توزیع برنولی به تعداد $n$ بار برقرار است؟ به کمک یک ماشین حساب یا کامپیوتر، مقادیر 
$\binom{n}{k}p^k(1-p)^{n-k}$
و
$e^{-np}{(np)^k\over k!}$
 را به ازای حالت های مختلف 
$n$
و
$p$
محاسبه کرده و خطای تقریب پواسون را به دست آورید.

الف)
$n=10\quad,\quad p=0.7\quad,\quad k=7$

ب)
$n=30\quad,\quad p=0.3\quad,\quad k=9$

پ)
$n=50\quad,\quad p=0.02\quad,\quad k=1$

ت)
$n=300\quad,\quad p=0.01\quad,\quad k=3$

ث)
$n=30\quad,\quad p=0.8\quad,\quad k=24$

ج)
$n=1000\quad,\quad p=0.5\quad,\quad k=1$

چ)
$n=1000\quad,\quad p=0.5\quad,\quad k=300$

ح)
$n=1000\quad,\quad p=0.5\quad,\quad k=490$

در کدام حالت تقریب پواسون، خطای کمتری دارد و چرا؟

%%%%%%%%%%%%%%%%%%%%%%%%%%%%%%%%%%%%%%

\Q
سکه‌ای را پرتاب می‌کنیم. اگر پشت آمد، آن را 9 بار دیگر پرتاب می‌کنیم و نتایج 10 پرتاب را در نظر می‌گیریم. اگر رو آمد، آن را 5 بار دیگر پرتاب می‌کنیم و نتایج 6 پرتاب را در نظر می‌گیریم. احتمال آن که در تمام پرتاب های سکه، دقیقأ 6 بار رو بیاید چقدر است؟

%%%%%%%%%%%%%%%%%%%%%%%%%%%%%%%%%%%%%%

\Q
یک آزمایش برنولی را که احتمال موفقیت در آن برابر 
$
40\%
$
است، $n$ بار تکرار می‌کنیم. اگر $k$، برابر تعداد موفقیت‌ها در این پرتاب ها باشد، $n$ حداقل چقدر باشد تا احتمال رخداد 
$
\{38\%<\frac{k}{n}<42\%\}
$
بیش از 
$
70\%
$
باشد؟

(راهنمایی: از قضیه‌ی دموآور-لاپلاس استفاده نمایید.)

(جدول مربوط به محاسبه‌ی تابع 
$
G^{-1}(x)
$
در صفحه‌ی بعد آمده است.
)

\newpage

%\newgeometry{left=0mm,right=0mm}

%\begin{figure}[h]
%\centering
%\includegraphics[width=220mm]{ginv.pdf}
%\end{figure}

%(تنها نامساوی مربوط به قضیه‌ی دموآور-لاپلاس را نوشته، اعداد را جایگذاری نموده و ساده کنید. محاسبه‌ی دقیق مقدار $n$ الزامی نیست.)
%سکه‌ای را پرتاب می‌کنیم. اگر رو آمد، یک تاس سالم را 5 بار پرتاب کرده و جمع اعداد این 5 پرتاب را می‌نویسیم.

\begin{table}[h]
\centering
\large
\lr{
\begin{tabular}{|c|c|c|c|c|c|c|c|}
\hline
$x$&$G^{-1}(x)$&$x$&$G^{-1}(x)$&$x$&$G^{-1}(x)$&$x$&$G^{-1}(x)$\\\hline
0.01&-2.3263&0.26&-0.6433&0.51&0.0251&0.76&0.7063\\\hline
0.02&-2.0537&0.27&-0.6128&0.52&0.0502&0.77&0.7388\\\hline
0.03&-1.8808&0.28&-0.5828&0.53&0.0753&0.78&0.7722\\\hline
0.04&-1.7507&0.29&-0.5534&0.54&0.1004&0.79&0.8064\\\hline
0.05&-1.6449&0.30&-0.5244&0.55&0.1257&0.80&0.8416\\\hline
0.06&-1.5548&0.31&-0.4959&0.56&0.1510&0.81&0.8779\\\hline
0.07&-1.4758&0.32&-0.4677&0.57&0.1764&0.82&0.9154\\\hline
0.08&-1.4051&0.33&-0.4399&0.58&0.2019&0.83&0.9542\\\hline
0.09&-1.3408&0.34&-0.4125&0.59&0.2275&0.84&0.9945\\\hline
0.10&-1.2816&0.35&-0.3853&0.60&0.2533&0.85&1.0364\\\hline
0.11&-1.2265&0.36&-0.3585&0.61&0.2793&0.86&1.0803\\\hline
0.12&-1.1750&0.37&-0.3319&0.62&0.3055&0.87&1.1264\\\hline
0.13&-1.1264&0.38&-0.3055&0.63&0.3319&0.88&1.1750\\\hline
0.14&-1.0803&0.39&-0.2793&0.64&0.3585&0.89&1.2265\\\hline
0.15&-1.0364&0.40&-0.2533&0.65&0.3853&0.90&1.2816\\\hline
0.16&-0.9945&0.41&-0.2275&0.66&0.4125&0.91&1.3408\\\hline
0.17&-0.9542&0.42&-0.2019&0.67&0.4399&0.92&1.4051\\\hline
0.18&-0.9154&0.43&-0.1764&0.68&0.4677&0.93&1.4758\\\hline
0.19&-0.8779&0.44&-0.1510&0.69&0.4959&0.94&1.5548\\\hline
0.20&-0.8416&0.45&-0.1257&0.70&0.5244&0.95&1.6449\\\hline
0.21&-0.8064&0.46&-0.1004&0.71&0.5534&0.96&1.7507\\\hline
0.22&-0.7722&0.47&-0.0753&0.72&0.5828&0.97&1.8808\\\hline
0.23&-0.7388&0.48&-0.0502&0.73&0.6128&0.98&2.0537\\\hline
0.24&-0.7063&0.49&-0.0251&0.74&0.6433&0.99&2.3263\\\hline
0.25&-0.6745&0.50&0.0000&0.75&0.6745&0.9999&3.7190\\\hline
\end{tabular}
}
\end{table}

%\restoregeometry



%%%%%%%%%%%%%%%%%%%%%%%%%%%%%%%%%%%%%%

\Q
یک تاس سالم را 6 بار پرتاب می‌کنیم.

الف) احتمال آن که جمع اعداد رو آمده در 6 پرتاب برابر 8 باشد چقدر است؟

ب) احتمال آن که در این 6 پرتاب، تمام اعداد 1 تا 6 ظاهر شوند چقدر است؟

%%%%%%%%%%%%%%%%%%%%%%%%%%%%%%%%%%%%%%

\Q
از کیسه‌ای که شامل 7 توپ آبی و 3 توپ سفید است، 1 توپ به تصادف برداشته، رنگ آن را یادداشت کرده و دوباره به کیسه بر می‌گردانیم. اگر این کار را 11 بار انجام دهیم، احتمال آن که از این 11 بار دقیقأ در 7 مرتبه، توپ آبی بیرون آمده باشد چقدر است؟

%%%%%%%%%%%%%%%%%%%%%%%%%%%%%%%%%%%%%%

\Q
یک کانال مخابراتی دارای ظرفیت 25 گیگابیت بر ثانیه است. در مجموع، 12 کاربر قصد استفاده از این کانال برای ارسال داده‌ی خود را دارند که هر کاربر، $2.5$ گیگابیت بر ثانیه از کانال را اشغال می‌کند و احتمال فعال بودن او، مستقل از سایرین برابر $p=0.6$ است. با چه احتمالی، برای تخصیص کانال به کاربران فعال، دچار کمبود ظرفیت کانال نخواهیم شد؟

%%%%%%%%%%%%%%%%%%%%%%%%%%%%%%%%%%%%%%

\Q
یک آزمایش برنولی را که احتمال موفقیت در آن برابر 
$
\frac{1}{3}
$
است، $n$ بار تکرار می‌کنیم. اگر $k$ تعداد موفقیت ها در $n$ آزمایش باشد، $n$ حداقل چقدر باشد تا احتمال رخداد 
$
\{\frac{97}{300}<\frac{k}{n}<\frac{103}{300}\}
$
برابر $99\%$ باشد؟

%%%%%%%%%%%%%%%%%%%%%%%%%%%%%%%%%%%%%%

\Q
آزمایشی را که احتمال موفقیت آن $p$ و احتمال شکست آن $1-p$ است، آنقدر تکرار می‌کنیم تا به $k$-امین موفقیت برسیم. متوسط تعداد آزمایش ها را تا حصول $k$-امین موفقیت به ازای 
$k=1$
و
$k=2$
به دست آورید.