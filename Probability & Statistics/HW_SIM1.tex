\documentclass[10pt,letterpaper]{article}
%\usepackage{toolsper}
\usepackage{amsmath,geometry,amssymb,xepersian}
\newcommand{\eqn}[2]{
\begin{equation}
\begin{split}
#1
\label{#2}
\end{split}
\end{equation}
}
%%%%%%%%%%%%%

%       \eqn{
%       x=x^2
%       }{label}

%%%%%%%%%%%%%
\newcommand{\feqn}[2]{
\begin{tcolorbox}[width=7in, colback=white]
\begin{equation}
\begin{split}
#1
\label{#2}
\end{split}
\end{equation}
\end{tcolorbox}
}
%%%%%%%%%%%%%%%
\newcommand{\hl}{
\begin{center}
\line(1,0){450}
\end{center}}
%%%%%%%%%%%%%%%
\newcommand{\qn}[2]{
\[
\begin{split}
#1
\label{#2}
\end{split}
\]
}
%\settextfont{B Nazanin}
\usepackage{lipsum}
\setlength{\parindent}{0mm}
\setlength{\parskip}{3mm}
\newcommand{\pic}[2]{
\begin{center}
\includegraphics[width=#2]{#1}
\end{center}
}
\begin{document}
\Large
\begin{center}
به نام او

تمرینات سری اول شبیه سازی درس احتمال مهندسی
\hl
\end{center}
سوال 1) سکه‌ی سالمی را (که احتمال پشت و رو برابر $1\over 2$ است) به تعداد n بار پرتاب می کنیم.

الف) منحنی تعداد دفعات رو آمدن سکه را به تعداد کل دفعات پرتاب سکه بر حسب $n$ به ازای $1\le n\le 100$ رسم کنید.

ب) رفتار این نمودار را با افزایش $n$ توصیف کرده و بیان کنید به چه عددی همگرا می شود. مشاهدات خود را توضیح دهید.

پ) همین نمودار را برای سکه غیر سالم که احتمال رو آمدن آن
$
{2\over 3}
$
است،  رسم کرده و بیان کنید به چه عددی همگرا می‌گردد. مشاهدات خود را توضیح دهید.

\textbf{
راهنمایی: ابتدا به کمک تابع 
randi()
در متلب، یک آرایه 
$1\times 100$
 از اعداد تصادفی 0 یا 1 بسازید (به عنوان مثال 0 را شیر و 1 را خط در نظر بگیرید). سپس هر بار، به ازای هر $n$ که
$1\le n \le 100$
، 
میانگین
$n$
 مقدار اول آرایه فوق را حساب کرده و آرایه‌ی $1\times 100$ جدید بسازید. آرایه‌ی جدید، پاسخ مسئله است که باید بر حسب $n$ از 1 تا 100 رسم گردد.
}

سوال 2) سکه ای با احتمال $p$ رو و با احتمال $1-p$ پشت می آید. آزمایشی را در نظر بگیرید که در آن، این سکه 100 بار پرتاب شده و نسبت تعداد رو آمدن ها به تعداد کل پرتاب ها (=100) به عنوان نتیجه آزمایش در نظر گرفته می شود.

اگر آزمایش فوق را $m$ بار تکرار کرده و نتیجه این $m$ آزمایش را در یک آرایه‌ی 
$
1\times m
$
 ذخیره کنیم،

الف) با فرض 
$
p=0.5
$
و
$
m=100
$
، این آرایه را رسم کنید (مشابه بند الف سوال 1). سپس به کمک قضیه دموآو-لاپلاس، منحنی گوسی را به عنوان تقریبی از توزیع دو جمله ای به همراه نمودار قبل رسم نموده و مشاهدات خود را توضیح دهید.

ب) بند الف را به ازای $m=1000$ تکرار کنید. چه تفاوتی مشاهده می شود؟

پ) بند الف را یک بار برای حالت 
$
p=0.2
$
و
$
m=1000
$
و بار دیگر برای حالت 
$
p=0.1
$
و
$
m=1000
$
شبیه سازی کنید و مشاهدات خود را توصیف کنید. چه تفاوتی با بند های پیشین مشاهده می شود؟

(یادآوری: توزیع دوجمله ای با پارامترهای $n$ و $p$ را می توان به صورت زیر با منحنی گوسی تقریب زد:
$$
\binom{n}{k}p^kq^{n-k}\approx {1\over\sqrt{2\pi npq}}e^{-{(k-np)^2\over 2npq}}
$$
که 
$
q=1-p
$.
)

\vspace{10mm}
\textit{
\hspace{10mm}
موفق و پیروز باشید
}

\textit{
\hspace{20mm}
سروش ضیایی و آرین ظروفی
}
\end{document}