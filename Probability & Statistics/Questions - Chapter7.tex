\chapter{شبیه سازی ها}

\Q
سکه‌ی سالمی را (که احتمال پشت و رو برابر $1\over 2$ است) به تعداد n بار پرتاب می کنیم.

الف) منحنی تعداد دفعات رو آمدن سکه را به تعداد کل دفعات پرتاب سکه بر حسب $n$ به ازای $1\le n\le 100$ رسم کنید.

ب) رفتار این نمودار را با افزایش $n$ توصیف کرده و بیان کنید به چه عددی همگرا می شود. مشاهدات خود را توضیح دهید.

پ) همین نمودار را برای سکه غیر سالم که احتمال رو آمدن آن
$
{2\over 3}
$
است،  رسم کرده و بیان کنید به چه عددی همگرا می‌گردد. مشاهدات خود را توضیح دهید.

\textbf{
راهنمایی: ابتدا به کمک تابع 
randi()
در متلب، یک آرایه 
$1\times 100$
 از اعداد تصادفی 0 یا 1 بسازید (به عنوان مثال 0 را شیر و 1 را خط در نظر بگیرید). سپس هر بار، به ازای هر $n$ که
$1\le n \le 100$
، 
میانگین
$n$
 مقدار اول آرایه فوق را حساب کرده و آرایه‌ی $1\times 100$ جدید بسازید. آرایه‌ی جدید، پاسخ مسئله است که باید بر حسب $n$ از 1 تا 100 رسم گردد.
}



\Q
سکه ای با احتمال $p$ رو و با احتمال $1-p$ پشت می آید. آزمایشی را در نظر بگیرید که در آن، این سکه 100 بار پرتاب شده و نسبت تعداد رو آمدن ها به تعداد کل پرتاب ها (=100) به عنوان نتیجه آزمایش در نظر گرفته می شود.

اگر آزمایش فوق را $m$ بار تکرار کرده و نتیجه این $m$ آزمایش را در یک آرایه‌ی 
$
1\times m
$
 ذخیره کنیم،

الف) با فرض 
$
p=0.5
$
و
$
m=100
$
، این آرایه را رسم کنید (مشابه بند الف سوال 1). سپس به کمک قضیه دموآو-لاپلاس، منحنی گوسی را به عنوان تقریبی از توزیع دو جمله ای به همراه نمودار قبل رسم نموده و مشاهدات خود را توضیح دهید.

ب) بند الف را به ازای $m=1000$ تکرار کنید. چه تفاوتی مشاهده می شود؟

پ) بند الف را یک بار برای حالت 
$
p=0.2
$
و
$
m=1000
$
و بار دیگر برای حالت 
$
p=0.1
$
و
$
m=1000
$
شبیه سازی کنید و مشاهدات خود را توصیف کنید. چه تفاوتی با بند های پیشین مشاهده می شود؟

(یادآوری: توزیع دوجمله ای با پارامترهای $n$ و $p$ را می توان به صورت زیر با منحنی گوسی تقریب زد:
$$
\binom{n}{k}p^kq^{n-k}\approx {1\over\sqrt{2\pi npq}}e^{-{(k-np)^2\over 2npq}}
$$
که 
$
q=1-p
$.
)

\Q
سوال 1) (تعبیر تجربی چگالی احتمال) می دانید که تابع چگالی احتمال دارای شهود تجربی است؛ به این معنا که مقدار چگالی احتمال در هر نقطه از دامنه‌ی متغیر تصادفی، نشان دهنده‌ی وزن احتمالاتی آن نقطه است. برای به تصویر کشیدن این شهود، متغیر تصادفی گوسی با میانگین صفر و واریانس 1 را در نظر بگیرید. تابع
\texttt{
randn()
}
در متلب، تحققی از چنین متغیر تصادفی ای را برآورده می کند. به کمک این تابع، 100000 تحقق مستقل از این متغیر تصادفی را تولید کرده و به کمک دستور 
\texttt{
histogram()
}
آرایه‌ی 100000 تایی را رسم کنید. سپس با حفظ این نمودار، نمودار منحنی گوسی با میانگین صفر و واریانس 1 را رسم کرده و دو نمودار را با هم مقایسه کنید. مشاهدات خود را توضیح دهید.

نکته مهم!

پارامتر 
\texttt{
`Normalization'
}
را روی مقدار 
\texttt{
`pdf'
}
تنظیم کنید. برای این کار، دستور 
\texttt{
histogram()
}
را به صورت 
\texttt{
histogram(.....,'Normalization','pdf')
}
استفاده کنید.


\Q
در مبحث توابعی از یک متغیر تصادفی آموختید که اگر متغیر تصادفی $X$ دارای چگالی احتمال  مشخص باشد، آنگاه می توان چگالی احتمال  متغیر تصادفی $Y=g(X)$ را با تنظیم تابع $g(\cdot)$ روی هر تابع دلخواهی تنظیم کرد. ثابت می‌شود که اگر $X$ متغیر تصادفی یکنواخت در بازه‌ی $[0,1]$ باشد، متغیر تصادفی 
$
Y=-\ln (1-X)
$
دارای توزیع نمایی با پارامتر $\lambda=1$ و توزیع 
$
f_Y(y)=e^{-y}
$
است. برای اثبات این امر،

الف) 100000 تحقق از متغیر تصادفی یکنواخت در بازه‌ی $[0,1]$ را به کمک دستور 
\texttt{rand()}
تولید کرده و هیستوگرام آن را رسم کنید. اسم بردار 100000 تایی را $X$ بگذارید.

ب) از روی بردار $X$، بردار 100000 تایی $Y=-\ln(1-x)$ را محاسبه کرده و هیستوگرام آن را رسم کنید. سپس چگالی احتمال متغیر تصادفی نمایی با پارامتر $\lambda=1$ را به همراه این هیستوگرام در یک نمودار نشان دهید. رابطه‌ی بین هیستوگرام و چگالی احتمال  توزیع نمایی را توجیه کنید.

برای رسم دو نمودار در یک شکل، از دستور 
\texttt{\lr{hold on}}
استفاده کنید.

\Q
در این شبیه سازی، قصد داریم پرتاب سکه را با جزئیات بیشتر مطالعه کنیم. می‌دانیم در 
$
n
$
بار پرتاب سکه‌ی سالم، احتمال 
$
k
$
بار رو آمدن برابر است با
$$
\binom{n}{k}(\frac{1}{2})^n.
$$
جهت تجربه‌ی عملی فرمول اخیر، آزمایشی را که در آن سکه
$
1000
$
بار پرتاب می‌شود،
$
1000
$
بار تکرار می‌کنیم (مشابه اینکه سکه ای را 1000000 بار پرتاب کرده و هر 1000 نتیجه‌ی پشت سر هم را خوشه‌بندی کنیم).

گامها:

گام 1) به کمک یک حلقه‌ی \lr{for}،
1000 عدد تصادفی 0 یا 1 تولید کرده و تعداد دفعاتی را که این رشته‌ی 1000تایی برابر 1 می‌شود (سکه رو می‌آید) را حساب کنید.

گام 2) گام1 را 1000 بار تکرار کرده و هر بار نتیجه را در یک آرایه‌ی 1000تایی ذخیره کنید.

گام 3) هیستوگرام نتایج را رسم نموده و آن را توجیه کنید (تعداد \lr{bin}های هیستوگرام را برابر $40$ قرار دهید).

سوال 2)

الف) می‌دانیم که چگالی احتمال یک متغیر تصادفی، میانگین تجربی تعداد دفعات رخداد مقادیر آن متغیر تصادفی را نشان می‌دهد؛ به طور مثال در توزیع یکنواخت در بازه‌ی $[0,1]$، تمام اعداد این بازه، دارای میانگین تجربی یکسان رخداد هستند. جهت تحقیق این امر، 10000 تحقق از توزیع یکنواخت در بازه‌ی $[0,1]$ را تولید کرده و هیستوگرام آن را رسم کنید. نتیجه را توجیه کنید.

ب) می‌دانیم که اگر 
$
X
$
از توزیع یکنواخت در بازه‌ی 
$
[0,1]
$
پیروی کند، 
$
Y=-\ln X
$
از توزیع نمایی با پارامتر 
$
\lambda=1
$
پیروی می‌کند و چگالی احتمال آن به صورت زیر است:
$$
f_Y(y)=\begin{cases}
e^{-y}&,\quad y>0\\
0&,\quad y\le0\\
\end{cases}.
$$
به کمک دستور هیستوگرام، 10000 تحقق از توزیع یکنواخت در بازه‌ی 
$
[0,1]
$
تولید کرده و در آرایه‌ای به نام
$
X
$
ذخیره کنید. سپس آرایه‌ی
$
Y
$
را از روی آرایه‌ی
$
X
$
به صورت
$
Y=-\ln X
$
بسازید. سپس هیستوگرام $Y$ را با تعداد \lr{bin} برابر 100 رسم کنید. جهت مشاهده‌ی رفتار چگالی احتمال، تابع
$
e^{-y}
$
را در بازه‌ی 
$
(0,10)
$
روی همان نمودار هیستوگرام رسم شده بکشید و نتیجه را توجیه کنید.

(
فرض کنید میخواهید هیستوگرام آرایه‌ی $x$ را رسم کنید. جهت رسم هیستوگرام در این سوال، از دستورهای زیر استفاده کنید:

%متلب:
%\begin{latin}
%\begin{minted}{matlab}
%histogram(x,'Normalization','pdf')
%\end{minted}
%\end{latin}
%پایتون:
%\begin{latin}
%\begin{minted}{python}
%hist(x,density='pdf')
%\end{minted}
%\end{latin}

)

%\chapter{پاسخ سوالات}
%
%\QA
%از اصل سوم احتمال، برای هر دو مجموعه‌ی ناسازگار 
%$A$
%و
%$B$
%داریم
%$$P(A\cup B)=P(A)+P(B).$$
%از آنجا که 
%$A$
%و
%$A'$
%طبق تعریف ناسازگارند، بنابراین
%$$P(A\cup A')=P(A)+P(A').$$
%از طرفی طبق تعریف،
%$$A\cup A'=S$$
%که $S$ فضای نمونه است. در نتیجه
%$$P(A)+P(A')=1.$$
%بر اساس اصل اول احتمال، احتمال هر مجموعه مقداری نامنفی است؛ در نتیجه
%$$P(A)=1-P(A')\le 1$$
%و اثبات کامل است
%$\blacksquare$

