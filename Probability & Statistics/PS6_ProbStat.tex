\documentclass{article}

\usepackage{amsmath,amssymb,geometry,tikz}
\usepackage{xepersian}

\setlength{\parindent}{0pt}
\setlength{\parskip}{3mm}

\newcounter{questionnumber}
\setcounter{questionnumber}{1}

\newcommand{\Q}{
\textbf{سوال \thequestionnumber)}
\stepcounter{questionnumber}
}

\newcommand{\eqn}[1]{
\begin{equation}\begin{split}
#1
\end{split}\end{equation}
}

\begin{document}
\LARGE
\begin{center}
\settextfont{IranNastaliq}

به نام زیبایی

%\begin{figure}[h]
%\centering
%\includegraphics[width=30mm]{kntu_logo.eps}
%\end{figure}

تمرینات سری ششم درس احتمال مهندسی

\end{center}
\hrulefill
\large

\Q

یک سکه‌ی سالم را پرتاب می‌کنیم. اگر رو بیاید، یک تاس را پرتاب کرده و عدد روی آن را یادداشت می‌کنیم. اگر سکه پشت بیاید، دو تاس را پرتاب کرده و جمع اعداد دو تاس را یادداشت می‌کنیم. احتمال آنکه عدد رو آمده برابر $n$ باشد چقدر است؟ (
$
2\le n\le 12
$
)

\Q

از کیسه‌ای که شامل 7 توپ سیاه و 10 توپ سفید است، 3 توپ به تصادف بیرون می‌آوریم. سپس از بین 3 توپ بیرون آمده، یکی را به تصادف بر می‌گزینیم. اگر بدانیم حداقل یک توپ از 3 توپ بیرون آمده سیاه است، احتمال آنکه توپ انتخابی از بین این 3 توپ، سفید باشد چقدر است؟

\Q

دو کیسه در اختیار داریم. کیسه‌ی 1 شامل 7 توپ سیاه و 10 توپ سفید و کیسه‌ی 2 شامل 4 توپ سیاه، 2 توپ سفید و 3 توپ قرمز است. ابتدا یکی از کیسه‌ها را به تصادف انتخاب کرده و سپس، توپی از آن به تصادف بیرون می‌آوریم. اگر بدانیم توپ انتخابی سفید نیست، با چه احتمالی از کیسه‌ی 2 انتخاب شده است؟
\end{document}