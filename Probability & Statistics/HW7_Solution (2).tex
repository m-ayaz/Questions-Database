\documentclass[10pt,letterpaper]{article}
\usepackage{toolsper}
%\settextfont{B Nazanin}
\usepackage{lipsum}
\setlength{\parindent}{0mm}
\setlength{\parskip}{3mm}
\newcommand{\pic}[2]{
\begin{center}
\includegraphics[width=#2]{#1}
\end{center}
}
\begin{document}
\Large
\begin{center}
به نام او

پاسخ تمرینات سری هفتم درس احتمال مهندسی
\hl
\end{center}
سوال 1) هر CDF باید سه خاصیت داشته باشد:
\[
\begin{split}
&F(-\infty)=0
\\&F(\infty)=1
\\&F(x) \text{صعودی}
\end{split}
\]
الف) شرط 
$
F(\infty)=1
$
هنگامی برآورده می شود که 
$$
\lim_{x\to \infty}1-e^{-kx^2}=1\implies k>0
$$
از طرفی به ازای هر 
$
k>0
$
، تابع 
$
1-e^{-kx^2}
$
روی مقادیر 
$
x>0
$
صعودی است؛ پس محدوده‌ی مناسب $k$، 
$
(0,\infty)
$
خواهد بود.

ب) صعودی بودن $F(x)$ الزام می دارد که 
$
k\ge 0
$. از طرفی، نباید مقدار CDF هیچ کجا از بازه‌ی $[0,1]$ تجاوز کند. در این صورت 
$
k\le 1
$
. 
همچنان یک شرط دیگر باید برآورده شود و آن پیوستگی از راست CDF در تمام نقاط است. این نوع پیوستگی در $x=1$ تنها زمانی رخ می دهد که $k=1$. پس بازه‌ی مناسب $k$ برابر 
$
\{1\}
$
است.

پ) از آنجا که 
$
F(\infty)=1
$
، باید داشته باشیم 
$
k=1
$
؛ ولی صعودی بودن $F(x)$ نقض می شود؛ زیرا
$$
0<x<{1\over 2}\implies x-x^2 \text{صعودی اکید}\implies e^{x-x^2}\text{صعودی اکید}\implies 1-e^{x-x^2}\text{نزولی اکید}
$$
پس در این مورد، k هیچ مقداری را نمی تواند داشته باشد.

ت) شرایط 
$
F(-\infty)=0
$
و
$
F(\infty)=1
$
به ازای 
$
k\ne0
$
برآورده می شوند. صعودی بودن نیز به ازای 
$
k\ge0
$
رخ می دهد؛ پس بازه‌ی مناسب برابر است با 
$
(0,\infty)
$
.

ث) با توجه به اینکه 
$
F(-\infty)=\cos{\pi\over k}=0
$
باید داشته باشیم
$$
{\pi\over k}=l\pi+{\pi\over 2}\implies k={2\over 2l+1}\quad,\quad l\in \Bbb Z
$$
از طرفی چون به ازای 
$
k<0
$
، تابع 
$
{\pi\over e^x+k}
$
دارای مجانب عمودی خواهد بود، بنابراین باید داشته باشیم
$$
k={2\over 2l+1}\quad,\quad l\in \Bbb Z^{\ge 0}
$$
%$
%
%$

سوال 2) الف) $F(x^2)$ در $-\infty$ دارای مقدار 1 (بجای 0) می شود؛ پس نمی‌تواند CDF باشد.

ب، پ، ت و ث) این توابع تمام شرایط CDF را برآورده می کنند.

سوال 3) با توجه به جبر مجموعه ها، با تعریف
$$
A=(-\infty,2]
$$
$$
B=(-\infty,1]
$$
خواهیم داشت
$$
\Pr\{1<x\le 2\}=\Pr\{A-B\}=\Pr\{A\}-\Pr\{B\}=F(2)-F(1)
$$
الف)
$$
\Pr\{1<x\le 2\}=e^{-k}-e^{-4k}
$$
ب)
$$
\Pr\{1<x\le 2\}=0
$$
پ) (!!)

ت) 
$$
\Pr\{1<x\le 2\}={e^2\over e^2+k}-{e\over e+k}
$$
ث)
$$
\Pr\{1<x\le 2\}=\cos {\pi\over e^2+k}-\cos {\pi\over e+k}
$$
\end{document}