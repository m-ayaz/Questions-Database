\documentclass{article}

\usepackage{amsmath,amssymb,geometry}
\usepackage{xepersian}

\setlength{\parindent}{0pt}
\setlength{\parskip}{3mm}

\newcounter{questionnumber}
\setcounter{questionnumber}{1}

\newcommand{\Q}{
\textbf{سوال \thequestionnumber)}
\stepcounter{questionnumber}
}

\newcommand{\eqn}[1]{
\[\begin{split}
#1
\end{split}\]
}

\begin{document}
\LARGE
\begin{center}
\settextfont{IranNastaliq}

به نام زیبایی

%\begin{figure}[h]
%\centering
%\includegraphics[width=30mm]{kntu_logo.eps}
%\end{figure}

کوئیز 2 درس احتمال مهندسی

\end{center}
\hrulefill
\large

\Q

فرض کنید که برنامه ي نوشته اید که اعداد 1 تا 9 را به صورت کاملا تصادفی در هر بار اجرا در 3 جایگاه (سه رقم) چاپ می کند. احتمال ظاهر شده اعداد با هر سه رقم فرد را محاسبه کنید.

پاسخ:

طبق اصل ضرب، تعداد تمام اعداد سه رقمی متمایزی که می توان به این روش ساخت، برابر است با
$
9^3=729
$.
تعداد ارقام فرد از بین اعداد 1 تا 9، برابر 5 است (ارقام 1، 3، 5، 7 و 9). در نتیجه، تعداد اعداد سه رقمی ای که تمام ارقام آن فرد هستند را می توان دوباره طبق اصل ضرب به
$
5^3=125
$
طریق ممکن ساخت. بنابراین احتمال مطلوب عبارتست از
$$
p=\frac{n(A)}{n(S)}=\frac{125}{729}
$$

\end{document}