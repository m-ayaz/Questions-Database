\documentclass{article}

\usepackage{amsmath,amssymb,geometry}
\usepackage{xepersian}

\setlength{\parindent}{0pt}
\setlength{\parskip}{3mm}

\newcounter{questionnumber}
\setcounter{questionnumber}{1}

\newcommand{\Q}{
\textbf{سوال \thequestionnumber)}
\stepcounter{questionnumber}
}

\newcommand{\eqn}[1]{
\begin{equation}\begin{split}
#1
\end{split}\end{equation}
}

\begin{document}
\LARGE
\begin{center}
\settextfont{IranNastaliq}

به نام زیبایی

%\begin{figure}[h]
%\centering
%\includegraphics[width=30mm]{kntu_logo.eps}
%\end{figure}

تمرینات سری چهارم درس احتمال مهندسی

\end{center}
\hrulefill
\large

\Q

تاس سالمی را 3 بار پرتاب می‌کنیم (احتمال رو آمدن هر عدد از 1 تا 6، برابر $\frac{1}{6}$ است).

الف) احتمال آنکه از این 3 بار، حداقل 2 بار عدد زوج بیاید چقدر است؟

ب) احتمال آنکه مجموع اعداد رو آمده در این 3 پرتاب برابر 5 باشد، چقدر است؟

پ) احتمال رو آمدن مضرب 3 در پرتاب اول چقدر است؟

\Q

در کیسه ای، 10 توپ آبی و 7 توپ قرمز موجود است. دو توپ به تصادف و بدون جایگذاری بر می‌داریم.

الف) اگر توپهای همرنگ نامتمایز باشند، تعداد حالات برداشتن دو توپ غیرهمرنگ چقدر است؟

ب) اگر توپهای همرنگ نامتمایز باشند، احتمال برداشتن دو توپ غیرهمرنگ چقدر است؟

پ) اگر توپهای آبی را از 1 تا 10 و توپهای قرمز را از 1 تا 7 شماره گذاری کنیم، احتمال آنکه توپ آبی شماره 4 و توپ آبی شماره 3 برداشته شود چقدر است؟

ت) اگر توپهای آبی را از 1 تا 10 و توپهای قرمز را از 1 تا 7 شماره گذاری کنیم، آیا احتمال برداشتن دو توپ غیرهمرنگ، با مقدار بدست آمده در قسمت الف تفاوت می‌کند؟ توضیح دهید.

\Q

قسمتهای ب)، پ) و ت) مسئله‌ی پیش را با فرض داشتن جایگذاری حل کنید؛ یعنی زمانی که توپ اول را برداشتیم، رنگ آن را یادداشت کرده، آنرا به کیسه بازگردانده و سپس توپ دوم را بر می‌داریم.
\end{document}