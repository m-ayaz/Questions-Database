\documentclass{article}

\usepackage{amsmath,amssymb,geometry}
\usepackage{xepersian}

\setlength{\parindent}{0pt}
\setlength{\parskip}{3mm}

\newcounter{questionnumber}
\setcounter{questionnumber}{1}

\newcommand{\Q}{
\textbf{سوال \thequestionnumber)}
\stepcounter{questionnumber}
}

\begin{document}
\LARGE
\begin{center}
\settextfont{IranNastaliq}

به نام زیبایی

%\begin{figure}[h]
%\centering
%\includegraphics[width=30mm]{kntu_logo.eps}
%\end{figure}

تمرینات سری اول درس احتمال مهندسی

\end{center}
\hrulefill
\large

\Q

موارد زیر را در یک مسئله‌ی احتمالاتی تعریف کنید:

الف) فضای نمونه

ب) پیشامد (واقعه)

پ) پیشامد (واقعه‌ی) ساده

\Q

آیا فضای نمونه در یک مسئله‌ی احتمالاتی، تنها مجموعه با احتمال یک است؟ پاسخ را برای هر دو حالتی که فضای نمونه متناهی یا نامتناهی باشد شرح دهید و در صورت لزوم، مثال بزنید.

\Q

اگر $A$ فضای نمونه‌ی آزمایش پرتاب سکه با رخدادهای پشت و رو و $B$ فضای نمونه‌ی پرتاب تاس با اعداد طبیعی 1 تا 6 باشد:

الف) حاصلضرب دکارتی $A$ و $B$ (
$A\times B$
)
را به دست آورید. این مجموعه، فضای نمونه‌ی چه آزمایشی است؟

ب) دو زیر مجموعه‌ی 3 عضوی از مجموعه‌ی 
$
A\times B
$
برگزینید که با یکدیگر ناسازگار باشند. آیا می‌توانید همین کار را برای زیرمجموعه‌های 7 عضوی تکرار کنید؟ چرا؟

\Q

با بهره گیری از جبر مجموعه ها و اصول کولموگروف احتمال، نشان دهید اگر $A$، $B$ و $C$ سه مجموعه‌ باشند به طوری که 
$
P(B\cap C)=0
$
، در این صورت
$$
P\left\{
A\cap(B\cup C)
\right\}
=
P\left\{
A\cap B
\right\}
+
P\left\{
A\cap C
\right\}
$$







\end{document}