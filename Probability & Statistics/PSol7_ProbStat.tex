\documentclass{article}

\usepackage{amsmath,amssymb,geometry,tikz}
\usepackage{xepersian}

\setlength{\parindent}{0pt}
\setlength{\parskip}{3mm}

\newcounter{questionnumber}
\setcounter{questionnumber}{1}

\newcommand{\Q}{
\textbf{سوال \thequestionnumber)}
\stepcounter{questionnumber}
}

\newcommand{\eqn}[1]{
\begin{equation}\begin{split}
#1
\end{split}\end{equation}
}

\begin{document}
\LARGE
\begin{center}
\settextfont{IranNastaliq}

به نام زیبایی

%\begin{figure}[h]
%\centering
%\includegraphics[width=30mm]{kntu_logo.eps}
%\end{figure}

پاسخ تمرینات سری هفتم درس احتمال مهندسی

\end{center}
\hrulefill
\large

\Q

دو حالت در نظر می‌گیریم:

حالت 1) لینک DZ خراب باشد. در این صورت، احتمال وجود مسیر بین A و Z برابر است با
$$
p[1-(1-p)(1-p^2)^2]=p-p(1-p)(1-p^2)^2
$$

حالت 2) لینک DZ سالم باشد. در این صورت، می‌توان نودهای D و Z را یکی در نظر گرفت (چرا که رسیدن به D، رسیدن به Z را نتیجه می‌دهد). در این صورت، احتمال وجود مسیر بین A و Z برابر است با
\eqn{
&1-(1-p)\left\{1-\left[1-(1-p)^2\right]\left[1-(1-p)(1-p^2)\right]\right\}
\\&=1-(1-p)\left\{1-(2p-p^2)(p+p^2-p^3)\right\}
\\&=p+p^2(1-p)(2-p)(1+p-p^2)
}{}
در نتیجه، احتمال وجود داشتن مسیر از A تا Z برابر است با
\eqn{
&
(1-p)[p+p^2(1-p)(2-p)(1+p-p^2)]+p^2+p^3(1-p)(2-p)^2
\\&=
p+p^2(1-p)^2(2-p)(1+p-p^2)+p^3(1-p)(2-p)^2
}{}

%حالت 1) لینک CD خراب باشد. در این صورت، احتمال وجود مسیر بین A و Z برابر است با
%$$
%1-(1-p^2)\left[1-p(1-p)(1-p^2)\right]
%$$
%
%اگر هر دو لینک CZ و DZ خراب باشند، مسیری از A تا Z وجود ندارد. در نتیجه، باید حداقل یک لینک از این 2 لینک سالم باشد. سه حالت ممکن است:
%
%1) لینک CZ سالم و لینک DZ خراب باشد.
%
%2) لینک CZ خراب و لینک DZ سالم باشد.
%
%3) لینک CZ سالم و لینک DZ سالم باشد.
%
%در حالت اول، پیشامد وجود داشتن مسیر از A تا Z، معادل با وجود داشتن مسیر از A تا C است. مکمل پیشامد اخیر، وجود نداشتن مسیر از A تا C است. چون از A تا C سه مسیر وجود دارد، هر سه باید خراب باشند. خراب بودن هر مسیر، مکمل سالم بودن آن است که سالم بودن هر مسیر نیز، سالم بودن تمام لینک‌های آن را می‌طلبد. با توضیحات پیش، برای محاسبه‌ی احتمال وجود داشتن مسیر از A تا C، مشروط بر اینکه لینک CZ سالم و لینک DZ خراب باشد، برابر است با
%$$
%1-(1-p^2)(1-p)(1-p^2).
%$$
%
%در حالت دوم، پیشامد وجود داشتن مسیر از A تا Z، معادل با وجود داشتن مسیر از A تا D است. احتمال این پیشامد، به طریق مشابه برابر است با
%$$
%1-(1-p)\left\{1-p\left[1-(1-p)(1-p^2)\right]\right\}.
%$$
%
%حالت سوم، معادل این است که دو نود C و D را به هم بچسبانیم (و لینک CD را از بین ببریم). در این صورت، احتمال وجود مسیر از A تا نود C (که همان D است) برابر است با
%$$
%1-(1-p)^2(1-p^2).
%$$
%
%در نتیجه، احتمال وجود مسیر از A تا Z به صورت زیر است:
%\eqn{
%p(1-p)[1-(1-p^2)(1-p)(1-p^2)+1-(1-p)\left\{1-p\left[1-(1-p)(1-p^2)\right]\right\}]
%}{}
%
%ابتدا باید تمام پیشامدها (مسیرها)ی ممکن از A تا Z را پیدا کنیم. با شمارش ساده‌ای، این مسیرها عبارتند از:
%
%ABCZ
%
%ACZ
%
%ADCZ
%
%ABCDZ
%
%ACDZ
%
%ADZ
%
%احتمال سالم بودن این مسیرها، به ترتیب برابراست با
%
%$
%p^3
%$
%
%$
%p^2
%$
%
%$
%p^3
%$
%
%$
%p^4
%$
%
%$
%p^3
%$
%
%$
%p^2
%$
%
%و احتمال مطلوب (معادل با سالم بودن حداقل یک مسیر از )

%\begin{figure}[h]
%\centering
%\includegraphics[width=50mm]{PS7_ProbStat_Q1.eps}
%\end{figure}

\Q

الف)
$$
\binom{10}{3}(\frac{1}{2})^3(\frac{1}{2})^7=\frac{120}{1024}
$$

ب) مکمل پیشامد مطلوب، آن است که دقیقا 1 یا صفر بار خط بیاید. در نتیجه احتمال مطلوب برابر است با
$$
1-\binom{10}{0}(\frac{1}{2})^0(\frac{1}{2})^{10}-\binom{10}{1}(\frac{1}{2})^1(\frac{1}{2})^9=1-\frac{1}{1024}-\frac{10}{1024}=\frac{1013}{1024}
$$

پ) اگر بدانیم در 5 پرتاب اول خط آمده است، احتمال مطلوب، معادل با احتمال رو آمدن دقیقأ دو خط در 5 پرتاب باقیمانده است. در نتیجه احتمال مطلوب برابر است با
$$
\binom{5}{2}(\frac{1}{2})^2(\frac{1}{2})^3=\frac{10}{32}
.
$$
از روش احتمال شرطی نیز به همین پاسخ می‌رسیم.

\Q

الف) جمع این 6 پرتاب زمانی 8 می‌شود که یا دقیقأ دو بار 2 و 4 بار 1 یا دقیقا یک بار 3 و 5 بار 1 داشته باشیم. در این صورت، احتمال مطلوب برابر است با
$$
\binom{6}{2}(\frac{1}{6})^2(\frac{1}{6})^4
+
\binom{6}{1}(\frac{1}{6})^1(\frac{1}{6})^5
=\frac{21}{46656}
$$

ب)
$$
\frac{6!}{6^6}=\frac{5!}{6^5}=\frac{5}{324}
$$

\end{document}