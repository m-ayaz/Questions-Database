\documentclass{article}

\usepackage{amsmath,amssymb,geometry,tikz}
\usepackage{xepersian}

\setlength{\parindent}{0pt}
\setlength{\parskip}{3mm}

\newcounter{questionnumber}
\setcounter{questionnumber}{1}

\newcommand{\Q}{
\textbf{سوال \thequestionnumber)}
\stepcounter{questionnumber}
}

\newcommand{\eqn}[1]{
\begin{equation}\begin{split}
#1
\end{split}\end{equation}
}

\begin{document}
\LARGE
\begin{center}
\settextfont{IranNastaliq}

به نام زیبایی

%\begin{figure}[h]
%\centering
%\includegraphics[width=30mm]{kntu_logo.eps}
%\end{figure}

تمرینات سری نهم درس احتمال مهندسی

\end{center}
\hrulefill
\large

\Q

با استفاده از قضیه‌ی تقریب پواسن، کمیت های
$
\binom{n}{k}p^k(1-p)^{n-k}
$
و
$
e^{-np}\frac{(np)^k}{k!}
$
را به ازای مقادیر مختلف داده شده برای
$
n
$،
$
p
$
و
$
k
$
محاسبه کرده و خطای نسبی را در هر مورد به دست آورید. از مقایسه‌ی خطاهای نسبی چه نتیجه‌ای می‌گیرید؟

الف)
$
n=300\quad,\quad p=0.01\quad,\quad k=3
$

ب)
$
n=30\quad,\quad p=0.8\quad,\quad k=24
$

\Q

برای هر یک از توابع زیر، محدوده مقادیر $k$ را به گونه ای تعیین کنید که تابع مورد نظر، یک تابع توزیع انباشته باشد. سپس، چگالی احتمال را بیابید.

الف)
$
F(x)=\frac{1}{e^{-kx}+1}
$

ب)
$
F(x)=\begin{cases}
0&,\quad x<0\\
1-e^{-x-k\sin x}&,\quad x\ge0
\end{cases}
$

پ)
$
F(x)=\begin{cases}
0&,\quad x<0\\
1+xe^{-kx}&,\quad x\ge 0
\end{cases}
$

ت)
$
F(x)=\begin{cases}
0&,\quad x<0\\
\frac{1}{2}&,\quad 0\le x<1\\
1-\frac{1}{2}e^{k-kx}&,\quad x\ge 1
\end{cases}
$

\Q

برای بخش های الف و ت سوال پیش، مقادیر میانه، صدکهای 25ام و 75ام و همچنین احتمال‌های
$
\Pr\{X=0\}
$
و
$
\Pr\{0< X\le 2\}
$
را بیابید.

\Q

یک تاس را پرتاب می‌کنیم. اگر زوج آمد، عدد آن را یادداشت می‌کنیم و اگر فرد آمد، عددی را به تصادف از بازه‌ی 
$
[1,6]
$
انتخاب کرده و آن را یادداشت می‌کنیم. اگر متغیر تصادفی 
$
X
$،
نشان دهنده‌ی عدد یادداشت شده باشد، چگالی احتمال و تابع توزیع انباشته‌ی آن را به دست آورده و مقدار 
$
\Pr\{1\le X\le 3\}
$
را بیابید.

\Q

یک تابع چگالی احتمال 
$
f(x)
$،
همواره مثبت بوده و به ازای عدد حقیقی و داده‌شده‌ی $a$، دارای خاصیت زیر است:
$$
f(x)=f(a-x).
$$
در این صورت، میانه‌ی آن را بیابید.

%\lr{
%\begin{table}[h]
%\LARGE
%\begin{tabular}{|c|c|c|c|}
%1&2&3&4\\\hline
%\end{tabular}
%\begin{tabular}{|c|c|}
%1&2\\\hline
%1&2\\\hline
%\end{tabular}
%\end{table}
%}

\end{document}