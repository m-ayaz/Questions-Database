\documentclass{article}

\usepackage{amsmath,amssymb,geometry}
\usepackage{xepersian}

\setlength{\parindent}{0pt}
\setlength{\parskip}{3mm}

\newcounter{questionnumber}
\setcounter{questionnumber}{1}

\newcommand{\Q}{
\textbf{سوال \thequestionnumber)}
\stepcounter{questionnumber}
}

\newcommand{\eqn}[1]{
\begin{equation}\begin{split}
#1
\end{split}\end{equation}
}

\begin{document}
\LARGE
\begin{center}
\settextfont{IranNastaliq}

به نام زیبایی

%\begin{figure}[h]
%\centering
%\includegraphics[width=30mm]{kntu_logo.eps}
%\end{figure}

تمرینات سری سوم درس احتمال مهندسی

\end{center}
\hrulefill
\large

\Q

در یک کتابخانه، سه کتاب فیزیک، دو کتاب رمان و چهار کتاب روان شناسی موجود است. مطلوبست تعداد حالات چیدن این کتاب ها در یک قفسه کنار هم چنانچه:

الف) تمام کتابهای هم نوع متمایز باشند (مثلا ترتیب دو کتاب رمان نسبت به هم مهم باشد).

ب) تمام کتابهای هم نوع نامتمایز باشند (مثلا ترتیب دو کتاب رمان نسبت به هم مهم نباشد).

\Q

در یک گل فروشی، سه گل بنفشه، چهار گل رز و 2 گل اقاقیا موجود است. به چند طریق می توان دسته گلی متشکل از 3 گل از بین گل های موجود برگزید، اگر گلهای هم نوع نامتمایز باشند؟

\Q

اعضای یک شرکت شامل 1 مدیرعامل، 2 منشی، 1 حسابدار و 5 نفر از سایر اعضای هیئت مدیره در یک میزگرد دارای 11 صندلی می‌نشینند. مطلوبست تعداد حالاتی که

الف) هر دو منشی کنار هم باشند.

ب) هیچ یک از اعضای هیئت مدیره (به جز مدیرعامل)، مجاور مدیرعامل نباشد.

پ) حسابدار کنار مدیرعامل بنشیند و تمام اعضای هیئت مدیره (به جز مدیرعامل) کنار هم باشند.

(راهنمایی: برای حل این سوال، به تمایز یا عدم تمایز اعضای هیئت مدیره یا منشی ها دقت کنید. آیا منطقی است متمایز باشند یا نباشند؟ همچنین دقت کنید که همواره دو صندلی از میزگرد خالی می مانند و باید در شمارش حالات محاسبه شوند.)

\end{document}