\documentclass{article}

\usepackage{amsmath,amssymb,geometry}
\usepackage{xepersian}

\setlength{\parindent}{0pt}
\setlength{\parskip}{3mm}

\newcounter{questionnumber}
\setcounter{questionnumber}{1}

\newcommand{\Q}{
\textbf{سوال \thequestionnumber)}
\stepcounter{questionnumber}
}

\newcommand{\eqn}[1]{
\[\begin{split}
#1
\end{split}\]
}

\begin{document}
\LARGE
\begin{center}
\settextfont{IranNastaliq}

به نام زیبایی

%\begin{figure}[h]
%\centering
%\includegraphics[width=30mm]{kntu_logo.eps}
%\end{figure}

کوئیز 3 درس احتمال مهندسی

\end{center}
\hrulefill
\large

\Q

از کیسه‌ای که دارای 40 مهره سیاه و 60 مهره قرمز است، 20 مهره بر می‌داریم. با چه احتمالی، از این 20 مهره، 5 مهره سیاه و 15 مهره قرمزند؟

پاسخ:

تعداد حالات برداشتن 20 توپ، برابر 
$
\binom{100}{20}
$
بوده و تعداد حالات مطلوب، برابر
$
\binom{40}{5}\binom{60}{15}
$
است؛ لذا احتمال مطلوب برابر 
$
\frac{\binom{40}{5}\binom{60}{15}}{\binom{100}{20}}
$
خواهد بود.
\end{document}