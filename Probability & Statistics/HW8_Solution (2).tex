\documentclass[10pt,letterpaper]{article}
\usepackage{toolsper}
%\settextfont{B Nazanin}
\usepackage{lipsum}
\setlength{\parindent}{0mm}
\setlength{\parskip}{3mm}
\newcommand{\pic}[2]{
\begin{center}
\includegraphics[width=#2]{#1}
\end{center}
}
\begin{document}
\Large
\begin{center}
به نام او

پاسخ تمرینات سری هشتم درس احتمال مهندسی
\hl
\end{center}
{\color{red}
نکته مهم:
$$
\int_a^b\delta(x-c)dx=\begin{cases}
1&,\quad a<c<b\\
0&,\quad \text{در غیر این صورت}
\end{cases}
$$
}

سوال 1) الف) به وضوح باید $k$ مثبت باشد در غیر این صورت این تابع همواره سطح زیر نامحدود خواهد داشت. از طرفی
$$
\int_{-\infty}^\infty f(x)dx=\int_1^\infty {1\over x^k}dx=\begin{cases}
\infty&,\quad0<k\le1\\
{1\over k-1}&,\quad k>1
\end{cases}
$$
پس $k=2$.

ب) این تابع نیز تمام شرایط pdf را برآورده می کند؛ به شرط آن که سطح زیر آن واحد باشد. در این صورت:
$$
\int_{-\infty}^\infty f(x)dx=\int_0^\infty kxe^{-x}dx=-k(x+1)e^{-x}|_0^\infty=k=1
$$
پس k فقط باید برابر 1 باشد.

پ) به ازای $k>\pi$، pdf مقادیر منفی را نیز اختیار می کند؛ پس $k\le \pi$. از طرفی
$$
\int_{-\infty}^\infty f(x)dx=\int_0^k\sin xdx=1-\cos k=1
$$
پس $k={\pi\over 2}$.

ت)
$
f(x)=\begin{cases}
k\delta(x-1)&,\quad x=1\\
x&,\quad 0<x<1\\
0&,\quad \text{سایر جاها}
\end{cases}
$
(به عبارت دیگر، تابع در نقطه‌ی $x=1$ دارای ضربه‌ای به مساحت k است)
ابتدا باید $k$ مثبت باشد تا مقدار چگالی احتمال همواره نامنفی باشد. همچنین
$$
\int_{-\infty}^\infty f(x)dx=\int_{1^-}^{1^+} k\delta(x-1)dx+\int_0^1 xdx=k+0.5=1
$$
پس مقدار k باید برابر $0.5$ باشد.

ث) سطح زیر این چگالی همواره برابر 1 است و فقط هنگامی نامنفی می شود که 
$
0\le k\le 1
$.

سوال 2) مکمل این حالت زمانی رخ می دهد که بیش از 65 قطعه خراب شوند. احتمال خرابی هر قطعه در بازه‌ی $
\left(0,{T\over 4}\right)
$
برابر است با:
$$
p=\int_{-\infty}^{T\over 4}f(x)dx=
\int_{0}^{T\over 4}{1\over T}e^{-{x\over T}}dx=
\int_{0}^{1\over 4}e^{-x}dx=1-e^{-0.25}
$$
بنابراین احتمال پیشامد مطلوب برابر است با:
\[
\begin{split}
P&=1-[\binom{70}{66}p^{66}(1-p)^4\\&+\binom{70}{67}p^{67}(1-p)^3
+\binom{70}{68}p^{68}(1-p)^2
\\&+\binom{70}{69}p^{69}(1-p)
+\binom{70}{70}p^{70}
]
\end{split}
\]

سوال 3) اگر تابع توزیع تجمعی را با $F(x)$ نشان دهیم در این صورت:
$$
\Pr\{X=1\}=F(1)-F(1^-)
$$
\[
\begin{split}
\Pr\{X<{1\over 2}\}&=\Pr\{X\le{1\over 2}\}-\Pr\{X={1\over 2}\}\\&=F(0.5)-[F(0.5)-F(0.5^-)]=F(0.5^-)
\end{split}
\]
الف)
$$
\Pr\left\{X<{1\over 2}\right\}=0\quad ,\quad \Pr\{X=1\}=0
$$
ب)
$$
\Pr\left\{X<{1\over 2}\right\}=1-{3\over 2}e^{-{1\over 2}}\quad ,\quad \Pr\{X=1\}=0
$$
پ)
$$
\Pr\left\{X<{1\over 2}\right\}=1-\cos{1\over 2}\quad ,\quad \Pr\{X=1\}=0
$$
ت)
$$
\Pr\left\{X<{1\over 2}\right\}={1\over 8}\quad ,\quad \Pr\{X=1\}={1\over 2}
$$
ث)
$$
\Pr\left\{X<{1\over 2}\right\}=1-k\quad ,\quad \Pr\{X=1\}=k
$$

سوال 4) طبق تعریف، برای صدک-u داریم:
$$
x_u=\inf\{x\ \ |\ \ F(x)=y\}
$$
که $F(x)$ تابع توزیع تجمعی متغیر تصادفی است. به طور ساده تر، باید معادله‌ی 
$
F(x_u)=u
$
را حل کنیم. بنابراین:

الف)
$$
F(x)=\begin{cases}
1&,\quad 0\le x\le 1\\
x&,\quad x>1\\
0&,\quad x<0
\end{cases}
$$
و در نتیجه
$
x_u=u
$
.

ب)
$$
F(x)=\begin{cases}
0&,\quad x<0\\
1-e^{-2x}&,\quad x\ge 0
\end{cases}
$$
پس
$
x_u={1\over 2}\ln {1\over 1-u}
$
\end{document}