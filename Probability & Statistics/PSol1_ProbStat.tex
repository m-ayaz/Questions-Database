\documentclass{article}

\usepackage{amsmath,amssymb,geometry}
\usepackage{xepersian}

\setlength{\parindent}{0pt}
\setlength{\parskip}{3mm}

\newcounter{questionnumber}
\setcounter{questionnumber}{1}

\newcommand{\Q}{
\textbf{سوال \thequestionnumber)}
\stepcounter{questionnumber}
}

\newcommand{\eqn}[1]{
\begin{equation}\begin{split}
#1
\end{split}\end{equation}
}

\begin{document}
\LARGE
\begin{center}
\settextfont{IranNastaliq}

به نام زیبایی

%\begin{figure}[h]
%\centering
%\includegraphics[width=30mm]{kntu_logo.eps}
%\end{figure}

پاسخ تمرینات سری اول درس احتمال مهندسی

\end{center}
\hrulefill
\large

\Q

الف) فضای نمونه عبارتست از مجموعه‌ی تمام برآمدها(رخدادها)یی که می‌توانند در یک مسئله‌ی احتمالاتی رخ دهند. به طور مثال، فضای نمونه‌ی پرتاب تاس،
$
\{1,2,3,4,5,6\}
$
است.


ب) به هر زیرمجموعه از فضای نمونه، یک پیشامد یا واقعه گفته می‌شود. در پرتاب تاس، واقعه‌ی روآمدن عدد زوج معادل مجموعه‌ی 
$
\{2,4,6\}
$
است.

پ) به هر زیرمجموعه‌ی تک عضوی از فضای نمونه، یک پیشامد ساده یا برآمد گفته می‌شود. در پرتاب تاس، 6 برآمد وجود دارد.

\Q

در صورتی که فضای نمونه متناهی باشد، پاسخ مثبت است؛ زیرا هر برآمد دارای احتمال مثبت است و در نتیجه، احتمال رخداد هر زیرمجموعه کمتر از 1 خواهد بود. در حالتی که فضای نمونه نامتناهی باشد، حذف یک برآمد با احتمال رخداد صفر از فضای نمونه، تغییر در احتمال آن ایجاد نمی‌کند. به طور مثال، فرض کنید بخواهیم عددی حقیقی را به تصادف کامل از بازه‌ی 
$
[0,1]
$
برگزینیم. در این صورت، احتمال اینکه این عدد برابر $0.5$ نباشد برابر 1 است.

\Q

الف)
\eqn{
A\times B=\{
&
(\text{H},1),
(\text{H},2),
(\text{H},3),
(\text{H},4),
(\text{H},5),
(\text{H},6),
\\&
(\text{T},1),
(\text{T},2),
(\text{T},3),
(\text{T},4),
(\text{T},5),
(\text{T},6)
\}
}

این مجموعه، فضای نمونه‌ی آزمایش پرتاب توأم تاس و سکه است (``یک سکه و یک تاس را به طور همزمان پرتاب میکنیم...'').

ب) به طور مثال
\eqn{
&S_1=\{(\text{T},2),(\text{H},5),(\text{T},6)\}
\\&S_2=\{(\text{H},2),(\text{T},5),(\text{H},6)\}
}

نمی‌توان همین کار را برای زیرمجموعه‌های 7 عضوی تکرار کرد؛ چرا که طبق اصل لانه‌ی کبوتری، حداقل دو عضو تکراری در این دو زیرمجموعه وجود خواهد داشت.



\Q

می‌دانیم
\eqn{
P\left\{
A\cap(B\cup C)
\right\}
&=
P\left\{
[A\cap B]\cup[A\cap C]
\right\}
}
از طرفی
\eqn{
[A\cap B]\cap[A\cap C]
=
A\cap B\cap C
=
A\cap [B\cap C]
=
A\cap \emptyset
=
\emptyset
}
بنابراین طبق اصل سوم کولموگروف،
\eqn{
P\left\{
A\cap(B\cup C)
\right\}
&=
P\left\{
[A\cap B]\cup[A\cap C]
\right\}
\\&=
P\left\{
A\cap B
\right\}
+
P\left\{
A\cap C
\right\}
.
}




\end{document}