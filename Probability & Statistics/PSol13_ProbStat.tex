\documentclass{article}

\usepackage{amsmath,amssymb,geometry,tikz}
\usepackage{xepersian}

\setlength{\parindent}{0pt}
\setlength{\parskip}{3mm}

\newcounter{questionnumber}
\setcounter{questionnumber}{1}

\newcommand{\Q}{
\textbf{پرسش \thequestionnumber)}
\stepcounter{questionnumber}
}

\newcommand{\eqn}[1]{
\begin{equation}\begin{split}
#1
\end{split}\end{equation}
}

\begin{document}
\LARGE
\begin{center}
\settextfont{IranNastaliq}

به نام زیبایی

%\begin{figure}[h]
%\centering
%\includegraphics[width=30mm]{kntu_logo.eps}
%\end{figure}

پاسخ تمرینات سری سیزدهم درس احتمال مهندسی

\end{center}
\hrulefill
\large

\Q

\eqn{
E\{X^n\}&=\int_1^\infty\int_1^\infty x^n(xy-1)\exp(1-xy)dxdy
\\&=
\int_1^\infty x^n\int_1^\infty (xy-1)\exp(1-xy)dydx
\\&=
\int_1^\infty x^n (-y)\exp(1-xy)\Big|_{y=1}^{y=\infty} dx
\\&=
\int_1^\infty x^n \exp(1-x)dx
}
در نتیجه
\eqn{
&E\{X\}=\int_1^\infty x\exp(1-x)dx=2
\\&E\{X^2\}=\int_1^\infty x^2\exp(1-x)dx=5
\\&\implies \sigma^2_X= E\{X^2\}-E^2\{X\}=1
}

همچنین
\eqn{
\Phi_X(s)&=E\{\exp(sX)\}=\int_1^\infty\int_1^\infty \exp(sx)(xy-1)\exp(xy-1)dxdy
\\&=
\int_1^\infty \exp(sx)\int_1^\infty (xy-1)\exp(xy-1)dydx
\\&=
\int_1^\infty \exp(sx)(-y)\exp(1-xy)\Big|_{y=1}^{y=\infty} dx
\\&=
\int_1^\infty \exp(sx)\exp(1-x)dx
\\&=
\int_1^\infty \exp(1+(s-1)x)dx
\\&=
\frac{1}{s-1}\exp(1+(s-1)x)\Big|_1^\infty
\\&=
\frac{1}{1-s}\exp(s)\quad,\quad \Re\{s\}<1
}
و به دلیل تقارن
$$
\sigma^2_X=\sigma^2_Y\quad,\quad \Phi_X(s)=\Phi_Y(s).
$$


\eqn{
\Pr\{XY\le a\}=\begin{cases}
\int_1^a \int_1^\frac{a}{y}(xy-1)\exp(1-xy)dxdy&,\quad a\ge 1\\
0&,\quad a\le 1
\end{cases}
}
بنابراین برای $a\ge 1$ داریم:
\eqn{
\Pr\{XY\le a\}&=\int_1^a \int_1^\frac{a}{y}(xy-1)\exp(1-xy)dxdy
\\&=
\int_1^a (-x)\exp(1-xy)\Big|_{x=1}^{x=\frac{a}{y}}dy
\\&=
\int_1^a \exp(1-y)-\frac{a}{y}\exp(1-a)dy
\\&=
\int_1^a \exp(1-y)-\frac{a}{y}\exp(1-a)dy
\\&=
1-\exp(1-a)-a\exp(1-a)\ln a
}
در نتیجه برای $a\ge 1$:
$$
f_{XY}(a)=\begin{cases}
(a-1)e^{1-a}\ln a&,\quad a\ge 1\\
0&,\quad a\le 1
\end{cases}
.
$$

\eqn{
\Pr\{\max\{X,Y\}\le a\}=\Pr\{X\le a,Y\le a\}.
}

برای $a\le 1$ احتمال فوق برابر صفر است و برای $a\ge 1$ داریم:
\eqn{
\Pr\{X\le a,Y\le a\}&=\int_1^a\int_1^a (xy-1)\exp(1-xy)dxdy
\\&=\int_1^a(-x)\exp(1-xy)\Big|_1^ady
\\&=\int_1^a\exp(1-y)-a\exp(1-ay)dy
\\&=1-2\exp(1-a)+\exp(1-a^2)
}
در نتیجه
$$
f_{\max\{X,Y\}}(a)=\begin{cases}
2\exp(1-a)-2a\exp(1-a^2)&,\quad a\ge 1\\
0&,\quad a\le 1
\end{cases}
.
$$

\Q

الف) حجم زیر چگالی احتمال باید 1 باشد؛ پس:
\eqn{
&\int_{x^2+y^2\le 1}\alpha+2(\frac{1}{\pi}-\alpha)(x^2+y^2)dxdy=1
\implies
\\&
\int_0^{2\pi}\int_0^1\left[\alpha+2(\frac{1}{\pi}-\alpha)r^2\right]rdrd\phi=1
\implies
\\&
\int_0^{2\pi}\frac{\alpha}{2}+2(\frac{1}{\pi}-\alpha)\frac{1}{4}d\phi=1
\implies
\\&
1=1.
}
در نتیجه، به ازای هر مقدار از $\alpha$ حجم زیر، واحد خواهد بود. از طرفی، باید چگالی احتمال همواره نامنفی باشد؛ پس:
\eqn{
&
\forall x,y\quad,\quad f(x,y)\ge 0\iff \min_{x,y}f(x,y)\ge 0
\\&\implies 
\begin{cases}
\alpha+2(\frac{1}{\pi}-\alpha)\ge 0&,\quad \frac{1}{\pi}-\alpha\le 0\\
\alpha\ge 0&,\quad \frac{1}{\pi}-\alpha\ge 0
\end{cases}
\\&\implies 
\begin{cases}
\frac{2}{\pi}-\alpha\ge 0&,\quad \frac{1}{\pi}-\alpha\le 0\\
\alpha\ge 0&,\quad \frac{1}{\pi}-\alpha\ge 0
\end{cases}
\\&\implies 
\begin{cases}
\frac{1}{\pi}\le \alpha\le \frac{2}{\pi}\\
0\le \alpha\le \frac{1}{\pi}
\end{cases}
}

در نتیجه، محدوده‌ی مقادیر مجاز 
$
\alpha
$
برابر
$
[0,\frac{2}{\pi}]
$
خواهد بود.

ب) دو متغیر تصادفی 
$
X
$
و
$
Y
$
مستقل اند؛ اگر و تنها اگر
$$
f_{X,Y}(x,y)=f_{X}(x)f_{Y}(y).
$$
از طرفی،
\eqn{
f_X(x)&=\int_{-\sqrt{1-x^2}}^{\sqrt{1-x^2}} \alpha+2(\frac{1}{\pi}-\alpha)(x^2+y^2)dy
\\&=
\sqrt{1-x^2}\left[
\frac{8}{3}\left(\frac{1}{\pi}-\alpha\right)x^2+\frac{4}{3\pi}+\frac{2}{3}\alpha
\right]
\quad,\quad |x|\le 1.
}
به دلیل تقارن،
\eqn{
f_Y(y)&=
\sqrt{1-y^2}\left[
\frac{8}{3}\left(\frac{1}{\pi}-\alpha\right)y^2+\frac{4}{3\pi}+\frac{2}{3}\alpha
\right]
\quad,\quad |y|\le 1؛
}
در نتیجه به ازای هر مقدار $\alpha$ داریم
$
f_{X,Y}(x,y)\ne f_X(x)f_Y(y)
$.

برای تحقیق ناهمبستگی، به دلیل تقارن مرکزی،
\eqn{
E\{X\}=E\{Y\}=0.
}
از طرفی
\eqn{
E\{XY\}=&
\int_{x^2+y^2\le 1}xy[\alpha+2(\frac{1}{\pi}-\alpha)(x^2+y^2)]dxdy
=
0
}
پس این دو متغیر تصادفی همواره ناهمبسته اند.

پ) اگر $a=b=0$، در اینصورت
$
\Pr\{aX+bY\ge 0\}=1
$
. در غیر اینصورت، به دلیل تقارن دایروی چگالی احتمال خواهیم داشت:
\eqn{
\Pr\{aX+bY\ge 0\}=\Pr\{X\ge 0\}=\frac{1}{2}
}

همچنین
\eqn{
\Pr\{XY\ge 0\}&=\Pr\{X\ge 0,Y\ge 0\text{ or }X\le 0,Y\le 0\}
\\&=
\Pr\{X\ge 0,Y\ge 0\}+\Pr\{X\le 0,Y\le 0\}
\\&=\frac{1}{4}+\frac{1}{4}=\frac{1}{2}
}
\end{document}