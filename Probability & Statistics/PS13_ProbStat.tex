\documentclass{article}

\usepackage{amsmath,amssymb,geometry,tikz}
\usepackage{xepersian}

\setlength{\parindent}{0pt}
\setlength{\parskip}{3mm}

\newcounter{questionnumber}
\setcounter{questionnumber}{1}

\newcommand{\Q}{
\textbf{پرسش \thequestionnumber)}
\stepcounter{questionnumber}
}

\newcommand{\eqn}[1]{
\begin{equation}\begin{split}
#1
\end{split}\end{equation}
}

\begin{document}
\LARGE
\begin{center}
\settextfont{IranNastaliq}

به نام زیبایی

%\begin{figure}[h]
%\centering
%\includegraphics[width=30mm]{kntu_logo.eps}
%\end{figure}

تمرینات سری سیزدهم درس احتمال مهندسی

\end{center}
\hrulefill
\large

\Q

برای چگالی احتمال توأم زیر، مقادیر
$
\sigma_X^2
$،
$
\sigma_Y^2
$،
$
\Phi_X(s)
$،
$
\Phi_Y(s)
$
و چگالی احتمال متغیرهای تصادفی $XY$ و 
$
\max\{X,Y\}
$
 را محاسبه کنید.
$$
f_{X,Y}(x,y)=\begin{cases}
(xy-1)e^{1-xy}&,\quad x\ge 1,y\ge 1\\
0&,\quad \text{سایر جاها}
\end{cases}
$$

\Q

چگالی احتمال توأم زیر برای دو متغیر تصادفی $X$ و $Y$ داده شده است:
$$
f_{X,Y}(x,y)=\begin{cases}
\alpha+2(\frac{1}{\pi}-\alpha)(x^2+y^2)&,\quad x^2+y^2\le 1\\
0&,\quad \text{سایر جاها}
\end{cases}
$$

الف) محدوده‌ی مقادیر مجاز 
$
\alpha
$
را بیابید.

ب) به ازای چه مقدار از 
$
\alpha
$
، دو متغیر تصادفی 
$
X
$
و
$
Y
$
مستقل اند؟ ناهمبسته اند؟

پ) احتمال های
$
\Pr\{aX+bY\ge 0\}
$
و
$
\Pr\{XY\ge 0\}
$
را بیابید.


\end{document}