\documentclass[10pt,letterpaper]{article}
%\usepackage{toolsper}
\usepackage{amsmath,geometry,amssymb,xepersian}
\newcommand{\eqn}[2]{
\begin{equation}
\begin{split}
#1
\label{#2}
\end{split}
\end{equation}
}
%%%%%%%%%%%%%

%       \eqn{
%       x=x^2
%       }{label}

%%%%%%%%%%%%%
\newcommand{\feqn}[2]{
\begin{tcolorbox}[width=7in, colback=white]
\begin{equation}
\begin{split}
#1
\label{#2}
\end{split}
\end{equation}
\end{tcolorbox}
}
%%%%%%%%%%%%%%%
\newcommand{\hl}{
\begin{center}
\line(1,0){450}
\end{center}}
%%%%%%%%%%%%%%%
\newcommand{\qn}[2]{
\[
\begin{split}
#1
\label{#2}
\end{split}
\]
}
%\settextfont{B Nazanin}
\usepackage{lipsum}
\setlength{\parindent}{0mm}
\setlength{\parskip}{3mm}
\newcommand{\pic}[2]{
\begin{center}
\includegraphics[width=#2]{#1}
\end{center}
}
\begin{document}
\Large
\begin{center}
به نام او

تمرینات سری دوم شبیه سازی درس احتمال مهندسی

\hrulefill
\end{center}
سوال 1) (تعبیر تجربی چگالی احتمال) می دانید که تابع چگالی احتمال دارای شهود تجربی است؛ به این معنا که مقدار چگالی احتمال در هر نقطه از دامنه‌ی متغیر تصادفی، نشان دهنده‌ی وزن احتمالاتی آن نقطه است. برای به تصویر کشیدن این شهود، متغیر تصادفی گوسی با میانگین صفر و واریانس 1 را در نظر بگیرید. تابع
\texttt{
randn()
}
در متلب، تحققی از چنین متغیر تصادفی ای را برآورده می کند. به کمک این تابع، 100000 تحقق مستقل از این متغیر تصادفی را تولید کرده و به کمک دستور 
\texttt{
histogram()
}
آرایه‌ی 100000 تایی را رسم کنید. سپس با حفظ این نمودار، نمودار منحنی گوسی با میانگین صفر و واریانس 1 را رسم کرده و دو نمودار را با هم مقایسه کنید. مشاهدات خود را توضیح دهید.

نکته مهم!

پارامتر 
\texttt{
`Normalization'
}
را روی مقدار 
\texttt{
`pdf'
}
تنظیم کنید. برای این کار، دستور 
\texttt{
histogram()
}
را به صورت 
\texttt{
histogram(.....,'Normalization','pdf')
}
استفاده کنید.

سوال 2) در مبحث توابعی از یک متغیر تصادفی آموختید که اگر متغیر تصادفی $X$ دارای pdf مشخص باشد، آنگاه می توان pdf متغیر تصادفی $Y=g(X)$ را با تنظیم تابع $g(\cdot)$ روی هر تابع دلخواهی تنظیم کرد. ثابت می‌شود که اگر $X$ متغیر تصادفی یکنواخت در بازه‌ی $[0,1]$ باشد، متغیر تصادفی 
$
Y=-\ln (1-X)
$
دارای توزیع نمایی با پارامتر $\lambda=1$ و توزیع 
$
f_Y(y)=e^{-y}
$
است. برای اثبات این امر،

الف) 100000 تحقق از متغیر تصادفی یکنواخت در بازه‌ی $[0,1]$ را به کمک دستور 
\texttt{rand()}
تولید کرده و هیستوگرام آن را رسم کنید. اسم بردار 100000 تایی را $X$ بگذارید.

ب) از روی بردار $X$، بردار 100000 تایی $Y=-\ln(1-x)$ را محاسبه کرده و هیستوگرام آن را رسم کنید. سپس pdf متغیر تصادفی نمایی با پارامتر $\lambda=1$ را به همراه این هیستوگرام در یک نمودار نشان دهید. رابطه‌ی بین هیستوگرام و pdf توزیع نمایی را توجیه کنید.

برای رسم دو نمودار در یک شکل، از دستور 
\texttt{\lr{hold on}}
استفاده کنید.

\vspace{10mm}
\textit{
\hspace{10mm}
موفق و پیروز باشید
}

\textit{
\hspace{20mm}
سروش ضیایی و آرین ظروفی
}
\end{document}