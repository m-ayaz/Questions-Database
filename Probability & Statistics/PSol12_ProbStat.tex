\documentclass{article}

\usepackage{amsmath,amssymb,geometry,tikz}
\usepackage{xepersian}

\setlength{\parindent}{0pt}
\setlength{\parskip}{3mm}

\newcounter{questionnumber}
\setcounter{questionnumber}{1}

\newcommand{\Q}{
\textbf{سوال \thequestionnumber)}
\stepcounter{questionnumber}
}

\newcommand{\eqn}[1]{
\begin{equation}\begin{split}
#1
\end{split}\end{equation}
}

\begin{document}
\LARGE
\begin{center}
\settextfont{IranNastaliq}

به نام زیبایی

%\begin{figure}[h]
%\centering
%\includegraphics[width=30mm]{kntu_logo.eps}
%\end{figure}

پاسخ تمرینات سری دوازدهم درس احتمال مهندسی

\end{center}
\hrulefill
\large

\Q

الف)

از 
$
\iint_{\Bbb R^2}f(x,y)dxdy=1
$
نتیجه می شود:
\eqn{
&\int_0^1\int_0^1xy+kx+kydxdy=1\implies
\\&\int_0^1\frac{1}{2}y+k\frac{1}{2}+kydy=1\implies
\\&\frac{1}{4}+k\frac{1}{2}+k\frac{1}{2}=1\implies
\\&k=\frac{3}{4}.
}
همچنین به ازای این مقدار $k$ داریم
$
f(x,y)\ge 0
$.
برای محاسبه‌ی توزیع تجمعی توأم، توجه داریم که اگر 
$
x<0
$
یا 
$
y<0
$
یا هردو، در این صورت
$
F(x,y)=0
$
.
بنابراین 
$
x>0,y>0
$
و در نتیجه
\eqn{
F(x,y)&=
\int_0^{\min(x,1)}\int_0^{\min(y,1)} uv+\frac{3u}{4}+\frac{3v}{4}dudv
\\&=
\frac{\min(x,1)\cdot \min(y,1)}{4}+\frac{3\min(x,1)}{8}+\frac{3\min(y,1)}{8}
\\&=
\begin{cases}
\frac{xy}{4}+\frac{3x+3y}{8}&,\quad 0\le x\le 1,0\le y\le 1\\
\frac{x}{4}+\frac{3x+3}{8}&,\quad 0\le x\le 1,y\ge 1\\
\frac{y}{4}+\frac{3y+3}{8}&,\quad x\ge 1,0\le y\le 1\\
1&,\quad x\ge 1,y\ge 1\\
0&,\quad \text{سایر جاها}
\end{cases}
}

ب)

از 
$
\iint_{\Bbb R^2}f(x,y)dxdy=1
$
نتیجه می شود:
\eqn{
&\int_0^\frac{\pi}{6}\int_0^\frac{\pi}{2}k\sin(x+3y)dxdy=1\implies
\\&\int_0^\frac{\pi}{6}k\cos(3y)+k\sin(3y)dy=1\implies
\\&k\frac{1}{3}\sin(3y)\Big|_0^\frac{\pi}{6}-k\frac{1}{3}\cos(3y)\Big|_0^\frac{\pi}{6}=1\implies
\\&k=\frac{3}{2}
}
همچنین به ازای این مقدار $k$ داریم
$
f(x,y)\ge 0
$.
برای محاسبه‌ی توزیع تجمعی توأم، توجه داریم که اگر 
$
x<0
$
یا 
$
y<0
$
یا هردو، در این صورت
$
F(x,y)=0
$
.
بنابراین 
$
x>0,y>0
$
و در نتیجه
\eqn{
F(x,y)&=
\int_0^{\min(y,\frac{\pi}{6})}\int_0^{\min(x,\frac{\pi}{2})} \frac{3}{2}\sin(u+3v)dudv
\\&=
\int_0^{\min(y,\frac{\pi}{6})} \frac{3}{2}\cos(u+3v)\Big|_{\min(x,\frac{\pi}{2})}^0dv
\\&=
\int_0^{\min(y,\frac{\pi}{6})} \frac{3}{2}\cos(3v)-\frac{3}{2}\cos\left[\min(x,\frac{\pi}{2})+3v\right]dv
\\&=
\frac{1}{2}\sin(3v)\Big|_0^{\min(y,\frac{\pi}{6})}-\frac{1}{2}\sin\left[\min(x,\frac{\pi}{2})+3v\right]\Big|_0^{\min(y,\frac{\pi}{6})}
\\&=
\frac{1}{2}\sin(\min(3y,\frac{\pi}{2}))-\frac{1}{2}\sin\left[\min(x,\frac{\pi}{2})+\min(3y,\frac{\pi}{2})\right]+
\frac{1}{2}\sin\left[\min(x,\frac{\pi}{2})\right]
\\&
%%%%%%%%%
\begin{cases}
\frac{1}{2}\sin(x)+\frac{1}{2}\sin(3y)-\frac{1}{2}\sin(x+3y)
&,\quad 0\le x\le \frac{\pi}{2},0\le y\le \frac{\pi}{6}\\
\frac{1}{2}-\frac{1}{2}\cos(x)+\frac{1}{2}\sin(x)
&,\quad 0\le x\le \frac{\pi}{2},y\ge \frac{\pi}{6}\\
\frac{1}{2}\sin(3y)-\frac{1}{2}\cos(3y)+\frac{1}{2}
&,\quad x\ge \frac{\pi}{2},0\le y\le \frac{\pi}{6}\\
1
&,\quad x\ge \frac{\pi}{2},y\ge \frac{\pi}{6}\\
0&,\quad \text{سایر جاها}
\end{cases}
}

%$
%f(x,y)=
%\begin{cases}
%k\sin(x+3y)&,\quad 0<x<\frac{\pi}{2},0<y<\frac{\pi}{6}\\
%0&,\quad \text{سایر جاها}
%\end{cases}
%$

%پ) (امتیازی)
%
%$
%f(x,y)=
%\begin{cases}
%\frac{1}{2}&,\quad |x|^k+|y|^k<1\\
%0&,\quad \text{سایر جاها}
%\end{cases}
%$

\Q

الف)
\eqn{
&
\Pr\{X\le 4,Y\le -2\}=0
}
\eqn{
\Pr\{X+Y\le 2\}&=
\int_0^\frac{\pi}{2}\int_0^{\min(\frac{\pi}{2},2-y)}\frac{1}{2}\sin(x+y)dxdy
\\&=
\int_0^\frac{\pi}{2}\frac{1}{2}\cos(x+y)\Big|_{\min(\frac{\pi}{2},2-y)}^0dy
\\&=
\int_0^\frac{\pi}{2}\frac{1}{2}\cos(y)-\frac{1}{2}\cos\left[\min(\frac{\pi}{2},2-y)+y\right]dy
\\&=
\int_0^{2-\frac{\pi}{2}}\frac{1}{2}\cos(y)-\frac{1}{2}\cos\left[\min(\frac{\pi}{2},2-y)+y\right]dy
\\&+
\int_{2-\frac{\pi}{2}}^\frac{\pi}{2}\frac{1}{2}\cos(y)-\frac{1}{2}\cos\left[\min(\frac{\pi}{2},2-y)+y\right]dy
\\&=
\int_0^{2-\frac{\pi}{2}}\frac{1}{2}\cos(y)-\frac{1}{2}\cos\left[\frac{\pi}{2}+y\right]dy
\\&+
\int_{2-\frac{\pi}{2}}^\frac{\pi}{2}\frac{1}{2}\cos(y)-\frac{1}{2}\cos\left[2-y+y\right]dy
\\&=
\int_0^{2-\frac{\pi}{2}}\frac{1}{2}\cos(y)+\frac{1}{2}\sin(y)dy
\\&+
\int_{2-\frac{\pi}{2}}^\frac{\pi}{2}\frac{1}{2}\cos(y)-\frac{1}{2}\cos(2)dy
\\&=
\frac{1}{2}+
\int_0^{2-\frac{\pi}{2}}\frac{1}{2}\sin(y)dy
\\&+
\int_{2-\frac{\pi}{2}}^\frac{\pi}{2}-\frac{1}{2}\cos(2)dy
\\&=
1-\frac{1}{2}\sin2 -\frac{\pi-2}{2}\cos2
}
\eqn{
\Pr\{X=4Y\}=0
}
احتمال اخیر صفر است؛ زیرا دو متغیر تصادفی، پیوسته بوده و چگالی احتمال توأم آنها، شامل هیچ ضربه ای روی خط $X=4Y$ نیست.

ب)
\eqn{
\Pr\{X\le 4,Y\le -2\}=0
}
\eqn{
\Pr\{X+Y\le 2\}=1
}
\eqn{
&\Pr\{X=4Y\}=\Pr\{X=4Y=-4\}\\&=\iint_{\Bbb R^2}\frac{1}{2}\delta\left(\sqrt{(x+4)^2+(y+1)^2}\right)dxdy=\frac{1}{2}
}
\end{document}