\documentclass{article}

\usepackage{amsmath,amssymb,geometry}
\usepackage{xepersian}

\setlength{\parindent}{0pt}
\setlength{\parskip}{3mm}

\newcounter{questionnumber}
\setcounter{questionnumber}{1}

\newcommand{\Q}{
\textbf{سوال \thequestionnumber)}
\stepcounter{questionnumber}
}

\newcommand{\eqn}[1]{
\[\begin{split}
#1
\end{split}\]
}

\begin{document}
\LARGE
\begin{center}
\settextfont{IranNastaliq}

به نام زیبایی

%\begin{figure}[h]
%\centering
%\includegraphics[width=30mm]{kntu_logo.eps}
%\end{figure}

کوئیزها

\end{center}
\hrulefill
\large


کوئیز 2)

یک عدد دو رقمی را به این صورت می سازیم که هر رقم آن، به صورت تصادفی از بین ارقام 1 تا 9 انتخاب شده باشد. با چه احتمالی، عدد ساخته شده بر 9 بخش پذیر است؟

کوئیز 5)

سه جعبه داریم که هر یک شامل 10 توپ هستند. در جعبه اول، 3 توپ آبی و 7 توپ قرمز، در جعبه دوم، 3 توپ سفید و 5 توپ آبی و در جعبه سوم، 1 توپ قرمز و 9 توپ سفید هستند. ابتدا یکی از جعبه ها را به تصادف انتخاب کرده و سپس توپی از آن جعبه بیرون می‌کشیم. اگر توپ مورد نظر سفید باشد، با چه احتمالی از جعبه دوم \underline{نیست}؟

\end{document}