\documentclass{article}

\usepackage{amsmath,amssymb,geometry}
\usepackage{xepersian}

\setlength{\parindent}{0pt}
\setlength{\parskip}{3mm}

\newcounter{questionnumber}
\setcounter{questionnumber}{1}

\newcommand{\Q}{
\textbf{سوال \thequestionnumber)}
\stepcounter{questionnumber}
}

\newcommand{\eqn}[1]{
\begin{equation}\begin{split}
#1
\end{split}\end{equation}
}

\begin{document}
\LARGE
\begin{center}
\settextfont{IranNastaliq}

به نام زیبایی

%\begin{figure}[h]
%\centering
%\includegraphics[width=30mm]{kntu_logo.eps}
%\end{figure}

پاسخ تمرینات سری سوم درس احتمال مهندسی

\end{center}
\hrulefill
\large

\Q

اگر کتابهای هم نوع متمایز باشند، اینکه مثلا دو کتاب رمان نسبت به هم در چه موقعیتی قرار میگیرند مهم است. با این فرض، تمام کتابها متمایزند و مجموع حالات مطلوب،
$
(3+2+4)!=9!
$
خواهد بود.

اگر تمام کتابهای هم نوع نامتمایز باشند، در این صورت دو حالت که دو کتاب رمان در دو موقعیت متفاوت نسبت به هم باشند (مثلا رمان بینوایان سمت چپ یا راست رمان گوژپشت نتردام باشد!)، یکبار شمرده می‌شوند. ترتیب کتابهای فیزیک و روانشناسی نیز به 
$
3!
$
و
$
4!
$
حالت مختلف تعیین می شود. در این صورت، تعداد کل حالات مطلوب برابر
$
\frac{9!}{3!\times 2!\times 4!}
$
خواهد بود.

\Q

این دسته گل می تواند شامل حالات زیر باشد:

- 3 بنفشه

- 2 بنفشه و 1 رز

- 2 بنفشه و 1 اقاقیا

- 1 بنفشه و 2 رز

- 1 بنفشه و 2 اقاقیا

- 1 بنفشه، 1 رز و 1 اقاقیا

- 3 رز

- 2 رز و 1 اقاقیا

- 1 رز و 2 اقاقیا

در نتیجه، مجموع کل حالات مطلوب عبارتست از
\eqn{
&
\binom{3}{3}+
\binom{3}{2}\binom{4}{1}+
\binom{3}{2}\binom{2}{1}+
\binom{3}{1}\binom{4}{2}+
\binom{3}{1}\binom{2}{2}+
\\&
\binom{3}{1}\binom{2}{1}\binom{4}{1}+
\binom{4}{3}+
\binom{4}{2}\binom{2}{1}+
\binom{4}{1}\binom{2}{2}=
\\&
1+12+6+18+3+24+4+12+4=84.
}{}

\Q

در حل مسائلی که با افراد سروکار دارند، اگر مسئله از نوع کلان نباشد (مانند شیوع افسردگی در یک جامعه که به طور نسبی، به تعداد افراد مربوط است نه به تک تک آنها)، باید افراد را متمایز دانست. مسئله‌ی پیش رو چنین حالتی دارد. به دلیل اینکه نشست در یک میزگرد اتفاق می افتد، ابتدا یک نفر (مثلا مدیرعامل) را در یک صندلی می‌نشانیم و سپس حالات نشستن سایر افراد را بررسی می کنیم (چرا؟).

الف)اگر هر دو منشی کنار هم باشند، ابتدا هر دو نفر را یک نفر (به نام دو-منشی) به حساب می آوریم و تعداد حالات حاصله را می‌شماریم. سپس تعداد حالات را در تعداد ترتیبات نشستن دو منشی نسبت به هم ضرب می کنیم. با این رویکرد، دو-منشی دو صندلی کنار هم اختیار می‌کند که معادل این است که یک صندلی به او اختصاص داده و از تمام صندلی ها 1 واحد کم کنیم. در این صورت، دو-منشی و سایر اعضا (به غیر از مدیرعامل)، باید 7 صندلی از 9 صندلی باقیمانده را تصاحب کنند. این کار، به دلیل تمایز اعضا، به 
$
\binom{9}{7}\times 7!
$
طریق ممکن امکان پذیر است. چون دو-منشی شامل دو حالت ترتیب نشستن منشی ها نسبت به هم است، تعداد کل حالات ممکن برابر 
$
2\times 7!\times \binom{9}{7}=362880
$
خواهد بود.

ب) در این حالت باید تمام اعضای هیئت مدیره، 5 صندلی از 8 صندلی باقیمانده (غیرمجاور با مدیرعامل و خود مدیرعامل) را به 
$
5!\times \binom{8}{5}=6720
$
طریق تصاحب کنند. سپس منشی ها و حسابدار می‌توانند 3 صندلی از 5 صندلی باقیمانده را به 
$
3!\times\binom{5}{3}=60
$
راه انتخاب کنند. تعداد کل حالات طبق اصل ضرب برابر 
$
403200
$
خواهد بود.

پ) حسابدار به دو حالت کنار مدیرعامل می‌نشیند و تمام اعضای هیئت مدیره (به جز مدیرعامل) را که کنار هم می‌نشینند، یک نفر به نام پنج-مدیر در نظر می‌گیریم. پنج-مدیر، 5 صندلی کنار هم انتخاب می‌کند که در این صورت، معادلأ با 3 نفر روبرو هستیم که باید بر روی 5 صندلی بنشینند. مشابه بخش الف و به دلیل تمایز منشی ها و اعضای هیئت مدیره، تعداد کل حالات مطلوب برابر است با
$
\binom{5}{3}\times 3!\times 5!\times 2=14400
$.
\end{document}