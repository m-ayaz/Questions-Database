\documentclass{article}

\usepackage{amsmath,amssymb,geometry,tikz}
\usepackage{xepersian}

\setlength{\parindent}{0pt}
\setlength{\parskip}{3mm}

\newcounter{questionnumber}
\setcounter{questionnumber}{1}

\newcommand{\Q}{
\textbf{سوال \thequestionnumber)}
\stepcounter{questionnumber}
}

\newcommand{\eqn}[1]{
\begin{equation}\begin{split}
#1
\end{split}\end{equation}
}

\begin{document}
\LARGE
\begin{center}
\settextfont{IranNastaliq}

به نام زیبایی

%\begin{figure}[h]
%\centering
%\includegraphics[width=30mm]{kntu_logo.eps}
%\end{figure}

تمرینات سری دهم درس احتمال مهندسی

\end{center}
\hrulefill
\large

\Q

فرض کنید متغیر تصادفی $X$، از توزیع نمایی با پارامتر $\lambda=1$ پیروی کند. در این صورت، چگالی احتمال متغیر تصادفی $Y$ را در حالت های زیر بیابید.

الف)
$
Y=e^X
$

ب)
$
Y=X^\alpha
$
که
$
\alpha
$
عدد ثابت مثبتی است.

پ)
$
Y=\lfloor X\rfloor
$

\Q

فرض کنید $X$، یک متغیر تصادفی باشد که از توزیع زیر پیروی می‌کند:
$$
f_X(x)=\begin{cases}
kx&,\quad 0<x<1\\
\frac{1}{2}\delta(x)&,\quad x=1\\
0&,\quad \text{جاهای دیگر}
\end{cases}
$$

الف) مقدار مناسب $k$ را بیابید.

ب) مقدار 
$
\mathbb{E}\{X\}
$
را محاسبه کنید.

پ) مقدار 
$
\mathbb{E}\{e^{aX}\}
$
را به دست آورید که $a$ عدد حقیقی دلخواهی است.

\Q

متغیر تصادفی $X$ از توزیع زیر پیروی می‌کند:
$$
f_X(x)=\begin{cases}
2xe^{-x^2}&,\quad x>0\\
0&,\quad \text{جاهای دیگر}\\
\end{cases}
$$
متغیر تصادفی
$
Y=X^2
$
مفروض است.

الف) چگالی احتمال $Y$ را به دست آورید.

ب) امید ریاضی $X$ را بیابید.

پ) امید ریاضی $Y$ را از روی چگالی احتمال آن و مقدار 
$
\mathbb{E}\{X^2\}
$
را از قضیه‌ی اساسی امید ریاضی محاسبه کرده و با هم مقایسه کنید.

ت) مقادیر
$
\Pr\{X<\frac{1}{2}\}
$
و
$
\Pr\{Y<\frac{1}{4}\}
$
را به ترتیب از روی چگالی های احتمال 
$
X
$
و
$
Y
$
به دست آورده و با هم مقایسه کنید.
\end{document}