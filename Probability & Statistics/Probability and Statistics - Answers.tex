\documentclass[10pt,letterpaper]{report}
%\usepackage{toolsper}
%\settextfont{B Nazanin}
\usepackage{amsmath,amssymb,graphicx,geometry,subcaption,tikz,diagbox,hyperref,ulem,xepersian}
\usepackage{lipsum}
\setlength{\parindent}{0mm}
\setlength{\parskip}{3mm}

\newcounter{questionanswernumber}
\setcounter{questionanswernumber}{1}

\newcommand{\QA}{
\textbf{پاسخ سوال \thequestionanswernumber)}
\stepcounter{questionanswernumber}
}

\newcommand{\pic}[2]{
\begin{center}
\includegraphics[width=#2]{#1}
\end{center}
}
\newcommand{\EX}{\mathbb{E}}
\newcommand{\eqn}[1]{
\[\begin{split}
#1
\end{split}\]
}

\settextfont{X Kamran}

%\settextfont{X Morvarid}

%\settextfont{B Nazanin}

%\newgeometry{left=5mm,right=5mm,top=5mm,bottom=5mm}

\begin{document}
\Large
%\begin{center}
%
%\hrulefill
%\end{center}

\QA
از اصل سوم احتمال، برای هر دو مجموعه‌ی ناسازگار 
$A$
و
$B$
داریم
$$P(A\cup B)=P(A)+P(B).$$
از آنجا که 
$A$
و
$A'$
طبق تعریف ناسازگارند، بنابراین
$$P(A\cup A')=P(A)+P(A').$$
از طرفی طبق تعریف،
$$A\cup A'=S$$
که $S$ فضای نمونه است. در نتیجه
$$P(A)+P(A')=1.$$
بر اساس اصل اول احتمال، احتمال هر مجموعه مقداری نامنفی است؛ در نتیجه
$$P(A)=1-P(A')\le 1$$
و اثبات کامل است
$\blacksquare$

\QA
طبق اصل ضرب، تعداد تمام اعداد سه رقمی متمایزی که می توان به این روش ساخت، برابر است با
$9^3=729$.
تعداد ارقام فرد از بین اعداد 1 تا 9، برابر 5 است (ارقام 1، 3، 5، 7 و 9). در نتیجه، تعداد اعداد سه رقمی ای که تمام ارقام آن فرد هستند را می توان دوباره طبق اصل ضرب به
$5^3=125$
طریق ممکن ساخت. بنابراین احتمال مطلوب عبارتست از
$$p=\frac{n(A)}{n(S)}=\frac{125}{729}$$

\QA
تعداد حالات برداشتن 20 توپ، برابر 
$\binom{100}{20}$
بوده و تعداد حالات مطلوب، برابر
$\binom{40}{5}\binom{60}{15}$
است؛ لذا احتمال مطلوب برابر 
$\frac{\binom{40}{5}\binom{60}{15}}{\binom{100}{20}}$
خواهد بود.



{\color{red}

پاسخ:

اصولأ در پرسشهای احتمالاتی، باید فضای نمونه و پیشامدها را در ابتدا به درستی تعریف کرد.  اینجا نیز چنین قاعده‌ای را پی می‌گیریم.

از آنجا که یک فرد خاص می‌تواند زن یا مرد باشد یا چشم آبی باشد یا نباشد، چهار پیشامد ممکن وجود دارد:

\eqn{
&
M=\text{
پیشامد مرد بودن
}
\\&
F=\text{
پیشامد زن بودن
}
\\&
B=\text{
پیشامد چشم آبی بودن
}
\\&
N=\text{
پیشامد چشم آبی نبودن
}
\\&
S_1=\text{
پیشامد اهل استان 1 بودن
}
\\&
S_2=\text{
پیشامد اهل استان 2 بودن
}
}

صورت سوال، اطلاعات احتمالاتی زیر را به ما می‌دهد:
\eqn{
&
P(S_1)=\frac{100}{1100}
\\&
P(S_2)=\frac{1000}{1100}
\\&
P(B|S_1)=\frac{20}{100}
\\&
P(B|S_2)=\frac{50}{1000}
\\&
P(M|S_1)=\frac{60}{100}
\\&
P(M|S_2)=\frac{350}{1000}
}

الف) احتمال مطلوب ما،
$P(S_1|B)$
است که به صورت زیر به دست می‌آید:
\eqn{
P(S_1|B)&=
\frac{P(S_1\cap B)}{P(B)}
\\&=
\underbrace{\frac{P(S_1)P(B|S_1)}{P(B)}}_{\text{قاعده‌ی بیز}}
\\&=
\frac{P(S_1)P(B|S_1)}{P(S_1)P(B|S_1)+P(S_2)P(B|S_2)}
\\&=
\frac{
\frac{100}{1100}\times \frac{20}{100}
}{
\frac{100}{1100}\times \frac{20}{100}+\frac{1000}{1100}\times \frac{50}{1000}
}
=\frac{2}{7}
}

ب) برای این بخش داریم:

\eqn{
P(S_2\cap N|F)&=
\frac{P(S_2\cap N\cap F)}{P(F)}
}
پیشامد $S_2\cap N\cap F$، پیشامد حالتی است که فرد انتخاب شده، زن بوده، از استان 2 انتخاب شود و چشم آبی نباشد. از آنجا که از جامعه‌ی 1100 نفری، 630 نفر چنین ویژگی‌ای دارند در نتیجه:
$$P(S_2\cap N\cap F)=\frac{630}{1100}$$
و می‌توان نوشت
\eqn{
P(S_2\cap N|F)&=
\frac{P(S_2\cap N\cap F)}{P(F)}
\\&=
\frac{P(S_2\cap N\cap F)}{P(S_1)P(F|S_1)+P(S_2)P(F|S_2)}
\\&=
\frac{P(S_2\cap N\cap F)}{
\frac{100}{1100}\times \frac{40}{100}+\frac{1000}{1100}\times \frac{650}{1000}
}
\\&=
\frac{\frac{630}{1100}}{
\frac{100}{1100}\times \frac{40}{100}+\frac{1000}{1100}\times \frac{650}{1000}
}
\\&=
\frac{21}{23}
}









}

\%\%\%\%\%\%\%\%\%\%\%\%\%\%\%\%\%\%\%\%\%\%\%\%\%\%\%\%\%\%
\%\%\%\%\%\%\%\%\%\%\%\%\%\%\%\%\%\%\%\%\%\%\%\%\%\%\%\%\%\%

\%\%\%\%\%\%\%\%\%\%\%\%\%\%\%\%\%\%\%\%\%\%\%\%\%\%\%\%\%\%
\%\%\%\%\%\%\%\%\%\%\%\%\%\%\%\%\%\%\%\%\%\%\%\%\%\%\%\%\%\%

\%\%\%\%\%\%\%\%\%\%\%\%\%\%\%\%\%\%\%\%\%\%\%\%\%\%\%\%\%\%
\%\%\%\%\%\%\%\%\%\%\%\%\%\%\%\%\%\%\%\%\%\%\%\%\%\%\%\%\%\%

\%\%\%\%\%\%\%\%\%\%\%\%\%\%\%\%\%\%\%\%\%\%\%\%\%\%\%\%\%\%
\%\%\%\%\%\%\%\%\%\%\%\%\%\%\%\%\%\%\%\%\%\%\%\%\%\%\%\%\%\%

\%\%\%\%\%\%\%\%\%\%\%\%\%\%\%\%\%\%\%\%\%\%\%\%\%\%\%\%\%\%
\%\%\%\%\%\%\%\%\%\%\%\%\%\%\%\%\%\%\%\%\%\%\%\%\%\%\%\%\%\%

\%\%\%\%\%\%\%\%\%\%\%\%\%\%\%\%\%\%\%\%\%\%\%\%\%\%\%\%\%\%
\%\%\%\%\%\%\%\%\%\%\%\%\%\%\%\%\%\%\%\%\%\%\%\%\%\%\%\%\%\%

\%\%\%\%\%\%\%\%\%\%\%\%\%\%\%\%\%\%\%\%\%\%\%\%\%\%\%\%\%\%
\%\%\%\%\%\%\%\%\%\%\%\%\%\%\%\%\%\%\%\%\%\%\%\%\%\%\%\%\%\%

\%\%\%\%\%\%\%\%\%\%\%\%\%\%\%\%\%\%\%\%\%\%\%\%\%\%\%\%\%\%
\%\%\%\%\%\%\%\%\%\%\%\%\%\%\%\%\%\%\%\%\%\%\%\%\%\%\%\%\%\%

\%\%\%\%\%\%\%\%\%\%\%\%\%\%\%\%\%\%\%\%\%\%\%\%\%\%\%\%\%\%
\%\%\%\%\%\%\%\%\%\%\%\%\%\%\%\%\%\%\%\%\%\%\%\%\%\%\%\%\%\%

\%\%\%\%\%\%\%\%\%\%\%\%\%\%\%\%\%\%\%\%\%\%\%\%\%\%\%\%\%\%
\%\%\%\%\%\%\%\%\%\%\%\%\%\%\%\%\%\%\%\%\%\%\%\%\%\%\%\%\%\%



%%%%%%%%%%%%%%%%%%%%%%%%%%%%%%%%%%%%%%
%%%%%%%%%%%%%%%%%%%%%%%%%%%%%%%%%%%%%%
%%%%%%%%%%%%%%%%%%%%%%%%%%%%%%%%%%%%%%
%%%%%%%%%%%%%%%%%%%%%%%%%%%%%%%%%%%%%%
%%%%%%%%%%%%%%%%%%%%%%%%%%%%%%%%%%%%%%
%%%%%%%%%%%%%%%%%%%%%%%%%%%%%%%%%%%%%%
%%%%%%%%%%%%%%%%%%%%%%%%%%%%%%%%%%%%%%
%%%%%%%%%%%%%%%%%%%%%%%%%%%%%%%%%%%%%%
%%%%%%%%%%%%%%%%%%%%%%%%%%%%%%%%%%%%%%
%%%%%%%%%%%%%%%%%%%%%%%%%%%%%%%%%%%%%%
%%%%%%%%%%%%%%%%%%%%%%%%%%%%%%%%%%%%%%
%%%%%%%%%%%%%%%%%%%%%%%%%%%%%%%%%%%%%%
%%%%%%%%%%%%%%%%%%%%%%%%%%%%%%%%%%%%%%
%%%%%%%%%%%%%%%%%%%%%%%%%%%%%%%%%%%%%%
%%%%%%%%%%%%%%%%%%%%%%%%%%%%%%%%%%%%%%
%%%%%%%%%%%%%%%%%%%%%%%%%%%%%%%%%%%%%%
%%%%%%%%%%%%%%%%%%%%%%%%%%%%%%%%%%%%%%
%%%%%%%%%%%%%%%%%%%%%%%%%%%%%%%%%%%%%%


\QA


سوال 1) الف) فضای نمونه، مجموعه‌ی تمام وقایع ساده‌ی محتمل است که عبارتست از:
$$
S=\{HHH,HHT,HTH,HTT,THH,THT,TTH,TTT\}
$$
ب) از آنجا که واقعه طبق تعریف یک زیر مجموعه از فضای نمونه است و فضای نمونه 8 عضوی است، این مسئله دارای $2^8=256$ واقعه محتمل است که اگر تهی را نامحتمل بگیریم، 255 وافعه‌ی محتمل خواهیم داشت.

پ) طبق تعریف کلاسیک احتمال، احتمال زیرمجموعه‌ی $A$ از مجموعه ی $S$ عبارتست از
$$
P(A)={n(A)\over n(S)}
$$
از طرفی واقعه‌ی اینکه در پرتاب اول و دوم سکه نتیجه یکسان باشد (در پرتاب سوم نتیجه دلخواه است)، دارای چهار عضو $HHH$، $HHT$، $TTT$ و $TTH$ است که نتیجه می دهد:
$$
P(A)={n(A)\over n(S)}={4\over8}={1\over2}
$$
سوال 2) الف و ب و پ)
\[
\begin{split}
&A\cap B=\{4\}
\\&A-B=\{1,5\}
\end{split}
\]
$$
A\times B=\{(1,2),(1,3),(1,4),(4,2),(4,3),(4,4),(5,2),(5,3),(5,4)\}
$$
ت) برای محاسبه‌ی 
$
(A\cup B)\cap C
$
داریم:
$$
A\cup B=\{1,2,3,4,5\}
$$
بنابراین
$$
(A\cup B)\cap C=\{2,5\}
$$
همچنین برای محاسبه‌ی $(A\cap C)\cup (B\cap C)$:
$$
A\cap C=\{5\}\quad,\quad B\cap C=\{2\}
$$
پس خواهیم داشت
$$
(A\cap C)\cup(B\cap C)=\{2,5\}
$$
که نتیجه می دهد:
$$
(A\cup B)\cap C=(A\cap C)\cup (B\cap C)
$$

سوال 3) از اصل 3 کولموگروف می توان دریافت که اگر دو مجموعه ی $S$ و $T$ ناسازگار باشند، خواهیم داشت:
$$
P(S\cup T)=P(S)+P(T)
$$
در این مسئله با تعریف
\[
\begin{split}
&S=A-B
\\&T=A\cap B
\end{split}
\]
می دانیم که مجموعه‌ی $A-B$ شامل عناصر $B$ نیست؛ در حالی که عناصر مجموعه ی $A\cap B$ در $B$ وجود دارند؛ پس نتیجه گیری زیر به دست می آید:
$$
[A-B]\cap[A\cap B]=\emptyset\implies P(A)=P([A-B]\cup[A\cap B])=P(A-B)+P(A\cap B)
$$



سوال 1) الف) از آنجا که سکه دارای 2 حالت و تاس دارای 6 حالت است، طبق اصل ضرب 12 حالت مختلف برای پیشامدهای ساده خواهیم داشت؛ یعنی فضای شدنی مسئله‌ی ما 12 حالتی است. از این 12 حالت فقط حالاتی که سکه رو بیاید و تاس یکی از اعداد 1-3-5 شود مدنظر است که تعداد این حالات خاص 3 تاست. در نتیجه احتمال مطلوب 
$
{3\over 12}={1\over 4}
$
خواهد بود.

ب) پیشامد اینکه سکه به رو بیفتد را با $A$ و اینکه تاس فرد شود را با $B$ نمایش می دهیم. هدف محاسبه‌ی 
$
P(A\cup B)
$
 که می دانیم:
$$
P(A\cup B)=P(A)+P(B)-P(A\cap B)
$$
از طرفی
$$
P(A)={1\over 2}\quad,\quad P(B)={1\over 2}\quad,\quad P(A\cap B)={1\over 4}
$$
بنابراین
$$
P(A\cup B)={3\over 4}
$$

سوال 2) الف)
$$
S=\{3,6,\text{پشت},\text{رو}\}
$$

ب) سکه زمانی رو می آید که تاس مضرب 3 نشود و خود سکه هم به رو بیفتد. احتمال اینکه تاس مضرب 3 نشود برابر $2\over 3$ و احتمال اینکه سکه در صورت پرتاب شدن به رو بیفتد برابر $1\over 2$ است؛ پس احتمال مطلوب برابر حاصلضرب دو احتمال قبلی یعنی $1\over 3$ خواهد بود.

پ) اگر پیشامد 1 آمدن تاس را با A و پشت آمدن سکه را با B نمایش دهیم، در این صورت مطلوبست
$$
P(A\cup B)=P(A)+P(B)-P(A\cap B)
$$
از طرفی
$$
P(A)={1\over 6}\quad,\quad P(B)={1\over 3}
$$
تاس با احتمال $1\over 6$، 1 می آید که در این صورت منجر به پرتاب سکه خواهد شد و سکه هم با احتمال $0.5$ به پشت می افتد؛ پس $P(A\cap B)$ برابر 
$
{1\over 6}\times{1\over 2}={1\over 12}
$
 و 
$
P(A\cup B)
$
برابر
$
5\over 12
$
خواهد بود.

سوال 3) هنگامی که از اشکال دوبعدی بهره می گیریم، جهت استفاده از مفهوم اندازه‌ی پیشامدها، باید مساحت آن ها را در نظر بگیریم.

الف) نقطه ای از داخل مربع به مساحت 4 انتخاب شده است. چون پیشامد مطلوب، انتخاب نقطه از داخل دایره است و دایره به طور کامل درون مربع قرار دارد، احتمال مطلوب عبارت است از:
$$
P(A)={\text{مساحت دایره}\over \text{مساحت مربع}}={\pi\over 4}
$$

ب) از آنجا که قطر ضخامتی ندارد (مساحت آن برابر صفر است؛ برای درک این موضوع، به جای قطر یک نوار نازک در نظر بگیرید و ضخامت آن را به سمت صفر میل دهید) احتمال مطلوب برابر 0 خواهد بود.

پ) مکمل این پیشامد عبارتست از اینکه فاصله ی نقطه از دست کم یکی از رأس های مربع کمتر از $0.5$ باشد. به ازای هر راس مربع، مکان هندسی نقاطی از داخل مربع که فاصله‌ی آنها از راس مورد نظر کمتر از $0.5$ باشد، یه ربع دایره به مرکز آن راس و شعاع $0.5$ داخل مربع خواهد بود. 4 راس در مربع داریم؛ پس 4 تا از این ربع دایره ها خواهیم داشت که همپوشانی ندارند؛ پس مساحت مکمل پیشامد مورد نظر عبارتست از:
$$
\text{مساحت پیشامد A'}=4\times \text{مساحت هر ربع دایره}={\pi\over 4}
$$
و برای احتمال مطلوب داریم:
$$
P(A)={\text{مساحت پیشامد A}\over \text{مساحت مربع}}={16-\pi\over 16}=1-{\pi\over 16}
$$

سوال 4) الف) یک عدد زمانی به 3 بخش پذیر است که جمع ارقام آن به 3 بخش پذیر باشد. مجموعه‌ی این اعداد عبارتست از:
$$
S=\{111,222,210,201,120,102\}\implies |S|=6
$$
ب) تمام اعداد 3 رقمی ای که با این ارقام ساخته می شوند، یا دارای صدگان 1 یا 2 هستند. تعداد اعداد سه رقمی و سه رقمی زوج که دارای صدگان 1 یا 2 باشند، به ترتیب برابر 9 و 6 خواهد بود. بنابراین احتمال مطلوب عبارتست از:
$$
P(A)={6+6\over 9+9}={2\over 3}
$$




%کوئیز 10)
%$
%X
%$
%می تواند مقادیر 0 و 1 و 
%$
%Y
%$
%می تواند مقادیر 0، 2، 4 و 6 را اختیار کند. در نتیجه
%\eqn{
%\Pr\{X=2Y\}&=\Pr\{X=2Y=0\}+\Pr\{X=2Y=1\}
%\\&=\Pr\{X=0,Y=0\}+\Pr\{X=1,Y=\frac{1}{2}\}
%\\&=\Pr\{i<4,i\in\{1,3,5\}\}+0
%\\&=\Pr\{i\in\{1,3\}\}=\frac{1}{3}
%}




%کوئیز 10)
%$
%X
%$
%می تواند مقادیر 0 و 1 و 
%$
%Y
%$
%می تواند مقادیر 0، 2، 4 و 6 را اختیار کند. در نتیجه
%\eqn{
%\Pr\{X=2Y\}&=\Pr\{X=2Y=0\}+\Pr\{X=2Y=1\}
%\\&=\Pr\{X=0,Y=0\}+\Pr\{X=1,Y=\frac{1}{2}\}
%\\&=\Pr\{i<4,i\in\{1,3,5\}\}+0
%\\&=\Pr\{i\in\{1,3\}\}=\frac{1}{3}
%}


\end{document}