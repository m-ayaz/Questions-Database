\documentclass{article}

\usepackage{amsmath,amssymb,geometry,minted}
\usepackage{xepersian}

\setlength{\parindent}{0pt}
\setlength{\parskip}{3mm}

\newcounter{questionnumber}
\setcounter{questionnumber}{1}

\newcommand{\Q}{
\textbf{سوال \thequestionnumber)}
\stepcounter{questionnumber}
}

\begin{document}
\LARGE
\begin{center}
\settextfont{IranNastaliq}

به نام زیبایی

%\begin{figure}[h]
%\centering
%\includegraphics[width=30mm]{kntu_logo.eps}
%\end{figure}

شبیه سازی سری اول درس احتمال مهندسی

\end{center}
\hrulefill
\large

\Q

در این شبیه سازی، قصد داریم پرتاب سکه را با جزئیات بیشتر مطالعه کنیم. می‌دانیم در 
$
n
$
بار پرتاب سکه‌ی سالم، احتمال 
$
k
$
بار رو آمدن برابر است با
$$
\binom{n}{k}(\frac{1}{2})^n.
$$
جهت تجربه‌ی عملی فرمول اخیر، آزمایشی را که در آن سکه
$
1000
$
بار پرتاب می‌شود،
$
1000
$
بار تکرار می‌کنیم (مشابه اینکه سکه ای را 1000000 بار پرتاب کرده و هر 1000 نتیجه‌ی پشت سر هم را خوشه‌بندی کنیم).

گامها:

گام 1) به کمک یک حلقه‌ی \lr{for}،
1000 عدد تصادفی 0 یا 1 تولید کرده و تعداد دفعاتی را که این رشته‌ی 1000تایی برابر 1 می‌شود (سکه رو می‌آید) را حساب کنید.

گام 2) گام1 را 1000 بار تکرار کرده و هر بار نتیجه را در یک آرایه‌ی 1000تایی ذخیره کنید.

گام 3) هیستوگرام نتایج را رسم نموده و آن را توجیه کنید (تعداد \lr{bin}های هیستوگرام را برابر $40$ قرار دهید).

سوال 2)

الف) می‌دانیم که چگالی احتمال یک متغیر تصادفی، میانگین تجربی تعداد دفعات رخداد مقادیر آن متغیر تصادفی را نشان می‌دهد؛ به طور مثال در توزیع یکنواخت در بازه‌ی $[0,1]$، تمام اعداد این بازه، دارای میانگین تجربی یکسان رخداد هستند. جهت تحقیق این امر، 10000 تحقق از توزیع یکنواخت در بازه‌ی $[0,1]$ را تولید کرده و هیستوگرام آن را رسم کنید. نتیجه را توجیه کنید.

ب) می‌دانیم که اگر 
$
X
$
از توزیع یکنواخت در بازه‌ی 
$
[0,1]
$
پیروی کند، 
$
Y=-\ln X
$
از توزیع نمایی با پارامتر 
$
\lambda=1
$
پیروی می‌کند و چگالی احتمال آن به صورت زیر است:
$$
f_Y(y)=\begin{cases}
e^{-y}&,\quad y>0\\
0&,\quad y\le0\\
\end{cases}.
$$
به کمک دستور هیستوگرام، 10000 تحقق از توزیع یکنواخت در بازه‌ی 
$
[0,1]
$
تولید کرده و در آرایه‌ای به نام
$
X
$
ذخیره کنید. سپس آرایه‌ی
$
Y
$
را از روی آرایه‌ی
$
X
$
به صورت
$
Y=-\ln X
$
بسازید. سپس هیستوگرام $Y$ را با تعداد \lr{bin} برابر 100 رسم کنید. جهت مشاهده‌ی رفتار چگالی احتمال، تابع
$
e^{-y}
$
را در بازه‌ی 
$
(0,10)
$
روی همان نمودار هیستوگرام رسم شده بکشید و نتیجه را توجیه کنید.

(
فرض کنید میخواهید هیستوگرام آرایه‌ی $x$ را رسم کنید. جهت رسم هیستوگرام در این سوال، از دستورهای زیر استفاده کنید:

متلب:
\begin{latin}
\begin{minted}{matlab}
histogram(x,'Normalization','pdf')
\end{minted}
\end{latin}
پایتون:
\begin{latin}
\begin{minted}{python}
hist(x,density='pdf')
\end{minted}
\end{latin}

)

\end{document}