\documentclass{article}

\usepackage{amsmath,amssymb,geometry}
\usepackage{xepersian}

\setlength{\parindent}{0pt}
\setlength{\parskip}{3mm}

\newcounter{questionnumber}
\setcounter{questionnumber}{1}

\newcommand{\Q}{
\textbf{سوال \thequestionnumber)}
\stepcounter{questionnumber}
}

\newcommand{\eqn}[1]{
\[\begin{split}
#1
\end{split}\]
}

\begin{document}
\LARGE
\begin{center}
\settextfont{IranNastaliq}

به نام زیبایی

%\begin{figure}[h]
%\centering
%\includegraphics[width=30mm]{kntu_logo.eps}
%\end{figure}

کوئیزهای 8 و 9 درس احتمال مهندسی (اضافی)

\end{center}
\hrulefill
\large

کوئیز 8)

تابع جرم احتمال متغیر تصادفی $X$ دارای خاصیت زیر است:
$$
6\Pr\{X=k+2\}-5\Pr\{X=k+1\}+\Pr\{X=k\}=0\quad,\quad k=1,2,\cdots
$$
همچنین 
$
\Pr\{X=1\}=\frac{7}{12}
$.
در این صورت، چگالی جرم احتمال متغیر $X$ را بیابید.

کوئیز 9)

متغیر تصادفی $X$ دارای چگالی احتمال زیر است:
$$
f(x)=\frac{a}{2}e^{-ax}+\frac{1}{2}e^{-x}\quad,\quad x>0.
$$
مقدار $a$ را به گونه ای بیابید به طوری که
$
\mathbb{E}\{X\}=5
$.


%کوئیز 10)
%$
%X
%$
%می تواند مقادیر 0 و 1 و 
%$
%Y
%$
%می تواند مقادیر 0، 2، 4 و 6 را اختیار کند. در نتیجه
%\eqn{
%\Pr\{X=2Y\}&=\Pr\{X=2Y=0\}+\Pr\{X=2Y=1\}
%\\&=\Pr\{X=0,Y=0\}+\Pr\{X=1,Y=\frac{1}{2}\}
%\\&=\Pr\{i<4,i\in\{1,3,5\}\}+0
%\\&=\Pr\{i\in\{1,3\}\}=\frac{1}{3}
%}

\end{document}