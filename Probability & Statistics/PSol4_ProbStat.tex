\documentclass{article}

\usepackage{amsmath,amssymb,geometry}
\usepackage{xepersian}

\setlength{\parindent}{0pt}
\setlength{\parskip}{3mm}

\newcounter{questionnumber}
\setcounter{questionnumber}{1}

\newcommand{\Q}{
\textbf{سوال \thequestionnumber)}
\stepcounter{questionnumber}
}

\newcommand{\eqn}[1]{
\begin{equation}\begin{split}
#1
\end{split}\end{equation}
}

\begin{document}
\LARGE
\begin{center}
\settextfont{IranNastaliq}

به نام زیبایی

%\begin{figure}[h]
%\centering
%\includegraphics[width=30mm]{kntu_logo.eps}
%\end{figure}

پاسخ تمرینات سری چهارم درس احتمال مهندسی

\end{center}
\hrulefill
\large

\Q

الف) پیشامد مطلوب، عبارتست از آنکه دقیقا 2 بار عدد زوج یا دقیقا 3 بار عدد زوج بیاید. احتمال عدد زوج آمدن برابر $0.5$ است. در نتیجه احتمال مطلوب برابر است با
$$
\binom{3}{2}(\frac{1}{2})^3+\binom{3}{3}(\frac{1}{2})^3=\frac{1}{2}.
$$

ب) مجموع اعداد رو آمده در این 3 پرتاب، در حالات زیر برابر 5 می شود:

- دو بار 1 و یکبار 3 بیاید.

- دوبار 2 و یکبار 1 بیاید.

هر یک از حالات فوق، دارای احتمال
$$
\binom{3}{1}(\frac{1}{6})^3=\frac{1}{72}
$$
هستند؛ در نتیجه احتمال مطلوب، برابر 
$
\frac{1}{72}+\frac{1}{72}=\frac{1}{36}
$
خواهد بود.

پ) برای رو آمدن مضرب 3، باید اعداد 3 و 6 ظاهر شوند. احتمال این موضوع برابر 
$
\frac{1}{3}
$
است و چون نتیجه‌ی سایر پرتاب ها مهم نیست، احتمال مطلوب نیز 
$
\frac{1}{3}
$
خواهد بود.

\Q

الف) طبق اصل ضرب، ابن کار به 
$
7\times 10=70
$
طریق ممکن (بدون احتساب ترتیب) و 140 طریق ممکن (با احتساب ترتیب) امکان پذیر است.

ب) تعداد کل حالات ممکن برای برداشتن 2 توپ، برابر 
$
\binom{17}{2}=136
$
است و در نتیجه، احتمال مطلوب برابر 
$
\frac{70}{136}
$
خواهد بود.

پ) از آنجا که فقط یک توپ آبی مشخص و یک توپ قرمز مشخص مد نظر ماست، تنها به یک حالت می‌توانیم این دو توپ را برداریم (دقت کنید چه ترتیب انتخاب را محسوب کنیم چه نکنیم، تعداد حالات را عوض می‌کند ولی در مقدار احتمال تأثیری ندارد). در نتیجه احتمال مطلوب برابر 
$
\frac{1}{136}
$
خواهد بود.

ت) خیر؛ چرا که متمایز کردن توپ ها (به کمک شماره گذاری آنها)، جزو مطلوبات مسئله نبوده است.

\Q

ب) برداشتن دو توپ با جایگذاری، مانند این است که دو توپ از دو کیسه‌ی کاملا مشابه (یک توپ از هر کیسه) برداریم. از آنجا که ترتیب انتخاب نباید مهم باشد، حالتی که توپ کیسه‌ی اول قرمز و توپ کیسه‌ی دوم آبی است، هم ارزند و باید هر دو شمرده شوند. در این صورت، تعداد حالات برداشتن دو توپ به این روش که یکی آبی و دیگری قرمز باشد،
$
140
$
تا است و همچنین، این دو توپ به 
$
{17\times 17}=289
$
حالت ممکن برداشته می‌شوند (اگر ترتیب انتخاب مهم نباشد). در نتیجه، احتمال مطلوب برابر 
$
\frac{140}{289}
$
خواهد بود.

پ) به طریق مشابه قبل، دو حالت امکان پذیر است که توپ های شماره دار خاصی از هر رنگ برداشته شوند. در نتیجه، احتمال مطلوب برابر 
$
\frac{2}{289}
$
خواهد بود.

ت) خیر. دلیل مشابه است؛ زیرا متمایز بودن توپها جزو مطلوبات مسئله نبوده است.
\end{document}