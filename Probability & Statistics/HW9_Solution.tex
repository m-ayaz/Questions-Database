\documentclass[10pt,letterpaper]{article} 
\usepackage{toolsper}
%\usepackage{graphicx}‎‎
%\usefonttheme{serif}‎
%\usepackage{ptext}‎
\usepackage{xepersian}
\settextfont{B Nazanin}
\usepackage{lipsum}
\setlength{\parindent}{0pt}
\newcommand{\pf}{$\blacksquare$}
\newcommand{\pic}[2]{
\begin{center}
\includegraphics[width=#2]{#1}
\end{center}
}
\begin{document}
\Large
\begin{center}
به نام خدا

پاسخ تمرینات سری نهم درس آمار و احتمال
\hl
\end{center}
سوال 1) الف) با توجه به اینکه 
$
\Pr\{XY>1\}=0
$
 ، برای $u<1$ خواهیم داشت:
\qn{
&\Pr\{XY<u\}=\Pr\left\{X<{u\over Y}\right\}
\\&=\int_{-\infty}^\infty \Pr\left\{X<{u\over y}|Y=y\right\}f_Y(y)dy
\\&=\int_0^1 \Pr\left\{X<{u\over y}|Y=y\right\}dy
\\&=\int_0^u \Pr\left\{X<{u\over y}\right\}dy
\\&+
\int_u^1 \Pr\left\{X<{u\over y}\right\}dy
\\&=u+\int_u^1 {u\over y}dy
\\&=u-u\ln u
}{}
بنابراین
\qn{
f_{XY}(u)={d\over du}F_{XY}(u)=
\begin{cases}
-\ln u&,\quad 0\le u<1
\\0&,\quad \text{\rl{در غیر این صورت}}
\end{cases}
}{}
ب) ابتدا، می دانیم
\qn{
\Pr\{X+Y<0\}=1-\Pr\{X+Y<2\}=0
}{}
بنابراین با فرض 
$
0<u<2
$
 خواهیم داشت:
\qn{
&\Pr\{X+Y<u\}=\Pr\left\{X<{u-Y}\right\}
\\&=\int_{-\infty}^\infty \Pr\left\{X<{u-y}|Y=y\right\}f_Y(y)dy
\\&=\int_0^1 \Pr\left\{X<{u-y}|Y=y\right\}dy
%\\&=\int_0^u \Pr\left\{X<{u-y}\right\}dy
%\\&+
%\int_u^1 \Pr\left\{X<{u\over y}\right\}dy
%\\&=u+\int_u^1 {u\over y}dy
%\\&=u-u\ln u
}{}
به ازای $u<1$:
\qn{
&\int_0^1 \Pr\left\{X<{u-y}|Y=y\right\}dy\\&=\int_0^u \Pr\left\{X<{u-y}|Y=y\right\}dy
\\&=\int_0^u u-ydy\\&={u^2\over 2}
}{}
به ازای $u>1$:
\qn{
&\int_0^1 \Pr\left\{X<{u-y}|Y=y\right\}dy\\&=\int_{u-1}^1 \Pr\left\{X<{u-y}|Y=y\right\}dy
\\&+
\int_0^{u-1} \Pr\left\{X<{u-y}|Y=y\right\}dy
\\&=\int_{u-1}^1 u-ydy\\&+u-1
\\&=u(2-u)-{1\over 2}+{(u-1)^2\over 2}+u-1
\\&=2u-{u^2\over 2}
}{}
بنابراین
\qn{
f(x)=
\begin{cases}
x&,\quad 0<x<1
\\2-x&,\quad 1\le x<2
\\0&,\quad\text{\rl{در غیر این صورت}}
\end{cases}
}{}
پ) به ازای $u>0$
\qn{
&\Pr\left\{{X\over Y}<u\right\}=\Pr\left\{X<uY\right\}
\\&=\int_{-\infty}^\infty \Pr\left\{X<uy|Y=y\right\}f_Y(y)dy
\\&=\int_0^1 \Pr\left\{X<uy|Y=y\right\}dy
%\\&=\int_0^u \Pr\left\{X<{u-y}\right\}dy
%\\&+
%\int_u^1 \Pr\left\{X<{u\over y}\right\}dy
%\\&=u+\int_u^1 {u\over y}dy
%\\&=u-u\ln u
}{}
به ازای $u<1$ با اندکی محاسبات
\qn{
\Pr\left\{{X\over Y}<u\right\}={u\over 2}
}{}
به ازای $u>1$ با اندکی محاسبات بیشتر
\qn{
\Pr\left\{{X\over Y}<u\right\}=1-{1\over 2u}
}{}
بنابراین
\qn{
f(x)=
\begin{cases}
{1\over 2}&,\quad 0<x<1
\\{1\over 2x^2}&,\quad 1\le x
\\0&,\quad\text{\rl{در غیر این صورت}}
\end{cases}
}{}
ت) با فرض 
$
0<u<1
$
 خواهیم داشت:
\qn{
&\Pr\left\{\max\{X,Y\}<u\right\}=\Pr\left\{X<u,Y<u\right\}
\\&=\Pr\left\{X<u\right\}\Pr\left\{Y<u\right\}
=u^2
}{}
بنابراین
\qn{
f(x)=
\begin{cases}
2x&,\quad 0<x<1
\\0&,\quad\text{\rl{در غیر این صورت}}
\end{cases}
}{}
ث) با فرض 
$
0<u<1
$
 خواهیم داشت:
\qn{
&\Pr\left\{\min\{X,Y\}<u\right\}=1-\Pr\left\{\min\{X,Y\}>u\right\}
\\&=1-\Pr\left\{X>u,Y>u\right\}
\\&=1-\Pr\left\{X>u\right\}\Pr\left\{Y>u\right\}
=1-(1-u)^2
}{}
بنابراین
\qn{
f(x)=
\begin{cases}
2-2x&,\quad 0<x<1
\\0&,\quad\text{\rl{در غیر این صورت}}
\end{cases}
}{}
اکنون اگر متغیرهای تصادفی $X$ و $Y$، نمایی با پارامتر 1 باشند، در این صورت با فرض 
$
u>0
$

ب)
\qn{
&\Pr\left\{X+Y<u\right\}=\Pr\left\{X<u-Y\right\}
\\&=\int_0^\infty e^{-y}\Pr\left\{X<u-y\right\}dy
\\&=\int_0^u e^{-y}\Pr\left\{X<u-y\right\}dy
\\&=\int_0^u e^{-y}(1-e^{y-u})dy
\\&=\int_0^u e^{-y}-e^{-u}dy
\\&=1-e^{-u}-ue^{-u}
}{}
در نتیجه
\qn{
f(x)=xe^{-x}\quad,\quad x>0
}{}
پ)
\qn{
&\Pr\left\{{X\over Y}<u\right\}=\Pr\left\{X<uY\right\}
\\&=\int_0^\infty e^{-y}\Pr\left\{X<uy\right\}dy
%\\&=\int_0^u e^{-y}\Pr\left\{X<u-y\right\}dy
\\&=\int_0^\infty e^{-y}(1-e^{-yu})dy
\\&=\int_0^\infty e^{-y}-e^{-(1+u)y}dy
\\&=1-{1\over u+1}
}{}
در نتیجه
\qn{
f(u)={1\over (u+1)^2}
\quad,\quad u>0
}{}
\newline
\newline
%سوال 2) با تعریف 
%$
%w=x-\rho y
%$
% می توان نوشت:
%$$
%w^2+(\sqrt{1-\rho^2} y)^2=x^2+y^2-2\rho xy
%$$
%بنابراین 
%\qn{
%f(x,y)=f(w,\sqrt{1-\rho^2}y)={1\over \sqrt{2\pi (1-\rho^2)}}e^{-{1\over 2}{1\over 1-\rho^2}(w^2+(\sqrt{1-\rho^2}y)^2)}
%}{}
سوال 2) در سوال 4 سری پیشین، ثابت شد هردوی $X$ و $Y$ دارای توزیع نرمال با میانگین صفر و واریانس 1 هستند. همچنین به ازای هر $Y=y$، توزیع $X$ دارای میانگین $\rho y$ و واریانس $1-\rho^2$ است؛ بنابراین:
\qn{
&\Pr\{X+Y<u\}=\Pr\{X<u-Y\}
\\&=\int_{-\infty}^{\infty}{1\over \sqrt{2\pi}}\exp\left[-{y^2\over 2}\right]F_X(u-y)dy
}{}
که در آن $F_X(x)$ توزیع تجمعی متغیر تصادفی $X$ به ازای هر $Y=y$ است. با مشتق گیری خواهیم داشت:
\qn{
&{d\over du}\Pr\{X+Y<u\}
=\int_{-\infty}^{\infty}{1\over \sqrt{2\pi}}\exp\left[-{y^2\over 2}\right]{1\over \sqrt{2\pi(1-\rho^2)}}\exp\left[-{(u-y-\rho y)^2\over 2(1-\rho^2)}\right]dy
\\&={1\over 2\pi\sqrt{1-\rho^2}}
\int_{-\infty}^{\infty}\exp\left[-{(u-y-\rho y)^2+(1-\rho^2)y^2\over 2(1-\rho^2)}\right]dy
%\\&={1\over 2\pi\sqrt{1-\rho^2}}
%\int_{-\infty}^{\infty}\exp\left[-{(u-y-\rho y)^2+(1-\rho^2)y^2\over 2(1-\rho^2)}\right]dy
\\&={1\over 2\pi\sqrt{1-\rho^2}}
\int_{-\infty}^{\infty}\exp\left[-{u^2-2(1+\rho)uy+y^2(2+2\rho)\over 2(1-\rho^2)}\right]dy
\\&={1\over 2\pi\sqrt{1-\rho^2}}
\int_{-\infty}^{\infty}\exp\left[-{{1-\rho\over 2}u^2+(2+2\rho)\left(y-{u\over 2}\right)^2\over 2(1-\rho^2)}\right]dy
\\&={1\over 2\pi\sqrt{1-\rho^2}}
\exp\left[-{u^2\over 4(1+\rho)}\right]
\int_{-\infty}^{\infty}\exp\left[-{\left(y-{u\over 2}\right)^2\over 1-\rho}\right]dy
\\&={1\over2\sqrt{\pi(1+\rho)}}
\exp\left[-{u^2\over 4(1+\rho)}\right]
%\\&={1\over 2\pi}
%\int_{-\infty}^{\infty}\exp\left[-\left(y-{u\over 2}\right)^2\right]e^{-u^2\over 4}dy={1\over 2\sqrt{\pi}}e^{-u^2\over 4}{1\over \sqrt\pi}
%\int_{-\infty}^{\infty}\exp\left[-\left(y-{u\over 2}\right)^2\right]dy
%\\&={1\over 2\sqrt{\pi}}e^{-u^2\over 4}
}{}
بنابراین $X+Y$، یک توزیع گوسی با میانگین صفر و واریانس $2+2\rho$ است.

این واریانس زمانی بیشینه است که $\rho=1$؛ به عبارت دیگر حالتی که تساوی $X=Y$ با احتمال 1 برقرار باشد. علت شهودی آن جالب است؛ زیرا زمانی که تساوی اخیر با احتمال 1 رخ دهد، تغییرات $X$ و $Y$ هم جهت یکدیگر است (به عبارت دیگر، هردو همزمان و به یک اندازه کم و زیاد می شوند).
\end{document}