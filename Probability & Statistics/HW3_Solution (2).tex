\documentclass[10pt,letterpaper]{article}
\usepackage{toolsper}
%\settextfont{B Nazanin}
\usepackage{lipsum}
\setlength{\parindent}{0mm}
\setlength{\parskip}{3mm}
\newcommand{\pic}[2]{
\begin{center}
\includegraphics[width=#2]{#1}
\end{center}
}
\begin{document}
\Large
\begin{center}
به نام او

پاسخ تمرینات سری سوم درس احتمال مهندسی
\hl
\end{center}
%\color{red}
سوال 1) 

الف) اگر جایگذاری داشته باشیم، پس از برداشتن گلوله‌ی اول به 8 حالت، گلوله‌ی دوم را نیز می توانیم به 8 حالت برداریم. در این صورت از آنجا که ترتیب برداشتن فرقی نمی کند، برداشتن دو گلوله مجموعا به 
$
8\times 8\over 2
$
طریق ممکن است. همچنین اینکه یکی از گلوله ها سفید و دیگری آبی باشد، به 
$
5\times 3
$
راه ممکن است؛ پس احتمال مطلوب برابر است با
$$
P={15\over 32}
$$
\textbf{نکته مهم!!}
ممکن است این گونه برداشت شود که پاسخ اصلی در یک ضریب 2 با پاسخ بالا تفاوت می کند؛ به طور مثال یک راه حل (که البته نادرست است!) به صورت زیر است:

\textbf{
پیشامد اینکه \underline{گلوله‌ی اول سفید و دومی آبی باشد}، 15 حالت متفاوت دارد. چون هر گلوله را به 8 حالت مستقل از دیگری بر می داریم، پاسخ 
$
3\times 5\over 8\times 8
$
می شود.
}

ایراد استدلال بالا این است که رنگ گلوله ها در ترتیب برداشته شدن گلوله ها اثرگذار بوده است. برای اینکه مشکل این استدلال رفع شود، باید پیشامد عکس هم در نظر گرفته شود؛ یعنی حالتی که \underline{گلوله‌ی اول آبی و دومی سفید باشد} تا محور زمان دیده نشود.

نوع دیگر استدلال (درست) چنین است: هنگامی که جایگذاری داشته باشیم، برداشتن گلوله های اول و دوم کاملا از هم مستقل می شود. پس مسئله معادل است با اینکه:

\textit{
دو کیسه داریم که هر یک شامل 5 سفید و 3 آبی است. از هر یک، یک گلوله بر می داریم. با چه احتمالی یکی سفید و دیگری آبی می شود؟
}

مسئله ی فوق، بعد زمان را به فضا تبدیل کرده است؛ یعنی به جای دو بار برداشتن گلوله ها از یک کیسه در زمانهای مختلف، دوتا را از دو کیسه همزمان برداشته ایم. در اینصورت پیشامد اینکه یکی آبی و دیگری سفید باشد، اجتماع دو پیشامد هم احتمال است که هر یک با احتمال
$
15\over 64
$
رخ می دهد؛ پس پاسخ درست 
$
15\over 32
$
است.

ب) اگر جایگذاری مجاز نباشد، دو گلوله را به 
$
\binom{8}{2}=28
$
طریق ممکن می توان برداشت که فقط حالاتی که یکی سفید و دیگری آبی باشد مطلوب است. این حالات مجموعا به 
$
\binom{5}{1}\binom{3}{1}=15
$
طریق ممکن امکان پذیرند؛ پس احتمال مطلوب برابر است با
$$
P={15\over 28}
$$

سوال 2) الف) مکمل این پیشامد، حالتی است که حداکثر یک بار رو ظاهر شود که برابر است با حالاتی که در 10 پرتاب دقیقا 1 رو یا دقیقا صفر رو ظاهر شود (همگی به پشت ظاهر شوند). مجموع احتمالات برابر است با
$$
p'=10\times \left({1\over 2}\right)^9\times \left({1\over 2}\right)^1 +\left({1\over 2}\right)^{10}={11\over 1024}
$$
بنابراین احتمال مطلوب برابر است با
$$
p=1-p'={1013\over 1024}
$$

ب) مشابه قسمت بالا، از آنجا که پرتاب های سکه از هم مستقل هستند، داریم:
$$
p=3\times \left({1\over 2}\right)^2\times \left({1\over 2}\right)^1 +\left({1\over 2}\right)^{3}={1\over2}
$$

پ) احتمال اینکه در پرتاب های زوج نتیجه رو باشد با اینکه پشت باشد، به دلیل تقارن مسئله یکسان است. از طرفی برای محاسبه‌ی احتمال اینکه در پرتاب های زوج نتیجه رو باشد داریم:
$$
p'=\left({1\over 2}\right)^5
$$
بنابراین احتمال مطلوب برابر است با
$$
p=2p'={1\over 16}
$$

سوال 3) این دسته گل را به 
$
\binom{20}{5}=15504
$
راه ممکن می توان برگزید.

الف) اگر دسته گل بخواهد شامل 2 نسترن و 2 بنفشه باشد، باید گل باقیمانده را از بین لاله ها و اقاقیاها به 12 طریق ممکن برداریم. این کار به 
$
\binom{3}{2}\binom{5}{2}\binom{12}{1}=180
$
حالت ممکن امکان پذیر است؛ بنابراین احتمال مطلوب برابر است با
$$
p={180\over 15504}\approx0.01
$$
(اگر فرض کرده اید دسته گل شامل حداقل 2 نسترن یا 2 بنفشه است نیز راه حل مورد قبول است!)

ب) فقط می توان از بین 7 گل بنفشه و اقاقیا انتخاب کرد که این به 
$
\binom{7}{5}=21
$
حالت ممکن است؛ پس:
$$
p={21\over 15504}\approx0.0014
$$
پ) احتمال مطلوب عبارتست از
\[
\begin{split}
p&={1\over 15504}{\binom{10}{2}\binom{5}{1}\binom{3}{1}\binom{2}{1}}
\\&+{1\over 15504}{\binom{10}{1}\binom{5}{2}\binom{3}{1}\binom{2}{1}}
\\&+{1\over 15504}{\binom{10}{1}\binom{5}{1}\binom{3}{2}\binom{2}{1}}
\\&+{1\over 15504}{\binom{10}{1}\binom{5}{1}\binom{3}{1}\binom{2}{2}}
\\&={50\over 323}\approx0.15
\end{split}
\]

سوال 4) الف) هر مجموعه n عضوی شامل $2^n$ زیر مجموعه‌ی متمایز است که 
$
\binom{n}{k}
$
تا از آنها k عضوی اند. پس احتمال مطلوب برابر است با
$$
p={\binom{n}{k}\over 2^n}
$$

ب) از آنجا که مجموع تمام احتمالات فوق برابر 1 است، داریم:
$$
\sum_{k=0}^n{\binom{n}{k}\over 2^n}=1
$$
که معادل گزاره ای است که میخواستیم ثابت کنیم.

سوال 5)

الف) در رشته متوالی لامپ ها، لامپ ها به صورت پشت سر هم به یکدیگر وصل شده اند؛ پس رشته زمانی روشن است که تمام لامپ ها سالم باشند. احتمال این امر برابر 
$
(1-p)^n
$
است.

ب) در رشته‌ی موازی لامپ ها، یکی از سرهای همه‌ی لامپ ها به یک نقطه و سر دیگر تمام لامپ ها به نقطه‌ی دیگر وصل شده اند؛ پس رشته زمانی خراب می شود که همه ی لامپ های آن خراب باشند. بنابراین احتمال روشن شدن رشته برابر 
$
1-p^n
$
است.
\end{document}