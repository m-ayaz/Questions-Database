\documentclass[10pt,letterpaper]{article}
\usepackage{toolsper}
%\usepackage{xepersian}
%\usepackage{amsmath}
%\settextfont{B Nazanin}
\usepackage{lipsum}
\setlength{\parindent}{0mm}
\setlength{\parskip}{3mm}
\newcommand{\pic}[2]{
\begin{center}
\includegraphics[width=#2]{#1}
\end{center}
}
\begin{document}
\Large
\begin{center}
به نام او

پاسخ تمرینات سری اول درس احتمال مهندسی

\hrulefill
\end{center}
سوال 1) الف) فضای نمونه، مجموعه‌ی تمام وقایع ساده‌ی محتمل است که عبارتست از:
$$
S=\{HHH,HHT,HTH,HTT,THH,THT,TTH,TTT\}
$$
ب) از آنجا که واقعه طبق تعریف یک زیر مجموعه از فضای نمونه است و فضای نمونه 8 عضوی است، این مسئله دارای $2^8=256$ واقعه محتمل است که اگر تهی را نامحتمل بگیریم، 255 وافعه‌ی محتمل خواهیم داشت.

پ) طبق تعریف کلاسیک احتمال، احتمال زیرمجموعه‌ی $A$ از مجموعه ی $S$ عبارتست از
$$
P(A)={n(A)\over n(S)}
$$
از طرفی واقعه‌ی اینکه در پرتاب اول و دوم سکه نتیجه یکسان باشد (در پرتاب سوم نتیجه دلخواه است)، دارای چهار عضو $HHH$، $HHT$، $TTT$ و $TTH$ است که نتیجه می دهد:
$$
P(A)={n(A)\over n(S)}={4\over8}={1\over2}
$$
سوال 2) الف و ب و پ)
\[
\begin{split}
&A\cap B=\{4\}
\\&A-B=\{1,5\}
\end{split}
\]
$$
A\times B=\{(1,2),(1,3),(1,4),(4,2),(4,3),(4,4),(5,2),(5,3),(5,4)\}
$$
ت) برای محاسبه‌ی 
$
(A\cup B)\cap C
$
داریم:
$$
A\cup B=\{1,2,3,4,5\}
$$
بنابراین
$$
(A\cup B)\cap C=\{2,5\}
$$
همچنین برای محاسبه‌ی $(A\cap C)\cup (B\cap C)$:
$$
A\cap C=\{5\}\quad,\quad B\cap C=\{2\}
$$
پس خواهیم داشت
$$
(A\cap C)\cup(B\cap C)=\{2,5\}
$$
که نتیجه می دهد:
$$
(A\cup B)\cap C=(A\cap C)\cup (B\cap C)
$$

سوال 3) از اصل 3 کولموگروف می توان دریافت که اگر دو مجموعه ی $S$ و $T$ ناسازگار باشند، خواهیم داشت:
$$
P(S\cup T)=P(S)+P(T)
$$
در این مسئله با تعریف
\[
\begin{split}
&S=A-B
\\&T=A\cap B
\end{split}
\]
می دانیم که مجموعه‌ی $A-B$ شامل عناصر $B$ نیست؛ در حالی که عناصر مجموعه ی $A\cap B$ در $B$ وجود دارند؛ پس نتیجه گیری زیر به دست می آید:
$$
[A-B]\cap[A\cap B]=\emptyset\implies P(A)=P([A-B]\cup[A\cap B])=P(A-B)+P(A\cap B)
$$

\end{document}