\documentclass[10pt,letterpaper]{article} 
\usepackage{toolsper}
%\usepackage{graphicx}‎‎
%\usefonttheme{serif}‎
%\usepackage{ptext}‎
%\usepackage{xepersian}
%\settextfont{B Nazanin}
\usepackage{lipsum}
\setlength{\parindent}{0pt}
\newcommand{\pf}{$\blacksquare$}
\newcommand{\EX}{\Bbb E}
\newcommand{\pic}[2]{
\begin{center}
\includegraphics[width=#2]{#1}
\end{center}
}
\begin{document}
\Large
\begin{center}
به نام خدا

پاسخ تمرینات سری یازدهم درس آمار و احتمال
\hl
\end{center}
سوال 1) الف) می دانیم این چگالی توزیع، روی یک مربع دوران یافته با طول قطر 2، مقدار ثابت $k$ دارد و سایر جاها صفر است؛ بنابراین چون مساحت مربع برابر 2 است، خواهیم داشت
$
k={1\over 2}
$
.

ب)
\qn{
\EX\{X\}&=\int_{-1}^1 \int_{|y|-1}^{1-|y|}xf(x,y)dxdy
\\&=\int_{-1}^1 \int_{|y|-1}^{1-|y|}kxdxdy
\\&=0
}{}
به طریق مشابه
$$
\EX\{Y\}=0
$$
%همچنین
%\qn{
%\EX\{X^2\}&=\int_{-1}^1 \int_{|y|-1}^{1-|y|}x^2f(x,y)dxdy
%\\&=\int_{-1}^1 \int_{|y|-1}^{1-|y|}kx^2dxdy
%\\&=\int_{-1}^1 \int_{0}^{1-|y|}x^2dxdy
%\\&=\int_{-1}^1 {(1-|y|)^3\over 3}dy
%\\&=2\int_{0}^1 {(1-y)^3\over 3}dy
%\\&=2{(1-y)^4\over 4}|_1^0
%\\&={1\over 2}
%}{}
%به طریق مشابه
%$$
%\EX\{Y^2\}={1\over 2}
%$$
%و در نتیجه
%$$
%\sigma_X=\sigma_Y={1\over \sqrt 2}
%$$
می دانیم
\qn{
\text{cov}(X,Y)&=\EX\{(X-\mu_X)(Y-\mu_Y)\}
\\&=\EX\{XY\}=\int_{-1}^1 \int_{|y|-1}^{1-|y|}xyf(x,y)dxdy
\\&={1\over 2}\int_{-1}^1 \int_{|y|-1}^{1-|y|}xydxdy=0
}{}
بنابراین کوواریانس و ضریب همبستگی هر دو برابر صفرند.

پ) دو متغیر تصادفی $X+Y$ و $X-Y$ از چرخش 45 درجه‌ی متغیرهای تصادفی $X$ و $Y$ و سپس مقیاس کردن (تجانس) به اندازه‌ی $\sqrt 2$ حاصل می شوند. بنابراین تابع توزیع توأم آنها نیز دچار چنین تبدیلی خواهد شد و خواهیم داشت
$$
f_{X+Y,X-Y}(u,v)=\begin{cases}
{1\over 4}&,\quad -1<u,v<1\\
0&,\quad \text{در غیر این صورت}
\end{cases}
$$
در این صورت، براحتی دیده می شود که این دو متغیر تصادفی مستقلند و هریک دارای توزیع یکنواخت بین $-1$ و 1 هستند.

ت) برای 
$
|x|>1
$
 داریم $f(x)=0$ و برای 
$
|x|<1
$
\qn{
f_X(x)&=\int_\Bbb R f(x,y)dy
\\&={1\over 2}\int_{|x|-1}^{1-|x|}dy
\\&=1-|x|
}{}
که میانگین آن صفر است و برای واریانس
\qn{
\sigma_X^2&=\EX\{X^2\}
\\&=\int_{-1}^1 x^2-x^2|x|dx
\\&=2\int_{0}^1 x^2-x^3dx
\\&={1\over 6}
}{}
\hl
سوال 2) 
\qn{
\Phi_X(s)&=\EX\{e^{sX}\}
\\&=\int_{-{1\over 2}}^{{1\over 2}} f(x)e^{sx}dx
\\&=\int_{-{1\over 2}}^{{1\over 2}} e^{sx}dx
\\&={1\over s}(e^{s\over 2}-e^{-{s\over 2}})
\\&={2\sinh {s\over 2}\over s}
}{}
سری تیلور $\sinh x$ برابر است با
$$
\sinh x=x+{x^3\over 3!}+{x^5\over 5!}+\cdots 
$$
بنابراین
$$
{\sinh x\over x}=1+{x^2\over 3!}+{x^4\over 5!}+\cdots 
$$
و
$$
{\sinh {s\over 2}\over {s\over 2}}=1+{s^2\over 4\times 3!}+{s^4\over16\times  5!}+\cdots 
$$
و در نتیجه
$$
{d^4\over ds^4}\Phi_X(s)\Big|_{x=0}={1\over 80}
$$
\hl
سوال 3) برای هر 
$
-{\pi\over 2}<u<{\pi\over 2}
$
:
\qn{
&\Pr\left\{\tan^{-1}{Y\over X}<u\right\}=
\Pr\left\{{Y\over X}<\tan u\right\}
\\&=\Pr\left\{{Y\over X}<\tan u,X>0\right\}+\Pr\left\{{Y\over X}<\tan u,X\le0\right\}
%\\&={1\over 2}\Pr\left\{{Y\over X}<\tan u|X>0\right\}+{1\over 2}\Pr\left\{{Y\over X}<\tan u|X<0\right\}
\\&=\Pr\{Y<X\tan u,X>0\}+\Pr\{Y>X\tan u,X<0\}
}{}
می دانیم
\qn{
\Pr\{Y<X\tan u,X>0\}={1\over 2\pi}\int_0^\infty\int_{-\infty}^{x\tan u} \exp(-{y^2\over 2}) \exp(-{x^2\over 2}) dydx
}{}
با تغییر متغیر $x\to -x$ و $y\to -y$، دیده می شود
$$
\Pr\{Y<X\tan u,X>0\}=\Pr\{Y>X\tan u,X<0\}
$$
بنابراین
$$
\Pr\left\{\tan^{-1}{Y\over X}<u\right\}={1\over \pi}\int_0^\infty\int_{-\infty}^{x\tan u} \exp(-{y^2\over 2}) \exp(-{x^2\over 2}) dydx
$$
و با مشتق گیری خواهیم داشت
\qn{
{d\over du}\Pr\left\{\tan^{-1}{Y\over X}<u\right\}&={1\over \pi}\int_0^\infty x(1+\tan^2 u)\exp(-{x^2\tan^2 u\over 2}) \exp(-{x^2\over 2}) dx
%\\&={1\over \pi \cos^2 u}\int_0^\infty \exp(-{x^2\over 2\cos^2 u})dx
\\&={1\over \pi}\exp(-{x^2(1+\tan^2 u)\over 2})\Big|_{x=\infty}^{x=0}
\\&={1\over \pi}
}{}
که نشان می دهد چگالی احتمال، مقدار یکنواخت $1\over \pi$ را در بازه‌ی 
$
[-{\pi\over 2},{\pi\over 2}]
$
 اختیار می کند.
\hl
سوال 4) می دانیم برای هر $y>0$
\qn{
f(y)&=\int_{0}^{\infty} f(x,y)dx
\\&=\int_{0}^{\infty} e^{-x(1+y)^2}dx
\\&={1\over (1+y)^2}e^{-x(1+y)^2}\Big|^{0}_{\infty}
\\&={1\over (1+y)^2}
}{}
همچنین
$$
f(x|y)f(y)=f(x,y)
$$
بنابراین
$$
f(x|y)=(1+y)^2e^{-x(1+y)^2}\quad,\quad x,y>0
$$
\hl
سوال 5) الف) ثابت می کنیم همگرایی، از نوع در احتمال است.
\qn{
\Pr\{|X_n-X|<\epsilon\}&=\Pr\{{1\over n}<\epsilon\}
\\&=\begin{cases}
1&,\quad n>{1\over \epsilon}\\
0&,\quad n\le {1\over \epsilon}
\end{cases}
}{}
که نشان می دهد برای هر $\epsilon>0$، می توان $n$ را چنان بزرگ انتخاب کرد که احتمال فوق برابر 1 شود.

ب) همگرایی از نوع در توزیع است؛ زیرا
$$
F_{X_n}(u)=1-e^{-{n+1\over n}u}\quad,\quad u>0
$$
و در نتیجه
$$
\lim_{n\to \infty}F_{X_n}(u)=F_X(u)\quad,\quad \forall u\in \Bbb R
$$

پ) طبق قضیه ی حد مرکزی، اگر $X_i$ ها، یکنواخت بین $-{1\over 2}$ و $1\over 2$ باشند، دنباله‌ی 
$
X_1+\cdots+X_n\over \sqrt n
$
، در احتمال به یک متغیر گوسی با میانگین صفر و واریانس $1\over 12$ میل می کند. چون در اینجا به دنبال همگرایی 
$
X_1+\cdots+X_n\over \sqrt n
$
هستیم، نتیجه می شود که دنباله‌ی فوق، در احتمال به یک توزیع گوسی با میانگین $1\over 2$ و واریانس صفر میل می کند (یعنی توزیعی که فقط مقدار $1\over 2$ را با احتمال 1 می پذیرد).
\end{document}