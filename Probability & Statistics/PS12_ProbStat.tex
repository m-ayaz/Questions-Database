\documentclass{article}

\usepackage{amsmath,amssymb,geometry,tikz}
\usepackage{xepersian}

\setlength{\parindent}{0pt}
\setlength{\parskip}{3mm}

\newcounter{questionnumber}
\setcounter{questionnumber}{1}

\newcommand{\Q}{
\textbf{سوال \thequestionnumber)}
\stepcounter{questionnumber}
}

\newcommand{\eqn}[1]{
\begin{equation}\begin{split}
#1
\end{split}\end{equation}
}

\begin{document}
\LARGE
\begin{center}
\settextfont{IranNastaliq}

به نام زیبایی

%\begin{figure}[h]
%\centering
%\includegraphics[width=30mm]{kntu_logo.eps}
%\end{figure}

تمرینات سری دوازدهم درس احتمال مهندسی

\end{center}
\hrulefill
\large

\Q

برای هر یک از توابع دومتغیره‌ی زیر، محدوده مقادیر $k$ را به گونه ای بیابید که تابع مورد نظر، چگالی احتمال توأم دو متغیر تصادفی باشد و سپس، توزیع تجمعی توأم را (در صورت وجود) بیابید.

الف)
$
f(x,y)=
\begin{cases}
xy+kx+ky&,\quad 0<x<1,0<y<1\\
0&,\quad \text{سایر جاها}
\end{cases}
$

ب)
$
f(x,y)=
\begin{cases}
k\sin(x+3y)&,\quad 0<x<\frac{\pi}{2},0<y<\frac{\pi}{6}\\
0&,\quad \text{سایر جاها}
\end{cases}
$

%پ) (امتیازی)
%
%$
%f(x,y)=
%\begin{cases}
%\frac{1}{2}&,\quad |x|^k+|y|^k<1\\
%0&,\quad \text{سایر جاها}
%\end{cases}
%$

\Q

برای هر یک از چگالی های احتمال زیر، مقادیر
$
\Pr\{X\le 4,Y\le -2\}
$
،
$
\Pr\{X+Y\le 2\}
$
و

$
\Pr\{X=4Y\}
$
را بیابید.

الف)
$
f_{XY}(x,y)=
\begin{cases}
\frac{1}{2}\sin (x+y)&,\quad 0<x<\frac{\pi}{2},0<y<\frac{\pi}{2}\\
0&,\quad \text{سایر جاها}
\end{cases}
$

ب)
$
f_{XY}(x,y)=
\begin{cases}
\frac{1}{2}\delta\left(\sqrt{(x+4)^2+(y+1)^2}\right)&,\quad x=-4,y=-1\\
\frac{1}{2}&,\quad 0<x<1,0<y<1\\
0&,\quad \text{سایر جاها}
\end{cases}
$

(دقت شود که همانگونه که 
$
\delta(x-x_0)
$
نشان دهنده‌ی ضربه ای در 
$
x=x_0
$
است، 
$
\delta(\sqrt{(x-x_0)^2+(y-y_0)^2})
$
نیز نشان دهنده‌ی ضربه ای در 
$
x=x_0,y=y_0
$
در دو بعد و دارای سطح زیر یک است.
)
\end{document}