\documentclass{article}

\usepackage{amsmath,amssymb,geometry,tikz}
\usepackage{xepersian}

\setlength{\parindent}{0pt}
\setlength{\parskip}{3mm}

\newcounter{questionnumber}
\setcounter{questionnumber}{1}

\newcommand{\Q}{
\textbf{سوال \thequestionnumber)}
\stepcounter{questionnumber}
}

\newcommand{\eqn}[1]{
\[\begin{split}
#1
\end{split}\]
}

\begin{document}
\LARGE
\begin{center}
\settextfont{IranNastaliq}

به نام زیبایی

%\begin{figure}[h]
%\centering
%\includegraphics[width=30mm]{kntu_logo.eps}
%\end{figure}

پاسخ تمرینات سری نهم درس احتمال مهندسی

\end{center}
\hrulefill
\large

\Q

الف)
\eqn{
&\text{Exact}=0.2252
\\&\text{Approx.}=0.2240
\\&\text{Rel. Error}=0.53\%
}

ب)
\eqn{
&\text{Exact}=0.1795
\\&\text{Approx}=0.0812
\\&\text{Rel. Error}=54.76\%
}

از مقایسه‌ی خطاهای نسبی نتیجه می‌شود که در حالاتی که شرایط قضیه‌ی تقریب پواسون برقرار است، خطای تقریب بسیار کوچک می‌شود.

\Q

تابع توزیع انباشته باید صعودی، در $-\infty$ برابر صفر و در $\infty$ برابر یک باشد. در این صورت:

الف)
$
\frac {dF(x)}{dx}=\frac{ke^{-kx}}{(e^{-kx}+1)^2}
$
برای نامنفی بودن مشتق باید داشته باشیم $k\ge 0$. از طرفی، به ازای $k=0$ خواهیم داشت 
$
F(x)=1
$
که نمی‌تواند یک توزیع انباشته باشد. در نتیجه
$
k>0
$
و
$$
f(x)=\frac{ke^{-kx}}{(e^{-kx}+1)^2}.
$$

ب)
\eqn{
f_X(x)=\frac{dF(x)}{dx}=
\begin{cases}
0&,\quad x<0\\
(1+k\cos x)e^{-x-k\sin x}&,\quad x\ge0
\end{cases}
}
در نتیجه باید به ازای هر $x$ داشته باشیم 
$
1+k\cos x\ge 0
$
که معادل است با
$
-1\le k\le 1
$
.
از طرفی، به ازای این مقادیر از $k$، داریم 
$
F(-\infty)=0
$
و
$
F(\infty)=1
$؛ در نتیجه
$
-1\le k\le 1
$
.
%$
%F(x)=\begin{cases}
%0&,\quad x<0\\
%1-e^{-x-k\sin x}&,\quad x\ge0
%\end{cases}
%$

پ) داریم
$
F(1)=1+e^{-k}>1
$
؛ پس این تابع هرگز نمی‌تواند یک توزیع انباشته باشد.
%$
%F(x)=\begin{cases}
%0&,\quad x<0\\
%1+xe^{-kx}&,\quad x\ge 0
%\end{cases}
%$

ت)
%$
%F(x)=\begin{cases}
%0&,\quad x<0\\
%\frac{1}{2}&,\quad 0\le x<1\\
%1-\frac{1}{2}e^{k-kx}&,\quad x\ge 1
%\end{cases}
%$
از 
$
F(\infty)=1
$
نتیجه می‌شود که 
$
k>0
$.
از طرفی به ازای هر مقدار مثبت از $k$، تابع زیر نامنفی است:
$$
f(x)=\frac{dF(x)}{dx}=
\begin{cases}
\frac{1}{2}\delta(x)&,\quad x=0\\
\frac{1}{2}ke^{k-kx}&,\quad x\ge 1
\end{cases}
\text{؛}
$$
پس محدوده مقادیر مجاز $k$، مقادیر مثبت خواهند بود.


\Q

میانه، صدک 50ام است.
\eqn{
\text{صدک $a$ ام}=\sup\{x:F(x)\le a\}
}

الف)
\eqn{
&\text{صدک $a$ ام}:
\\&F(x)=\frac{a}{100}\implies
\frac{1}{1+\exp(-kx)}=\frac{a}{100}\implies
\exp(-kx)=\frac{100}{a}-1
\\&\implies x=-\frac{1}{k}\ln (\frac{100}{a}-1)
\implies\\&
x_{25}=-\frac{1}{k}\ln 3
\quad,\quad
x_{50}=0
\quad,\quad
x_{75}=\frac{1}{k}\ln 3=-x_{25}
}

\eqn{
&\Pr\{X=0\}=F(0)-\lim_{x\to 0^-} F(x)=\frac{1}{2}-\frac{1}{2}=0
\\&\Pr\{0< X\le 2\}=F(2)-F(0)=\frac{1}{1+e^{-2k}}-\frac{1}{2}=\frac{\tanh k}{2}
}

ت)
\eqn{
&\text{صدک $a$ ام}:
\\&F(x)\le\frac{a}{100}\implies
\begin{cases}
0&,\quad x<0\\
\frac{1}{2}&,\quad 0\le x<1\\
1-\frac{1}{2}e^{k-kx}&,\quad x\ge 1
\end{cases}\le\frac{a}{100}
\\&\implies
\begin{cases}
x<0&,\quad a=0\\
x\le0&,\quad 0<a<50\\
x\le1&,\quad a=50\\
x=1-\frac{1}{k}\ln(2-\frac{a}{50})&,\quad 50<a<100\\
\end{cases}
\\&
\implies
x_{25}=0
\quad,\quad
x_{50}=1
\quad,\quad
x_{75}=1+\frac{1}{k}\ln 2
}

\eqn{
&\Pr\{X=0\}=F(0)-\lim_{x\to 0^-} F(x)=\frac{1}{2}-0=\frac{1}{2}
\\&\Pr\{0< X\le 2\}=F(2)-F(0)=1-\frac{1}{2}e^{-k}-\frac{1}{2}=\frac{1}{2}(1-e^{-k})
}
%برای بخش های الف و ت سوال پیش، مقادیر میانه، صدکهای 25ام و 75ام و همچنین احتمال‌های
%$
%\Pr\{X=0\}
%$
%و
%$
%\Pr\{0< X\le 2\}
%$
%را بیابید.

\Q

%یک تاس را پرتاب می‌کنیم. اگر زوج آمد، عدد آن را یادداشت می‌کنیم و اگر فرد آمد، عددی را به تصادف از بازه‌ی 
%$
%[1,6]
%$
%انتخاب کرده و آن را یادداشت می‌کنیم. اگر متغیر تصادفی 
%$
%X
%$،
%نشان دهنده‌ی عدد یادداشت شده باشد، چگالی احتمال و تابع توزیع انباشته‌ی آن را به دست آورده و مقدار 
%$
%\Pr\{1\le X\le 3\}
%$
%را بیابید.

برای این متغیر تصادفی داریم
\eqn{
\Pr\{X=2\}&=
\Pr\{X=2|\text{2 آمدن تاس}\}\Pr\{\text{2 آمدن تاس}\}
\\&+
\Pr\{X=2|\text{2 نیامدن تاس}\}\Pr\{\text{2 نیامدن تاس}\}
\\&=1\times\frac{1}{6}+0\times\frac{5}{6}=\frac{1}{6}
}
به طریق مشابه
$
\Pr\{X=4\}=\Pr\{X=6\}=\frac{1}{6}
$.
اکنون، توزیع انباشته‌ی $X$ را می‌نویسیم:
\eqn{
\Pr\{X\le x\}&=
\Pr\{X\le x|\text{زوج آمدن تاس}\}\Pr\{\text{زوج آمدن تاس}\}
\\&+
\Pr\{X\le x|\text{زوج نیامدن تاس}\}\Pr\{\text{زوج نیامدن تاس}\}
\\&=
\frac{1}{2}\Pr\{X\le x|\text{زوج آمدن تاس}\}
+
\frac{1}{2}\Pr\{X\le x|\text{زوج نیامدن تاس}\}
}
به ازای 
$
x<2
$
،
\eqn{
\Pr\{X\le x\}&=
\frac{1}{2}\Pr\{\emptyset|\text{زوج آمدن تاس}\}
+
\frac{1}{2}\Pr\{1\le X\le x|\text{زوج نیامدن تاس}\}
\\&=\frac{x-1}{10}.
}
به ازای 
$
2\le x<4
$
،
\eqn{
\Pr\{X\le x\}&=
\frac{1}{2}\Pr\{X=2|\text{زوج آمدن تاس}\}
+
\frac{1}{2}\Pr\{1\le X\le x|\text{زوج نیامدن تاس}\}
\\&=\frac{1}{6}+\frac{x-1}{10}.
}
به ازای 
$
4\le x<6
$
،
\eqn{
\Pr\{X\le x\}&=
\frac{1}{2}\Pr\{X=2\text{ یا }X=4|\text{زوج آمدن تاس}\}
+
\frac{1}{2}\Pr\{1\le X\le x|\text{زوج نیامدن تاس}\}
\\&=\frac{1}{3}+\frac{x-1}{10}.
}
به ازای 
$
x=6
$،
\eqn{
\Pr\{X\le x\}&=1
}
در نتیجه
\eqn{
F(x)=\begin{cases}
0&,\quad x<1\\
\frac{x-1}{10}&,\quad 1\le x<2\\
\frac{1}{6}+\frac{x-1}{10}&,\quad 2\le x<4\\
\frac{1}{3}+\frac{x-1}{10}&,\quad 4\le x<6\\
1&,\quad x\ge6\\
\end{cases}
}
و متعاقبأ
\eqn{
f_X(x)=\frac{dF(x)}{dx}=\begin{cases}
\frac{1}{10}&,\quad 1<x<6\\
\frac{1}{6}\delta(x-2)&,\quad x=2\\
\frac{1}{6}\delta(x-4)&,\quad x=4\\
\frac{1}{6}\delta(x-6)&,\quad x=6
\end{cases}
\text{؛}
}
در نتیجه
$$
\Pr\{1\le X\le 3\}=
\Pr\{X\le 3\}-\Pr\{X<1\}=F(3)-\lim_{x\to 1^-}F(x)
=\frac{1}{6}+\frac{1}{5}=\frac{11}{30}
$$

\Q

\textbf{
\{\{\{
بعد از طرح این تمرین، مشابه همین تمرین در کلاس به عنوان تمرین امتیازی مطرح شد! در نتیجه فعلأ حل نمی‌کنیم و منتظر پاسخهای دوستان میمونیم :)
\}\}\}
}
\end{document}