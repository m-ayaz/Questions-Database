\documentclass{article}

\usepackage{amsmath,amssymb,geometry}
\usepackage{xepersian}

\setlength{\parindent}{0pt}
\setlength{\parskip}{3mm}

\newcounter{questionnumber}
\setcounter{questionnumber}{1}

\newcommand{\Q}{
\textbf{سوال \thequestionnumber)}
\stepcounter{questionnumber}
}

\newcommand{\eqn}[1]{
\[\begin{split}
#1
\end{split}\]
}

\begin{document}
\LARGE
\begin{center}
\settextfont{IranNastaliq}

به نام زیبایی

%\begin{figure}[h]
%\centering
%\includegraphics[width=30mm]{kntu_logo.eps}
%\end{figure}

پاسخ تمرینات سری دوم درس احتمال مهندسی

\end{center}
\hrulefill
\large

\Q

اگر پیشامدهای ابتلا به کرونا و آنفلوآنزا را به ترتیب با 
$
A
$
و
$
B
$
نشان دهیم، طبق فرض مسئله داریم
\eqn{
&P(A)=0.07,
\\&P(B)=0.19,
\\&P(A\cup B)=0.2
.
}
در این صورت

الف) خواسته‌ی مسئله، 
$
P(A\cap B)
$
است که برابر است با
$$
P(A\cap B)=P(A)+P(B)-P(A\cup B)=0.07+0.19-0.2=0.06
$$

ب) مطلوبست 
$
P(A-B)
$
. در این صورت
$$
P(A-B)=P(A)-P(A\cap B)=0.07-0.06=0.01.
$$

\Q

الف)
$$
P(A)=P(2)+P(3)+P(5)+P(7)=0.1+0.1+0.1+0.1=0.4
$$

ب)
\eqn{
&A-B=\{2\}
\\&A\cap B=\{3,5,7\}
\\&\implies
\\&P(A-B)=0.1
\\&P(A\cap B)=0.3
}

پ) داریم
\eqn{
&P(A-B)=0.1
\\&P(A)=0.4
\\&P(A\cap B)=0.3
}
بنابراین درستی رابطه‎‌ی زیر مشاهده می‌شود:
\eqn{
P(A-B)=P(A)-P(A\cap B).
}
علت درستی این رابطه آن است که دو مجموعه‌ی 
$
A-B
$
و
$
A\cap B
$
ناسازگارند؛ در نتیجه طبق اصل سوم کولموگروف
$$
P(A\cap B)+P(A-B)=P([A\cap B]\cup[A-B])=P(A).
$$

\Q
نامساوی سمت راست به سادگی از 
$$
P(A\cap B)=P(A)+P(B)-P(A\cup B)
$$
و 
$
P(A\cap B)\ge 0
$
نتیجه می‌شود. برای نامساوی سمت چپ باید اثبات کنیم
\eqn{
P(A\cap B)\le\frac{1}{4\max\{1-P(A),1-P(B)\}}
}
به دلیل تقارن مسئله، فرض می‌کنیم
$
P(A)\ge P(B)
$.
در نتیجه
\eqn{
P(A\cap B)\le\frac{1}{4[1-P(B)]}.
}
از طرفی
\eqn{
&(\frac{1}{2}-P(B))^2\ge0\implies
\\&
4P^2(B)-4P(B)+1\ge0\implies
\\&
1\ge 4P(B)-4P^2(B)\implies
\\&
1\ge 4P(B)[1-P(B)]\implies
\\&
P(B)\le\frac{1}{4[1-P(B)]}.
}
همچنین می دانیم 
$
A\cap B\ \ \ \subseteq\ \ \  B
$
. در نتیجه
\eqn{
P(A\cap B)\le P(B)\le\frac{1}{4[1-P(B)]}
}
و اثبات کامل می شود
$
\blacksquare
$

\end{document}