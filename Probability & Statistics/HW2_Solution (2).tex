\documentclass[10pt,letterpaper]{article}
\usepackage{toolsper}
%\usepackage{xepersian}
%\usepackage{amsmath}
%\settextfont{B Nazanin}
\usepackage{lipsum}
\setlength{\parindent}{0mm}
\setlength{\parskip}{3mm}
\newcommand{\pic}[2]{
\begin{center}
\includegraphics[width=#2]{#1}
\end{center}
}
\begin{document}
\Large
\begin{center}
به نام او

پاسخ تمرینات سری دوم درس احتمال مهندسی

\hrulefill
\end{center}
سوال 1) الف) از آنجا که سکه دارای 2 حالت و تاس دارای 6 حالت است، طبق اصل ضرب 12 حالت مختلف برای پیشامدهای ساده خواهیم داشت؛ یعنی فضای شدنی مسئله‌ی ما 12 حالتی است. از این 12 حالت فقط حالاتی که سکه رو بیاید و تاس یکی از اعداد 1-3-5 شود مدنظر است که تعداد این حالات خاص 3 تاست. در نتیجه احتمال مطلوب 
$
{3\over 12}={1\over 4}
$
خواهد بود.

ب) پیشامد اینکه سکه به رو بیفتد را با $A$ و اینکه تاس فرد شود را با $B$ نمایش می دهیم. هدف محاسبه‌ی 
$
P(A\cup B)
$
 که می دانیم:
$$
P(A\cup B)=P(A)+P(B)-P(A\cap B)
$$
از طرفی
$$
P(A)={1\over 2}\quad,\quad P(B)={1\over 2}\quad,\quad P(A\cap B)={1\over 4}
$$
بنابراین
$$
P(A\cup B)={3\over 4}
$$

سوال 2) الف)
$$
S=\{3,6,\text{پشت},\text{رو}\}
$$

ب) سکه زمانی رو می آید که تاس مضرب 3 نشود و خود سکه هم به رو بیفتد. احتمال اینکه تاس مضرب 3 نشود برابر $2\over 3$ و احتمال اینکه سکه در صورت پرتاب شدن به رو بیفتد برابر $1\over 2$ است؛ پس احتمال مطلوب برابر حاصلضرب دو احتمال قبلی یعنی $1\over 3$ خواهد بود.

پ) اگر پیشامد 1 آمدن تاس را با A و پشت آمدن سکه را با B نمایش دهیم، در این صورت مطلوبست
$$
P(A\cup B)=P(A)+P(B)-P(A\cap B)
$$
از طرفی
$$
P(A)={1\over 6}\quad,\quad P(B)={1\over 3}
$$
تاس با احتمال $1\over 6$، 1 می آید که در این صورت منجر به پرتاب سکه خواهد شد و سکه هم با احتمال $0.5$ به پشت می افتد؛ پس $P(A\cap B)$ برابر 
$
{1\over 6}\times{1\over 2}={1\over 12}
$
 و 
$
P(A\cup B)
$
برابر
$
5\over 12
$
خواهد بود.

سوال 3) هنگامی که از اشکال دوبعدی بهره می گیریم، جهت استفاده از مفهوم اندازه‌ی پیشامدها، باید مساحت آن ها را در نظر بگیریم.

الف) نقطه ای از داخل مربع به مساحت 4 انتخاب شده است. چون پیشامد مطلوب، انتخاب نقطه از داخل دایره است و دایره به طور کامل درون مربع قرار دارد، احتمال مطلوب عبارت است از:
$$
P(A)={\text{مساحت دایره}\over \text{مساحت مربع}}={\pi\over 4}
$$

ب) از آنجا که قطر ضخامتی ندارد (مساحت آن برابر صفر است؛ برای درک این موضوع، به جای قطر یک نوار نازک در نظر بگیرید و ضخامت آن را به سمت صفر میل دهید) احتمال مطلوب برابر 0 خواهد بود.

پ) مکمل این پیشامد عبارتست از اینکه فاصله ی نقطه از دست کم یکی از رأس های مربع کمتر از $0.5$ باشد. به ازای هر راس مربع، مکان هندسی نقاطی از داخل مربع که فاصله‌ی آنها از راس مورد نظر کمتر از $0.5$ باشد، یه ربع دایره به مرکز آن راس و شعاع $0.5$ داخل مربع خواهد بود. 4 راس در مربع داریم؛ پس 4 تا از این ربع دایره ها خواهیم داشت که همپوشانی ندارند؛ پس مساحت مکمل پیشامد مورد نظر عبارتست از:
$$
\text{مساحت پیشامد A'}=4\times \text{مساحت هر ربع دایره}={\pi\over 4}
$$
و برای احتمال مطلوب داریم:
$$
P(A)={\text{مساحت پیشامد A}\over \text{مساحت مربع}}={16-\pi\over 16}=1-{\pi\over 16}
$$

سوال 4) الف) یک عدد زمانی به 3 بخش پذیر است که جمع ارقام آن به 3 بخش پذیر باشد. مجموعه‌ی این اعداد عبارتست از:
$$
S=\{111,222,210,201,120,102\}\implies |S|=6
$$
ب) تمام اعداد 3 رقمی ای که با این ارقام ساخته می شوند، یا دارای صدگان 1 یا 2 هستند. تعداد اعداد سه رقمی و سه رقمی زوج که دارای صدگان 1 یا 2 باشند، به ترتیب برابر 9 و 6 خواهد بود. بنابراین احتمال مطلوب عبارتست از:
$$
P(A)={6+6\over 9+9}={2\over 3}
$$
%چون بزرگترین عددی که می توان ساخت برابر 222 است، تنها اعدادی به 3 بخش پذیرند که جمع ارقامشان 3 یا 6 باشد. 
\end{document}