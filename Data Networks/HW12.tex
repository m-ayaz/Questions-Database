\documentclass[10pt,letterpaper]{article}
\usepackage{amsmath,amssymb,geometry,graphicx}
\usepackage{enumitem}
%\settextfont{B Nazanin}
\usepackage{lipsum}
%\setlength{\parskip}{3mm}
\setlength{\parindent}{0mm}
\newcommand{\wid}{0.49\textwidth}
\newcommand{\widone}{60mm}
\begin{document}
\Large
\begin{center}
In the name of beauty

The assignment \#12 of the ComNet course

\hrulefill
\end{center}


Q1) 

\begin{enumerate}[label=\alph*)]
\item
At which layer is SSL implemented and why?
\item
Why does sequence numbering prevent an intruder from re-ordering seg-
ments? What happens if the sequence numbers are not encrypted by hash
(i.e. only the data and the MAC key are included in hash calculations)?
\end{enumerate}

Q2)

\begin{enumerate}[label=\alph*)]
\item
For what reason, is network-layer security said to provide blanket coverage and what does that mean anyway?
\item
What is Security Policy Database (SPD) and what does it stand for?
\end{enumerate}

Q3)

\begin{enumerate}[label=\alph*)]
\item
What problem can be caused when a duplicate key is used in WEP?
\item
Since MK is a shared key between the client and the authentication server, how can we generate a shared key between the client and the access point?
\end{enumerate}

Q4)

Assume that an attacker wants to perform a DoS (Denial of Service) attack by sending TCP ACK segments to an internal network. A possible solution is to configure the internal firewall to block (i.e. drop) all the incoming TCP ACK segments. What problem does this solution make and how to bypass it?

\end{document}