\documentclass[10pt,letterpaper]{article}
\RequirePackage{amsthm,amssymb,amsmath,graphicx}
\RequirePackage[top=2cm, bottom=2cm, left=2.5cm, right=3cm]{geometry}
\usepackage{caption}
\usepackage{subcaption}
\usepackage[pagebackref=false,colorlinks,linkcolor=black,citecolor=magenta]{hyperref}
\RequirePackage{MnSymbol}
\newcommand{\eqn}[2]{
\begin{equation}
\begin{split}
#1
\label{#2}
\end{split}
\end{equation}
}
%%%%%%%%%%%%%

%       \eqn{
%       x=x^2
%       }{label}

%%%%%%%%%%%%%
\newcommand{\feqn}[2]{
\begin{tcolorbox}[width=7in, colback=white]
\begin{equation}
\begin{split}
#1
\label{#2}
\end{split}
\end{equation}
\end{tcolorbox}
}
%%%%%%%%%%%%%%%
\newcommand{\hl}{
\begin{center}
\line(1,0){450}
\end{center}}
\setlength{\parindent}{0pt}
\newcommand{\nl}{\newline\newline}
\newcommand{\pic}[1]{
\begin{center}
\includegraphics[width=130mm]{#1}
\end{center}
}
%\settextfont{B Nazanin}
\usepackage{lipsum}
\begin{document}
\Large
\begin{center}
The assignment \#9 of the \textbf{ComNet} course
\hl
\end{center}
Q1) 

a. Why does not a base station as a part of wireless networks have a counterpart in wired networks? What is it responsible for?

b. Why does 802.11 use a a link-layer acknowledgment/retransmission (ARQ) scheme while Ethernet does not?
\nl
Q2)

a) Under Ethernet's CSMA/CD, multiple access protocol, a station begins transmitting as soon as the channel is sensed idle. With 802.11 CSMA/CA, however, the station refrains from transmitting while counting down, even when it senses the channel to be idle.Why do CSMA/CD and CDMA/CA take such different approaches here?

b) Consider the 802.11 CSMA/CA once again. Assume that two stations $A$ and $B$ want to send frames. If $A$ and $B$ have experienced $n_A$ and $n_B$ collisions respectively and both, simuletaneously execute the binary exponential back-off for a radom delay before retransmission, what is the probability that the stations $A$ and $B$ do not collide in transmission due to equally generated random delays?
\nl
Q3)

Consider Figure 6.12. of the textbook. Assume that a host $H1$ broadcasts an RTS control frame to all the nodes, including an AP, to further send a DATA frame to another host, say $H2$. The AP, having received the RTS control frame, broadcasts a CTS control frame to $H1$
 which is correctly received by \textbf{all other nodes} in the network, but lost on its way back to $H1$. The host $H1$ waits for a while, then retransmits the RTS control frame. The new RTS control frame is now errorneously detected as DATA in the AP. Explain what happens next.
%\nl
%Q4)
%
%Based on CDMA scheme, assume that the chipping rate is $M$ times faster than the bit rate and the channel reverts each of the $M$ bits of the encoded bit with probability $p$. The channel output is then multiplied in the code, is summed up in its digits and compared to 0-threshold, all at the receiver. The original bit is declared as 0 if the result falls below the 0-threshold and declared as $1$ if otherwise.
%Assume that station $A$ wants to send frames to station $B$. The channel is idle with probability $p$ and the station $A$ senses the channel condition at each time slot with this ratio (i.e. at the $p$ fraction of a whole number of time slots, the channel is sensed idle by the station $A$). Each frame exactly fits each time slot (tightly!). If binary exponential back-off is used, and the probability that an ACK is successfully received by $A$ is $1\over 2$ when $A$ has sent a frame,
\end{document}