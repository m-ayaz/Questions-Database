\documentclass[10pt,letterpaper]{article} 
\usepackage{tikz}
\usepackage{toolsper}
%\usepackage{graphicx}‎‎
%\usefonttheme{serif}‎
%\usepackage{ptext}‎
\usepackage{xepersian}
\settextfont{B Nazanin}
\usepackage{lipsum}
\setlength{\parindent}{0pt}
\newcommand{\nl}{\newline\newline}
\newcommand{\pic}[2]{
\begin{center}
\includegraphics[width=#2]{#1}
\end{center}
}
\newcommand{\EX}{\Bbb E}
\newcommand{\mA}{\lr{mA}}
\newcommand{\mB}{\lr{mB}}
\newcommand{\mC}{\lr{mC}}
\newcommand{\mD}{\lr{mD}}
\newcommand{\mE}{\lr{mE}}
\newcommand{\mF}{\lr{mF}}
\begin{document}
\Large
\begin{center}
به نام خدا

امتحان پایان ترم شبکه های مخابراتی
\hl
\end{center}
سوال 1) درستی یا نادرستی هر یک از عبارات زیر را با بیان دلایل کافی، تحقیق کنید (10 نمره)

1-1) با استفاده از تکنیک \lr{Tunneling} در لایه‌ی شبکه، دیگر نیازی به استفاده از \lr{IPv4} در هیچ یک از نودهای میانی نیست.
\\
\\
\\
1-2) در الگوریتم های مسیریابی متمرکز (\lr{Global} یا \lr{Centralized})، هر یک از نود ها، فقط با در دست داشتن اطلاعاتی درباره‌ی همسایگان خود، اقدام به استفاده از الگوریتم های مسیریابی مانند \lr{Distance Vector} می کند.
\\
\\
\\
1-3) در \lr{Slotted ALOHA} نرخ موثر ارسال برای هر نود، کمتر از \lr{ALOHA} است؛ زیرا به صورت نامتمرکز (\lr{Decentralized}) پیاده سازی می شود.
\\
\\
\\
1-4) در هردو دیاگرام فضا زمان شکلهای زیر، \lr{Collision Detection} در نودهای \lr{A} و  \lr{B}صورت می گیرد.
\pic{Q11}{110mm}
1-5)	پروتکل \lr{MPLS}  در لایه بین \lr{Transport}  و \lr{IP}  قرار دارد و روتینگ انجام شده در آن از نوع \lr{Connectionless}   می باشد.
\\
\\
\\
1-6)	هدف از انجام پروتکل \lr{spanning tree}  بالا بردن سرعت ارسال بسته ها در حالت   \lr{Multicast} می باشد.
\\
\\
\\
1-7)	اگر علی دارای زوج کلید های خصوصی و عمومی باشد و سارا و علی و دیگران یک \lr{Hash function} یکسان را استفاده کنند. در صورتیکه سارا دسترسی به \lr{Certificate} کلید عمومی علی داشته باشد، علی امکان انجام  \lr{Authentication}  سارا را نخواهد داشت. 
\\
\\
\\

1-8)	اگر طول فریمهای \lr{CTS} و \lr{RTS} برابر طول فریمهای \lr{Data} و \lr{ACK} باشند همچنان بهره گیری از فریمهای \lr{CTS} و \lr{RTS} مفید می باشد.
\\
\\
\\
1-9)	اگر یک ارتباط \lr{TCP} روی \lr{Mobile TCP} برقرار شده باشد، قسمتی از ارتباط \lr{TCP} که بین طرف دوم ارتباط با میزبان موبایل می باشد از طریق شبکه میزبان موبایل برقرار می شود ولی انتقال داده به صورت مستقیم بدون در میان بودن شبکه میزبان موبایل انجام می شود. 
\\
\\
\\
1-10)	در یک شبکه با \lr{N} کاربر که هر کاربر با بقیه نیاز به ارتباط دارند و همه کاربران دسترسی به هر داده انتقالی در شبکه را دارند تعداد کلید مورد نیاز برای جلوگیری از افشا شدن هیچ یک از ارتباطات در حالت استفاده از کلید متقارن کمتر از حالت کلید عمومی می باشد. 
\\
\\
\\
\hl
 
سوال 2) در شبکه‌ی زیر، هزینه‌ی تمام لینک ها برابر 1 است:
\pic{Q2}{80mm}
الف) فرض کنید بخواهیم از الگوریتم \lr{Link-State} برای مسیریابی استفاده کنیم. همچنین فرض کنید در زمان صفر، تمام نودها از هزینه‌ی تمام لینک ها آگاهی دارند. اگر هر تکرار از الگوریتم \lr{Dijkstra}، به طور متوسط تاخیری برابر $0.5$ میلی ثانیه داشته باشد، پس از چه مدت، الگوریتم مسیریابی در تمام نودها خاتمه می یابد؟ (3 نمره)
\vspace{10cm}

ب) اکنون فرض کنید بخواهیم از الگوریتم \lr{Distance Vector} برای مسیریابی استفاده کنیم و در زمان صفر، تمام نودها فقط از هزینه‌ی لینک ها تا همسایگان خود آگاهی دارند. اگر هر تکرار از الگوریتم \lr{Distance Vector}، تاخیری برابر $0.5$ میلی ثانیه داشته باشد و ارسال جدول مسیریابی هر نود به همسایگانش، به اندازه‌ی $10$ میکروثانیه طول بکشد، پس از چه مدت، الگوریتم مسیریابی در تمام نودها خاتمه می یابد؟(3 نمره)

\vspace{10cm}

پ) فرض کنید پس از اتمام الگوریتم \lr{Distance Vector}، نود \lr{b}، از طریق نود \lr{a} به نود \lr{c} بسته ارسال می کند. اگر هزینه‌ی لینک بین \lr{a} و \lr{c} به 100 افزایش یابد، الگوریتم \lr{Distance Vector} دچار چه مشکلی می شود و برای رفع آن چه باید کرد؟ (4 نمره)

\vspace{10cm}
\hl
سوال 3) شبکه‌ی زیر را در نظر بگیرید:

با فرض آن که نود 1، می‌خواهد بسته ای را به اندازه‌ی \lr{1Mbytes} در بین تمام نودها \lr{Broadcast} کند، چه مقدار داده در تمام لینک ها به طور کل جابجا می شود اگر

الف) از \lr{Uncontrolled Flooding} استفاده شود؟ (2 نمره)

\vspace{10cm}

ب) از \lr{Controlled Flooding} با رهیافت \lr{Reverse Path Forwarding} استفاده شود؟ (3 نمره)
\pic{Q3}{80mm}
\vspace{10cm}
پ) از \lr{minimum spanning tree} با نقطه‌ی مرکزی نود 2 استفاده شود؟ (همچنین مراحل ساختن چنین درختی را ذکر کنید. طبیعتأ، پاسخ ممکن است یکتا نباشد!) (5 نمره)
\vspace{10cm}
\hl
سوال 4) فرض کنید $n$ نود میخواهند با استفاده از \lr{Slotted ALOHA} روی یک کانال ارسال کنند. فرض کنید نود \lr{A} از بین این $n$ نود، دارای احتمال ارسال $p_\text{\lr{A}}$ در هر شیار زمانی است و سایر نودها، دارای احتمال ارسال $p$ هستند. نرخ متوسط تمام نودها را به دست آورید و سپس با یافتن مقداری از $p$ که بازدهی را برای $p_\text{\lr{A}}$ داده شده ماکزیمم می کند ماکزیمم بازدهی را برای تعداد نامحدود نود ها بدست آورید؟
\vspace{10cm}
\hl

سوال 5) 7-	در شبکه زیر اگر در شبکه \lr{AS2} و \lr{AS3} از الگوریتم \lr{OSPF}  و در شبکه \lr{AS1} و \lr{AS4} از الگوریتم \lr{RIP}  برای روتینگ  \lr{Intra}  استفاده شود. روتینگ بین شبکه ای هم ازپروتکل \lr{eBGP}  و \lr{iBGP}  استفاده می کند. اگر در ابتدا بین شبکه های \lr{AS2}  و \lr{AS4} لینک فیزیکی وجود نداشته باشد. در اینصورت:
\pic{Q6}{110mm}

1)	روتر \lr{3c} اطلاعات   \lr{prefix x} رااز چه پروتکل روتینگ بدست می آورد؟
\\
\\
2)	روتر  \lr{3a} در مورد \lr{prefix x}  از چه پروتکل روتینگ اطلاع می یابد؟
\\
\\
3)	روتر  \lr{1c} در مورد \lr{prefix x}  از چه پروتکل روتینگ اطلاع می یابد؟
\\
\\
4)	روتر \lr{1d}  در مورد \lr{prefix x} از چه پروتکل روتینگ اطلاع می یابد؟
جوابها را با دلایل کافی بیان کنید؟ (6 نمره)



\end{document}