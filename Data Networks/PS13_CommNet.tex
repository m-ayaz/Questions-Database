\documentclass{article}

\usepackage{amsmath,amssymb,geometry,enumitem,graphicx}

\setlength{\parindent}{0pt}
\setlength{\parskip}{1mm}

\newcounter{questionnumber}
\setcounter{questionnumber}{1}

\newcommand{\Q}{
\textbf{Question \thequestionnumber)}
\stepcounter{questionnumber}
}

\newcommand{\eqn}[1]{
\[\begin{split}
#1
\end{split}\]
}

\begin{document}
\LARGE
\begin{center}

In the name of beauty

%\begin{figure}[h]
%\centering
%\includegraphics[width=30mm]{kntu_logo.eps}
%\end{figure}

13th Problem Set of Computer Networks

\end{center}
\hrulefill
\Large

\Q

\begin{enumerate}[label=\alph*)]
\item
At which layer is SSL implemented and why?
\item
Why does sequence numbering prevent an intruder from re-ordering seg-
ments? What happens if the sequence numbers are not encrypted by hash
(i.e. only the data and the MAC key are included in hash calculations)?
\end{enumerate}

\Q

\begin{enumerate}[label=\alph*)]
\item
For what reason, is network-layer security said to provide blanket coverage and what does that mean anyway?
\item
What is Security Policy Database (SPD) and what does it stand for?
\end{enumerate}

\Q

\begin{enumerate}[label=\alph*)]
\item
What problem can be caused when a duplicate key is used in WEP?
\item
Since MK is a shared key between the client and the authentication server, how can we generate a shared key between the client and the access point?
\end{enumerate}

\Q

Assume that an attacker wants to perform a DoS (Denial of Service) attack by sending TCP ACK segments to an internal network. A possible solution is to configure the internal firewall to block (i.e. drop) all the incoming TCP ACK segments. What problem does this solution make and how to bypass it?

\end{document}