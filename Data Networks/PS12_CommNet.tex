\documentclass{article}

\usepackage{amsmath,amssymb,geometry,enumitem,graphicx}

\setlength{\parindent}{0pt}
\setlength{\parskip}{3mm}

\newcounter{questionnumber}
\setcounter{questionnumber}{1}

\newcommand{\Q}{
\textbf{Question \thequestionnumber)}
\stepcounter{questionnumber}
}

\newcommand{\eqn}[1]{
\[\begin{split}
#1
\end{split}\]
}

\begin{document}
\LARGE
\begin{center}

In the name of beauty

%\begin{figure}[h]
%\centering
%\includegraphics[width=30mm]{kntu_logo.eps}
%\end{figure}

12th Problem Set of Computer Networks

\end{center}
\hrulefill
\Large

\Q

A block-cipher encryptor deploys encryption of long messages in 8-bits chunks. This encyption scheme, breaks any 8-bits chunk into two consecutive 4-bits-long strings and maps the strings according to the following randomly-generated mapping table:
\begin{table}[h]
\centering
\Large
\begin{tabular}{cc}
plaintext&ciphertext\\\hline\hline
0000&0011\\\hline
0001&0100\\\hline
0010&0001\\\hline
0011&0010\\\hline
0100&0111\\\hline
0101&0101\\\hline
0110&0110\\\hline
0111&0000\\\hline
\end{tabular}
\begin{tabular}{cc}
plaintext&ciphertext\\\hline\hline
1000&1100\\\hline
1001&1000\\\hline
1010&1101\\\hline
1011&1001\\\hline
1100&1110\\\hline
1101&1010\\\hline
1110&1111\\\hline
1111&1011\\\hline
\end{tabular}
\end{table}
\begin{enumerate}[label=\alph*)]
\item
How likely is it for a nefarious intruder to break an arbitrarily-generated 8-bits ciphertext using a brute-force attack without any knowledge of the mapping table?
\item
How likely is it for a nefarious intruder to break an arbitrarily-generated 8-bits ciphertext using a brute-force attack having exposed exactly $k$ entries of the mapping table to him?
\end{enumerate}

\newpage

\Q

In the following encryption process, assume the circle blocks of + sign mean string concatenation. Also suppose that $K_A^+$, $K_A^-$, $K_B^+$, $K_B^-$ denote Alice's public and private key and Bob's public and private key. Also the Hash function block is denoted by $H(\cdot)$. Assume Alice generates a random key $S$ for using in both the Hash function and string concatenation (as indicated in the figure), before sending messages to Bob.
\begin{figure}[htbp]
\centering
\includegraphics[width=110mm]{q6_1.pdf}
\end{figure}
\begin{enumerate}[label=\alph*)]
\item
Sketch the necessary block diagram of message decryption at receiver (Bob's side).
\item
Which of confidentiality, end-to-end authentication and message integrity have not been preserved?
\item
Make as minimal as possible changes in the sender's (Alice's) encryption block diagram to fulfil the options of part b that were not preserved.
\end{enumerate}

%\Q
%
%The following encryption process is implemented at sender:
%
%
%\begin{enumerate}[label=\alph*)]
%\item
%Which one of confidentiality or message integrity (or both) is compromised in this structure? why?
%\item
%How can the problem of the structure be solved minimally?
%\end{enumerate}
%For guaranteeing message integrity between Alice and Bob, the following hash function is implemented at Alice side:
%$$
%H(\overline{a_0a_1a_2a_3a_4a_5})=
%$$

\newpage

\Q

Assume in a transmission system, that plaintexts can only be 8-digit numbers (each digit can be 0 to 9) and a mapping function maps these 8-digit plaintexts to 10-digits number $a_0a_1a_2\cdots a_9$ where $$a_i=\text{number of }i\text{s in the input}$$
(e.g. 54756473 is mapped to 0001221200). If the 8-digit plain text is $33543612$, what is the probability that an intruder compromises the message having its mapping output?

\Q

Recall the discussion about RSA (page 684, Public Key Encryption). Assume Alice encrypts her message by a public key $(77,13)$ and Bob and Jake each use a private key of $(77,37)$ and $(77,97)$ for decryption, respectively. Show that if Alice encrypts the number $2$ and sends it on the channel, then both Bob and Jake can decrypt it uniquely by calculating the details of RSA.



\end{document}